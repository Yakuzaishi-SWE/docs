\section{Consuntivo di periodo} \label{section:consuntivo}
Questa sezione riporta le spese effettivamente sostenute dal gruppo \groupName.
Vengono riportate le ore ed i costi impiegati in ciascun ruolo per svolgere le attività pianificate.
Viene inoltre presentato un bilancio in termini di costo, dato dalla differenza tra il consuntivo di periodo ed il preventivo, che potrà essere:
\begin{itemize}
    \item \textbf{Positivo:} Se la spesa effettiva è minore di quanto preventivato;
    \item \textbf{In pareggio:} Se la spesa effettiva è uguale a quanto preventivato;
    \item \textbf{Negativo:} Se la spesa effettiva è maggiore di quanto preventivato.
\end{itemize}
Il bilancio viene riportato tra parentesi a fianco dei valori rilevati dal consuntivo di periodo. 
\\Se il valore tra parentesi non è presente indica che l'aspettativa del preventivo è stata rispettata.
\pagebreak
%%%%%%%%%%%%%%%%%%%%%%%%%%%%%%%%%%%%%%%%%%%%%%%%%%%%%%%%%%%%%%

\subsection{Analisi preliminare} \label{subsection:consuntivo_analisi}
Di seguito viene presentato il consuntivo di periodo dell'analisi preliminare.
\begin{table}[H]
    \centering
    \renewcommand{\arraystretch}{1.8}
    \rowcolors{2}{green!100!black!40}{green!100!black!30}
    \begin{tabular}{c|c|c}
      \rowcolor[HTML]{125E28} 
      \multicolumn{1}{c}{\color[HTML]{FFFFFF}\textbf{Ruolo}}
      & \multicolumn{1}{c}{\color[HTML]{FFFFFF}\textbf{Ore}}
      & \multicolumn{1}{c}{\color[HTML]{FFFFFF}\textbf{Costo (€)}}\\
      \hline
      Responsabile      & 26 (+3) & 780,00 (+90,00)\\
      Amministratore    & 45 (+3) & 900,00 (+60,00)\\
      Analista          & 79 (-4) & 1950,00 (-100,00)\\
      Progettista       & - & -\\
      Programmatore     & - & -\\
      Verificatore      & 65 (-6) & 975,00 (-90,00)\\
      \textbf{Totale Consuntivo} & \textbf{214} & \textbf{4605,00}\\
      \textbf{Totale Preventivo} & \textbf{218} & \textbf{4645,00}\\
      \textbf{Bilancio} & \textbf{-4} & \textbf{-40,00}\\
    \end{tabular}
    \caption{Analisi preliminare - Consuntivo di periodo}
  \end{table}


\subsubsection{Ragione degli scostamenti} \label{subsubsection:ragione_scostamenti}
\begin{itemize}
    \item \textbf{Responsabile (+3 ore):} Inizialmente, a causa dell'inesperienza, tale figura ha riscontrato problemi nella suddivisione del carico del lavoro, richiedendo un'analisi più minuziosa;
    \item \textbf{Amministratore (+3 ore):} La stesura delle \docNameNdP ha richiesto più tempo del previsto data l'elevata dipendenza con gli altri documenti;
    \item \textbf{Analista (-4 ore):} Grazie alle conoscenze pregresse di alcuni membri del gruppo si è riusciti ad ottimizzare i tempi previsti per l'\docNameAdR;
    \item \textbf{Verificatore (-6 ore):} Durante questo periodo, avendo solamente documentazione da verificare, il controllo è risultato più rapido del previsto.
\end{itemize}

\subsubsection{Considerazioni rispetto al preventivo} \label{subsubsection:considerazioni_finali}
Il bilancio risulta essere positivo rispetto al preventivo per questo periodo. Non si ritiene comunque necessaria alcuna ripianificazione del prossimo periodo in quanto la somma risparmiata non è abbastanza significativa.
Inoltre, avendo raggiunto tutti gli obiettivi precedentemente pianificati, l'avanzamento delle attività non ha subito alcun rallentamento.
\pagebreak
%%%%%%%%%%%%%%%%%%%%%%%%%%%%%%%%%%%%%%%%%%%%%%%%%%%%%%%%%%%%%%

\subsection{Preventivo a finire} \label{subsection:preventivo_a_finire}
Di seguito viene riportato il preventivo a finire mediante una tabella.
Per ogni periodo che è stato ultimato viene riportato il preventivo ed il consuntivo di periodo.
Se il valore del consuntivo di un periodo non risulta essere ancora presente, per il conteggio del totale verrà utilizzato il valore del preventivo.
\begin{table}[H]
    \centering
    \renewcommand{\arraystretch}{1.8}
    \rowcolors{2}{green!100!black!40}{green!100!black!30}
    \begin{tabular}{c|c|c}
      \rowcolor[HTML]{125E28} 
      \multicolumn{1}{c}{\color[HTML]{FFFFFF}\textbf{Periodo}}
      & \multicolumn{1}{c}{\color[HTML]{FFFFFF}\textbf{Preventivo}}
      & \multicolumn{1}{c}{\color[HTML]{FFFFFF}\textbf{Consuntivo}}\\
      \hline
      Analisi preliminare     & 4645,00 & 4605,00\\

      \textbf{Totale} & \textbf{14225,00} & \textbf{14185,00}\\
    \end{tabular}
    \caption{Preventivo a finire}
  \end{table}
