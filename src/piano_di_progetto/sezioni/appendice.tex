\appendix

\section{Attualizzazione dei rischi} \label{section:attualizzazione_dei_rischi}
\begin{table}[H]
  \centering
  \renewcommand{\arraystretch}{1.8}
  \rowcolors{2}{green!100!black!40}{green!100!black!30}
  \begin{tabular}{c|p{6cm}|p{6cm}}
    \rowcolor[HTML]{125E28}
    \color[HTML]{FFFFFF}\textbf{ID}
    & \multicolumn{1}{c}{\color[HTML]{FFFFFF}\textbf{Descrizione}}
    & \multicolumn{1}{c}{\color[HTML]{FFFFFF}\textbf{Mitigazione}}\\
    \hline
    \rowcolor[HTML]{6BC26B}
    \multicolumn{3}{c}{\textbf{Analisi preliminare}}\\
    \hline
    \textbf{RT1} & Alcuni componenti hanno avuto difficoltà nell'uso del template LaTex e nell'impostare Visual Studio Code. & I componenti con difficoltà sono stati seguiti da quelli più esperti.\\
    \textbf{Conclusioni} & \multicolumn{2}{l}{Dopo aver spiegato in maniera dettagliata come utilizzare l'ambiete scelto non sono risultati altri problemi in merito.}\\
    \hline
    \rowcolor[HTML]{6BC26B}
    \multicolumn{3}{c}{\textbf{Progettazione della Technology Baseline}}\\
    \hline
    \textbf{RI1} & Alcuni componenti hanno sostenuto degli esami universitari. & Le attività sono state distribuite ed organizzate in maniera equa.\\
    \textbf{Conclusioni} & \multicolumn{2}{l}{L'organizzazione dei compiti è stata efficace poiché ha permesso agli interessati di prepararsi adeguatamente agli esami da sostenere, c'è stato un lieve scostamento sulla tabella di marcia.}\\
    \hline
    \textbf{RR2} & Durante un incontro con il proponente è stato suggerito un diverso metodo per la gestione degli smart contract. & Essendo stata suggerita prima dell'inizio della codifica del POC non ha causato grandi problemi.\\
    \textbf{Conclusioni} & \multicolumn{2}{l}{Avendo ritenuto il metodo suggerito migliore, abbiamo sostituito il precedente con quello proposto.}\\
  \end{tabular}
  \caption{Attualizzazione dei rischi}
\end{table}
Impegni
Modifiche requisiti
Inesperienza