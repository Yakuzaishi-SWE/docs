\section{Modello di sviluppo} \label{section:modello_di_sviluppo}
Il gruppo ha deciso di lavorare secondo il \textbf{modello incrementale} per il \textit{ciclo di vita}\glo{} del software per i seguenti motivi:
\begin{itemize}
  \item Può produrre valore ad ogni incremento, aiutando a fissare meglio i requisiti per gli incrementi successivi;
  \item Ogni incremento riduce il rischio di fallimento;
  \item Le funzionalità principali sono sviluppare nei primi incrementi, rendendole via via più stabili.
\end{itemize}
Nel modello incrementale i requisiti vengono classificati in base alla loro importanza strategica. In questo modo, quelli più importanti vengono 
trattati prima. Questo ne aumenta la chiarezza e la facilità di soddisfazione. I requisiti meno importanti invece vengono soddisfatti in seguito, 
venendo così inseriti in un sistema già stabilizzato.\\
Il metodo di lavoro sarà quindi questo:
\begin{itemize}
  \item In ogni fase di lavoro vengono prefissati degli incrementi che devono essere prodotti entro una scadenza decisa dal gruppo;
  \item Il lavoro viene diviso tra i membri del gruppo;
  \item Al termine del periodo prefissato, servirà una riunione per analizzare il lavoro svolto da ogni membro, riscontrare problemi e difficoltà;
  \item Sarà compito dei \textit{verificatori} controllare il lavoro svolto dagli altri membri del gruppo e sollevare eventuali incongruenze o errori;
  \item Alla fine di questa verifica, seguirà una nuova discussione di gruppo per stabilire se gli obiettivi dell'incremento sono stati soddisfatti.
\end{itemize}

\begin{figure}[H]
	\centering
  \includegraphics[scale=0.6]{immagini/modello_incrementale.png}
  \caption{Modello incrementale - tratto da: Ian Sommerville, \textit{Software Engineering}, 8th ed.}
\end{figure}
\pagebreak

%%%%%%%%%%%%%%%%%%%%%%%%%%%%%%%%%%%%%%%%%%%%%%%%%%%%%%%%%%%%%%%%%%%%%%%%%%%

\subsection{Incrementi individuati} \label{subsection:incrementi}
In seguito è riportata la tabella con indicati gli incrementi individuati, con il rispettivo obiettivo e i requisiti e casi d'uso ad esso associati. 
I requisiti riportati includono tutti i requisiti figli. Tutti i requisiti non riportati sono da intendersi soddisfatti, in parte, da 
ogni incremento. \\
Ogni requisito e caso d'uso è definito tramite il suo codice identificativo, per maggiori informazioni fare riferimento all'\docNameVersionAdR{}.

\begin{table}[H]
  \centering
  \renewcommand{\arraystretch}{1.8}
  \rowcolors{2}{green!100!black!40}{green!100!black!30}
  \begin{tabular}{c|p{8cm}|c|c}
    \rowcolor[HTML]{125E28}
    \color[HTML]{FFFFFF}\textbf{Incr.}
    & \multicolumn{1}{c}{\color[HTML]{FFFFFF}\textbf{Obiettivo}}
    & \color[HTML]{FFFFFF}\textbf{Requisiti}
    & \color[HTML]{FFFFFF}\textbf{Casi d'uso}\\
    \hline
    I	& Studio di Fantom come blockchain di riferimento e relative differenze con Ethereum.	& R1V1, R1V2, R2V4, R1V6 & - \\
    II & Studio librerie e pattern per lo sviluppo Frontend di React. & R1V3, R3V5, R1V7 & - \\
    III	& Stesura smart contract per pagamento singolo. & R1F3, R1F5.1, R1F7 & UC2.1, UC5.1, UC6 \\
    IV & Sviluppo struttura del front-end in React e stile SCSS. &	R1F1, R1F2, R1F8 & UC1, UC2, UC3, UC6, UC7 \\
    V	& Integrazione dell'estensione Metamask e interfaccia per interazione con lo smart contract. & R1F4, R1F5 & UC4, UC5 \\
    VI & Aggiunta di Paginazione e Filtrazione dell'elenco delle transazioni per maggiore fruibilità. & R2F8.1, R1F8.2 & UC7.1, UC7.2, UC7.3 \\
    VII	& Aggiunta funzione pagamento condiviso money box nello smart contract. & R2F5.2, R2F2.2.4	UC2.2, UC2.2.4 \\
    VIII & Sviluppo moduli front-end per l'interazione con un pagamento money box. & R2F2.2.1, R2F2.2.2, R2F2.2.5 & UC2.2.1, UC2.2.2, UC2.2.3 \\
    IX & Gestione attesa caricamenti ed errori. & - & UC4.1, UC4.2, UC4.3, UC4.4, UC5.2 \\
    X & Aggiunta funzione dello smart contract per la conversione del denaro in stable coin. & R2F9 & - \\
    XI & Integrazione funzionalità d'identificazione grafica per i dati presenti in blockchain. & R3F2.2.3 & - \\
  \end{tabular}
  \caption{Incrementi individuati}
\end{table}