\section{Introduzione} \label{introduzione}

\subsection{Scopo del documento} \label{intro_scopo_documento}
Il fine del seguente documento è quello di fornire un prospetto riguardo la pianificazione dettagliata e le modalità attraverso cui verrà sviluppato il progetto.\\
All'interno vengono anche riportate le problematiche che il team potrebbe incontrare lungo tutto il periodo.\\
Al termine sono presenti i preventivi e consuntivi di periodo.

\subsection{Scopo del capitolato} \label{intro_scopo_capitolato}
L'avvento delle tecnologie BlockChain\glo\ ha portato e porterà nei prossimi anni a grandi cambiamenti nella società.\\
In particolare, ha aperto le porte a una nuova forma di finanza, la cosiddetta “DeFi” (Finanza Decentralizzata) che ha permesso a chiunque sia dotato di connessione
 internet di creare un Wallet\glo\ e possedere quindi criptovalute\glo.\\
Questo ha delineato due profili critici strettamente legati: da un lato il controllo del proprio portafoglio è passato completamente nelle mani dell'utente, 
dall'altro lato questo comporta la mancanza di un ente terzo che si occupi di gestire transazioni e offrire garanzie.\\
Nel capitolato in questione si vuole proprio risolvere questo problema, in uno scenario che comprende un e-commerce\glo\ basato su BlockChain\glo\ in cui si 
vuole tutelare entrambe le parti coinvolte in un acquisto tramite criptovalute.\\
Il fine del progetto è la realizzazione di un prototipo di una piattaforma integrabile con un “crypto-ecommerce”\glo\, che si occupi di gestire gli ordini dalle 
fasi di pagamento alla consegna.

\subsection{Glossario} \label{intro_glossario}
\gloDesc

\subsection{Riferimenti} \label{intro_riferimenti}
\subsubsection{Riferimenti normativi} \label{intro_riferimenti_normativi}
\begin{itemize}
  \item \textbf{\docNameVersionNdP};
  \item \textbf{Regolamento del progetto didattico:}\\\url{https://www.math.unipd.it/~tullio/IS-1/2021/Dispense/PD2.pdf}
\end{itemize}
\subsubsection{Riferimenti informativi} \label{intro_riferimenti_informativi}
\begin{itemize}
  \item \textbf{\docNameVersionAdR}
  \item \textbf{Capitolato d'appalto C2 - ShopChain:}\\\url{https://www.math.unipd.it/~tullio/IS-1/2021/Progetto/C2.pdf}
\end{itemize}