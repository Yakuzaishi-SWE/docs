\section{Introduzione} \label{section:introduzione}

\subsection{Scopo del documento} \label{subsection:intro_scopo_documento}
Il fine del seguente documento è quello di fornire un prospetto riguardante la pianificazione dettagliata e le modalità attraverso cui verrà sviluppato il progetto.\\
All'interno vengono inoltre riportate le problematiche che il team potrebbe incontrare lungo tutto il periodo.\\
Al termine sono presenti i preventivi e consuntivi di periodo.

\subsection{Scopo del capitolato} \label{subsection:intro_scopo_capitolato}
L'avvento delle tecnologie BlockChain\glo{} ha portato e porterà nei prossimi anni a grandi cambiamenti nella società.\\
In particolare, ha aperto le porte ad una nuova forma di finanza, la cosiddetta “DeFi” (Finanza Decentralizzata) che ha permesso a chiunque sia dotato di connessione
 internet di creare un Wallet\glo{} e possedere quindi criptovalute\glo{}.\\
Questo ha delineato due profili critici strettamente legati: da un lato il controllo del proprio portafoglio è passato completamente nelle mani dell'utente, 
dall'altro lato questo comporta la mancanza di un ente terzo che si occupi di gestire transazioni e offrire garanzie.\\
Nel capitolato in questione si vuole proprio risolvere questo problema, in uno scenario che comprende un e-commerce\glo{} basato su BlockChain\glo{} in cui si 
vogliono tutelare entrambe le parti coinvolte in un acquisto tramite criptovalute.\\
Il fine del progetto è la realizzazione di un prototipo di una piattaforma integrabile con un “crypto-ecommerce”\glo{}, che si occupi di gestire gli ordini dalle 
fasi di pagamento alla consegna.

\subsection{Glossario} \label{subsection:intro_glossario}
\gloDesc

\subsection{Riferimenti} \label{subsection:intro_riferimenti}
\subsubsection{Riferimenti normativi} \label{subsubsection:intro_riferimenti_normativi}
\begin{itemize}
  \item \textbf{\docNameVersionNdP{}};
  \item \textbf{Regolamento del progetto didattico:}
  \begin{center}
    \url{https://www.math.unipd.it/~tullio/IS-1/2021/Dispense/PD2.pdf}
  \end{center}
\end{itemize}
\subsubsection{Riferimenti informativi} \label{subsubsection:intro_riferimenti_informativi}
\begin{itemize}
  \item \textbf{\docNameVersionAdR{};}
  \item \textbf{Capitolato d'appalto C2 - ShopChain:}
  \begin{center}
    \url{https://www.math.unipd.it/~tullio/IS-1/2021/Progetto/C2.pdf}
  \end{center}
  \item \textbf{Software Engineering - Ian Sommerville - 10th Edition:}
  \begin{itemize}
    \item Capitolo 22 - Project management;
    \item Capitolo 23 - Project planning.
  \end{itemize}
  \item \textbf{Il ciclo di vita del software - slide T5 del corso di Ingegneria del Software:}
  \begin{center}
    \url{https://www.math.unipd.it/~tullio/IS-1/2021/Dispense/T05.pdf}
  \end{center}
  \item \textbf{Gestione di Progetto - slide T6 del corso Ingegneria del Software:}
  \begin{center}
    \url{https://www.math.unipd.it/~tullio/IS-1/2021/Dispense/T06.pdf}
  \end{center}
\end{itemize}

\subsection{Scadenze} \label{subsection:intro_scadenze}
La pianificazione presentata in questo documento si basa sulle scadenze che il gruppo \groupName{} si impegna a rispettare, esse sono riassunte nella seguente tabella:
\begin{table}[H]
  \centering
  \renewcommand{\arraystretch}{1.8}
  \rowcolors{2}{green!100!black!40}{green!100!black!30}
  \begin{tabular}{p{7cm}|p{2cm}|p{2cm}}
    \rowcolor[HTML]{125E28} 
    \multicolumn{1}{c}{\color[HTML]{FFFFFF}\textbf{Revisione}}
    & \multicolumn{1}{c}{\color[HTML]{FFFFFF}\textbf{Acronimo}}
    & \multicolumn{1}{c}{\color[HTML]{FFFFFF}\textbf{Data}}\\
    \hline
    \RTB{} & RTB & 2022/02/22 \\
    \PB{} & PB & 2022/04/04 \\
    \CA{} & CA & 2022/04/25 \\
  \end{tabular}
  \caption{Scadenze}
\end{table}

\subsection{Note sulle tabelle} \label{subsection:intro_note_tabelle}
Per chiarezza grafica e per aumentare la leggibilità delle tabelle, verrà utilizzato un trattino (-) nelle celle in cui è presente uno 0 (zero).
Inoltre si precisa che laddove venga indicato un costo all'interno di una tabella, quest'ultimo verrà espresso in euro (€).