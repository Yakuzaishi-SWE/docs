\section{Metriche di qualità} \label{section:metriche_qualita}

\subsection{Metriche per la qualità di processo} \label{subsection:qualita_processo}
La presente sezione espone le metriche selezionate dal gruppo \groupName{} per misurare il raggiungimento degli obiettivi di qualità del processo.
\paragraph*{Parametri per comprendere le metriche scelte:}
\begin{itemize}
    \item Budget At Completion (BAC): È il budget stimato in preventivo per il completamento del progetto;
    \item Actual Calendar Days (ACD): È il numero effettivo di giorni impiegati in un determinato periodo;
    \item Planned Calendar Days (PCD): È la durata preventivata di un determinato periodo;
    \item Number of Requirements Added (NRA): Numero di requisiti aggiunti in un determinato periodo;
    \item Number of Requirements Changed (NRC): Numero di requisiti modificati in un determinato periodo;
    \item Number of Requirements Removed (NRR): Numero di requisiti rimossi in un determinato periodo;
    \item Total Number of Initial Requirements (TNIR): Numero totale dei requisiti all'inizio di un determinato periodo.
\end{itemize}

\subsubsection{MPC01: Earned Value (EV)}
È il valore del lavoro effettuato fino ad un determinato periodo.

\subsubsection{MPC02: Actual Cost (AC)}
È il costo effettivamente sostenuto fino ad un determinato periodo.

\subsubsection{MPC03: Planned Value (PV)}
È il costo stimato per il completamento di un determinato periodo.

\subsubsection{MPC04: Cost Variance (CV)}
Metrica che indica, in percentuale, se il gruppo è regolare rispetto al budget prestabilito.\\
La formula adottata è la seguente:
\begin{center}
    $CV = (\frac{EV}{AC}-1)*100$
\end{center}
Se il valore risulta negativo si è fuori budget.

\subsubsection{MPC05: Schedule Variance (SV)}
Metrica che indica, in percentuale, se il gruppo è regolare rispetto i limiti temporali imposti.\\
La formula adottata è la seguente:
\begin{center}
    $SV = (1-\frac{ACD}{PCD})*100$
\end{center}
Se il valore risulta negativo si è indietro rispetto i tempi preventivati.

\subsubsection{MPC06: Estimated At Completion (EAC)}
Revisione del valore stimato per la realizzazione del progetto in un determinato periodo, ossia il BAC rivisto allo stato corrente del progetto.\\
La formula adottata è la seguente:
\begin{center}
    $EAC = AC + ETC$
\end{center}

\subsubsection{MPC07: Estimate To Complete (ETC)}
Valore stimato del costo del lavoro rimanente in un determinato periodo.\\
La formula adottata è la seguente:
\begin{center}
    $ETC = BAC - EV$
\end{center}

\subsubsection{MPC08: Requirements Stability Index (RSI)}
Indice che misura la variazione dei requisiti nel tempo.\\
La formula adottata è la seguente:
\begin{center}
    $RSI = (1-\frac{NRA + NRC + NRR}{TNIR})*100$
\end{center}

\subsubsection{MPC09: Passed Tests}
Indica la percentuale dei test che sono stati superati fino ad un determinato periodo.\\
La formula adottata è la seguente:
\begin{center}
    $Passed\ Tests = \frac{Number\ of\ Passed\ Tests}{Tot.\ Number\ of\ Tests}*100$
\end{center}

\subsubsection{MPC10: Metrics Satisfied}
Indice che rappresenta la percentuale di metriche di qualità soddisfatte in un determinato periodo.
La formula adottata è la seguente:
\begin{center}
    $Metrics\ Satisfied = \frac{Number\ of\ Metrics\ Satisfied}{Tot.\ Number\ of\ Metrics}*100$
\end{center}

\subsubsection{MPC11: Risks Found}
Indica il numero di rischi incontrati in un determinato periodo.

\vspace{2cm}

\pagebreak
%%%%%%%%%%%%%%%%%%%%%%%%%%%%%%%%%%%%%%%%%%%%%%%%
\subsection{Metriche per la qualità di prodotto} \label{subsection:qualita_prodotto}
La presente sezione espone le metriche selezionate dal gruppo \groupName{} per misurare il raggiungimento degli obiettivi di qualità del prodotto.

\subsubsection{MPD01: Indice di Gulpease}
Indice che riporta il grado di leggibilità di un testo redatto in lingua italiana.
La formula adottata è la seguente:
\begin{center}
    $GULP = 89 + \displaystyle \frac{300*(totale\ frasi)-10*(totale\ lettere)}{(totale\ parole)}$
\end{center}
L’indice così calcolato può assumere valori tra 0 e 100:
\begin{itemize}
    \item \textbf{GULP $<$ 80:} indica una leggibilità difficile per un utente con licenza elementare;
    \item \textbf{GULP $<$ 60:} indica una leggibilità difficile per un utente con licenza media;
    \item \textbf{GULP $<$ 40:} indica una leggibilità difficile per un utente con licenza superiore.
\end{itemize}

Per facilitare il calcolo di questo indice, il gruppo si è appoggiato al sito \href{https://farfalla-project.org/readability_static/}, copiando
l'intero contenuto dei documenti in formato pdf escludendo la prima pagina e gli indici.

\subsubsection{MPD02: Errori ortografici}
Il controllo ortografico viene eseguito secondo quanto descritto in \ref{verifica_ortografica} del presente documento.

\subsubsection{MPD03: Copertura requisiti obbligatori}
Indice che misura in ogni istante la percentuale di requisiti obbligatori soddisfatti.
La formula adottata è la seguente:
\begin{center}
    $CRO = \displaystyle \frac{ROC}{RO}*100$
\end{center}
dove:
\begin{itemize}
    \item \textbf{ROC:} indica il numero di requisiti obbligatori coperti dall'implementazione;
    \item \textbf{RO:} indica il numero complessivo di requisiti obbligatori.
\end{itemize}

\subsubsection{MPD04: Copertura requisiti opzionali accettati}
Indice che misura in ogni istante la percentuale di requisiti opzionali accettati soddisfatti.
La formula adottata è la seguente:
\begin{center}
    $CRA = \displaystyle \frac{RAC}{RA}*100$
\end{center}
dove:
\begin{itemize}
    \item \textbf{ROC:} indica il numero di requisiti opzionali accettati coperti dall'implementazione;
    \item \textbf{RO:} indica il numero complessivo di requisiti opzionali accettati.
\end{itemize}

\subsubsection{MPD05: Densità di failure}
Percentuale che indica l'affidabilità del prodotto e si può ricavare dalla percentuale di test falliti sui test eseguiti.
La formula adottata è la seguente:
\begin{center}
    $DF = \displaystyle \frac{Tfail}{Ttot}*100$
\end{center}
dove:
\begin{itemize}
    \item \textbf{Tfail} indica il numero di test eseguiti sul programma ma falliti;
    \item \textbf{Ttot} indica il numero complessivo di test eseguiti sul programma.
\end{itemize}

\subsubsection{MPD06: Numero di bug}
Numero di righe di codice che potrebbero comportare un comportamento imprevisto o diverso da quello desiderato.
La metrica viene rilevata automaticamente mediante l’utilizzo di {SonarCloud}\glo.

\subsubsection{MPD07: Remediation effort}
Lavoro necessario a risolvere bug. Il valore viene misurato in ore di lavoro.
La metrica viene rilevata automaticamente mediante l’utilizzo di {SonarCloud}\glo.

\subsubsection{MPD08: Complessità cognitiva}
Misura della difficoltà di comprensione del codice.
La metrica viene rilevata automaticamente mediante l’utilizzo di {SonarCloud}\glo.

\subsubsection{MPD09: Complessità ciclomatica}
Indice software utilizzato per stimare la complessità di un programma attraverso l’analisi del codice sorgente.
Viene calcolata impiegando il grafo del flusso di controllo, dove:
\begin{itemize}
    \item i nodi sono gruppi indivisibili di istruzioni;
    \item un arco connette due nodi se le istruzioni di un nodo possono essere eseguite subito dopo le istruzioni dell’altro.
\end{itemize}
Un valore troppo elevato di tale metrica è il risultato di codice troppo complesso, difficilmente testabile e manutenibile.
Esistono due metodologie per calcolare la complessità ciclomatica; il primo adotta la formula:
\begin{center}
    $V(G) = \displaystyle E - N + 2P$
\end{center}
dove:
\begin{itemize}
    \item \textbf{E} è il numero di archi;
    \item \textbf{N} è il numero di nodi;
    \item \textbf{N} è il numero di di componenti connesse.
\end{itemize}
La metrica viene rilevata automaticamente mediante l’utilizzo di {SonarCloud}\glo.

\subsubsection{MPD10: Facilità di utilizzo}
Numero di click necessari con cui l'utente raggiunge la funzionalità cercata.

\subsubsection{MPD11: Facilità di apprendimento}
Numero di minuti necessari all'utente per apprendere le funzionalità del prodotto software.

\subsubsection{MPD12: Tempo medio di risposta}
Tempo medio impiegato dal software per rispondere a una richiesta utente o svolgere un'attività di sistema.

\subsubsection{MPD13: Numero di code smell}
Numero di code smell individuati, corrispondenti a righe di codice difficilmente estendibili.
La metrica viene rilevata automaticamente mediante l’utilizzo di {SonarCloud}\glo.

\subsubsection{MPD14: Technical Debt}
Lavoro necessario a sanare code smell. Il valore viene misurato in ore di lavoro.
La metrica viene rilevata automaticamente mediante l’utilizzo di {SonarCloud}\glo.

\subsubsection{MPD15: Densità di duplicazione}
Percentuale di righe di codice ripetute tra i vari file sorgente.
Un alto valore di questo indice significa una bassa qualità del codice prodotto e una sua bassa ottimizzazione.
La metrica viene rilevata automaticamente mediante l’utilizzo di {SonarCloud}\glo.

\subsubsection{MPD16: Versioni browser supportate}
Percentuale di versioni di browser supportate dal prodotto.
Calcolabile con la seguente formula:
\begin{center}
    $BS = \displaystyle \frac{Bsup}{Btot}*100$
\end{center}
dove:
\begin{itemize}
    \item \textbf{Bsup} indica il numero di browser in cui il prodotto può essere eseguito;
    \item \textbf{Btot} indica il numero complessivo di browser presi in considerazione.
\end{itemize}

\subsubsection{MPD17: Successo dei test}
Percentuale di test implementati e superati correttamente.
La formula adottata è la seguente:
\begin{center}
    $PST = \displaystyle \frac{Tsup}{Ttot}*100$
\end{center}
dove:
\begin{itemize}
    \item \textbf{Tsup} indica il numero di test superati;
    \item \textbf{Ttot} indica il numero complessivo di test implementati.
\end{itemize}
La metrica viene rilevata automaticamente mediante l’utilizzo di {SonarCloud}\glo.

\subsubsection{MPD18: Line coverage}
Percentuale di linee di codice verificate da test automatici.
La metrica viene rilevata automaticamente mediante l’utilizzo di {SonarCloud}\glo.

\subsubsection{MPD19: Branch coverage}
Percentuale del rapporto tra tutte le possibili deviazioni del flusso del codice (costrutti if else, operatori ternari, lancio di eccezioni e altre istruzioni similari).
La metrica viene rilevata automaticamente mediante l’utilizzo di {SonarCloud}\glo.

\subsubsection{MPD20: Code coverage}
Percentuale di copertura del codice sorgente da parte dei test. Si ricava da diversi fattori di coverage secondo la seguente formula:
\begin{center}
    $coverage = \displaystyle \frac{CT + CF + LC}{2 * B + EL}$
\end{center}
dove:
\begin{itemize}
    \item \textbf{CT} indica i branch valutati positivi almeno una volta;
    \item \textbf{CF} indica i branch valutati negativi almeno una volta;
    \item \textbf{LC} indica le righe di codice coperte da test;
    \item \textbf{B} indica il numero totale di branch;
    \item \textbf{EL} indica il numero totale di righe di codice da coprire coi test.
\end{itemize}
La metrica viene rilevata automaticamente mediante l’utilizzo di {SonarCloud}\glo.
