\section{Metriche di qualità} \label{section:metriche_qualita}

\subsection{Metriche per la qualità di processo} \label{subsection:qualita_processo}
La presente sezione espone le metriche selezionate dal gruppo \groupName{} per misurare il raggiungimento degli obiettivi di qualità del processo.
\paragraph*{Parametri per comprendere le metriche scelte:}
\begin{itemize}
    \item Budget At Completion (BAC): È il budget stimato in preventivo per il completamento del progetto;
    \item Actual Calendar Days (ACD): È il numero effettivo di giorni impiegati in un determinato periodo;
    \item Planned Calendar Days (PCD): È la durata preventivata di un determinato periodo;
    \item Number of Requirements Added (NRA): Numero di requisiti aggiunti in un determinato periodo;
    \item Number of Requirements Changed (NRC): Numero di requisiti modificati in un determinato periodo;
    \item Number of Requirements Removed (NRR): Numero di requisiti rimossi in un determinato periodo;
    \item Total Number of Initial Requirements (TNIR): Numero totale dei requisiti all'inizio di un determinato periodo.
\end{itemize}

\subsubsection{MPC01: Earned Value (EV)}
È il valore del lavoro effettuato fino ad un determinato periodo.

\subsubsection{MPC02: Actual Cost (AC)}
È il costo effettivamente sostenuto fino ad un determinato periodo.

\subsubsection{MPC03: Planned Value (PV)}
È il costo stimato per il completamento di un determinato periodo.

\subsubsection{MPC04: Cost Variance (CV)}
Metrica che indica, in percentuale, se il gruppo è regolare rispetto al budget prestabilito.\\
La formula adottata è la seguente:
\begin{center}
    $CV = \displaystyle (\frac{EV}{AC}-1)*100$
\end{center}
Se il valore risulta negativo si è fuori budget.

\subsubsection{MPC05: Schedule Variance (SV)}
Metrica che indica, in percentuale, se il gruppo è regolare rispetto i limiti temporali imposti.\\
La formula adottata è la seguente:
\begin{center}
    $SV = \displaystyle (1-\frac{ACD}{PCD})*100$
\end{center}
Se il valore risulta negativo si è indietro rispetto i tempi preventivati.

\subsubsection{MPC06: Estimated At Completion (EAC)}
Revisione del valore stimato per la realizzazione del progetto in un determinato periodo, ossia il BAC rivisto allo stato corrente del progetto.\\
La formula adottata è la seguente:
\begin{center}
    $EAC = AC + ETC$
\end{center}

\subsubsection{MPC07: Estimate To Complete (ETC)}
Valore stimato del costo del lavoro rimanente in un determinato periodo.\\
La formula adottata è la seguente:
\begin{center}
    $ETC = BAC - EV$
\end{center}

\subsubsection{MPC08: Requirements Stability Index (RSI)}
Indice che misura la variazione dei requisiti nel tempo.\\
La formula adottata è la seguente:
\begin{center}
    $RSI = \displaystyle (1-\frac{NRA + NRC + NRR}{TNIR})*100$
\end{center}

\subsubsection{MPC09: Passed Tests}
Indica la percentuale dei test che sono stati superati fino ad un determinato periodo.\\
La formula adottata è la seguente:
\begin{center}
    $Passed\ Tests = \displaystyle \frac{Number\ of\ Passed\ Tests}{Tot.\ Number\ of\ Tests}*100$
\end{center}

\subsubsection{MPC10: Metrics Satisfied}
Indice che rappresenta la percentuale di metriche di qualità soddisfatte in un determinato periodo.
La formula adottata è la seguente:
\begin{center}
    $Metrics\ Satisfied = \displaystyle \frac{Number\ of\ Metrics\ Satisfied}{Tot.\ Number\ of\ Metrics}*100$
\end{center}

\subsubsection{MPC11: Risks Found}
Indica il numero di rischi incontrati in un determinato periodo.

\pagebreak
%%%%%%%%%%%%%%%%%%%%%%%%%%%%%%%%%%%%%%%%%%%%%%%%
\subsection{Metriche per la qualità di prodotto} \label{subsection:qualita_prodotto}
La presente sezione espone le metriche selezionate dal gruppo \groupName{} per misurare il raggiungimento degli obiettivi di qualità del prodotto.

\subsubsection{MPD01: Indice di Gulpease}
Indice che riporta il grado di leggibilità di un testo redatto in lingua italiana.
La formula adottata è la seguente:
\begin{center}
    $GULP = 89 + \displaystyle \frac{300*(totale\ frasi)-10*(totale\ lettere)}{(totale\ parole)}$
\end{center}
L'indice così calcolato può assumere valori tra 0 e 100:
\begin{itemize}
    \item \textbf{GULP $<$ 80:} indica una leggibilità difficile per un utente con licenza elementare;
    \item \textbf{GULP $<$ 60:} indica una leggibilità difficile per un utente con licenza media;
    \item \textbf{GULP $<$ 40:} indica una leggibilità difficile per un utente con licenza superiore.
\end{itemize}

Per facilitare il calcolo di questo indice, il gruppo si è appoggiato al sito \href{https://farfalla-project.org/readability_static/}, copiando
l'intero contenuto dei documenti in formato pdf escludendo la prima pagina e gli indici.

\subsubsection{MPD02: Errori ortografici}
Il controllo ortografico viene eseguito secondo quanto descritto in \ref{verifica_ortografica} del presente documento.

\subsubsection{MPD03: Copertura requisiti obbligatori}
Indice che misura in ogni istante la percentuale di requisiti obbligatori soddisfatti.
La formula adottata è la seguente:
\begin{center}
    $CROB = \displaystyle \frac{ROBC}{ROB}*100$
\end{center}
dove:
\begin{itemize}
    \item \textbf{ROBC:} indica il numero di requisiti obbligatori coperti dall'implementazione;
    \item \textbf{ROB:} indica il numero complessivo di requisiti obbligatori.
\end{itemize}

\subsubsection{MPD04: Copertura requisiti desiderabili}
Indice che misura in ogni istante la percentuale di requisiti desiderabili soddisfatti.
La formula adottata è la seguente:
\begin{center}
    $CRD = \displaystyle \frac{RDC}{RD}*100$
\end{center}
dove:
\begin{itemize}
    \item \textbf{RDC:} indica il numero di requisiti desiderabili coperti dall'implementazione;
    \item \textbf{RD:} indica il numero complessivo di requisiti opzionali.
\end{itemize}

\subsubsection{MPD05: Copertura requisiti opzionali}
Indice che misura in ogni istante la percentuale di requisiti opzionali soddisfatti.
La formula adottata è la seguente:
\begin{center}
    $CROP = \displaystyle \frac{ROPC}{ROP}*100$
\end{center}
dove:
\begin{itemize}
    \item \textbf{ROPC:} indica il numero di requisiti opzionali accettati coperti dall'implementazione;
    \item \textbf{ROP:} indica il numero complessivo di requisiti opzionali accettati.
\end{itemize}

\subsubsection{MPD06: Facilità di utilizzo}
Numero di click necessari con cui l'utente raggiunge la funzionalità cercata.

\subsubsection{MPD07: Versioni browser supportate}
Percentuale di versioni di browser supportate dal prodotto.
Calcolabile con la seguente formula:
\begin{center}
    $BS = \displaystyle \frac{Bsup}{Btot}*100$
\end{center}
dove:
\begin{itemize}
    \item \textbf{Bsup} indica il numero di browser in cui il prodotto può essere eseguito;
    \item \textbf{Btot} indica il numero complessivo di browser presi in considerazione.
\end{itemize}
Dove in \textbf{Btot} i browser presi in considerazione sono i seguenti:
\begin{itemize}
    \item Chrome;
    \item Firefox;
    \item Microsoft Edge;
    \item Brave.
\end {itemize} 
L'integrazione con Metamask in Safari non è supportata, di conseguenza l'applicativo non funziona in tale browser.

\subsubsection{MPD08: Solidity Statement Coverage}
Viene utilizzato per calcolare il numero di istruzioni nel codice sorgente che sono state eseguite.\\
Lo scopo dello Statement Coverage è quello di coprire tutti i possibili percorsi, righe e istruzioni nel codice sorgente.\\
Il valore è stato calcolato per mezzo di solidity-coverage.

\subsubsection{MPD09: Solidity Branch Coverage}
Lo scopo del Branch Coverage è garantire che ogni condizione decisionale venga eseguita almeno una volta.\\
Il valore è stato calcolato per mezzo di solidity-coverage.

\subsubsection{MPD10: Solidity Function Coverage}
Indica la misura in cui le funzioni presenti nel codice sorgente sono coperte durante il test.\\
Tutte le funzioni presenti nel codice sorgente vengono testate durante l'esecuzione del test.\\
Il valore è stato calcolato per mezzo di solidity-coverage.

\subsubsection{MPD11: Solidity Line Coverage}
Il Line Coverage di un programma è il numero di righe eseguite diviso per il numero totale di righe.\\
Vengono considerate solo le righe che contengono istruzioni eseguibili.
Il valore è stato calcolato per mezzo di solidity-coverage.
