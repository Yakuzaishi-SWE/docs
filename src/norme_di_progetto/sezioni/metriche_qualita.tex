\section{Metriche di qualità} \label{section:metriche_qualita}

\subsection{Metriche per la qualità di processo} \label{subsection:qualita_processo}

%%%%%%%%%%%%%%%%%%%%%%%%%%%%%%%%%%%%%%%%%%%%%%%%

\subsection{Metriche per la qualità di prodotto} \label{subsection:qualita_prodotto}
La presente sezione espone le metriche selezionate dal gruppo \groupName\ per misurare il raggiungimento degli obiettivi di qualità del prodotto.
    
    \subsubsection{MPD01: Indice di Gulpease}
    Indice che riporta il grado di leggibilità di un testo redatto in lingua italiana.
    La formula adottata è la seguente: 
    \begin{center}
        $GULP = 89 + \displaystyle \frac{300*(totale\ frasi)-10*(totale\ lettere)}{(totale\ parole)}$
    \end{center}
    L’indice così calcolato può assumere valori tra 0 e 100:
    \begin{itemize}
        \item \textbf{GULP $<$ 80:} indica una leggibilità difficile per un utente con licenza elementare;
        \item \textbf{GULP $<$ 60:} indica una leggibilità difficile per un utente con licenza media;
        \item \textbf{GULP $<$ 40:} indica una leggibilità difficile per un utente con licenza superiore;
    \end{itemize}

    \subsubsection{MPD02: Errori ortografici}
    Il controllo ortografico viene eseguito secondo quanto descritto in \ref{verifica_ortografica} del presente documento.

    \subsubsection{MPD03: Copertura requisiti}
    Indice che misura in ogni istante la percentuale di requisiti soddisfatti.
    La formula adottata è la seguente:
    \begin{center}
        $R = \displaystyle \frac{RC}{R}*100$
    \end{center}
    dove:
    \begin{itemize}
        \item \textbf{RC:} indica il numero di requisiti coperti dall'implementazione;
        \item \textbf{R:} indica il numero complessivo di requisiti.
    \end{itemize}