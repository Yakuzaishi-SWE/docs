\section{Metriche di qualità} \label{section:metriche_qualita}

\subsection{Metriche per la qualità di processo} \label{subsection:qualita_processo}
La presente sezione espone le metriche selezionate dal gruppo \groupName{} per misurare il raggiungimento degli obiettivi di qualità del processo.
    \paragraph*{Parametri per comprendere le metriche scelte:}
    \begin{itemize}
        \item Budget At Completion (BAC): È il budget stimato in preventivo per il completamento del progetto;
        \item Actual Calendar Days: È il numero effettivo di giorni impiegati in un determinato periodo;
        \item Planned Calendar Days: È la durata preventivata di un determinato periodo;
        \item Number of Requirements Added (NRA): Numero di requisiti aggiunti in un determinato periodo;
        \item Number of Requirements Changed (NRC): Numero di requisiti modificati in un determinato periodo;
        \item Number of Requirements Removed (NRR): Numero di requisiti rimossi in un determinato periodo;
        \item Total Number of Initial Requirements (TNIR): Numero totale dei requisiti all'inizio di un determinato periodo.
    \end{itemize}

    \subsubsection{MPC01: Cost Variance (CV)}
    Metrica che indica, in percentuale, se il gruppo è regolare rispetto al budget prestabilito.\\
    La formula adottata è la seguente:
    \begin{center}
        $CV = (\frac{EV}{AC}-1)*100$
    \end{center}
    Se il valore risulta negativo si è fuori budget.
    
    \subsubsection{MPC02: Schedule Variance (SV)}
    Metrica che indica, in percentuale, se il gruppo è regolare rispetto i limiti temporali imposti.\\
    La formula adottata è la seguente:
    \begin{center}
        $SV = (1-\frac{Actual\ Calendar\ Days}{Planned\ Calendar\ Days})*100$
    \end{center}
    Se il valore risulta negativo si è indietro rispetto i tempi preventivati.

    \subsubsection{MPC03: Estimated At Completion (EAC)}
    Revisione del valore stimato per la realizzazione del progetto in un determinato periodo, ossia il BAC rivisto allo stato corrente del progetto.\\
    La formula adottata è la seguente:
    \begin{center}
        $EAC = AC + ETC$
    \end{center}

    \subsubsection{MPC04: Estimate To Complete (ETC)}
    Valore stimato del costo del lavoro rimanente in un determinato periodo.\\
    La formula adottata è la seguente:
    \begin{center}
        $ETC = BAC - EV$
    \end{center}

    \subsubsection{MPC05: Earned Value (EV)}
    È il valore del lavoro effettuato fino ad un determinato periodo.

    \subsubsection{MPC06: Actual Cost (AC)}
    È il costo effettivamente sostenuto fino ad un determinato periodo.

    \subsubsection{MPC07: Planned Value (PV)}
    È il costo stimato per il completamento di un determinato periodo.

    \subsubsection{MPC08: Requirements Stability Index (RSI)}
    Indice che misura la variazione dei requisiti nel tempo.\\
    La formula adottata è la seguente:
    \begin{center}
        $RSI = (1-\frac{NRA + NRC + NRR}{TNIR})*100$
    \end{center}
    
    \subsubsection{MPC09: Passed Tests}
    Indica la percentuale dei test che sono stati superati fino ad un determinato periodo.\\
    La formula adottata è la seguente:
    \begin{center}
        $Passed\ Tests = \frac{Number\ of\ Passed\ Tests}{Tot.\ Number\ of\ Tests}*100$
    \end{center}
    
    \subsubsection{MPC10: Metrics Satisfied}
    Indice che rappresenta la percentuale di metriche di qualità soddisfatte in un determinato periodo.
    La formula adottata è la seguente:
    \begin{center}
        $Metrics\ Satisfied = \frac{Number\ of\ Metrics\ Satisfied}{Tot.\ Number\ of\ Metrics}*100$
    \end{center}

    \subsubsection{MPC11: Risks Found}
    Indica il numero di rischi incontrati in un determinato periodo.

    \vspace{2cm}

%%%%%%%%%%%%%%%%%%%%%%%%%%%%%%%%%%%%%%%%%%%%%%%%
\subsection{Metriche per la qualità di prodotto} \label{subsection:qualita_prodotto}
La presente sezione espone le metriche selezionate dal gruppo \groupName\ per misurare il raggiungimento degli obiettivi di qualità del prodotto.
    
    \subsubsection{MPD01: Indice di Gulpease}
    Indice che riporta il grado di leggibilità di un testo redatto in lingua italiana.
    La formula adottata è la seguente: 
    \begin{center}
        $GULP = 89 + \displaystyle \frac{300*(totale\ frasi)-10*(totale\ lettere)}{(totale\ parole)}$
    \end{center}
    L’indice così calcolato può assumere valori tra 0 e 100:
    \begin{itemize}
        \item \textbf{GULP $<$ 80:} indica una leggibilità difficile per un utente con licenza elementare;
        \item \textbf{GULP $<$ 60:} indica una leggibilità difficile per un utente con licenza media;
        \item \textbf{GULP $<$ 40:} indica una leggibilità difficile per un utente con licenza superiore.
    \end{itemize}

    \subsubsection{MPD02: Errori ortografici}
    Il controllo ortografico viene eseguito secondo quanto descritto in \ref{verifica_ortografica} del presente documento.

    \subsubsection{MPD03: Copertura requisiti}
    Indice che misura in ogni istante la percentuale di requisiti soddisfatti.
    La formula adottata è la seguente:
    \begin{center}
        $R = \displaystyle \frac{RC}{R}*100$
    \end{center}
    dove:
    \begin{itemize}
        \item \textbf{RC:} indica il numero di requisiti coperti dall'implementazione;
        \item \textbf{R:} indica il numero complessivo di requisiti.
    \end{itemize}