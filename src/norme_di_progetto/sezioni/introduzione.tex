\section{Introduzione} \label{section: introduzione}

\subsection {Scopo del documento}
Il fine del seguente documento è quello di normare le regole e le procedure fondamentali che ciascun membro del gruppo \groupName si impegna a visionare e rispettare.
Allo stato attuale il documento risulta incompleto, dato che il team di sviluppo nella redazione dei vari documenti ha deciso di intraprendere un approccio di tipo incrementale, aggiornando il documento di volta in volta al seguito di ogni decisione presa dal gruppo.
Le norme già presenti possono subire cambiamenti (aggiunte, rimozioni, modifiche), e dovranno essere comunicate dal \roleProjectManagerLow\ di Progetto a ciascun membro del gruppo.

\subsection{Scopo del capitolato}
L'avvento delle tecnologie BlockChain\glo\ ha portato e porterà nei prossimi anni a grandi cambiamenti nella società. 
In particolare, ha aperto le porte a una nuova forma di finanza, la cosiddetta "DeFi" (Finanza Decentralizzata) che ha permesso a chiunque sia dotato di connessione internet di creare un Wallet\glo\ e possedere quindi criptovalute\glo.
Questo ha delineato due profili critici strettamente legati; da un lato il controllo del proprio portafoglio è passato completamente nelle mani dell'utente, dall'altro lato questo comporta la mancanza di un ente terzo che si occupi di gestire transazioni e offrire garanzie.
\newline
Nel capitolato in questione si vuole proprio risolvere questo problema, in uno scenario che comprende un e-commerce\glo\ basato su BlockChain\glo\ in cui si vuole tutelare entrambe le parti coinvolte in un acquisto tramite criptovalute.
\newline
Il fine del progetto è la realizzazione di un prototipo di una piattaforma integrabile con un "crypto-ecommerce\glo", che si occupi di gestire gli ordini dalle fasi di pagamento alla consegna.

\subsection{\docNameGlo}
I termini utilizzati in questo documento potrebbero generare dubbi riguardo al loro significato, richiedendo pertanto una definizione al fine di evitare ambiguità.\\
Tali termini vengono contrassegnati da una G maiuscola finale a pedice della parola.\\
La loro spiegazione è riportata nel \docNameVersionGlo

\subsection{Riferimenti}

\subsubsection{Riferimenti normativi}
\begin{itemize}
    \item \textbf{Capitolato d'appalto C2 - ShopChain:}
    \begin{center}
        \url{https://www.math.unipd.it/~tullio/IS-1/2021/Progetto/C2.pdf}
    \end{center}
\end{itemize}

\subsubsection{Riferimenti informativi}
\begin {itemize}
    \item Corso tenuto alla laurea Magistrale in informatica dell' Università di Verona riguardante le blockchain\glo:
    \begin{center}
        \url{https://www.di.univr.it/?ent=avviso&dest=&id=157271}
    \end{center}
    
    \item Software Engineering - Ian Sommerville - 10 th Edition:\\
    Parte 4 - Software management:
    \begin{itemize}
        \item Capitolo 25 - Configuration management:
    
    \begin{itemize}
        \item Paragrafo 25.1 - Version management (da pag. 735 a 740);
        \item Paragrafo 25.2 - System building (da pag. 740 a 745).
    \end{itemize}
\end{itemize}
    \item Guida \LaTeX:
    \begin{center}
        \url{http://www.lorenzopantieri.net/LaTeX_files/LaTeXimpaziente.pdf}
    \end{center}
\end{itemize}