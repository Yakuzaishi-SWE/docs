\section{Introduzione} \label{section: introduzione}

\subsection {Scopo del documento}
Il fine del seguente documento è quello di normare le regole e le procedure fondamentali che ciascun membro del gruppo \groupName{} si impegna a visionare e rispettare.
Allo stato attuale il documento risulta incompleto, dato che il team di sviluppo nella redazione dei vari documenti ha deciso di intraprendere un approccio di tipo incrementale, aggiornando il documento di volta in volta al seguito di ogni decisione presa dal gruppo.
Le norme già presenti possono subire cambiamenti (aggiunte, rimozioni, modifiche), e dovranno essere comunicate dal \roleProjectManagerLow{} di progetto a ciascun membro del gruppo.

\vspace{1cm}

\subsection{Scopo del capitolato}
L'avvento delle tecnologie blockchain\glo{} ha portato e porterà nei prossimi anni a grandi cambiamenti nella società.
In particolare, ha aperto le porte a una nuova forma di finanza, la cosiddetta "DeFi" (Finanza Decentralizzata) che ha permesso a chiunque sia dotato di connessione internet di creare un wallet\glo{} e possedere quindi criptovalute\glo{}.
Questo ha delineato due profili critici strettamente legati; da un lato il controllo del proprio portafoglio è passato completamente nelle mani dell'utente, dall'altro ha comportato la mancanza di un ente terzo che si occupi di gestire transazioni e offrire garanzie.
\newline
Il capitolato in questione si pone quindi il seguente obiettivo: in un e-commerce\glo{} basato su blockchain\glo{}, dove il pagamento avviene tramite cryptovalute\glo{}, si vuole tutelare le parti coinvolte nell'acquisto.
\newline
Il fine del progetto è la realizzazione di un prototipo di una piattaforma integrabile con un "crypto-ecommerce\glo{}", che si occupi di gestire gli ordini dalla fase di pagamento fino alla consegna del prodotto.

\vspace{1cm}

\subsection{\docNameGlo}
I termini utilizzati in questo documento potrebbero generare dubbi riguardo al loro significato, richiedendo pertanto una definizione al fine di evitare ambiguità.\\
Tali termini vengono contrassegnati da una G maiuscola finale a pedice della parola.\\
La loro spiegazione è riportata nel \docNameVersionGlo{}.

\pagebreak
\subsection{Riferimenti}

\subsubsection{Riferimenti normativi}
\begin{itemize}
    \item \textbf{Regolamento del progetto didattico:}
          \begin{center}
              \url{https://www.math.unipd.it/~tullio/IS-1/2021/Dispense/PD2.pdf}
          \end{center}
    \item \textbf{Capitolato d'appalto C2 - ShopChain:}
          \begin{center}
              \url{https://www.math.unipd.it/~tullio/IS-1/2021/Progetto/C2.pdf}
          \end{center}
\end{itemize}

\subsubsection{Riferimenti informativi}
\begin {itemize}
\item \textbf{Corso sulla blockchain\glo{} tenuto alla laurea Magistrale in informatica dell' Università di Verona:}
\begin{center}
    \url{https://www.di.univr.it/?ent=avviso&dest=&id=157271}
\end{center}

\item \textbf{Software Engineering - Ian Sommerville - 10 th Edition:\\
    Parte 4 - Software management:}
\begin{itemize}
    \item Capitolo 25 - Configuration management:

          \begin{itemize}
              \item Paragrafo 25.1 - Version management (da pag. 735 a 740);
              \item Paragrafo 25.2 - System building (da pag. 740 a 745).
          \end{itemize}
\end{itemize}
\item \textbf{Guida \LaTeX:}
\begin{center}
    \url{http://www.lorenzopantieri.net/LaTeX_files/LaTeXimpaziente.pdf}
\end{center}
\item \textbf{Standard ISO/IEC 12207:}
\begin{center}
    \url{https://www.math.unipd.it/~tullio/IS-1/2009/Approfondimenti/ISO_12207-1995.pdf}
    \begin{itemize}
        \item Capitolo 5 - Primary life cycle processes (da pag. 10 a 24);
        \item Capitolo 6 - Supporting life cycle processes (da pag. 28 a 41);
        \item Capitolo 7 - Organizational life cycle processes (da pag. 42 a 47).
    \end{itemize}
\end{center}
\item \textbf{Standard ISO/IEC 9126:}
\begin{center}
    \url{http://www.colonese.it/00-Manuali_Pubblicatii/07-ISO-IEC9126_v2.pdf}
    \begin{itemize}
        \item Capitolo 2 - Il modello ISO/IEC 9126 (da pag. 12 a 24);
        \item Capitolo 4 - Utilizzo del modello:
              \begin{itemize}
                  \item Paragrafo 4.1 - Processo di sviluppo e qualità (da pag. 31 a 34);
                  \item Paragrafo 4.2 - Implementazione del modello in progetti reali (da pag. 34 a 37);
                  \item Paragrafo 4.3 - Esempi di metriche (pag. 37).
              \end{itemize}
    \end{itemize}

\end{center}
\end{itemize}