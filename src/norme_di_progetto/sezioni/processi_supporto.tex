\section{Processi di supporto}\label{section:processi_supporto}
\subsection{Documentazione}\label{subsection: documentazione}
\subsubsection{Scopo}\label{subsubsection: scopo}
Scopo di questa sezione è quello di normare la documentazione dei vari processi e le attività di sviluppo.
Ci occuperemo quindi di definire le norme per la definizione della struttura che i vari documenti redatti dal gruppo \groupName{} dovranno avere.
\subsubsection{Aspettative}
Le aspettative del gruppo \groupName{} in questo processo sono:
\begin{itemize}
\item Delineare una valida e chiara struttura dei documenti;
\item Definire una convenzione che accomuni tutti i tipi di documentazione redatta.
\end {itemize}
\subsubsection{Descrizione}
Lo scopo della documentazione è quello di trascrivere i fatti accaduti e le decisioni prese dal gruppo durante l'intera durata del progetto.
\subsubsection{Ciclo di vita del documento}
Le fasi del ciclo di vita del documento sono le seguenti:
\begin{figure}[htbp]
\centering
\includegraphics[scale = 0.5]{../template/images/NdP/CicloDiVitaDocumento.png}
\caption{Descrizione del ciclo di vita del documento}
\end {figure}
\subsubsection{Template}
Il team di sviluppo ha deciso di realizzare un template mediante l'utilizzo di \LaTeX{} in modo da avere una struttura uniforme di tutte le pagine nei vari documenti.
Un altro scopo del template è quello di velocizzare la stesura dei documenti stessi, dato che con la struttura già disponibile, i membri del gruppo che si occupano di redigere i documenti dovranno solamente occuparsi della stesura degli stessi, senza badare alla parte grafica.
Il template in particolare definisce la prima pagina, il registro delle modifiche e la presentazione del documento.
\subsubsection{Struttura del documento}
In questa sezione viene descritta la struttura dei documenti. Questi ultimi sono stati suddivisi in due macrofamiglie la cui struttura differisce in alcune parti. Le due macrofamiglie vengono identificate come \textit{verbali} e \textit{documentazione}.
\paragraph{\textit{Verbali}}
I \textit{verbali} vengono redatti successivamente alla riunione del gruppo oppure dopo un incontro con il proponente, essi contengono il rapporto dettagliato delle tematiche discusse durante il meeting.
Prevedono una singola stesura, dato che contengono delle decisioni che non vengono modificate successivamente.

\subparagraph{Prima pagina}
La struttura di ogni prima pagine di un \docNameVLow{} è la seguente:
\begin{itemize}
    \item \textbf{Logo}: il logo del gruppo \groupName{} è collocato in alto ed è centrato verticalmente;
    \item \textbf{Titolo}: collocato sotto al logo del gruppo, centrato verticalmente, composto dalla data e dal tipo di \docNameVLow{} interno/esterno;
    \item \textbf{Nome del capitolato}: collocato sotto il nome del documento, centrato verticalmente vi è il nome del capitolato in questione;
    \item \textbf{E-mail del gruppo}: collocata sotto il nome del capitolato, centrata verticalmente vi è la e-mail del gruppo \groupName{};
    \item \textbf{Informazioni su documento}:
          \begin{itemize}
              \item \textbf{\roleProjectManager{}}: indica il nome del \roleProjectManagerLow{} in quel preciso momento;
              \item \textbf{\textit{Redattori}}: indica chi si è occupato della redazione del documento;
              \item \textbf{\textit{Verificatori}}: indica chi sono stati i verificatori in quel preciso documento;
              \item \textbf{Uso}: indica se il documento è destinato a uso interno o esterno;
              \item \textbf{Destinatari}: indica a chi è destinato il documento;
          \end{itemize}
    \item \textbf{Sommario}: contenente una breve descrizione del documento.
          \end {itemize}

          \subparagraph{Contenuti}
          Collocati nella seconda pagina del \docNameVLow{}. Si tratta di un indice che raccoglie i vari punti trattati nel \docNameVLow{}.

          \subparagraph{Generale}
          A seguire i contenuti, la sezione sopracitata contiene:
          \begin {itemize}
    \item \textbf{Informazioni sulla riunione}: contenenti data, ora di inizio, ora di fine, luogo e partecipanti dell'incontro;
    \item \textbf{Ordine del giorno}: contenente i temi trattati durante la riunione.
          \end {itemize}

          \subparagraph{Resoconto}
          Ultima sezione dei \textit{verbali}, contiene:
          \begin{itemize}
              \item \textbf{Verbale e azioni da intraprendere}: riporta in maniera dettagliata i punti discussi durante la riunione;
              \item \textbf{Tracciamento delle decisioni}: riassunto in forma tabellare delle decisioni prese.
          \end{itemize}

          \subparagraph{Struttura delle pagine}
          Ogni pagina di contenuto è strutturata come segue:
          \begin {itemize}
    \item \textbf{Intestazione} in alto composta da:
          \begin {itemize}
    \item \textbf{Nome del gruppo} in alto a sinistra;
    \item \textbf{Logo del gruppo} in alto a destra.
\end{itemize}
\item \textbf{Piè di pagina} in basso composto da:
\begin{itemize}
\item \textbf{Nome del documento} in basso a sinistra;
\item \textbf{Numero di pagina} in basso a destra, contenente sia il numero di pagina corrente sia il numero di pagine totali.
\end {itemize}
\item \textbf{Contenuto} compreso tra l'intestazione e il piè di pagina.
\end {itemize}




\paragraph{Documentazione}
Fanno parte di questa famiglia tutti i documenti diversi dai \textit{verbali}.
\subparagraph{Prima pagina}
La struttura di ogni prima pagina dei vari documenti è la seguente:
\begin {itemize}
\item \textbf{Logo}: il logo del gruppo \groupName{} è collocato in alto ed è centrato verticalmente;
\item \textbf{Nome del documento}: collocato sotto il logo del gruppo, centrato verticalmente vi è il nome del documento in questione;
\item \textbf{Nome del capitolato}: collocato sotto il nome del documento, centrato verticalmente vi è il nome del capitolato in questione;
\item \textbf{E-mail del gruppo}: collocata sotto il nome del capitolato, centrata verticalmente vi è la e-mail del gruppo \groupName{};
\item \textbf{Informazioni sul documento}:
\begin{itemize}
    \item \textbf{\roleProjectManager}: indica il nome del \roleProjectManagerLow{} in quel preciso momento;
    \item \textbf{\textit{Redattori}}: indica chi si è occupato della redazione del documento;
    \item \textbf{\textit{Verificatori}}: indica chi si è occupato della verifica in quel preciso documento;
    \item \textbf{Uso}: indica se il documento è destinato a uso interno o esterno;
    \item \textbf{Destinatari}: indica a chi è destinato il documento;
    \item \textbf{Versione}: indica la versione del documento.
\end{itemize}
\item \textbf{Sommario}: spiega brevemente le finalità del documento.
\end {itemize}
\subparagraph{Registro delle modifiche}
Il file \textit{changelog.tex} contiene il registro delle modifiche, presente in ogni singolo documento, eccezion fatta per i \textit{verbali}.
Esso corrisponde a una tabella nella quale si tiene traccia di ogni attività svolta (stesura, modifica, verifica, approvazione) e viene aggiornato ogni volta che avviene una delle attività precedentemente citate.
La tabella del registro delle modifiche contiene le seguenti voci:
\begin{itemize}
    \item \textbf{Versione}: versione attuale del documento;
    \item \textbf{Data}: data in cui accade un determinato evento;
    \item \textbf{Autore}: indica chi ha apportato dei cambiamenti al documento;
    \item \textbf{Ruolo}: ruolo che l'autore svolgeva in quel preciso momento;
    \item \textbf{Descrizione}: breve descrizione della modifica effettuata.
          \end {itemize}
          \subparagraph{Indice}
          L'indice, posto in seguito al registro delle modifiche, deve contenere tutte le varie sezioni di cui si compone il documento.
          \subparagraph{Elenco delle tabelle}
          L'elenco delle tabelle, successivo all'indice, deve contenere tutte le tabelle che compaiono nel documento.\\
          Questo non sarà presente qualora il documento in questione non prevede che vi siano tabelle al suo interno.
          \subparagraph{Elenco delle figure}
          L'elenco delle tabelle, posto a seguito dell'elenco delle tabelle, deve contenere tutte le figure che compaiono nel documento.\\
          Questo non sarà presente qualora il documento in questione non prevede che vi siano immagini al suo interno.

          \subparagraph{Struttura delle pagine}
          Ogni pagina di contenuto è strutturata come segue:
          \begin {itemize}
    \item \textbf{Intestazione} in alto composta da:
          \begin {itemize}
    \item \textbf{Nome del gruppo} in alto a sinistra;
    \item \textbf{Logo del gruppo} in alto a destra.
\end{itemize}
\item \textbf{Piè di pagina} in basso composto da:
\begin{itemize}
\item \textbf{Nome del documento} in basso a sinistra;
\item \textbf{Numero di pagina} in basso a destra, contenente sia il numero di pagina corrente sia il numero di pagine totali.
\end {itemize}
\item \textbf{Contenuto} compreso tra l'intestazione e il piè di pagina.
\end {itemize}

\subsubsection{Convenzioni}
\paragraph{Nomi dei file}
Di seguito viene descritta la rappresentazione dei nomi dei file, i quali sono validi per tutti i documenti:
\begin{itemize}
\item I nomi dei file iniziano tutti con la lettera minuscola;
\item Se il nome comprende più parole allora ognuna di esse è separata dal simbolo '\_';
\end {itemize}
Esempi corretti:
\begin{itemize}
\item introduzione;
\item norme\textunderscore di\textunderscore progetto.
\end {itemize}
Esempi non corretti:
\begin{itemize}
\item Norme\textunderscore di\textunderscore Progetto (sono presenti lettere maiuscole);
\item NormeDiProgetto (sono presenti lettere maiuscole e sono assenti i caratteri separatori).
\end {itemize}

\paragraph{Stile di testo}
Nella sezione sottostante vengono riportati i vari stili di testo utilizzati nei documenti e i contesti in cui essi sono utilizzati:
\begin {itemize}
\item \textbf{Grassetto}: lo stile grassetto viene utilizzato per indicare i termini negli elenchi puntati e per i titoli delle sezioni;
\item \textbf{Corsivo}: lo stile corsivo viene utilizzato per indicare il nome del gruppo, per il nome del proponente e per le parole di particolare rilevanza all'interno dei documenti;
\item \textbf{Link}: i link rappresentano dei collegamenti esterni al documento, essi verranno rappresentati di colore blu e sottolineati;
\item \textbf{Nome dei documenti:} il nome dei documenti viene rappresentato in corsivo. Se si fa un riferimento specifico a un particolare documento bisogna indicarne anche la versione (sempre in corsivo);
\item \textbf{Collegamenti interni:} le parole che si riferiscono a una parte del documento vanno sottolineate.
\end {itemize}
\paragraph{\docNameGlo}
Le norme relative al \docNameGloLow{} sono:
\begin{itemize}
    \item Ogni parola presente nel \docNameGloLow{} viene contrassegnata con una 'G' a pedice;
    \item Se un termine compare nella sua stessa definizione all'interno del \docNameGloLow{} esso viene contrassegnato.
\end{itemize}

\paragraph{Elenchi puntati e numerati}
Di seguito viene la descrizione di come il team di sviluppo ha deciso di utilizzare gli elenchi puntati e numerati.
\begin {itemize}
\item Ogni punto dell'elenco inizia con la lettera maiuscola;
\item Alla fine di ogni punto vi è un ';';
\item Dopo l'ultima voce vi è un '.';
\item Se vi è un concetto da spiegare esso viene scritto in grassetto seguito da ':' e segue la spiegazione di esso.
\end {itemize}
\paragraph{Sigle}
In questa sezione vengono raccolte e spiegate le sigle presenti nei vari documenti:
\begin{itemize}
    \item Sigle che rappresentano i ruoli che ogni membro del gruppo a rotazione deve svolgere:
          \begin{table}[H]
              \centering
              \renewcommand{\arraystretch}{1.8}
              \rowcolors{2}{green!100!black!40}{green!100!black!30}
              \begin{tabular}{c|c}
                  \rowcolor[HTML]{125E28}
                  \multicolumn{1}{c}{\color[HTML]{FFFFFF}\textbf{Ruolo}}
                                 & \multicolumn{1}{c}{\color[HTML]{FFFFFF}\textbf{Sigla}} \\
                  \hline
                  Responsabile   & Re                                                     \\
                  Amministratore & Am                                                     \\
                  Analista       & An                                                     \\
                  Progettista    & Pt                                                     \\
                  Programmatore  & Pr                                                     \\
                  Verificatore   & Ve
              \end{tabular}
              \caption{Sigle dei ruoli}
          \end{table}
\end{itemize}
\paragraph{Formato della data}
Il team di sviluppo ha deciso di adottare la seguente convenzione per rappresentare le date che compaiono nei vari documenti:
\begin{center}
    \large{\textbf{YYYY-MM-DD}}
\end{center}
dove:
\begin{itemize}
    \item \textbf{YYYY}: indica l'anno;
    \item \textbf{MM}: indica il mese;
    \item \textbf{DD}: indica il giorno.
\end{itemize}

\subsubsection{Elementi grafici}
\paragraph{Tabelle}
Ad eccezione del registro delle modifiche, le tabelle di ogni documento seguono le seguenti convenzioni:
\begin{itemize}
    \item Ogni tabella contiene al di sotto di essa, in posizione centrale, una didascalia descrittiva;
    \item Ogni tabella viene identificata con un numero progressivo a partire da 1, che la identifica univocamente all'interno del documento;
    \item Per chiarezza grafica e per aumentarne la leggibilità, verrà usato un trattino (-) nelle celle in cui è presente uno 0 (zero). Inoltre si precisa che laddove venga indicato un costo all'interno di una tabella, quest'ultimo verrà espresso in euro (€).
\end{itemize}
\paragraph{Immagini}
Le immagini sono collocate al centro della pagina in questione e al di sotto di ciascuna di esse è presente, in posizione centrale, una descrizione esplicativa e un numero progressivo a partire da 1, che la identifica univocamente all'interno del documento.
\paragraph{Diagrammi}
Sia i diagrammi UML\glo{} che i diagrammi di Gantt\glo{} vengono riportati come immagini, quindi sono soggetti alle regole sopra riportate.
\subsubsection{Verifica ortografica} \label{verifica_ortografica}
% ESEMPIO: Per la verifica ortografica dei documenti scritti in LATEX si utilizzerà lo strumento di correzione automatica presente nell'editor TexmakerG oppure lo strumento Aspell (versione > 0.60.5). Il primo fornisce avvisi di errore istantaneamente; il secondo deve essere utilizzato da terminale lanciando il comando direttamente sul file in formato tex: aspell –mode=tex –lang=it check NomeFile.tex
Per la verifica ortografica dei documenti scritti in \LaTeX{} si utilizzerà lo strumento di correzione automatica \textbf{LanguageTool}:
\begin{center}\url{https://languagetool.org}\end{center}

\subsubsection{Strumenti}
Di seguito sono riportati gli strumenti utilizzati dal team di sviluppo in fase di redazione dei documenti:
\begin{itemize}
    \item \textbf{\LaTeX}: linguaggio di markup per la preparazione di testi, basato sul programma di composizione tipografica TEX;
          \begin{center}\url{https://www.latex-project.org}\end{center}
    \item \textbf{Visual Studio Code}: IDE\glo{} versatile e gratuito, scelto per la stesura del codice \LaTeX{} necessario a produrre i documenti in quanto, grazie all'estensione \textbf{LaTeX Workshop}, comprende un compilatore integrato, l'autocompletamento dei comandi \LaTeX{} e la segnalazione di errori di sintassi;
          \begin{center}\url{https://code.visualstudio.com}\end{center}
    \item \textbf{Diagrams.net}: software di disegno grafico multipiattaforma gratuito e open source. Il gruppo ha scelto questo strumento per la progettazione di diagrammi UML\glo{} in quanto il software è gratuito e presenta tutte le funzionalità necessarie allo scopo.
          \begin{center}\url{https://www.diagrams.net}\end{center}
\end{itemize}

\vspace{2cm}

\subsection{Gestione della configurazione}\label{subsection:gestione_configurazione}
\subsubsection{Scopo}
Scopo di questa sezione è definire come il team \groupName{} ha deciso di affrontare la tematica della gestione della configurazione, ovvero come il team di sviluppo ha deciso di mantenere tracciata la documentazione redatta ed il codice sviluppato.

\subsubsection{Aspettative}
Le aspettative del gruppo \groupName{} nell'utilizzo di questo processo sono:
\begin{itemize}
    \item Possibilità di tracciare tutte le modifiche effettuate;
    \item Possibilità di ripristino, qualora fosse necessario, a una versione precedente;
    \item Possibilità di condivisione dei file configurati tra tutti i membri del gruppo;
    \item Possibilità di individuare e correggere eventuali errori e/o conflitti.
\end{itemize}

\subsubsection{Descrizione}
Il processo di gestione della configurazione ha lo scopo di mantenere organizzata e tracciabile la documentazione redatta e il codice sviluppato, creando una storia per ciascun file prodotto.
In particolare si vuole gestire la struttura e la disposizione delle varie parti di ogni file all'interno di repository\glo{} facilmente accessibili e navigabili.

\subsubsection{Versionamento}
\paragraph{Codice di versione}
Ogni versione di documento è identificata tramite un codice numerico di tre cifre:
\begin{center}
    \Large \textbf{[X].[Y].[Z]}
\end{center}
il cui significato viene spiegato nel seguito:
\begin{itemize}
    \item \textbf{X}: versione stabile, sottoposta ad approvazione del \roleProjectManagerLow{} del documento;
    \item \textbf{Y}: versione controllata, sottoposta a revisione da parte del \roleVerifierLow{} del documento;
    \item \textbf{Z}: versione modificata dal redattore del documento, seguita da una verifica di quanto scritto.
\end{itemize}
Per quanto riguarda il prodotto software, oltre a quanto descritto precedentemente, vengono aggiunti ancora 2 parametri trasformando il codice in un codice numerico di 5 cifre come segue:
\begin{center}
    \Large \textbf{[X].[Y].[Z]-[A].[B]}
\end{center}
in cui il significato dei primi tre numeri rimane inalterato rispetto a quanto spiegato prima, mentre per i due successivamente aggiunti:
\begin{itemize}
    \item \textbf{A}: Indica una versione completa e funzionante del prodotto che supera tutti i test, soddisfa le metriche e implementa i requisiti obbligatori;
    \item \textbf{B}: Cresce al raggiungimento degli obiettivi degli incrementi pianificati nel \docNamePdPLow{}.
\end{itemize}

\subparagraph{Metriche del codice di versione}
Le cifre del codice di versione precedentemente descritte seguono una particolare metrica di avanzamento:
\begin{enumerate}
    \item Tutte le cifre iniziano dal valore 0;
    \item Ciascuna cifra aumenta di un'unità ogni qual volta viene compiuta un'operazione sul documento e in particolare:
          \begin{itemize}
              \item Se la cifra \textbf{X} viene modificata, le cifre \textbf{Y} e \textbf{Z} ritornano al valore 0;
              \item Se la cifra \textbf{Y} viene modificata, la cifra \textbf{Z} ritorna al valore 0;
              \item Le cifre \textbf{A} e \textbf{B} possono solo incrementare il loro valore.
          \end{itemize}
\end{enumerate}

\paragraph{Sistemi software utilizzati}
Per gestire i repository\glo{} Git\glo{} si è scelto di utilizzare il servizio offerto da GitHub\glo{} in quanto:
\begin{itemize}
    \item Gran parte dei membri del gruppo hanno già precedentemente familiarizzato con questo software;
    \item Offre un servizio multipiattaforma dall'utilizzo tramite linea di comando, alla web app, fino ad arrivare alle applicazioni desktop e mobile;
    \item Possibilità di suddividere il tempo in milestone\glo{};
    \item Possibilità di organizzare e suddividere il lavoro tramite la creazione di issues\glo{} assegnabili alle milestone\glo{}.
\end{itemize}


\subsubsection{Struttura del repository}\label{paragraph: repository}
Al fine di organizzare meglio il lavoro si è deciso di creare quattro repository\glo{} distinti, tutti pubblici, con lo scopo di tenere separati i documenti e il software:
\begin{itemize}
    \item \textbf{Yakuzaishi-SWE/docs}: per il versionamento dei documenti;
    \item \textbf{Yakuzaishi-SWE/shopchain}: per il versionamento del codice backend\glo{};
    \item \textbf{Yakuzaishi-SWE/shopchain-frontend}: per il versionamento del codice frontend\glo{};
    \item \textbf{Yakuzaishi-SWE/smart-contract}: per il versionamento del codice dello smart contract\glo{}.

\end{itemize}

\paragraph{Yakuzaishi-SWE/docs}
L'organizzazione del repo è così riassunta:
\begin{itemize}
    \item \textbf{Branch\glo{} master}: è il branch\glo{} principale in cui è presente la sola documentazione, in formato tex, pronta alla revisione;
    \item \textbf{Branch\glo{} dev}: è un branch\glo{} intermedio tra il master e i branch dei singoli documenti. Utile per mantenere pulito il master ed evitare confusione;
    \item \textbf{Branch\glo{} derivanti dal dev}: sono diversi in numero e ciascuno di questi ha un nome parlante riferito al singolo documento. Ognuno è dedicato alla stesura del documento da cui prende il nome e, quando il lavoro su uno di questi branch\glo{} sarà finito e sottoposto a revisione, verrà eseguito un merge\glo{} del branch\glo{} in questione con il dev;
    \item \textbf{Branch\glo{} gh-pages} è un branch\glo{} nel quale vengono caricati i file compilati in formato pdf della documentazione presente nel master. I file presenti in questo branch\glo{} vengono automaticamente aggiunti e resi consultabili in formato pdf alla seguente pagina web:
          \begin{center}
              \url{https://yakuzaishi-swe.github.io/docs/}.
          \end{center}
\end{itemize}

\subparagraph{Gestione dei cambiamenti}
La separazione del flusso di lavoro tra i vari documenti da redarre permette una notevole diminuzione dei conflitti. Il punto focale è che il branch\glo{} master rimanga pulito da ogni tipo di errore, per cui non è utilizzabile da nessun membro del gruppo finché ciascun \roleProjectManagerLow{} non abbia dato l'approvazione al corrispettivo documento. Solo in quel momento è permesso il merge\glo{} di uno dei branch\glo{} minori nel master. I cambiamenti da gestire sui documenti possono essere:
\begin{itemize}
    \item \textbf{Modifiche minori}: riguardano errori grammaticali, lessicali o di sintassi, che possono essere corretti dai \textit{redattori} o dai \textit{verificatori} senza l'approvazione del \roleProjectManagerLow;
    \item \textbf{Modifiche generali}: riguardano cambiamenti più generali come la struttura del documento o convenzioni da utilizzare e richiedono il consulto con il \roleProjectManagerLow, il quale potrà accettare o declinare la proposta di modifica.
\end{itemize}

%%%%%%%%%%%%%%%%%%%%%%%%%%%%%%%%%%%%%%%%%%%%%
%				SISTEMARE
\paragraph{Yakuzaishi-SWE/shopchain, Yakuzaishi-SWE/shopchain-frontend e
    \\ Yakuzaishi-SWE/smart-contract}
Yakuzaishi-SWE/shopchain, Yakuzaishi-SWE/shopchain-frontend\glo{} e Yakuzaishi-SWE/smart-contract\glo{} sono organizzate allo stesso modo. In particolare:
\begin{itemize}
    \item \textbf{Branch\glo{} master}: è il branch\glo{} principale in cui sono presenti i soli file pronti per le release di consegna;
    \item \textbf{Branch\glo{} dev}: è un branch\glo{} intermedio tra il master e i branch delle singole features. Utile per mantenere pulito il master ed evitare confusione;
    \item \textbf{Branch\glo{} derivanti dal dev}: sono diversi in numero e ciascuno di questi ha un nome parlante riferito alla singola feature. Ognuno è dedicato allo sviluppo della feature da cui prende il nome e, quando il lavoro su uno di questi branch\glo{} sarà finito e sottoposto a revisione, verrà eseguito un merge\glo{} del branch\glo{} in questione con il dev.
\end{itemize}

\subparagraph{Gestione dei cambiamenti}
La separazione del flusso di lavoro tra le varie features da sviluppare permette una notevole diminuzione dei conflitti. Il punto focale è che il branch\glo{} master rimanga pulito da ogni tipo di errore, per cui non è utilizzabile da nessun membro del gruppo finché ciascun \roleProjectManagerLow{} non abbia dato l'approvazione per il rilascio. Solo in quel momento è permesso il merge\glo{} di uno dei branch\glo{} minori nel master.
%%%%%%%%%%%%%%%%%%%%%%%%%%%%%%%%%%%%%%%%%%%%%

\vspace{2cm}

\subsection{Gestione della qualità}\label{subsection: gestione_qualita}
\subsubsection{Scopo}
Lo scopo di questa sezione è la definizione del modo in cui il gruppo si impegna nella gestione della qualità.
\subsubsection{Aspettative}
Le aspettative del team di sviluppo durante questo processo sono le seguenti:
\begin {itemize}
\item Conseguimento della qualità del prodotto, in linea con le richieste del proponente;
\item Buona qualità di organizzazione del gruppo;
\item Prova oggettiva della qualità del prodotto.
\end {itemize}
\subsubsection{Descrizione}
La gestione di qualità è un processo che viene descritto nel \docNamePdQLow{}, esso infatti ha il compito di determinare le metriche e le modalità usate per valutare la qualità di prodotti e processi.
Il team di sviluppo ritiene fondamentale raggiungere una buona qualità del prodotto, e ha ritenuto che il modo migliore per raggiungere questo obiettivo è mediante un approccio di tipo sistematico, ovvero assegnando a ciascun componente del gruppo un compito e un ruolo, svolgendo così diversi processi simultaneamente e ottenendo così un ottimizzazione delle risorse, effettuando infine dei test sul prodotto finito.
\subsubsection{Attività}
Affinché il prodotto software e la documentazione redatta raggiungano una buona qualità, ogni componente del gruppo \groupName{} deve:
\begin{itemize}
\item Attenersi al \docNamePdQLow{};
\item Porsi obiettivi incrementali, in modo da ridurre il margine di errore;
\item Creare un avanzamento continuo della formazione personale, mediante conoscenze pregresse, conoscenze di altri membri del gruppo o mediante autoformazione.
\end {itemize}


\vspace{2cm}

\subsection{Verifica} \label{subsection: Verifica}
\subsubsection{Scopo}
Il processo di verifica ha come obiettivo la realizzazione di prodotti corretti, coesi e completi. Tale processo prende in input ciò che è già stato prodotto e lo restituisce in uno stato conforme alle aspettative. Per ottenere tale risultato ci si affida a processi di analisi e test così da accertare che non siano stati introdotti errori durante lo sviluppo del software e la stesura dei documenti. Scopo di questa sezione è definire come il gruppo ha deciso di attuare il processo di verifica.

\subsubsection{Aspettative}
Le aspettative del gruppo \groupName{} nell'utilizzo di questo processo sono:
\begin{itemize}
    \item Verifica di ciascuna fase, rispettando criteri precisi, consistenti e modificabili qualora dovesse essere necessario;
    \item Svolgimento di una verifica attenta al fine di ottenere successo in fase di validazione;
    \item Le attività svolte durante il processo di verifica devono essere rese quanto più possibile automatizzate;
    \item Devono essere rispettati gli obiettivi di copertura indicati nel \docNamePdQLow{}.
\end{itemize}

\subsubsection{Descrizione}
Durante questo processo, il compito dei verificatori è quello di effettuare l'analisi dei prodotti del team. Tale analisi si differenzia in due diverse tipologie:
\begin{itemize}
    \item \textbf{Analisi statica}: processo di valutazione di un sistema o di un suo componente basato sulla sua forma, struttura, contenuto, documentazione. Questo tipo di analisi viene generalmente svolto tramite ispezioni e revisioni e può essere svolta sui documenti così come sul software o su parti di esso;
    \item \textbf{Analisi dinamica}: processo di valutazione di un sistema software o di un suo componente basato sull'osservazione del suo comportamento in esecuzione. Questo tipo di analisi viene generalmente chiamato testing e per motivi logici non può essere svolta sui documenti.
\end{itemize}

\subsubsection{Verifica della documentazione}
Il processo di verifica della documentazione può essere svolto mediante l'uso di strumenti automatici, oppure può essere condotta a mano. In questo secondo caso si possono utilizzare due diverse tecniche di analisi descritte nel seguito:
\begin{itemize}
    \item \textbf{Walktrough}: Il \roleVerifierLow{} esegue un controllo completo del documento alla ricerca di eventuali errori;
    \item \textbf{Inspection}: Il \roleVerifierLow{} esegue un controllo mirato ai soli punti del documento nei quali vi è una più ampia possibilità di errore. Il controllo mirato avviene grazie all'esperienza del \roleVerifierLow{} e a una lista di controllo\glo{}.
\end{itemize}
La tecnica \textbf{Walkthrough} risulta molto onerosa a causa dell'adozione di una ricerca esaustiva degli errori. Questa metodologia sarà applicata come fase iniziale di verifica della documentazione. Con l'avanzare dell'attività di progetto e il ripetersi del processo di verifica sulla documentazione, si stilerà una lista di controllo\glo{} grazie alla quale, successivamente, sarà possibile passare alla tipologia di verifica \textbf{Inspection} che risulta essere più rapida.

\subsubsection{Verifica del codice}
Il processo di verifica del codice farà uso di test automatici appositamente creati e si servirà inoltre di analisi statica e dinamica per quanto riguarda il codice scritto e in particolare:
\begin{itemize}
    \item \textbf{Analisi statica}: verificatori e programmatoti durante le fasi di stesura e verifica del codice si accertano che siano stati rispettati i principi di buona programmazione preimpostati dal team;
    \item \textbf{Analisi dinamica}: verificatori e programmatori durante la fase di test vanno alla ricerca di bug\glo{} ed errori tramite l'esecuzione del software.
\end{itemize}

\paragraph{Test}
Come esposto in precedenza, questa attività ha lo scopo di evidenziare eventuali bug\glo{} e/o errori riscontrabili a runtime. L'esecuzione dei test deve essere ripetibile ed è quindi buona norma renderla quanto più possibile automatizzata.

\subparagraph{Classificazione dei test}
I test sono identificati tramite un codice descritto dalla seguente espressione regolare:
\begin{center}
    \large{\textbf{T[TipologiaTest][ImportanzaRequisito]*[TipologiaRequisito]*[IdNumerico]}}
\end{center}
dove:
\begin{itemize}[label={}]
    \item \textbf{TipologiaTest} può assumere uno tra i seguenti valori letterali:
          \begin{table}[H]
              \centering
              \renewcommand{\arraystretch}{1.8}
              \rowcolors{2}{green!100!black!40}{green!100!black!30}
              \begin{tabular}{c|c|p{12cm}}
                  \rowcolor[HTML]{125E28}
                  \multicolumn{1}{c}{\color[HTML]{FFFFFF}\textbf{Sigla}}
                             & \multicolumn{1}{c}{\color[HTML]{FFFFFF}\textbf{Significato}}
                             & \multicolumn{1}{c}{\color[HTML]{FFFFFF}\textbf{Descrizione}}                                                                                                                                                                                   \\
                  \hline
                  \textbf{A} & Test di accettazione                                         & Vengono eseguiti immediatamente prima del rilascio del prodotto                                                                                                                 \\
                  \textbf{I} & Test di integrazione                                         & Servono a testare che i moduli quando integrati funzionino come previsto, ovvero testare che i moduli che funzionano bene individualmente non abbiano problemi quando integrati \\
                  \textbf{S} & Test di sistema                                              & Condotti su un sistema integrato completo, servono per valutare la conformità del sistema ai requisiti specificati                                                              \\
                  \textbf{U} & Test di unità                                                & Condotti su una parte di codice, servono per verificare che tale parte funzioni correttamente                                                                                   \\
              \end{tabular}
              \caption{Tipologie test}
          \end{table}
    \item \textbf{ImportanzaRequisito} e \textbf{TipologiaRequisito} sono presenti solo nei codici per i test di sistema e accettazione. Questi campi identificano il requisito che si vuole testare, possono assumere i valori riportati e precedentemente descritti al paragrafo \ref{subparagraph:Classificazione dei requisiti}.\\\\
    \item \textbf{IdNumerico} è un valore crescente che parte da 1 e serve per distinguere i diversi test riguardanti la stessa tipologia.
\end{itemize}


\pagebreak

\subsection{Validazione}\label{subsection: validazione}
\subsubsection{Scopo}
Scopo di questa sezione è definire come il team di sviluppo ha deciso di affrontare il processo di validazione, ovvero per assicurarsi che il prodotto sviluppato sia in linea con i requisiti concordati con il proponente e che soddisfi i bisogni del committente.
\subsubsection{Aspettative}
Le aspettative del team di sviluppo nell'utilizzo di questo processo sono le seguenti:
\begin{itemize}
    \item Assicurarsi che il prodotto software sia conforme con i requisiti riportati nell'\docNameAdRLow;
    \item  Dimostrare la correttezza delle attività svolte in fase di verifica.
\end{itemize}
\subsubsection{Descrizione}
Tale processo consiste nell'esaminare il prodotto nella fase di verifica e assicurarsi che esso sia in linea con i requisiti concordati con il proponente e che soddisfi i bisogni del committente.
Sarà il \roleProjectManagerLow{} di progetto che avrà la responsabilità di controllare i risultati decidendo se:
\begin{itemize}
    \item Accettare il prodotto;
    \item Rifiutarlo richiedendo una nuova verifica.
\end{itemize}