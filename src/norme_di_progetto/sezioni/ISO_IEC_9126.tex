\section{Standard ISO/IEC 9126} \label{section:ISO_IEC_9126}
\href{http://www.colonese.it/00-Manuali_Pubblicatii/07-ISO-IEC9126_v2.pdf}{ISO/IEC 9126} è uno standard internazionale per la valutazione della qualità software.
Il gruppo \groupName{} ha deciso di fare riferimento a questo standard poiché considerato un caposaldo in materia di qualità.
Questo standard permette di diffondere una comprensione comune degli obiettivi di un progetto software.

\subsection{Modello della qualità del software} \label{subsection:modello_qualitaSW}
Il modello di qualità stabilito dallo standard si articola in sei caratteristiche principali, ognuna delle quali a sua volta presenta delle sottocaratteristiche, misurabili tramite delle metriche di qualità.
Di seguito sono esposte le sei caratteristiche sopra citate:
\begin{itemize}
    \item \textbf{Funzionalità:} insieme di attributi riguardanti un insieme di funzioni e le loro proprietà.
          Tali funzioni mirano a soddisfare requisiti stabiliti o implicitamente dedotti.
          Le sottocaratteristiche della funzionalità sono:
          \begin{itemize}
              \item \textbf{Adeguatezza:} capacità del prodotto di fornire un insieme di funzioni in grado di svolgere compiti e soddisfare obiettivi prefissati;
              \item \textbf{Accuratezza:} capacità del prodotto di fornire i risultati desiderati con la precisione richiesta;
              \item \textbf{Interoperabilità:} capacità del prodotto di interagire ed operare con uno o più sistemi;
              \item \textbf{Sicurezza:} capacità del prodotto di proteggere informazioni e dati monitorando gli accessi ad essi;
              \item \textbf{Aderenza alle funzionalità:} grado di adesione del prodotto agli standard scelti dal gruppo, alle convenzioni e ai regolamenti;
          \end{itemize}
    \item \textbf{Affidabilità:} insieme di attributi riguardanti la capacità del prodotto di mantenere un dato livello di performance sotto condizioni di esecuzione prestabilite.
          Le sottocaratteristiche dell'affidabilità sono:
          \begin{itemize}
              \item \textbf{Maturità:} capacità del prodotto di evitare errori e risultati non corretti durante l'esecuzione;
              \item \textbf{Tolleranza ai guasti:} capacità del prodotto di conservare il livello di prestazioni anche in caso di malfunzionamenti o di uso inappropriato del prodotto;
              \item \textbf{Recuperabilità:} capacità di un prodotto di ripristinare il livello di prestazioni e di recupero delle informazioni rilevanti, a seguito di un malfunzionamento. Il periodo di inaccessibilità del prodotto a seguito di un errore è valutato proprio dalla recuperabilità;
              \item \textbf{Aderenza all'affidabilità:} grado di adesione del prodotto a standard, regole e convenzioni inerenti all'affidabilità.
          \end{itemize}
    \item \textbf{Efficienza:} insieme di attributi riguardanti il rapporto tra il livello delle prestazioni e la quantità di risorse usate durante la loro esecuzione, sotto condizioni prestabilite.
          Le sottocaratteristiche dell'efficienza sono:
          \begin{itemize}
              \item \textbf{Comportamento rispetto al tempo:} capacità di un prodotto di fornire appropriati tempi di risposta e di elaborazione e quantità di lavoro eseguendo le funzionalità richieste in date condizioni di lavoro;
              \item \textbf{Utilizzo delle risorse:} capacità di un prodotto di usare appropriati numero e tipo di risorse durante la fase di esecuzione sotto condizioni di utilizzo date;
              \item \textbf{Aderenza all'efficienza:} grado di adesione del prodotto a standard, regole e convenzioni inerenti all'efficienza.
          \end{itemize}
    \item \textbf{Usabilità:} insieme di attributi riguardanti lo sforzo necessario all'utilizzo del prodotto e la valutazione individuale su tale uso da parte di un insieme di utenti.
          Le sottocaratteristiche dell'usabilità sono:
          \begin{itemize}
              \item \textbf{Comprensibilità:} facilità di comprensione dei concetti base del prodotto, per permettere all'utente di capire se il prodotto è appropriato;
              \item \textbf{Apprendibilità:} facilità con cui un utente medio può comprendere il funzionamento del prodotto ed imparare ad usarlo;
              \item \textbf{Operabilità:} misura della possibilità degli utenti di usare il prodotto in vari contesti e di adattarlo ai propri scopi;
              \item \textbf{Attrattività:} misura della gradevolezza e dell'essere "attraente" del prodotto durante l'uso;
              \item \textbf{Aderenza all'usabilità:} grado di adesione del prodotto a standard, regole e convenzioni inerenti all'usabilità.
          \end{itemize}
    \item \textbf{Manutenibilità:} insieme di attributi riguardanti lo sforzo richiesto per apportare modifiche specifiche al prodotto.
          Le sottocaratteristiche della manutenibilità sono:
          \begin{itemize}
              \item \textbf{Analizzabilità:} misura della difficoltà incontrata nel diagnosticare un errore nel prodotto;
              \item \textbf{Modificabilità:} facilità di apportare modifiche al prodotto originale o di introdurre nuove funzionalità; per modifiche si intendono cambiamenti al codice, alla progettazione o alla documentazione;
              \item \textbf{Stabilità:} capacità del prodotto di evitare effetti indesiderati a causa di modifiche;
              \item \textbf{Provabilità:} capacità del prodotto di essere verificato;
              \item \textbf{Aderenza alla manutenibilità:} grado di adesione del prodotto a standard, regole e convenzioni inerenti alla manutenibilità.
          \end{itemize}
    \item \textbf{Portabilità:} insieme di attributi riguardanti la capacità del software di essere trasferito da un ambiente di esecuzione ad un altro.
          Le sottocaratteristiche della portabilità sono:
          \begin{itemize}
              \item \textbf{Adattabilità:} capacità del prodotto di essere adattato per differenti ambienti operativi senza richiedere azioni specifiche diverse da quelle previste dal prodotto per quell'attività; include la scalabilità delle capacità interne del prodotto;
              \item \textbf{Installabilità:} capacità del prodotto di essere installato in un dato ambiente;
              \item \textbf{Coesistenza:} capacità di un prodotto di coesistere con altre applicazioni in ambienti comuni e condividere le risorse;
              \item \textbf{Sostituibilità:} capacità di sostituire un altro software specifico indipendente, per lo stesso scopo e nello stesso ambiente di sviluppo;
              \item \textbf{Aderenza alla portabilità:} capacità del prodotto di aderire a standard e convenzioni relative alla portabilità.
          \end{itemize}
\end{itemize}

\vspace{2cm}

\subsection{Qualità interna} \label{subsection:qualita_interna}
\subsubsection{Metriche per la qualità interna}
Le metriche interne si applicano al prodotto non eseguibile durante la progettazione e la codifica. Sono anche dette misure statiche.
Le misure effettuate permettono di prevedere il livello di qualità esterna ed in uso del prodotto finale, poiché gli attributi interni influiscono su quelli esterni e quelli in uso.
Le metriche interne permettono di individuare eventuali problemi che potrebbero inficiare la qualità finale del prodotto prima che sia realizzato il prodotto eseguibile.

\vspace{2cm}

\subsection{Qualità esterna} \label{subsection:qualita_esterna}
\subsubsection{Metriche per la qualità esterna}
Le metriche esterne misurano i comportamenti del prodotto sulla base dei test, dall'operatività e dall'osservazione durante la sua esecuzione, in funzione degli obiettivi stabiliti.
Le metriche esterne sono scelte in base alle caratteristiche che il prodotto finale dovrà dimostrare durante la sua esecuzione.

\vspace{2cm}

\subsection{Qualità in uso} \label{subsection:qualita_uso}
La qualità in uso rappresenta il punto di vista dell'utente sul prodotto. Il livello di qualità in uso è raggiunto quando sono state conseguite sia la qualità esterna che quella interna.
\subsubsection{Metriche per la qualità in uso}
La qualità in uso permette di abilitare specificati utenti ad ottenere dati obiettivi con efficacia, produttività, sicurezza e soddisfazione.
\begin{itemize}
    \item \textbf{Efficacia:} capacità del prodotto di consentire agli utenti di raggiungere gli obiettivi specificati con accuratezza e completezza;
    \item \textbf{Produttività:} capacità di consentire agli utenti di adoperare un'appropriata quantità di risorse rispetto all'efficacia ottenuta in uno dato contesto d'uso;
    \item \textbf{Soddisfazione:} capacità del prodotto di corrispondere alle aspettative degli utenti;
    \item \textbf{Sicurezza:} capacità del prodotto di raggiungere accettabili livelli di rischio di danni a persone, software, apparecchiature tecniche e all'ambiente d'uso.
\end{itemize}