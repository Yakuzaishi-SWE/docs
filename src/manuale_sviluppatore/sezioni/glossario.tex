\newglossaryentry{Criptovaluta}
{
  name={Criptovaluta},
  description={Traduzione in italiano del termine inglese cryptocurrency e si riferisce ad una rappresentazione digitale di valore basata sulla crittografia. Si tratta di una risorsa digitale paritaria e decentralizzata. Al mondo esistono oltre 13000 criptovalute.}
}

\newglossaryentry{E-Commerce}
{
  name={E-Commerce},
  description={Insieme di attività di vendita e acquisto di prodotti effettuato tramite internet.}
}

\newglossaryentry{Smart contract}
{
  name={Smart contract},
  description={Sono essenzialmente dei programmi salvati sulla blockchain che convertono i tradizionali contratti nella loro controparte digitale. Hanno quindi il vero e proprio obiettivo di digitalizzare i termini di un accordo in codice che viene eseguito quando i termini del contratto vengono rispettati.}
}

\newglossaryentry{Wallet}
{
  name={Wallet},
  description={Letteralmente un portafoglio per le criptovalute, che si traduce fisicamente in un dispositivo, un supporto fisico, un programma o un servizio che memorizza le chiavi pubbliche e/o private per le transazioni di criptovaluta.}
}

\newglossaryentry{Fantom}
{
  name={Fantom},
  description={Blockchain relativamente recente nata nel dicembre 2019 e si basa su un proprio algoritmo di consenso personalizzato. Vanta come punto di forza la velocità relativa alle transazioni, dichiara inferiore ai due secondi dagli sviluppatori stessi.}
}

\newglossaryentry{FTM}
{
  name={FTM},
  description={Simbolo di valuta di Fantom.}
}

\newglossaryentry{Blockchain}
{
  name={Blockchain},
  description={Letteralmente significa "catena di blocchi", è una struttura dati decentralizzata i cui dati sono raggruppati in blocchi concatenati in ordine cronologico e la cui integrità è garantita dall'uso della crittografia. Il suo contenuto è "immutabile". Una volta scritto tramite un processo normato non è più modificabile né eliminabile, a meno di non invalidare l'intero processo.}
}

\newglossaryentry{Observer}
{
  name={Observer},
  description={È l’oggetto osservatore nel design pattern Observer\glo.}
}

\newglossaryentry{Observable}
{
  name={Observable},
  description={È l’oggetto osservato nel design pattern Observer\glo, detto anche ”Subject”.}
}

\newglossaryentry{Observer design pattern}
{
  name={Observer design pattern},
  description={È un design pattern software utilizzato per far si che un oggetto observable\glo\ (”osservato”) notifichi modifiche interne all’observer\glo\ (”osservatore”), che reagirà di conseguenza, solitamente chiamando un suo metodo interno.}
}

\newglossaryentry{Repository}
{
  name={Repository},
  description={Ambiente di un sistema informativo, in cui vengono gestiti i metadati, attraverso tabelle relazionali; l’insieme di tabelle, regole e motori di calcolo tramite cui si gestiscono i metadati prende il nome di metabase.}
}

\newglossaryentry{Git}
{
  name={Git},
  description={Software di controllo di versione distribuito utilizzabile da interfaccia a riga di comando, creato da Linus Torvalds nel 2005.}
}

\newglossaryentry{GitHub}
{
  name={GitHub},
  description={Servizio di hosting per sviluppatori. Fornisce uno strumento di controllo versione e permette lo sviluppo distribuito del software.}
}

\newglossaryentry{Pattern architetturale}
{
  name={Pattern architetturale},
  description={È una modellazione architetturale di un prodotto software. Ne esistono diversi e permettono di separare il comportamento delle componenti che lo compongono, aumentando il disaccoppiamento e favorendo lo unit-test delle varie componenti.}
}

\newglossaryentry{Model-View-Controller(MVC)}
{
  name={Model-View-Controller(MVC)},
  description={Pattern architetturale molto diffuso nello sviluppo di software. Consiste nel dividere il prodotto in tre parti fondamentali: il modello, il controller, e la vista. Il controller si occupa di far comunicare vista e modello, ricevendo gli input dell’utente attraverso la vista e avvisando il modello dei cambiamenti avvenuti. La view in questo caso è in grado di visualizzare direttamente i dati contenuti nel modello.}
}

\newglossaryentry{Model-View-ViewModel(MVVM)}
{
  name={Model-View-ViewModel(MVVM)},
  description={Pattern architetturale derivato dal Model-View-Controller(MVC)\glo\ che prevede l’utilizzo di un View-Model per far comunicare vista e modello. In questo caso la vista non ha alcuna comunicazione con il modello quindi il disaccoppiamento è maggiore rispetto a MVC.}
}

\newglossaryentry{Stato interno}
{
  name={Stato interno},
  description={Un componente React può possedere uno stato interno (statefull) oppure no (stateless). Questo stato è un insieme più o meno numeroso di dati, i quali possono essere modificati attraverso interazioni dell’utente con questo componente (nella vista). La modifica di tali dati causa la rirenderizzazione del componente stesso e dei suoi figli.}
}

\newglossaryentry{Metamask}
{
  name={Metamask},
  description={Metamask è un software wallet per criptovalute usato per interagire con la blockchain Ethereum. Esso lascia accedere gli utenti ai loro walllet Etherium tramite un'estensione browser o una app mobile, che sono poi usate per interagire con applicazioni decentralizzate.}
}

\newglossaryentry{Docker}
{
  name={Docker},
  description={È una piattaforma open source di "containerizazzione": permette agli sviluppatori di racchiudere applicazioni in "container", componenti eseguibili standardizzati che uniscono il codice sorgente dell'app con le librerie di sistema e le dipendenze richieste per eseguire quel codice in qualsiasi ambiente.}
}

\newglossaryentry{BIOS}
{
  name={BIOS},
  description={È il firmware utilizzato per fornire servizi runtime ai sistemi operativi e programmi, e per eseguire l'Inizializzazione hardware durante lo start-up della macchina.}
}

\newglossaryentry{DApp}
{
  name={DApp},
  description={È un'applicazione che può operare autonomamente, tipicamente tramite l'uso di smartcontracts eseguiti su una blockchain completamente decentralizzata.}
}

\newglossaryentry{MobX}
{
  name={MobX},
  description={È una libreria per gestire in modo reattivo gli stati di una applicazione. Nonostante sia usabile in modo indipendente da React, sono comunemente usati assieme.}
}

\newglossaryentry{Testnet}
{
  name={Testnet},
  description={Nel gergo crypto, una testnet è una istanza di una blockchain usata per testing e sperimentazioni, senza il rischio di pagamenti e di intaccare la rete principale. Le cryptovalute delle testnet sono diverse da quelle delle reti principali, non hanno valore, e sono gratuitamente ottenibili.}
}

\newglossaryentry{Stable Coin}
{
  name={Stable Coin},
  description={Una criptovaluta con un valore che riflette completamente quello di una valuta reale, molto spesso il dollaro, per garantire una maggiore stabilità nelle transazioni.}
}

\newglossaryentry{TheGraph}
{
  name={TheGraph},
  description={È un protocollo di indicizzazione per organizzare dati blockchain e renderli facilmente accessibili tramite GraphQL\glo.}
}

\newglossaryentry{GraphQL}
{
  name={GraphQL},
  description={Linguaggio open source per richieste e manipolazione di dati per APIs\glo, è stata rilasciata pubblicamente nel 2015.}
}

\newglossaryentry{API}
{
  name={API},
  description={Le API, Application Programming Interface, sono connessioni tra diversi computers o tra diversi programmi, e definiscono come le parti comunicano tra loro tramite richieste e risposte.}
}