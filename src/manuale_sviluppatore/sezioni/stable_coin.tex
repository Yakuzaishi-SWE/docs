\section{Conversione a Stable Coin}\label{section:conversione_stable}

La funzionalità di convertire l'ammontare depositato sul contratto in stable coin è un'interessante funzionalità che il gruppo ha individuato ed aggiunto ai requisiti.\\
Il gruppo ha condotto quindi uno studio relativamente ai passaggi da affrontare per attuare il meccanismo di conversione dei token\glo{}.
Prima di procedere è necessario descrivere e comprendere alcune componenti fondamentali che sono necessarie:

\begin{itemize}
    \item Wrapped FTM: è un token crypto ancorato al valore del Fantom. Viene chiamato così oerchè l'asset originale viene messo in un wrapper che consente di crearne una versione su un'altra blockchain. Questo permette di creare ponti tra diverse blockchain e utilizzare asset non nativi su  un'altra blockchain. Ovviamente in genere può essere riscattato per il corrispettivo in qualsiasi momento. Il processo di "wrapping" permette quindi di rendere il token Fantom conforme allo standard ERC-20 e poter essere, ad esempio nel nostro caso, convertito;
    \item Stable Coin: è un token che ha valore fissato con un rapporto 1:1 con una valuta FIAT\glo{};
    \item Liquidity Pool: è un insieme di fondi bloccati in uno smart contract che generalmente contengono due token diversi. Hanno diversi casi d'uso tra cui la possibilità di fungere da punto di scambio tra i due token;
    \item Factory Contract: è un contratto che si occupa di gestire ed eseguire oprazioni sulle coppie di token. In particolare la sua funzione principale è quella di creare uno smart contract associato alla coppia di token scelta. Quest'ultima funzione è necessaria se si vuole generare una propria liquidity pool;
    \item Router Contract: è un contratto che fornisce un supporto a tutte quelle che sono le funzionalità di trading dei token e gestione della liquidità.
\end{itemize}

Le componenti citate sono fondamentali per poter effettuare la conversione in stable coin. 
Il gruppo ha deciso di sviluppare un primo PoC di questa funzionalità direttamente sulla rete fantom di test. La rete di test presenta però delle mancanze rispetto alla rete principale, vengono a mancare infatti i contratti di alcuni token, come ad esempio USDT (scelto dal gruppo), e quelli che si occupa di gestire le liquidity pool di WFTM e eventuali stable coin.
Il gruppo ha deciso di creare un token che simuli il comportamento di USDT (è possibile visualizzare il contratto associato al seguente link):

\href{}{link}

Purtroppo non è possibile replicare il funzionamento di un vero USDT e il suo relativo vincolo al valore del dollaro, ma è sufficiente per dimostrare come avvenga il processo di conversione.\\

Sulla rete Fantom il provider dei contratti per la gestione delle coppie di token e delle pool di liquidità è SpookySwap. In particolare come anticipato, è necessario riferirsi ai due contratti UniswapV2Factory (Contract Factory) e UniswapV2Router02 (Contract Router), è possibile visualizzarne i riferimenti al seguente link della documentazione:

\begin{center}
    \href{https://docs.spooky.fi/Resources/contracts}{https://docs.spooky.fi/Resources/contracts}
\end{center}

Tramite il contratto UniswapV2Factory abbiamo creato la pair WFTM/USDT (usando il token ERC-20 da noi creato).\\ Dopo aver creato la coppia è stata generata la corrispondente liquidity pool in cui è possibile fornire o rimuovere liquidità. 
è importante ricordare che il valore del token USDT da noi creato è associato al rapporto delle quantità dei due token presenti nella pool e non al dollaro.\\

A questo punto abbiamo proceduto ad inserire all'interno del contratto la chiamata al contratto UniswapV2Router02 per effettuare la conversione tra i due token.
Il metodo da chiamare è UniswapV2Router02.swapETHForExactTokens, che prende in input:

\begin{itemize}
    \item Equivalente di USDT da richiedere alla pool;
    \item Array contente gli indirizzi dei due contratti dei due token;
    \item Indirizzo a cui ritornare gli USDT richiesti (in questo caso è necessario indicare l'indirizzo del contratto);
    \item Un timestamp che indichi il timeout dopo il quale la transazione deve essere annullata.
\end{itemize}

Per approfondire ulteriormente il metodo, si veda la relativa documentazione:

\begin{center}
    \href{https://docs.uniswap.org/protocol/V2/reference/smart-contracts/router-02\#swaptokensforexacttokens}{https://docs.uniswap.org/protocol/V2/reference/smart-contracts/router-02\#swaptokensforexacttokens}
\end{center}

Poichè la conversione viene effettuata dal router su una pool, che verrà utilizzata da più utenti, il valore della conversione può cambiare rapidamente anche solo di poco. Il router in tal caso tornerà un ammontare di FTM da restituire al mittente della transazione nel caso in cui la conversione sia a suo sfavore. è necessario tenere traccia di questo comportamento e gestire la restituzione di tale importo.

%{screen con il codice}

