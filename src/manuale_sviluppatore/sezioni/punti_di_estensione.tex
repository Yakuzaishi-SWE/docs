\section{Punti di estensione} \label{section:punti_estensione}

ShopChain è un'applicazione progettata pensando anche ad eventuali operazioni future di manutenzione o
estensione del prodotto: durante lo sviluppo dell'applicazione ShopChain sono stati esplorati vari spunti per nuove features e ampliamenti, 
che però a causa del poco tempo rimasto non sono state implementate in tempo.
Sono state individuate principalmente tre macro aree che potranno essere ampliate senza grandi difficoltà.

\subsection{Smart Contract}

\subsubsection{Stable Coin}

\subsubsection{TheGraph protocol}

\subsection{Chain differenti}

\subsection{Frontend}

\subsubsection{Immagine MoneyBox}

Il gruppo aveva pensato di associare ad ogni MoneyBox una immagine generata automaticamente, o presa da un pool di immagini definito. Se si sceglie di generarla automaticamente, si può prendere come seed di generazione uno tra i seguenti: l'id dell'ordine associato alla MoneyBox, 
il numero del blocco relativo alla transazione in blockchain, il timestamp della creazione. 
Questo contribuisce ad una maggiore sicurezza nel condividere il link alla suddetta MoneyBox con amici, che potranno vedere a colpo d'occhio se il link è corretto tramite l'immagine incorporata.

\subsubsection{Riflettere i cambiamenti del contratto}

Con i cambiamenti che possono venire apportati agli smartcontract è bene che il lato frontend dell'applicazione sia aggiornato di conseguenza.
Nel caso dei cambiamenti proposti finora i cambiamenti individuati sono i seguenti:
\begin{itemize}
    \item \textbf{Stable Coin}: nelle pagine di pagamento, di dettaglio dell'ordine e di dettagli della MoneyBox, viene mostrato anche l'ammontare in stable coin dei vari pagamenti, per una migliore visione dei costi da parte dell'utente;
    \item \textbf{TheGraph}: le due pagine di elenco transazioni dovranno essere modificate per sfruttare appieno tutte le funzionalità portate dalla implementazione dei protocolli TheGraph;
    \item \textbf{Chain differenti}:  in un pop-up all'ingresso dell'applicazione, l'utente potrà selezionare la chain (e quindi la cryptovaluta correlata) con la quale procedere al pagamento. Tale scelta dovrà essere riportata anche nella breadcrumb come reminder testuale.
\end{itemize}

