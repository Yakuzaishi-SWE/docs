\section{Architettura} \label{section:architettura}

Per lo sviluppo generale dell'applicazione, è stata seguita una architettura DApp, (Fully) Decentralised Application.
Essa consiste in un frontend Web che effettua chiamate dirette ad una infrastruttura decentralizzata backend, nel 
nostro caso una gerarchia di smartcontracts eseguiti su blockchain Fantom; 
questa struttura ricorda quindi una architettura client-server, senza un supporto intermedio per le operazioni.
\\
\\
Il pattern architetturale scelto dal gruppo per lo sviluppo del frontend è il Model-View-ViewModel. Il
seguente pattern è tra i più diffusi nello sviluppo delle web application e permette di scrivere codice
facilmente mantenibile e riusabile; questo è possibile grazie al forte disaccoppiamento che sussiste tra
logica di presentazione e di business. Inoltre l'MVVM è risultato il più adatto per essere utilizzato con
React, libreria impiegata per lo sviluppo dell'UI e che renderizza le componenti in base al loro stato
interno.

\begin{figure}[H]
    \centering
    \includegraphics[scale=0.3]{immagini/mvvm.png}
    \caption{Model-View-ViewModel di ShopChain}
\end{figure}

Il passaggio dei dati dal Model alle varie componenti grafiche avviene attraverso l'utilizzo di un Context
React, al quale viene passato un'istanza del ViewModel. L'utilizzo di un Context React ci permette di
accedere al valore corrente del ViewModel in qualsiasi porzione della View, senza doverlo passare di
componente in componente attraverso le props (ossia gli argomenti dei componenti che compongono la
vista). Nella radice dell'applicazione viene infatti creata un'istanza del ViewModel, che viene passata
ad un Context.Provider, che fa da contenitore per tutta la View. All'interno di tale contenitore ogni
componente puo' utilizzare un hook per accedere al Context React ed utilizzare il valore più recente del ViewModel.
\\
\\
E' stato scelto di utilizzare un Context React per il passaggio dei dati in quanto la nostra applicazione è
molto profonda e non risultava conveniente passare i dati per molti componenti rischiando, nel peggiore
dei casi, di doverli utilizzare nell'ultimo della gerarchia.
Per poter fare in modo che una componente della View si renderizzi non solo al cambiamento del
suo stato interno ma anche al cambiamento dei dati nel Model, abbiamo utilizzato la libreria Mobx.
Questa ci permette di implementare l'observer pattern, non supportato di default da React. A tale
scopo, Mobx permette di segnare delle classi (o attributi di esse) come "observable" e di costruire
dei componenti della View come "observer". Quest'ultimi vengono automaticamente ri-renderizzati al
cambiamento di un qualsiasi attributo observable.

\subsection{Diagrammi delle classi}

\subsubsection{Classi Frontend}

\begin{figure}[H]
    \centering
    \includegraphics[scale = 0.5]{immagini/providerstore.png}
    \caption{La gerarchia di ProviderStore}
\end{figure}

\begin{figure}[H]
    \centering
    \includegraphics[scale = 0.5]{immagini/rootstore.png}
    \caption{Rootstore, le classi per gli ordini e la gestione dei contratti}
\end{figure}

\begin{landscape}
\begin{figure}[H]
    \centering
    \includegraphics[scale = 0.6]{immagini/rsviewmodel.png}
    \caption{Le classi viewmodel dipendenti da RootStore}
\end{figure}
\end{landscape}

\subsubsection{Classi Solidity}

\begin{figure}[H]
    \centering
    \includegraphics[scale = 0.45]{immagini/smartcontract.png}
    \caption{Le classi Solidity}
\end{figure}

Il diagramma delle classi Solidity, che gestiscono gli smartcontracts, è stato creato seguendo le indicazioni di \href{https://github.com/naddison36/sol2uml}{\textbf{sol2uml}}.
Esso è costituito da due classi:
\begin{itemize}
    \item \textbf{OrderManager}, si occupa della gestione di tutti i pagamenti singoli, salvati in più \textbf{map};
    \item \textbf{MoneyBoxManager}, si occupa della gestione delle Moneybox, anche esse mappate; essendo un'estensione di OrderManager, ne eredita tutti i metodi.
\end{itemize}
Tra i metodi, \textit{refund(string memory id)} si occupa di ritornare ai vari proprietari degli ordini 
(e a tutti i partecipanti nel caso di una MoneyBox) le somme di FTM interne al particolare contratto, identificato tramite \textit{id}.
\\
Per maggiori informazioni sulla parte dell'applicazione relativa alla Blockchain, si invita a consultare la sezione §\ref{section:blockchain}.



\subsection{Diagrammi di sequenza}

\begin{figure}[H]
    \centering
    \includegraphics[scale = 0.45]{immagini/diagrammaMeta.png}
    \caption{Diagramma di sequenza per la connessione a Metamask}
\end{figure}

Il diagramma di sequenza sopra riportato è cosi descritto:

\begin{enumerate}
    \item La connessione a MetaMask inizia quando nella \textbf{ConnnectMetamaskView} mediante l'apposito pulsante viene fatto l'onclick e viene chiamato il metodo \textit{connect()} sul \textbf{ConnectMetamaskViewModel};
    \item Esso a sua volta chiama il metodo \textit{connect()} sul \textbf{ProviderStore} che a sua volta fa la chiamata asincrona del metodo \textit{connect} sul \textbf{ProviderRepo} che a sua volta fa la chiamata asincrona del metodo \textit{connect} su \textbf{MetamaskInpageProvider} e la connessione è così stabilita;
    \item Successivamente dato che potrebbero esserci più account connessi il \textbf{ProviderStore} chiama il metodo asincrono \textit{getAccounts} sul \textbf{ProviderRepo} che lo chiama a sua volta sul \textbf{MetaMaskInpageProvider}, il quale ritorna tutti gli indirizzi degli account connessi mediante un array di stringhe;
    \item Succcessivamente il \textbf{ProviderStore} setta l'indirizzo mediante il metodo \textit{setAddress} al quale viene passato in input il primo carattere dell'array di stringhe ritornato nel punto precedente;
    \item Infine la \textbf{NavViewModel} prende gli indirizzi connessi mediante il metodo \textit{getAddress} chiamato sul \textbf{ProviderStore}, e li ritorna alla \textbf{AddressView} per essere visualizzati.
\end{enumerate}

\begin{figure}[H]
    \centering
    \includegraphics[scale = 0.6]{immagini/refund.png}
    \caption{Diagramma di sequenza della funzione di rimborso}
\end{figure}

Il diagramma di sequenza sopra riportato, molto semplice grazie ai pattern architetturali utilizzati, è cosi descritto:

\begin{enumerate}
    \item Dalla \textbf{OrderDetailsView}, dopo aver cliccato l'apposito bottone, parte la chiamata della funzione \textit{refund()} verso \textbf{OrderDetailsViewModel} per effettuare il rimborso dell'ordine corrente;
    \item Dal view-model, vengono chiamate le funzioni \textit{refund()} fino ad arrivare a Web3Contract, che si occuperà di contattare lo smartcontract nella Blockchain;
    \item Al ritorno delle varie funzioni, l'utente visualizzerà un messaggio di avvenuto rimborso dei pagamenti.
\end{enumerate}

\clearpage
\subsection{Architettura di dettaglio}

\subsubsection{Repository pattern}

Il repository pattern ci ha permesso di racchiudere tutte le interazioni con sorgenti esterne in un oggetto unico, mantenendo la separazione tra implementazione e interfaccia.

\subsubsection{Singleton}

Sono state implementate alcune classi seguendo il design pattern Singleton,
 ciò consente di garantire che una classe abbia una sola istanza, fornendo al contempo un punto di accesso globale alla stessa.
Il design singleton è stato usato per avere un'unica fonte detentrice dello stato di ogni modello.
Le classi più importanti create seguendo il pattern singleton sono:

\begin{itemize}
    \item \textbf{RootStore}: 
    \item \textbf{ProviderStore}: 
    \item \textbf{W3Store}: classe che gestisce la connessione a Metamask.
\end{itemize}

\subsubsection{Observer pattern}

Attraverso Mobx abbiamo usato l'observer pattern in quasi tutte le classi del frontend in modo tale che, alla modifica di una classe observable, i cambiamenti si rifletteranno nelle classi observer.