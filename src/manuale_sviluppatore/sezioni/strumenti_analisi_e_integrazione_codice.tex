\section{Strumenti di analisi e integrazione del codice}\label{strumenti_analisi_integrazione_codice}

\begin{table}[H]
	\centering
	\renewcommand{\arraystretch}{1.8}
	\rowcolors{2}{green!100!black!40}{green!100!black!30}
	\begin{tabular}{c | c | c}
		\rowcolor[HTML]{125E28}
		\multicolumn{1}{c}{\color[HTML]{FFFFFF} \textbf{Strumento}} &
        \multicolumn{1}{c}{\color[HTML]{FFFFFF} \textbf{Versione}} & 
		\multicolumn{1}{c}{\color[HTML]{FFFFFF} \textbf{Descrizione}}   \\ \hline
        \rowcolor[HTML]{1c9c3e}
        \multicolumn{3}{c}{\color[HTML]{FFFFFF} \textbf{Analisi Statica}} \\ \hline
        ESLint & 8.9.0 & \shortstack{Strumento di analisi statica utilizzato per la segnalazione\\ degli errori di sintassi, per avere regole d’indentazione\\ uguali in tutti i file e per notifiche altri problemi comuni} \\ \hline
        \rowcolor[HTML]{1c9c3e}
        \multicolumn{3}{c}{\color[HTML]{FFFFFF} \textbf{Analisi Dinamica}} \\ \hline
        Jest & 27.5.1 & \shortstack{Framework javascript per il testing, creato appositamente per React} \\ \hline
        Chai & 4.3.6 & \shortstack{Libreria javascript per il testing} \\ \hline
        Mocha & 9.2.2 & \shortstack{Libreria javascript per il testing} \\ \hline
    \end{tabular}
    \caption{Strumenti per l’analisi e l’integrazione del codice}
\end{table}



