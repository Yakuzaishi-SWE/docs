\section{Inizializzazione}

Per utilizzare l’applicazione web è necessario:
\begin{itemize}
    \item Clonare il repository;
    \item Installare le dipendenze;
    \item Avviare la web app.
\end{itemize}


\subsection{Clonare il repository}

\begin{enumerate}
    \item Scaricare il codice come file .zip direttamente dal repository shopchain-frontend:
            \begin{center}
                \href{https://github.com/Yakuzaishi-SWE/shopchain-frontend}{https://github.com/Yakuzaishi-SWE/shopchain-frontend}
            \end{center}
            Oppure, con Git\glo installato in locale, è possibile clonare il repository con il comando:
            \begin{center}
                \textcolor{red}{git clone https://github.com/Yakuzaishi-SWE/shopchain-frontend}
            \end{center}
    \item Se non clonato tramite Git, estrarre il file .zip.
\end{enumerate}

\subsection{Installare le dipendenze}

\begin{enumerate}
    \item Spostarsi all'interno del repository clonato tramite terminale con il comando:
    \begin{center}
        \textcolor{red}{cd percorso/shopchain-frontend}
    \end{center}
    \item Eseguire il comando:
    \begin{center}
        \textcolor{red}{npm i}
    \end{center}
    \item Attendere il termine della procedura di installazione delle dipendenze.
\end{enumerate}

\subsection{Avviare la web app}

\begin{enumerate}
    \item Dalla cartella, eseguire il comando:
    \begin{center}
        \textcolor{red}{npm start}
    \end{center}
    \item il browser di default si aprirà in automatico con l'applicativo eseguito.
\end{enumerate}