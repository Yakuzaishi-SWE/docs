\section{ShopChain}
Il sito di ShopChain presenta uno schema a tre pannelli che verrà mantenuto in ogni pagina interna allo stesso.\\
La figura seguente evidenzia in maniera specifica i tre pannelli usando i colori
\begin{itemize}
    \item \textbf{Giallo}: navbar
    \item \textbf{Verde}: menu di navigazione
    \item \textbf{Rosso}: contenuto
\end{itemize}
\begin{figure}[H]
    \centering
    \includegraphics[scale=0.2]{immagini/trePannelli.png}
    \caption{Schema a tre pannelli}
\end{figure}


    \subsection{Navbar}
    La navbar presenta la scritta Shopchain cliccabile che rimanda direttamente alla home, lo stato della connessione a MetaMask e l'indirizzo del portafoglio connesso.
    \begin{figure}[H]
        \centering
        \includegraphics[scale=0.5]{immagini/navbar.png}
        \caption{navbar}
    \end{figure}

    La prima cosa da fare una volta all'interno del sito è effettuare quindi la connessione a MetaMask. Per farlo e sufficiente cliccare il bottone "Connect MetaMask". A questo punto si aprirà un popup dell'estensione che chiederà di inserire la password scelta al momento dell'iscrizione. Una volta inserita la Password la connessione sarà stabilita.\\

    L'icona alla sinistra dell'indirizzo del portafoglio segnala appunto lo stato della connessione alla rete Fantom\glo{}.\\
    Posizionandosi sopra l'icona sarà possibile reperire eventuali messaggi di errore:
    \begin{figure}[H]
        \centering
        \includegraphics[scale=0.4]{immagini/stateSignal.png}
        \caption{messaggi di stato}
    \end{figure}

    \subsection{Menù e Contenuto}
    Il menù di navigazione presenta le seguenti sezioni:
    \begin{itemize}
        \item \textbf{Home};
        \item \textbf{Checkout Page};
        \item \textbf{Your Transactions};
        \item \textbf{Inbound Transactions};
    \end{itemize}

    Tutte le sezioni sopracitate vengono poi visualizzate all'interno del pannello dei contenuti.

    
        \subsubsection{Home}
        La home di ShopChain contiene solamente alcune informazioni riguardanti il sito stesso.
        \begin{figure}[H]
            \centering
            \includegraphics[scale=0.2]{immagini/Home.png}
            \caption{HomePage di ShopChain}
        \end{figure}
        Potrebbe risultare interessante dare uno sguardo alle statistiche riguardanti il contratto:
        \begin{figure}[H]
            \centering
            \includegraphics[scale=0.4]{immagini/ContractDetails.png}
            \caption{statistiche contratto}
        \end{figure}

        Esse rappresentano gli ordini effettuati su shopchain da quando è in funzione e quanto denaro è attualmente bloccato all'interno del contratto rispettivamente per ordine singolo per le icone di sinistra e per MoneyBox le icone di destra.

        \subsubsection{Checkout Page}

        \subsubsection{Transazioni}


            \paragraph{Dettagli della transazione}

            Dopo aver cliccato il bottone, all'utente verrà presentato il riepilogo dell'ordine, e verra messa a disposizione la possibilità
            di pagare interamente l'ordine o di creare una MoneyBox per lo stesso.

            %immagine pagina dettagli transazione

            \paragraph{Visualizzazione Transazioni}

            Tornando alla pagina Home, rimangono due bottoni nel menù di sinistra: "Your Transactions" e "Inbound Transactions".
            Cliccando su di essi si apriranno, rispettivamente, le pagine per visualizzare le transazioni in uscita dal proprio account e quelle in entrata.


                \subparagraph{Transazioni in uscita}

                \begin{figure}[H]
                    \centering
                    \includegraphics[scale=0.4]{immagini/transactionview.jpg}
                    \caption{La vista delle transazioni in uscita}
                \end{figure}

                La pagina delle transazioni in uscita, "Your Transactions", è divisa in due sezioni:
                \begin{itemize}
                    \item Yuor Transactions: in questa sezione sono riportati gli ordini \textbf{creati} dall'utente;
                    \item Your Contributions: in questa sezione sono riportati gli ordini di tipo MoneyBox a cui l'utente ha pertecipato.
                \end{itemize}

                \paragraph{Transazioni in entrata}

                \begin{figure}[H]
                    \centering
                    \includegraphics[scale=0.4]{immagini/inboundtransactions.jpg}
                    \caption{La vista delle transazioni in entrata}
                \end{figure}

                Viceversa, la pagine delle transazioni in entrata, "Inbound Transactions", mostra solo una sezione con le transazioni effettuate verso il wallet dell'utente.

                \subparagraph{Ordinamento transazioni}

                L'applicazione offre la possibilità di ordinare l'elenco delle transazioni in due possibili modalità, combinabili tra loro: 
                "Order Type", il tipo dell'ordine, e "Order State", lo stato corrente dell'ordine.
                La scelta è effettuata tramite un apposito menu a tendina.

                \begin{figure}[H]
                    \centering
                    \includegraphics[scale=0.4]{immagini/ordertype.jpg}
                \caption{Menu per la scelta del tipo di ordine}
                \end{figure}

                Il tipo di ordine potrà essere:
                \begin{itemize}
                    \item All: tutti i tipi di ordine;
                    \item Single Payment: pagamento singolo;
                    \item MoneyBox: pagamento creato come MoneyBox dall'utente.
                \end{itemize}

                \begin{figure}[H]
                    \centering
                    \includegraphics[scale=0.4]{immagini/orderstate.jpg}
                    \caption{Menu per la scelta dello stato dell'ordine}
                \end{figure}

                Lo stato dell'ordine invece:
                \begin{itemize}
                \item All: tutti i tipi di stato;
                    \item To Pay: il pagamento deve ancora essere completato;
                    \item Paid but Locked: il pagamento è stato effettuato interamente, ma non ancora sbloccato;
                    \item Unlocked: il pagamento è stato effettuato e sbloccato;
                    \item Refunded: il pagamento è stato rimborsato.
                \end{itemize}

                L'ordinamento delle transazioni è disponibile sia per le transazioni in entrata che per quelle in uscita, e funziona in identico modo in entrambe le pagine.

                \paragraph{Transazione in dettaglio}

                \begin{figure}[H]
                    \centering
                    \includegraphics[scale=0.4]{immagini/transactionsmall.jpg}
                    \caption{Due esempi di transazione nella lista "Your Transactions"}
                \end{figure}

                In dettaglio, le transazioni nelle liste sopracitate presentano all'utente le proprie informazioni più importanti.

                \begin{itemize}
                    \item l'id univoco del pagamento, cliccabile, che rimanda alla pagina di riepilogo dell'ordine, mostrata nella sezione ??;
                    \item l'indirizzo wallet dell'utente;
                    \item un'icona per identificare il tipo di pagamento, singolo o MoneyBox;
                    \item un'icona a forma di simbolo del dollaro per mostrare visivamente se la cifra richiesta è stata pagata interamente o meno;
                    \item un'icona a forma di lucchetto per mostrate visivamente lo stato del pagamento;
                    \item l'ammontare in FTM pagato.
                \end{itemize}

                L'icona di identificazione può assumere due forme: un icona di un uomo per il pagamento singolo, e l'icona di un maialino per il pagamento MoneyBox.

                \begin{figure}[H]
                    \centering
                    \begin{minipage}{0.45\textwidth}
                        \centering
                        \includegraphics[scale=1]{immagini/uomo.jpg} 
                        \caption{Icona pagamento singolo}
                    \end{minipage}\hfill
                    \begin{minipage}{0.45\textwidth}
                    \centering
                        \includegraphics[scale=1]{immagini/piggy.jpg} 
                        \caption{Icona pagamento MoneyBox}
                    \end{minipage}
                \end{figure}

                L'icona di stato dell'ammontare pagato può assumere i seguenti colori: \textbf{verde}, se l'ammontare richiesto è stato raggiunto, o \textbf{giallo} nel caso sia ancora da raggiungere.\\

                L'icona di stato del pagamento avrà un colore diverso a seconda dello stato corrente del pagamento: \textbf{verde} per i pagamenti effettuati e sbloccati, 
                \textbf{giallo} per i pagamenti completati non ancora sbloccati o non ancora interi, \textbf{rosso} per i pagamenti rimborsati.

