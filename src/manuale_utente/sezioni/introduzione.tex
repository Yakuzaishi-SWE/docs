\section{Introduzione}\label{section:introduzione}

\subsection{Scopo del documento}
Lo scopo del seguente documento è quello di illustrare le istruzioni per l'utilizzo e le funzionalità fornite dall' applicazione.
L'utente sarà quindi a conoscenza dei requisiti minimi necessari per il corretto funzionamento di \textit{ShopChain}.

\subsection{Scopo del prodotto}
L'obiettivo richiesto dall'azienda proponente è la realizzazione di una web app\glo{} che permetta la gestione dei pagamenti per una piattaforma e-commerce\glo{}, tramite l'uso di smart contracts\glo{} basati sulla blockchain\glo{} Fantom\glo{}.
La decisione di sviluppare l'applicazione per la blockchain\glo{} Fantom\glo{} è stata presa dopo una attenta analisi delle possibili blockchain\glo{} adatte alla costruzione di smart contract\glo{}.
L'applicazione sarà inoltre dotata di una funzione MoneyBox\glo{} per gestire pagamenti da più utenti per uno stesso prodotto.

\subsection{Glossario}
I termini utilizzati in questo documento potrebbero generare dubbi riguardo al loro significato, richie-
dendo pertanto una definizione al fine di evitare ambiguità. Tali termini vengono contrassegnati da
una G maiuscola finale a pedice della parola. Il glossario è presente alla fine del documento.

\subsection{Requisiti}
Per il corretto funzionamento dell'applicazione è necessario disporre di una connessione internet, ed utilizzare un browser che supporti l'estensione Metamask\glo{}.
L'applicazione è stata testata sui seguenti browser:
\begin{itemize}
    \item Mozilla Firefox, versione 97;
    \item Google Chrome, versione 98;
    \item Micosoft Edge, versione 98;
    \item Brave, versione 1.36.
\end{itemize}
Su Safari l'applicativo non funziona in quanto Metamask non è supportato dallo stesso.