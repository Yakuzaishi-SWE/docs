\section{Introduzione}\label{section:introduzione}

\subsection{Scopo del documento}
Lo scopo del seguente documento è quello di illustrare le funzionalità fornite dall'applicazione e le istruzioni per l'utilizzo della stessa.
L'utente sarà quindi a conoscenza dei requisiti minimi necessari per il corretto funzionamento di \textit{ShopChain}.

\subsection{Glossario}
I termini utilizzati in questo documento potrebbero generare dubbi riguardo al loro significato, richie-
dendo pertanto una definizione al fine di evitare ambiguità. Tali termini vengono contrassegnati da
una G maiuscola finale a pedice della parola. Alla fine del documento stesso è possibile reperire tale Glossario con i termini di interesse.

\subsection{Cos'è ShopChain}
\textit{ShopChain} è una piattaforma sviluppata per affiancare degli e-commerce\glo{} capace di fornire un sistema di pagamento sicuro attraverso le funzionalità fornite dafli smart contract\glo{} caricati sulla rete Fantom\glo{}, in cui viene utilizzata l'omonima criptovaluta\glo{}.\\\\
In particolare quando l'utente effettua un acquisto, scegliendo \textit{ShopChain} come modalità di pagamento, usufruirà di un contratto capace di trattenere il denaro fino al momento dell'arrivo del prodotto acquistato. L'utente potrà effettuare lo sblocco del denaro al wallet\glo{} del venditore mediante un codice di sblocco. Qualora l'ordine non dovesse essere recapitato al destinatario, questo avrà la possibilità di richiedere un rimborso pari all'ammontare speso, direttamente dal contratto, ancora prima che il venditore abbia ricevuto il denaro.\\\\
Il sistema di pagamento prevede due differenti modalità:
\begin{itemize}
    \item \textbf{Pagamento singolo};
    \item  \textbf{Pagamento Moneybox}.
\end{itemize}
Tali modalità verranno meglio approfondite alla sezione §\ref{subsection: CheckoutPage}.

\subsection{Requisiti}
Per il corretto funzionamento dell'applicazione è necessario disporre di una connessione internet, e utilizzare un browser che supporti l'estensione Metamask.
L'applicazione è stata testata sui seguenti browser:
\begin{itemize}
    \item Mozilla Firefox, versione 97;
    \item Google Chrome, versione 98;
    \item Micosoft Edge, versione 98;
    \item Brave, versione 1.36.
\end{itemize}
Su Safari l'applicativo non funziona in quanto Metamask non è supportato dallo stesso.\\
Maggiori dettagli su Metamask verranno discussi nel seguito.



