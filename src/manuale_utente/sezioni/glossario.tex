\newglossaryentry{Criptovaluta}
{
	name={Criptovaluta},
	description={Traduzione in italiano del termine inglese cryptocurrency e si riferisce ad una rappresentazione digitale di valore basata sulla crittografia. Si tratta di una risorsa digitale paritaria e decentralizzata. Al mondo esistono oltre 13000 criptovalute\glo}
}

\newglossaryentry{E-Commerce}
{
	name={E-Commerce},
	description={Insieme di attività di vendita e acquisto di prodotti effettuato tramite internet}
}

\newglossaryentry{Smart contract}
{
	name={Smart contract},
	description={Sono essenzialmente dei programmi salvati sulla blockchain\glo{} che convertono i tradizionali contratti nella loro controparte digitale. Hanno quindi il vero e proprio obiettivo di digitalizzare i termini di un accordo in codice che viene eseguito quando i termini del contratto vengono rispettati}
}

\newglossaryentry{Wallet}
{
	name={Wallet},
	description={Letteralmente un portafoglio per le criptovalute\glo{}, che si traduce fisicamente in un dispositivo, un supporto fisico, un programma o un servizio che memorizza le chiavi pubbliche e/o private per le transazioni di criptovaluta\glo}
}

\newglossaryentry{Fantom}
{
	name={Fantom},
	description={Blockchain\glo{} relativamente recente nata nel dicembre 2019 e si basa su un proprio algoritmo di consenso personalizzato. Vanta come punto di forza la velocità relativa alle transazioni, dichiara inferiore ai due secondi dagli sviluppatori stessi}
}