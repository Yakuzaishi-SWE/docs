\section{Introduzione}\label{section:introduzione}
\subsection{Scopo del documento}
Lo scopo di questo documento è quello di descrivere e motivare le scelte architetturali che il gruppo \groupName{} ha deciso di prendere in fase di progettazione e codifica del prodotto.
Vengono quindi riportati i diagrammi delle classi e di sequenza per descrivere architettura e funzionalità principali del prodotto. Infine è presente una sezione dedicata ai requisiti che il gruppo \groupName{} è riuscito a soddisfare, così da avere un' ampia visione sullo stato di avanzamento del lavoro.
\subsection{Scopo del prodotto}
L'obiettivo richiesto dall'azienda proponente è la realizzazione di una web app\glo{} che permetta la gestione dei pagamenti per una piattaforma e-commerce\glo{}, tramite l'uso di smart contracts\glo{} basati sulla blockchain\glo{} Fantom\glo{}.
La decisione di sviluppare l'applicazione per la blockchain\glo{} Fantom\glo{} è stata presa dopo una attenta analisi delle possibili blockchain\glo{} adatte alla costruzione di smart contract\glo{}.
L'applicazione sarà inoltre dotata di una funzione MoneyBox\glo{} per gestire pagamenti da più utenti per uno stesso prodotto.

\subsection{Riferimenti}
\subsubsection{Riferimenti normativi}
\begin{itemize}
    \item Capitolato d'appalto C2 - ShopChain: Exchange platform on blockchain.
    \begin{center}
        \url{https://www.math.unipd.it/~tullio/IS-1/2021/Progetto/C2.pdf}
    \end{center}
    \end{itemize} 
\subsubsection{Riferimenti informativi}
\begin{itemize}
    \item Slide P2 del corso di ingegneria del software - Diagrammi delle classi;
    \begin{center}
        \url{https://www.math.unipd.it/~rcardin/swea/2021/Diagrammi%20delle%20Classi_4x4.pdf}
    \end{center}
    \item Slide P5 del corso di ingegneria del software - Diagrammi di sequenza.
    \begin{center}
        \url{https://www.math.unipd.it/~rcardin/swea/2022/Diagrammi%20di%20Sequenza.pdf}
    \end{center}
\end{itemize}    