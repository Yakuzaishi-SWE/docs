\section{Resoconto}
\subsection{Verbale e azioni da intraprendere}

\begin{enumerate}
	\item \textbf{Macchine produttive}:
		\begin{itemize}
			\item non c'è una categoria specifica di macchina produttiva, i controlli sono legati al  solo prodotto finito. Per semplificare il progetto, la macchina sarà adibita alla produzione di una sola tipologia di prodotto.
		\end{itemize}
	\item \textbf{Admin}:
		\begin{itemize}
			\item si occupa unicamente di censire macchine produttive e caratteristiche dei prodotti, l'unico controllo sui dati inseriti è di tipo numerico;
		\end{itemize}
	\item \textbf{Carte di controllo}:
		\begin{itemize}
			\item Utilizzare carte di controllo lineari come mostrato nelle slide, altre tipologie di carte di controllo possono risultare complicate nel riconoscere visivamente eventuali "anomalie". Ogni caratteristica del prodotto ha la propria carta di controllo;
			\item La visualizzazione delle carte di controllo va effettuata singolarmente poiché in un monitor standard si rischia di togliere visione al dato, per questo motivo le carte "ruoteranno" tra di loro. Abbiamo la possibilità di sperimentare funzioni proattive aumentando la visibilità di valori anomali.
		\end{itemize}
	\item \textbf{Web app}:
		\begin{itemize}
			\item si puo sviluppare una singola web app\glo{} ma è consigliata la divisione tra admin e utente in modo da sviluppare usando diverse tecnologie. In particolare il lato admin deve avere un servizio di autenticazione e non deve funzionare offline. L'accesso utente può avvenire tramite servizi con API token;
		\end{itemize}
	\item \textbf{Punti anomali}:
		\begin{itemize}
			\item basta evidenziarli con un colore diverso e segnare con un tag "anomalo" sul db;
		\end{itemize}
	\item \textbf{Client}:
		\begin{itemize}
			\item semplice, capibile da chiunque;
		\end{itemize}
	\item \textbf{Dati}:
		\begin{itemize}
			\item possiamo testare su dati storici reali reperibili dal sito fornito nella presentazione del capitolato ma è consigliato creare un tester con 2 valori: media e deviazione standard. Il tester genera valori causali da questi 2 valori. Possibile effettuare multithreading con più generatori di dati;
		\end{itemize}
	\item \textbf{Disponibilità azienda}:
		\begin{itemize}
			\item possibilità di contattare tramite suite Google per organizzare incontri online di circa 15/30min.
		\end{itemize}
\end{enumerate}

\pagebreak

\subsection{Tracciamento delle decisioni}

\begin{table}[H]
	\centering
	\renewcommand{\arraystretch}{1.8}
	\rowcolors{2}{green!100!black!40}{green!100!black!30}
	\begin{tabular}{c | p{10cm}}
		\rowcolor[HTML]{125E28}
		\multicolumn{1}{c}{\color[HTML]{FFFFFF} \textbf{ID}} &
		\multicolumn{1}{c}{\color[HTML]{FFFFFF} \textbf{Decisione}} \\
		\hline
		VE-2021-11-09-1 & Chiariti i dubbi, sembra interessante, papabile scelta. \\
	\end{tabular}
\end{table}