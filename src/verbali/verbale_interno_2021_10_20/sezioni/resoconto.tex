\section{Resoconto}
\subsection{Verbale e azioni da intraprendere}

\begin{enumerate}
	\item \textbf{Nome del gruppo};
	\begin{itemize}
		\item Dopo una discussione tra i membri del gruppo, tra le varie proposte si è deciso il nome Yakuzaishi;
	\end{itemize}
	\item \textbf{Logo del gruppo};
	\begin{itemize}
		\item Prendendo spunto dal nome del gruppo, si è creato un logo ispirato al bastone di Asclepio, riadattato in chiave veneta;
	\end{itemize}
	\item \textbf{Creazione mail di gruppo};
	\begin{itemize}
		\item Dopo vari tentativi, si è riusciti a trovare un indirizzo GMail\glo\ libero, yakuzaishi.swe@gmail.com;
	\end{itemize}
	\item \textbf{Creazione repository su GitHub};
	\begin{itemize}
		\item E' stato scelto GitHub\glo{} come servizio di hosting per la repo pubblica, ed è quindi stata creata la stessa;
	\end{itemize}

\end{enumerate}

\pagebreak

\subsection{Tracciamento delle decisioni}

\begin{table}[H]
	\centering
	\renewcommand{\arraystretch}{1.8}
	\rowcolors{2}{green!100!black!40}{green!100!black!30}
	\begin{tabular}{c | p{8cm}}
		\rowcolor[HTML]{125e28}
		\multicolumn{1}{c}{\color[HTML]{FFFFFF} \textbf{ID}} &
		\multicolumn{1}{c}{\color[HTML]{FFFFFF} \textbf{Decisione}} \\
		\hline
		VI-2021-10-20-1 & Abbiamo deciso il nome del team, Yakuzaishi, e il relativo logo.  \\ \hline
		VI-2021-10-20-2 & L'indirizzo mail del gruppo sarà: yakuzaishi.swe@gmail.com \\ \hline
		VI-2021-10-20-3 & Il sistema di versionamento che verrà utilizzato sarà Git\glo{} e la repository\glo{} verrà creata su GitHub\glo.  \\ \hline
		VI-2021-10-20-4 & Gli strumenti che utilizzeremo sono: Telegram\glo, Discord\glo, Google Drive\glo\ per la condivisione di materiale utile, LATEX\glo\ per la stesura dei documenti.  \\
	\end{tabular}
\end{table}