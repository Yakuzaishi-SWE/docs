\section{Resoconto}
\subsection{Verbale e azioni da intraprendere}

\begin{enumerate}
	\item \textbf{Codice di Sblocco};
	\begin{itemize}
		\item Non importa come viene generato;
		\item Una possibilità è prendere il numero del blocco nella blockchain\glo{} come codice;
		\item Deve effettuare lo sblocco solo chi deve pagare;
		\item Il codice deve essere usato solo come conferma di sblocco;
	\end{itemize}
	\item \textbf{Creazione degli Smartcontract};
	\begin{itemize}
		\item Ci sono due strade: lo smart contract\glo{} unico fabbrica smartcontracts personalizzati, o uno smartcontract contiene tutte le transazioni;
		\item Per complessità e per scalabilità conviene la strada di uno smartcontract dedicato per ogni MoneyBox;
	\end{itemize}
	\item \textbf{Metriche per la sicurezza};
	\begin{itemize}
		\item Sono state individuate vari possibili controlli di sicurezza, che saranno analizzati ed esplorati;
		\item Riguardo alla sicurezza della blockchain\glo{}, la sua piattaforma stessa la garantisce;
	\end{itemize}
	\item \textbf{Pagamento delle gas fee};
	\begin{itemize}
		\item Mettere qualche fondo all'interno dello smartcontract cosi ha lui stesso liquidità per pagare le gas fee,
		 ci sarà una percentuale inviata al wallet\glo{} nostro (ovvero ai proprietari di ShopChain), più una piccola percentuale per le gas fee;
	\end{itemize}
	\item \textbf{Codici di Sicurezza per la MoneyBox};
	\begin{itemize}
		\item E' una "traduzione" dell'indirizzo alla MoneyBox\glo{} in qualcosa di identificabile visivamente;
		\item Sarebbe piu' sicura una chiave per accedere ma e' stata ritenuta troppo invasiva a lato utente;
		\item Buona idea, da esplorare;
	\end{itemize}
	\item \textbf{E-Commerce}\glo{};
	\begin{itemize}
		\item Conviene lanciare una chiamata "finta" con i dati richiesti per l'inizio dell'operazione;
		\item Come requisito opzionale si possono mettere i dettagli dell'ordine;
	\end{itemize}
\end{enumerate}

\pagebreak

\subsection{Tracciamento delle decisioni}

\begin{table}[H]
	\centering
	\renewcommand{\arraystretch}{1.8}
	\rowcolors{2}{green!100!black!40}{green!100!black!30}
	\begin{tabular}{c | p{10cm}}
		\rowcolor[HTML]{125E28}
		\multicolumn{1}{c}{\color[HTML]{FFFFFF} \textbf{ID}} &
		\multicolumn{1}{c}{\color[HTML]{FFFFFF} \textbf{Decisione}} \\
		\hline
		VE-2022-01-17-1 & Devono essere analizzati i vari metodi degli smartcontract per capire quanto sono legati tra loro gli smartcontract delle moneybox\glo{} e delle transazioni singole; \\ \hline
		VE-2022-01-17-2 & Verranno analizzati vari possibili controlli di sicurezza per l'applicazione\\ 
		VE-2022-01-17-3 & Le gas fee saranno pagate dallo smartcontract stesso \\ 
		VE-2022-01-17-4 & Verrà esplorata la funzione di traduzione visiva degli indirizzi alle MoneyBox\glo{} \\
	\end{tabular}
\end{table}