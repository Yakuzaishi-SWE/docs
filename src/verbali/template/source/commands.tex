% ==================================================================================================
% COMANDI DA RIDEFINIRE
% ==================================================================================================

% Nome/Versione/Data del documento
\newcommand{\documentName}{ERRORE}
\newcommand{\documentVersion}{ERRORE}
\newcommand{\documentDate}{ERRORE}

% Editori del documento
\newcommand{\documentEditors}{ERRORE}

% Verificatore del documento
\newcommand{\documentVerifiers}{ERRORE}

% Approvatore del documento
\newcommand{\documentApprovers}{ERRORE}

% Uso del documento
\newcommand{\documentUsage}{ERRORE}

% Destinatari del documento
\newcommand{\documentAddressee}{ERRORE}

% Sommario del documento
\newcommand{\documentSummary}{ERRORE}

% Vesione dei documenti
\newcommand{\docVersionGlo}{\textit{v1.0.0}} % Glossario
\newcommand{\docVersionNdP}{\textit{v1.0.0}} % Norme di Progetto
\newcommand{\docVersionSdF}{\textit{v1.0.0}} % Studio di Fattibilità
\newcommand{\docVersionPdP}{\textit{v1.0.0}} % Piano di Progetto
\newcommand{\docVersionPdQ}{\textit{v1.0.0}} % Piano di Qualifica
\newcommand{\docVersionST}{\textit{v1.0.0}} % Specifica Tecnica
\newcommand{\docVersionDdP}{\textit{v1.0.0}} % Definizione di Prodotto
\newcommand{\docVersionAdR}{\textit{v1.0.0}} % Analisi dei Requisiti
\newcommand{\docVersionMU}{\textit{v1.0.0}} % Manuale Utente
\newcommand{\docVersionMS}{\textit{v1.0.0}} % Manuale Sviluppatore

% ==================================================================================================
% COMANDI DA NON RIDEFINIRE
% ==================================================================================================

% Stile font
\newcommand{\code}[1]{\texttt{\textbackslash{}#1}} % Testo stile codice
\newcommand{\glo}{\ped{\textbf{\tiny G}}} % Testo glossario
\newcommand{\es}[1]{(e.g. #1)} % Esempio e.g.

% Nome del progetto
\newcommand{\projectName}{BlockChange}
\newcommand{\groupEmail}{\textit{\href{mailto:yakuzaishi.swe@gmail.com}{yakuzaishi.swe@gmail.com}}}
\newcommand{\groupName}{Yakuzaishi}

% Componenti del gruppo
\newcommand{\team}{Francesco Bugno, Luca Busacca, Luca Carturan, Michele Filosofo, Dario Furlan, Francesco Mattarello, Matteo Midena}

% Ruoli del progetto
\newcommand{\roleProjectManager}{\textit{Responsabile}}
\newcommand{\roleAdministrator}{\textit{Amministratore}}
\newcommand{\roleAnalyst}{\textit{Analista}}
\newcommand{\roleProgrammer}{\textit{Programmatore}}
\newcommand{\roleDesigner}{\textit{Progettista}}
\newcommand{\roleVerifier}{\textit{Verificatore}}

% Referenti e commitenti (M = master, S = slave)
\newcommand{\proposerName}{Fabio Pallaro - Sync Lab S.r.l.}
\newcommand{\commitNameM}{Prof. Tullio Vardanega}
\newcommand{\commitNameS}{Prof. Riccardo Cardin}

% Nome dei documenti
\newcommand{\docNameGlo}{\textit{Glossario}} % Glossario
\newcommand{\docNameNdP}{\textit{Norme di Progetto}} % Norme di Progetto
\newcommand{\docNameSdF}{\textit{Studio di Fattibilità}} % Studio di Fattibilità
\newcommand{\docNameST}{\textit{Specifica Tecnica}} % Specifica Tecnica
\newcommand{\docNameDdP}{\textit{Definizione di Prodotto}} % Definizione di Prodotto
\newcommand{\docNamePdP}{\textit{Piano di Progetto}} % Piano di Progetto
\newcommand{\docNamePdQ}{\textit{Piano di Qualifica}} % Piano di Qualifica
\newcommand{\docNameAdR}{\textit{Analisi dei Requisiti}} % Analisi dei Requisiti
\newcommand{\docNameMU}{\textit{Manuale Utente}} % Manuale Utente
\newcommand{\docNameMS}{\textit{Manuale Sviluppatore}} % Manuale Sviluppatore

% Nome e versione dei documenti
\newcommand{\docNameVersionGlo}{\docNameGlo{} \docVersionGlo} % Glossario
\newcommand{\docNameVersionNdP}{\docNameNdP{} \docVersionNdP} % Norme di Progetto
\newcommand{\docNameVersionSdF}{\docNameSdF{} \docVersionSdF} % Studio di Fattibilità
\newcommand{\docNameVersionPdP}{\docNamePdP{} \docVersionPdP} % Piano di Progetto
\newcommand{\docNameVersionPdQ}{\docNamePdQ{} \docVersionPdQ} % Piano di Qualifica
\newcommand{\docNameVersionAdR}{\docNameAdR{} \docVersionAdR} % Analisi dei Requisiti
\newcommand{\docNameVersionMU}{\docNameMU{} \docVersionMU} % Manuale Utente
\newcommand{\docNameVersionMS}{\docNameMS{} \docVersionMS} % Manuale Sviluppatore

% Nome delle revisioni
\newcommand{\RTB}{\textit{Requirements and Technology Baseline}}
\newcommand{\RP}{\textit{Product Baseline}}
\newcommand{\CA}{\textit{Customer Acceptance}}

% Descrizione del glossario
\newcommand{\gloDesc}{I termini utilizzati in questo documento potrebbero generare dubbi riguardo al loro significato, richiedendo pertanto una definizione al fine di evitare ambiguità. Tali termini vengono contrassegnati da una G maiuscola finale a pedice della parola. La loro spiegazione è riportata nel \docNameVersionGlo{}.}

% Comandi per la generazione di grafici a torta e a barre
\newcommand{\pie}[3][]{
	\begin{scope}[#1]
		\pgfmathsetmacro{\curA}{90}
		\pgfmathsetmacro{\r}{1}
		\def\c{(0,0)}
		\node[pie title] at (90:1.3) {#2};
		\foreach \v\s in{#3}{
			\pgfmathsetmacro{\deltaA}{\v/100*360}
			\pgfmathsetmacro{\nextA}{\curA + \deltaA}
			\pgfmathsetmacro{\midA}{(\curA+\nextA)/2}
			
			\path[slice,\s] \c
			-- +(\curA:\r)
			arc (\curA:\nextA:\r)
			-- cycle;
			\pgfmathsetmacro{\d}{max((\deltaA * -(.5/50) + 1) , .5)}
			
			\begin{pgfonlayer}{foreground}
				\path \c -- node[pos=\d,pie values,values of \s]{$\v\%$} +(\midA:\r);
			\end{pgfonlayer}
			
			\global\let\curA\nextA
		}
	\end{scope}
}

\newcommand{\legend}[2][]{
	\begin{scope}[#1]
		\path
		\foreach \n/\s in {#2}
		{
			++(0,-5pt) node[\s,legend box] {} +(5pt,0) node[legend label] {\n} % ++(x,y) -> y: spazio tra i vari campi della legenda, x: l'obliquità
		}
		;
	\end{scope}
}