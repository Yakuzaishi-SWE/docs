\section{Resoconto}
\subsection{Verbale e azioni da intraprendere}

\begin{enumerate}
	\item \textbf{Modello di sviluppo e pianificazione}
	\begin{itemize}
		\item Ok modello incrementale;
		\item Buona idea dividere in milestone interne.
	\end{itemize}
	\item \textbf{Tecnologie con cui sviluppare}
	\begin{itemize}
		\item Back-end: Express.js (javascript è buono per comunicare con la blockchain);
		\item Front-end: Angular (buono a livello didattico), React (più facile di angular), Vue.js (buono per interfacce di poche pagine).
	\end{itemize}
	\item \textbf{Proof of Concept}
	\begin{itemize}
		\item Landing page dove far partire il pagamento;
		\item No moneybox;
		\item Smart contract minimo;
		\item Pagina per lo sblocco dei soldi da parte del compratore.
	\end{itemize}
	\item \textbf{Casi d'uso}
	\begin{itemize}
		\item Il venditore deve vedere tutte le transazioni? Si.
	\end{itemize}
	\item \textbf{Consigli del proponente}
	\begin{itemize}
		\item Forse un database non è necessario;
		\item C'è un esempio dello smart contract da realizzare all'interno della documentazione solidity (\url{https://docs.soliditylang.org/en/v0.8.10/solidity-by-example.html#safe-remote-purchase})
	\end{itemize}
\end{enumerate}
\pagebreak

\subsection{Tracciamento delle decisioni}

\begin{table}[H]
	\centering
	\renewcommand{\arraystretch}{1.8}
	\rowcolors{2}{green!100!black!40}{green!100!black!30}
	\begin{tabular}{c | p{10cm}}
		\rowcolor[HTML]{125E28}
		\multicolumn{1}{c}{\color[HTML]{FFFFFF} \textbf{ID}} &
		\multicolumn{1}{c}{\color[HTML]{FFFFFF} \textbf{Decisione}} \\
		\hline
		VE-2021-12-20-1 & Utilizziamo un modello incrementale e creiamo milestone interne per scandire meglio il tempo. \\ \hline
		VE-2021-12-20-2 & Le tecnologie che andremo ad usare probabilmente saranno Express.js e React. \\ \hline
		VE-2021-12-20-3 & Chiarito cosa è richiesto nel Proof of Concept.
	\end{tabular}
\end{table}
