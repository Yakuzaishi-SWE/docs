\section{Resoconto}
\subsection{Verbale e azioni da intraprendere}

\begin{enumerate}
	\item \textbf{Modello di sviluppo e pianificazione}
	\begin{itemize}
		\item Il gruppo Yakuzaishi ha proposto come modello di sviluppo il modello incrementale. Il proponente si è trovato in accordo con la decisione, sottolineando che per questo tipo di progetti che non sono ben defniti dall'inizio, il modello incrementale è un buon approccio allo sviluppo;
		\item In fase di pianificazione è buona norma dividere l'intero periodo con milestone interne.
	\end{itemize}
	\item \textbf{Tecnologie con cui sviluppare}
	\begin{itemize}
		\item Back-end: Il gruppo ha proposto Express.js ed il proponente si è trovato in accordo poiché javascript è buono per comunicare con la blockchain;
		\item Front-end: A seguito di una discussione tra i vari framework ci è stata lasciata la possibilità di scelta tra Angular (buono a livello didattico), React (più facile di angular ma meno utilizzato) e Vue.js (buono per interfacce di poche pagine).
	\end{itemize}
	\item \textbf{Proof of Concept}\\
		Per quanto riguarda il proof of concept si è stabilito che questo dovrà essere un eseguibile che dovrà implementare:
	\begin{itemize}
		\item Landing page da dove far partire il pagamento e contenente le informazioni di riepilogo dell'ordine;
		\item Pagina per lo sblocco dei soldi da parte del compratore;
		\item implementazione di smart contract minima;
		\item Non c'è bisogno di implementare la funzionalità di moneybox.
	\end{itemize}
	\item \textbf{Casi d'uso}
	\begin{itemize}
		\item A seguito di una discussione su uno specifico caso d'uso è emerso che potrebbe essere conveniente per i venditore essere a conoscenza di tutte le transazioni.
	\end{itemize}
	\item \textbf{Consigli del proponente}
	\begin{itemize}
		\item Si può valutare di escludere un database poiché si potrebbero registrare tutti i dati di cui si ha bisogno nella blockchain stessa;
		\item È possibile trovare un esempio simile allo smart contract da realizzare all'interno della documentazione Solidity al seguente link:\\
		\begin{center}
			\url{https://docs.soliditylang.org/en/v0.8.10/solidity-by-example.html#safe-remote-purchase}
		\end{center}
	\end{itemize}
\end{enumerate}
\pagebreak

\subsection{Tracciamento delle decisioni}

\begin{table}[H]
	\centering
	\renewcommand{\arraystretch}{1.8}
	\rowcolors{2}{green!100!black!40}{green!100!black!30}
	\begin{tabular}{c | p{10cm}}
		\rowcolor[HTML]{125E28}
		\multicolumn{1}{c}{\color[HTML]{FFFFFF} \textbf{ID}} &
		\multicolumn{1}{c}{\color[HTML]{FFFFFF} \textbf{Decisione}} \\
		\hline
		VE-2021-12-20-1 & Utilizziamo un modello incrementale e creiamo milestone interne per scandire meglio il tempo. \\ \hline
		VE-2021-12-20-2 & Le tecnologie che andremo ad usare probabilmente saranno Express.js e React. \\ \hline
		VE-2021-12-20-3 & Chiarito cosa è richiesto nel Proof of Concept.
	\end{tabular}
\end{table}
