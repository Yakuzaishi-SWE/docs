\section{Resoconto}
\subsection{Verbale e azioni da intraprendere}

\begin{enumerate}
	\item \textbf{Dimostrazione dell'applicazione};
	\begin{itemize}
		\item Durante la dimostrazione sono stati individuati diversi possibili miglioramenti riguardanti
		 il front end e la presentazione grafica dell'applicazione, tra cui la separazione del 
		 resoconto degli ordini e alcuni bug grafici delle icone e bottoni presenti;
		\item  stato deciso quindi di separare la pagina di resoconto degli ordini in due pagine distinte, per ordini in entrata ed uscita dall'indirizzo dato.
	\end{itemize}
	\item \textbf{Dubbi sulle tecnologie}
	\begin{itemize}
		\item Durante l'incontro è stata messa in discussione l'effettiva 
		utilità di un backend\glo{} per l'applicazione, e si è quindi optato per far
		 gestire tutti i dati da memorizzare alla blockchain\glo{} Fantom\glo{}.
	\end{itemize}
\end{enumerate}

\pagebreak

\subsection{Tracciamento delle decisioni}

\begin{table}[H]
	\centering
	\renewcommand{\arraystretch}{1.8}
	\rowcolors{2}{green!100!black!40}{green!100!black!30}
	\begin{tabular}{c | p{10cm}}
		\rowcolor[HTML]{125E28}
		\multicolumn{1}{c}{\color[HTML]{FFFFFF} \textbf{ID}} &
		\multicolumn{1}{c}{\color[HTML]{FFFFFF} \textbf{Decisione}} \\
		\hline
		VE-2022-02-16-1 & Verrà rimosso il back-end dell'applicazione in quanto non necessario \\ \hline
		VE-2022-02-16-2 & La pagina di resoconto degli ordini verrà separata in due pagine distinte per compratore e venditore \\ \hline
		VE-2022-02-16-3 & La pagina delle transazioni dovrà prevedere diversi tipi di filtro \\ \hline
		VE-2022-02-16-4 & Verranno corretti i bug\glo{} riscontrati nel frontend\glo{} dell'applicazione \\
	\end{tabular}
\end{table}