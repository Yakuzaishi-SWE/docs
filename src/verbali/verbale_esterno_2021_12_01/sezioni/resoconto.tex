\section{Resoconto}
\subsection{Verbale e azioni da intraprendere}

\begin{enumerate}
	\item \textbf{Quale BlockChain\glo{} utilizzare:}
	\begin{itemize}
		\item \textbf{Fantom\glo{}}: approvato come scelta dal proponente, da la possibilità di smart contract\glo{} in Solidity\glo{} e si può utilizzare \textit{web3.js}\glo{} per comunicare tra \textit{Metamask}\glo{} e \textit{BlockChain}\glo{};
		\item \textbf{Polygon\glo{}}: scartata dal proponente.
	\end{itemize}
	\item \textbf{Tecnologie per Solidity\glo{}:}
	\begin{itemize}
		\item \textbf{Remix\glo{}:} possibilità di effettuare test in locale o su Testnet, consigliato per il lavoro di gruppo;
		\item \textbf{Truffle\glo/Ganache\glo{}:} utile per fare test in locale ma a questo punto meglio Remix\glo{}.
	\end{itemize}
	\item \textbf{Multiwallet/Moneybox\glo{}:} 
	\begin{itemize}
		\item Salvadanaio comune su wallet\glo{} temporaneo (che sarà un portafogli temporaneo), se non ha abbastanza soldi da errore. Possibile metodo \textit{close} che ritorna i soldi e chiude il salvadanaio (anche in caso di resto dopo la transazione);
		\item Togliamo completamente la parte mobile per inserirlo, il ricevente dovrà inserire un codice da pc con il suo Metamask\glo{}.
	\end{itemize}
	\item \textbf{Consigli del proponente:}
	\begin{itemize}
		\item Sito documentazione solidity\glo{} ufficiale: https://docs.soliditylang.org/en/latest/;
		\item Ci è stato fornito accesso al server Discord\glo{} aziendale.
	\end{itemize}
	
\end{enumerate}

\pagebreak

\subsection{Tracciamento delle decisioni}

\begin{table}[H]
	\centering
	\renewcommand{\arraystretch}{1.8}
	\rowcolors{2}{green!100!black!40}{green!100!black!30}
	\begin{tabular}{c | c}
		\rowcolor[HTML]{125e28}
		\multicolumn{1}{c}{\color[HTML]{FFFFFF} \textbf{ID}} &
		\multicolumn{1}{c}{\color[HTML]{FFFFFF} \textbf{Decisione}} \\
		\hline
		VE-2021-12-01-1 & Studieremo un'implementazione su blockchain\glo{} Fantom\glo{}\\ \hline
		VE-2021-12-01-2 & Confermata l'implementazione del multiwallet\glo{}\\ \hline
		VE-2021-12-01-3 & Utilizzeremo Remix\glo{} per sviluppare in Solidity\glo{}\\

	\end{tabular}
\end{table}