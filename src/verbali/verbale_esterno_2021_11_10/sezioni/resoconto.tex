\section{Resoconto}
\subsection{Verbale e azioni da intraprendere}

\begin{enumerate}
	\item \textbf{Ambiente di test:}
	\begin{itemize}
		\item Sono da scartare tecnologie blockchain\glo{} che non hanno ambiente di test e che non implementano smart contract\glo{}.
		In fase di testing non ci sono preferenze sulla rete da utilizzare, i test vanno dunque effettuati su una test net scelta da noi.
	\end{itemize}

	\item \textbf{Tipo di protocollo:}
	\begin{itemize}
		\item Per quanto riguarda il protocollo la rete main test è su proof of work\glo{}, sulle reti di test potremmo utilizzare la proof of authority\glo{}.
	\end{itemize}

	\item \textbf{Volatilità della cryptovaluta\glo:}
	\begin{itemize}
		\item Per quanto riguarda il fattore volatilità, da quando l'utente effettua l'acquisto a quando l'importo arriva al venditore, la moneta può variare di valore.
		 La soluzione proposta è che quando si attiva lo smart contract\glo{} la valuta viene convertita in stable coin\glo{}.
	\end{itemize}

	\item \textbf{Forme di tutela:}
	\begin{itemize}
		\item Il lato di gestione della tutela del cliente non è di nostra pertinenza.
		\item Diamo per scontato che quando l'utente apre il pacco scansiona il qr code\glo{} e sblocca i soldi che vanno al venditore.
		 Per agevolare lo sblocco da parte del cliente si può implementare una meccanica di reward.
	\end{itemize}

	\item \textbf{Multiwallet:}
	\begin{itemize}
		\item Proposta del gruppo verso il committente è stata quella di implementare un pagamento diviso tra più utenti.
		 L'idea è stata definita buona ma troppo complessa da implementare nell'attuale progetto ed è stata considerata come un ottimo spunto per un progetto futuro.
	\end{itemize}

	\item \textbf{Web app:}
	\begin{itemize}
		\item è un'applicazione di sostegno agli e-commerce\glo{} che aderiranno a questa modalità di pagamento alla quale si viene reindirizzati durante fase di pagamento (in stile PayPal).
		\item Necessario lato Admin alla quale possono accedere i venditori per il monitoraggio e la gestione di eventuale eliminazione delle transazioni. 
		\item Necessario lato utente che permetta la scanzione del qr code\glo{} in fase di consegna così da confermare la ricezione del pacco da parte del cliente e sbloccare la fase di spostamento di denaro dallo smart contract\glo{} al wallet del venditore.
	\end{itemize}

	\item \textbf{Richieste aggiuntive dell'azienda:}
	\begin{itemize}
		\item L'azienda ha mostrato interesse ad una possibile implementazione del progetto utilizzando blockchain\glo{} differenti da quella proposta quali Cardano\glo{} e Avalance\glo{}.
		 Sottolineando però che potrebbe risultare più difficile trovare documentazione a rigurdo essendo le cryptovalute precedentemente citate più giovani e meno studiate di Ethereum\glo{}.
	\end{itemize}

	\item \textbf{Disponibilità azienda:}
	\begin{itemize}
		\item L'azienda mette a disposizione link alla documentazione necessaria e video di un corso di circa 40-50 ore in cui descrivono i meccanismi della blockchain\glo{}.
		
	\end{itemize}

	\item \textbf{Contatti Azienda:}
	\begin{itemize}
		\item Canale Discord\glo{} sulla formazione delle blockchain\glo{} in cui inseriscono informazioni e risorse formative.
	\end{itemize}

	\item \textbf{Varie ed eventuali:}
	\begin{itemize}
		\item è stato proposto dal gruppo di usare una rete come Ethereum\glo{} per i test e per la fase di sviluppo e di prevedere che in fase di produzione venga fatto un tentativo di prova in una blockchain\glo{} differente (tipo Avalanche) per valutarne la fattibilità.
	\end{itemize}

\end{enumerate}

\pagebreak

\subsection{Tracciamento delle decisioni}

\begin{table}[H]
	\centering
	\renewcommand{\arraystretch}{1.8}
	\rowcolors{2}{green!100!black!40}{green!100!black!30}
	\begin{tabular}{c | c}
		\rowcolor[HTML]{125e28}
		\multicolumn{1}{c}{\color[HTML]{FFFFFF} \textbf{ID}} &
		\multicolumn{1}{c}{\color[HTML]{FFFFFF} \textbf{Decisione}} \\
		\hline
		VE-2021-11-10-1 & Chiariti i dubbi, decisamente interessante, papabile scelta. \\ \hline

	\end{tabular}
\end{table}