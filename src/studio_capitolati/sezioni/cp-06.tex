\section{Capitolato C6 - Smart4Energy}\label{section:c6}

\subsection{Informazioni generali}
	\begin {itemize}
		\item \textbf{Nome:} \textit{Smart4Energy};
		\item \textbf{Proponente:} \textit{Socomec};
		\item \textbf{Committente:} \textit{Prof. Tullio Vardanega e Prof. Riccardo Cardin}.
	\end{itemize}

	\subsection{Descrizione del capitolato}
		L’obiettivo del capitolato è quello di cambiare il modo in cui le persone si interfacciano agli UPS\glo\ dell'azienda e il modo in cui viene erogato il servizio di assistenza.

	\subsection{Finalità del progetto}
		Il capitolato è diviso in due parti principali:
		\begin {itemize}
			\item \textbf{Virtual Display:}
			\begin{itemize}
				\item Realizzare un’applicazione mobile in grado di visualizzare lo stato di funzionamento dell’UPS in tempo reale;
				\item L’applicazione dovrà connettersi all’UPS\glo{} e leggere le informazioni di stato tramite protocollo Modbus\glo\ e interfaccia WiFi/Ethernet.
				\item (opzionale) Sviluppo della comunicazione App-UPS\glo{} tramite protocollo BLE\glo.
			\end{itemize}
			\item \textbf{Temote Support:}
			\begin{itemize}
				\item Rendere disponibili su console/testo, al tecnico remoto, le stesse informazioni che il cliente vede in locale con un tempo di refresh non superiore a 5 secondi, e le eventuali informazioni del cliente connesso;
				\item (opzionale) Sviluppo dell’interfaccia grafica con un look and feel simile a quello dell’App per il tecnico remoto
			\end{itemize}
			\item (opzionale) Analisi e sviluppo di una connessione sicura tra App e tecnico remoto tramite utilizzo di	tecnologie per cifratura del canale, certificati, autenticazione/autorizzazioni utenti e dispositivi;
			\item (opzionale) Arricchire l’esperienza d’uso del supporto remoto tramite comunicazione audio e video tra	tecnico e cliente.
		\end {itemize}

	\subsection {Tecnologie interessate}
		\begin {itemize}
			\item \textbf{WebRTC:} tecnologia open source\glo{} che consente ai browser di effettuare in tempo reale la videochat;
			\item Non suggeriscono ulteriori tecnologie per lo sviluppo dell'applicazione.
		\end {itemize}

	\subsection{Aspetti positivi}
		\begin{itemize}
			\item Utilizzo (e apprendimento) di linguaggi di programmazione per lo sviluppo mobile;
			\item Utilizzo dei protocolli Modbus\glo{} e BLE\glo{};
			\item Potrebbe essere molto utile all'azienda proponente.
		\end{itemize}

	\subsection {Criticità e fattori di rischio}
		\begin{itemize}
			\item Test non effettuabili su UPS\glo{} reale;
			\item L'azienda è più specializzata in ambito elettronico che informatico.
	\end{itemize}

\subsection {Conclusioni}
	Il team ha dimostrato da subito scarso interesse riguardo questo capitolato, ritenendolo poco stimolante rispetto le altre proposte.