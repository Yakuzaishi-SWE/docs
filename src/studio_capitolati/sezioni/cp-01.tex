\section{Capitolato C1 - Bot4Me}\label{section:c1}

\subsection{Informazioni generali}
	\begin {itemize}
		\item \textbf{Nome:} \textit{Bot4Me};
		\item \textbf{Proponente:} \textit{Imola Informatica};
		\item \textbf{Committente:} \textit{Prof. Tullio Vardanega e Prof. Riccardo Cardin}.
	\end{itemize}

	\subsection{Descrizione generale}
		L’obiettivo del capitolato è quello di realizzare un'applicazione chatbot in grado di interpretare un flusso testuale (chat) che permetta ai dipendenti dell'azienda di eseguire operazioni di gestione e consuntivazione nel modo più intuitivo e user-friendly possibile.
		L'applicazione chatbot deve essere in grado, ad esempio, di eseguire le seguenti operazioni:
		\begin {itemize}
			\item Effettuare le operazioni di check-in e check-out sull’applicativo EMT;
			\item Inserire un consuntivo dell’attività giornaliera svolta sull’applicativo EMT;
			\item Aprire il cancello della sede aziendale;
			\item Creare una nuova riunione su un applicativo per videoconferenze;
			\item Ricercare dei documenti sul repository\glo{} aziendale;
			\item Creare dei ticket di tracciamento per bug\glo{} o progetti.
		\end{itemize}

	\subsection{Finalità del progetto}
		L'azienda propone il seguente flusso di operazioni per l'attività dell'applicazione:
		\begin {itemize}
			\item Alla prima interazione terze il sistema deve chiedere all’utente le credenziali del dipendente, per poi salvarle nella base di dati al fine di non chiederle per ogni interazione;
			\item Le comunicazioni tra app e server avvengono nel momento in cui l'utente invia il messaggio;
			\item Il server prende in carico la richiesta;
			\item Esegue le operazioni;
			\item Conferma la corretta esecuzione con un messaggio verso l'utente;
			\item In caso di errore o di mancata interpretazione del messaggio, il server procede mediante una segnalazione verso l'utente.
		\end {itemize}

	\subsection {Tecnologie interessate}
		L'azienda non suggerisce l'utilizzo di specifiche tecnologie all'interno del documento.

	\subsection{Aspetti positivi}
		\begin{itemize}
			\item Progetto innovativo;
			\item Potrebbe essere molto utile all'azienda proponente.
		\end{itemize}
	\subsection {Criticità e fattori di rischio}
		\begin{itemize}
			\item Scarsa reperibilità dell'azienda;
			\item Non all'altezza degli altri capitolati.
	\end{itemize}
\subsection {Conclusioni}
Nonostante il team di sviluppo ritenenesse stimolante ed innovativo questo capitolato, esso è stato scartato perchè ritenuto meno interessante rispetto ad altri progetti.