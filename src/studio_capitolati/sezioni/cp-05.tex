\section{Capitolato C5 - Login Warrior}\label{section:c5}

\subsection{Informazioni generali}
    \begin {itemize}
        \item \textbf{Nome:} \textit{Login Warrior};
        \item \textbf{Proponente:} \textit{Zucchetti S.p.A};
        \item \textbf{Committente:} \textit{Prof. Tullio Vardanega e Prof. Riccardo Cardin}.
    \end{itemize}

\subsection{Descrizione del capitolato}
    L’obiettivo del capitolato è quello di costruire un sistema di analisi esplorativa dei dati ottenuti dalle login per poter costantemente studiare i pattern d’uso regolare e i pattern di attacco al fine di migliorare l'esperienza d'uso degli utenti.\\
    Bisognerà quindi sviluppare un’applicazione di visualizzazione dei dati di login a supporto della fase esplorativa (EDA\glo).\\
    L’applicazione dovrà presentare i dati con degli appositi grafici che aiutino a distinguere i casi sospetti dai casi leciti.

\subsection{Finalità del progetto}
    L’analisi dei dati avviene sfruttando tecniche di statistica, machine learning\glo\ e intelligenza artificiale\glo.\\
    Vi si possono distinguere quattro fasi fondamentali: l'acquisizione dei dati, la loro pulizia, l'analisi dei dati ed infine l'interpretazione.\\
    Per quanto riguarda l'interpretazione dei dati, nella prima fase esplorativa, ci si focalizza nella visualizzazione grafica del dato.\\
    A tal proposito l'applicazione da sviluppare dovrà presentare almeno le seguenti visualizzazioni:
    \begin {itemize}
        \item \textbf{Scatter plot:} permette di visualizzare dati su due dimensioni (fino ad un massimo di cinque dimensioni);
        \item \textbf{Parallel coordinates:} permette di visualizzare più assi e con gli opportuni ordinamenti facilita l’identificazione di casi che non rientrano nella normalità;
        \item \textbf{Force-directed graph:} si basa sul concetto di distanza tra i punti. Ogni elemento viene posizionato vicino ad altri con le stesse caratteristiche e allontanato da quelli diversi;
        \item \textbf{Sankey diagram:} ordina gli effetti di ogni dimensione esplorata, evidenziando gli effetti in cascata.
    \end {itemize}

\subsection {Tecnologie interessate}
    L'applicazione dovrà utilizzare le seguenti tecnologie:
    \begin {itemize}
        \item \textbf{HTML\glo}/\textbf{CSS\glo}/\textbf{Javascript\glo} con libreria \textbf{D3.js\glo} per la visualizzazione dei dati di login.
    \end {itemize}

\subsection{Aspetti positivi}
    \begin{itemize}
        \item L’azienda proponente ha una lunga storia alle spalle ed è una delle maggiori esponenti nel proprio settore;
        \item La prossimità geografica dell’azienda alle sedi universitarie, fattore che renderebbe agevoli eventuali incontri di presenza;
        \item Nel capitolato si tratta di machine learning e di intelligenza artificiale, argomenti molto interessanti in un futuro ambito lavorativo;
        \item Gli strumenti HTML/CSS\glo{}/JavaScript sono già stati visti nel corso di Tecnologie Web;
        \item La libreria D3.js\glo{} è ben documentata e largamente fornita di esempi sulle visualizzazioni richieste;
        \item L'azienda metterà a disposizione i dati da presentare in formato CSV\glo;
        \item L'azienda è aperta alle proposte da parte nostra e all'eventuale aggiunta di esse come requisiti opzionali.
    \end{itemize}

\subsection {Criticità e fattori di rischio}
    \begin{itemize}
        \item Il capitolato proposto si focalizza maggiormente sull'aspetto algoritmico matematico dell'analisi e visualizzazione dei dati più che nello sviluppo dell'applicativo web;
        \item Altri capitolati hanno per argomento situazioni più stimolanti e concrete in confronto all'analisi ed elaborazione dei dati;
        \item La libreria D3.js\glo{} essendo molto vasta potrebbe richiedere diverso tempo per imparare a padroneggiarla.
    \end{itemize}
    
\subsection {Conclusioni}
Il progetto, nonostante fosse di interesse comune, non è stato scelto in quanto si è deciso di dedicarsi ad un altro capitolato che raccoglieva maggiori stimoli e interessi da parte dei membri.