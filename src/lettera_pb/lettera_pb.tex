\documentclass[a4paper, 10pt]{article}
\usepackage[utf8]{inputenc}
\usepackage[italian]{babel}
\usepackage[margin=0.8in]{geometry}
\usepackage{graphicx}
\usepackage{float}
\usepackage[table,xcdraw]{xcolor}
\usepackage[hidelinks]{hyperref}

\begin{document}
\begin{center}
\includegraphics[width=0.3\textwidth]{../template/images/logo.png}
\hspace{3cm}
\includegraphics[width=0.2\textwidth]{../template/images/logoUnipd.png}\\
\end{center}
\begin{flushright}
    \
    \textbf{}\\
    Padova - 11 Giugno 2022
\end{flushright}
Egregio Prof. Vardanega,\\
Egregio Prof. Cardin,\\\\
Con la presente il gruppo Yakuzaishi intende comunicarvi la partecipazione alla Customer Acceptance, al fine di esporvi il prodotto completo da voi commissionato, denominato:
\begin{center}
    \textbf{ShopChain}
\end{center}
proposto dall'azienda Sync Lab S.r.l.\\\\
Nel repository all'indirizzo:
\begin{center}
    \href{https://yakuzaishi-swe.github.io/docs/CA/}{\underline{https://yakuzaishi-swe.github.io/docs/CA/}}\\ 
\end{center}
potrete trovare tutti i documenti che trattano in maniera approfondita gli aspetti di pianificazione, normazione, qualifica e analisi.\\\\
Nella cartella \textit{Documenti Esterni} saranno presenti i file:
\begin{itemize}
    \item \textit{analisi\_dei\_requisiti\_v3.0.0};
    \item \textit{piano\_di\_progetto\_v3.0.0};
    \item \textit{piano\_di\_qualifica\_v3.0.0};
    \item \textit{glossario\_v2.0.0};
    \item \textit{manuale\_utente\_v2.0.0};
    \item \textit{manuale\_sviluppatore\_v1.0.0}.
\end{itemize}
Nella cartella \textit{Documenti Interni} saranno presenti i file:
\begin{itemize}
    \item \textit{norme\_di\_progetto\_v3.0.0}.
\end{itemize}
Saranno inoltre presenti, nelle rispettive cartelle, i seguenti verbali di periodo:
\begin{itemize}
    \item \textit{verbale\_interno\_2022\_05\_18};
    \item \textit{verbale\_interno\_2022\_05\_24};
    \item \textit{verbale\_interno\_2022\_06\_10};
    \item \textit{verbale\_esterno\_2022\_06\_07};
    \item \textit{verbale\_esterno\_2022\_06\_10};
\end{itemize}
Per completezza, inseriamo di seguito la lista dei membri del gruppo.
\begin{table}[H]
    \centering
    \renewcommand{\arraystretch}{1.8}
    \rowcolors{2}{green!100!black!40}{green!100!black!30}
    \begin{tabular}{l|cccccc|c}
      \rowcolor[HTML]{125E28} 
      \multicolumn{1}{c}{\color[HTML]{FFFFFF}\textbf{Nome}} 
      & \color[HTML]{FFFFFF}\textbf{Matricola}\\
      \hline
      Bugno Francesco & 1170991\\
      Busacca Luca & 1227589\\
      Carturan Luca & 1094033\\
      Filosofo Michele & 1094325\\
      Furlan Dario & 1223861\\
      Mattarello Francesco & 1172722\\
      Midena Matteo & 1227272\\
      \hline             
    \end{tabular}
  \end{table}
\textbf{}\\
Cordiali saluti,\\
\begin{flushright}
    \
    Responsabile di Progetto\\
    Luca Carturan \hspace{1.1cm}\hspace{1cm}
\end{flushright}
\end{document}
