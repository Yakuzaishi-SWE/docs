\section{Specifica dei test}\label{section:specifica_test}

%%%%%%%%%%%%%%%%%%%%%%%%%%%%%%%%%%%%%%%%%%%%%%%%%%%%%%%%%%%%%%%%%%%%%%%%%%%%%%%
\subsection{Test di accettazione}\label{subsection:test_accettazione}
I test di accettazione sono necessari per dimostrare che il prodotto soddisfi i requisiti minimi concordati con il proponente. \\
Essi si compongono dei test di sistema e vengono eseguiti durante il collaudo finale sia dai membri del
gruppo che dall'azienda proponente, sotto supervisione del gruppo stesso.

%%%%%%%%%%%%%%%%%%%%%%%%%%%%%%%%%%%%%%%%%%%%%%%%%%%%%%%%%%%%%%%%%%%%%%%%%%%%%%%
\subsection{Test di sistema}\label{subsection:test_sistema}
\begin{table}[H]
  \centering
  \renewcommand{\arraystretch}{1.8}
  \rowcolors{2}{green!100!black!40}{green!100!black!30}
  \begin{tabular}{c|p{8cm}|c|c}
    \rowcolor[HTML]{125E28}
    \color[HTML]{FFFFFF}\textbf{ID Test}
              & \multicolumn{1}{c}{\color[HTML]{FFFFFF}\textbf{Descrizione}}
              & \color[HTML]{FFFFFF}\textbf{ID Requisiti}
              & \color[HTML]{FFFFFF}\textbf{Stato}                                                                                                  \\
    \hline
    TS1F1     & Verificare che la richiesta di checkout da un E-Commerce\glo{} avvenga correttamente.                                     & R1F1     & NI \\
    TS1F2     & Verificare che l'utente visualizzi correttamente le diverse tipologie di pagamento.                                       & R1F2     & NI \\
    TS1F2.1   & Verificare che l'utente possa scegliere la tipologia di pagamento unico correttamente.                                    & R1F2.1   & NI \\
    TS2F2.2   & Verificare che l'utente possa scegliere la tipologia di pagamento MoneyBox\glo{} correttamente.                           & R2F2.2   & NI \\
    TS2F2.2.1 & Verificare che l'utente possa visualizzare lo stato di completamento della MoneyBox\glo{} correttamente.                  & R2F2.2.1 & NI \\
    TS2F2.2.2 & Verificare che l'utente possa visualizzare l'invito di partecipazione alla MoneyBox\glo{} correttamente.                  & R2F2.2.2 & NI \\
    TS3F2.2.3 & Verificare che l'utente possa visualizzare una traduzione visiva della MoneyBox\glo{} correttamente.                      & R3F2.2.3 & NI \\
    TS2F2.2.4 & Verificare che in caso di chiusura della MoneyBox\glo{} i soldi vengano restituiti correttamente.                         & R2F2.2.4 & NI \\
    TS2F2.2.5 & Verificare che l'utente possa visualizzare l'elenco delle transazione dei partecipanti alla MoneyBox\glo{} correttamente. & R2F2.2.5 & NI \\
    TS1F3     & Verificare che l'utente possa visualizzare il totale dell'ordine correttamente.                                           & R1F3     & NI \\
    TS1F4     & Verificare che l'utente possa effettuare la connessione a Metamask\glo{} correttamente.                                   & R1F4     & NI \\
  \end{tabular}
\end{table}
\begin{center}
  \textit{\small Continua nella pagina successiva}
\end{center}
\begin{table}[H]
  \centering
  \renewcommand{\arraystretch}{1.8}
  \rowcolors{2}{green!100!black!40}{green!100!black!30}
  \begin{tabular}{c|p{8cm}|c|c}
    \rowcolor[HTML]{125E28}
    \color[HTML]{FFFFFF}\textbf{ID Test}
              & \multicolumn{1}{c}{\color[HTML]{FFFFFF}\textbf{Descrizione}}
              & \color[HTML]{FFFFFF}\textbf{ID Requisiti}
              & \color[HTML]{FFFFFF}\textbf{Stato}                                                                                                 \\
    \hline
    TS1F5     & Verificare che l'utente possa pagare correttamente.                                                                       & R1F5     & NI \\
    TS1F5.1   & Verificare che il versamento per il pagamento unico copra l'intera somma richiesta.                                       & R1F5.1   & NI \\
    TS2F5.2   & Verificare che il versamento per il pagamento tramite MoneyBox\glo{} possa coprire solo una parte della somma richiesta.  & R2F5.2   & NI \\
    TS1F5.3   & Verificare che l'utente possa visualizzare correttamente il codice di sblocco generato dal sistema.                       & R1F5.3   & NI \\
    TS1F5.4   & Verificare che l'utente possa visualizzare correttamente un messaggio d'errore nel caso in cui la transazione fallisca.   & R1F5.4   & NI \\
    TS1F6     & Verificare che la richiesta di rimborso funzioni correttamente.                                                           & R1F6     & NI \\
    TS1F7     & Verificare che l'acquirente possa sbloccare, correttamente, i fondi dallo Smart Contract\glo{} dopo avvenuta ricezione.   & R1F7     & NI \\
    TS1F8     & Verificare che l'utente possa visualizzare le transazioni.                                                                & R1F8     & NI \\
    TS2F8.1   & Verificare che il venditore possa visualizzare le transazioni in entrata pagate.                                          & R2F8.1   & NI \\
    TS2F8.1.1 & Verificare che il venditore possa visualizzare le transazioni in entrata pagate ma non sbloccate.                         & R2F8.1.1 & NI \\
    TS2F8.1.2 & Verificare che il venditore possa visualizzare le transazioni in entrata pagate e sbloccate.                              & R2F8.1.2 & NI \\
    TS2F8.1.3 & Verificare che il venditore possa visualizzare le transazioni in entrata pagate cancellate.                               & R2F8.1.3 & NI \\
    TS1F8.2   & Verificare che il proprietario dell'ordine possa visualizzare le transazioni in uscita.                                   & R1F8.2   & NI \\
    TS2F8.2.1 & Verificare che il proprietario dell'ordine possa visualizzare le transazioni in uscita non pagate.                        & R2F8.2.1 & NI \\
    TS2F8.2.2 & Verificare che il proprietario dell'ordine possa visualizzare le transazioni in uscita pagate ma non sbloccate.           & R2F8.2.2 & NI \\
    \end{tabular}
\end{table}
\begin{center}
  \textit{\small Continua nella pagina successiva}
\end{center}
\begin{table}[H]
  \centering
  \renewcommand{\arraystretch}{1.8}
  \rowcolors{2}{green!100!black!40}{green!100!black!30}
  \begin{tabular}{c|p{8cm}|c|c}
    \rowcolor[HTML]{125E28}
    \color[HTML]{FFFFFF}\textbf{ID Test}
           & \multicolumn{1}{c}{\color[HTML]{FFFFFF}\textbf{Descrizione}}
           & \color[HTML]{FFFFFF}\textbf{ID Requisiti}
           & \color[HTML]{FFFFFF}\textbf{Stato}                                                                                      \\
    \hline
    TS2F8.2.3 & Verificare che il proprietario dell'ordine possa visualizzare le transazioni in uscita pagate e sbloccate.                & R2F8.2.3 & NI \\
    TS2F8.2.4 & Verificare che il proprietario dell'ordine possa visualizzare le transazioni in uscita cancellate.                        & R2F8.2.4 & NI \\
    TS2F9  & Verificare che si possa convertire correttamente l'ammontare depositato in stable coin\glo{}.                    & R2F9  & NI \\
    TS3F10 & Verificare che, dopo la transazione, una fee percentuale venga destinata al fondo ShopChain correttamente.       & R3F10 & NI \\
    TS1F11 & Verificare che l'utente possa visualizzare correttamente l'indirizzo del suo wallet\glo{}.                             & R1F11 & NI \\
    TS1F11.1 & Verificare che l'utente possa visualizzare correttamente l'indirizzo del suo wallet\glo{} in forma testuale.         & R1F11.1 & NI \\
    TS1F11.2 & Verificare che l'utente possa visualizzare correttamente un avviso della mancata connessione a Metamask\glo{}.         & R1F11.2 & NI \\
    TS3F11.3 & Verificare che l'utente possa visualizzare correttamente l'indirizzo del suo wallet\glo{} sotto forma di sequenza di emoji.  & R3F11.3 & NI \\
    TS3F12 & Verificare che gli utenti partecipanti ad una MoneyBox\glo{} possano ricevere una notifica al suo completamento.  & R3F12 & NI \\
    TS1F13 & Verificare che l'utente possa visualizzare correttamente il dettaglio dell'errore della transazione.  & R1F13 & NI \\
    TS1F14 & Verificare che l'utente possa visualizzare correttamente lo stato di connessione a Metamask\glo{}.  & R1F14 & NI \\
    TS1F14.1 & Verificare che l'utente possa visualizzare un suggerimento se connesso correttamente.  & R1F14.1 & NI \\
    TS1F14.2 & Verificare che l'utente possa visualizzare nel caso in cui non sia installato Metamask\glo{}.  & R1F14.2 & NI \\
    TS1F14.3 & Verificare che l'utente possa visualizzare un errore nel caso in cui la blockchain\glo{} selezionata non sia corretta. & R1F14.3 & NI \\
  \end{tabular}
\end{table}
\begin{center}
  \textit{\small Continua nella pagina successiva}
\end{center}
\begin{table}[H]
  \centering
  \renewcommand{\arraystretch}{1.8}
  \rowcolors{2}{green!100!black!40}{green!100!black!30}
  \begin{tabular}{c|p{8cm}|c|c}
    \rowcolor[HTML]{125E28}
    \color[HTML]{FFFFFF}\textbf{ID Test}
           & \multicolumn{1}{c}{\color[HTML]{FFFFFF}\textbf{Descrizione}}
           & \color[HTML]{FFFFFF}\textbf{ID Requisiti}
           & \color[HTML]{FFFFFF}\textbf{Stato}                                                                                      \\
    \hline
    TS1F14.4 & Verificare che l'utente possa visualizzare un errore nel caso in cui non abbia connesso un account a ShopChain.  & R1F14.4 & NI \\
    TS1F15 & Verificare che l'utente possa visualizzare un messaggio di avviso nel caso in cui la transazione sia già presente in blockchain\glo{}.  & R1F15 & NI \\
    TS1F16 & Verificare che la richiesta di un ordine venga caricata correttamente su una blockchain\glo{} pubblica.                  & R1V1  & NI \\
    TS1F17 & Verificare che il pagamento e le transazioni siano gestite correttamente dallo Smart Contract\glo{}.                     & R1V2  & NI \\
  \end{tabular}
  \caption{Test di sistema}
\end{table}

%%%%%%%%%%%%%%%%%%%%%%%%%%%%%%%%%%%%%%%%%%%%%%%%%%%%%%%%%%%%%%%%%%%%%%%%%%%%%%%
\subsection{Test d'integrazione}\label{subsection:test_integrazione}
\begin{table}[H]
  \centering
  \renewcommand{\arraystretch}{1.8}
  \rowcolors{2}{green!100!black!40}{green!100!black!30}
  \begin{tabular}{c|p{8cm}|c}
    \rowcolor[HTML]{125E28}
    \color[HTML]{FFFFFF}\textbf{Codice}
        & \multicolumn{1}{c}{\color[HTML]{FFFFFF}\textbf{Descrizione}}
        & \color[HTML]{FFFFFF}\textbf{Stato}                                                                                       \\
    \hline
    TI1 & Verificare che il collegamento con Metamask\glo{} avvenga correttamente.                                                  & NI \\
    TI2 & Verificare che il collegamento tra Front-End\glo{} e Smart Contract\glo{} tramite libreria Web3.js\glo{} avvenga correttamente. & NI \\
  \end{tabular}
  \caption{Test d'integrazione}
\end{table}

\vspace{1cm}
%%%%%%%%%%%%%%%%%%%%%%%%%%%%%%%%%%%%%%%%%%%%%%%%%%%%%%%%%%%%%%%%%%%%%%%%%%%%%%%
\subsection{Test di unità}\label{subsection:test_unita}
Per garantire il corretto funzionamento di ogni minimo componente autonomo del sistema vengono eseguiti i test di unità. \\
Fino a questo momento sono stati presi in considerazione esclusivamente i test eseguiti su linguaggio Solidity\glo{}.
I test di unità rimanenti verranno stabiliti durante il periodo di \textit{Progettazione di dettaglio e codifica requisiti obbligatori}.

\subsubsection{Test di unità - Solidity - OrderManager}\label{subsubsection:TUO}
\begin{table}[H]
  \centering
  \renewcommand{\arraystretch}{1.8}
  \rowcolors{2}{green!100!black!40}{green!100!black!30}
  \begin{tabular}{c|p{10cm}|c}
    \rowcolor[HTML]{125E28}
    \color[HTML]{FFFFFF}\textbf{ID Test}
         & \multicolumn{1}{c}{\color[HTML]{FFFFFF}\textbf{Descrizione}}
         & \color[HTML]{FFFFFF}\textbf{Stato}                                                                                                    \\
    \hline
    TUO01 & Si verifica che il contatore degli ordini sia inizializzato a zero.                                                               & S \\
    TUO02 & Si verifica che l'ordine creato sia correttamente.                                                                                & S \\
    TUO03 & Si verifica che il contratto abbia ricevuto l'ammontare corretto.                                                                 & S \\
    TUO04 & Si verifica il caso fallimentare in cui l'utente tenta di inviare un importo inferiore a quello richiesto.                        & S \\
    TUO05 & Si verifica il caso fallimentare in cui l'utente prova a registrare un ordine che ha come venditore se stesso.                    & S \\
    TUO06 & Si verifica il caso fallimentare in cui l'utente prova a inviare un ammontare pari a zero.                                        & S \\
    TUO07 & Si verifica il caso fallimentare in cui l'utente prova a inviare un ammontare negativo.                                           & S \\
    TUO08 & Si verifica il caso fallimentare in cui l'utente non ha abbastanza fondi.                                                         & S \\
    TUO09 & Si verifica il caso fallimentare in cui il frontend\glo{} richiede la creazione di un ordine con un ID precedentemente usato.     & S \\
    TUO10 & Si verifica che le informazioni dell'ordine siano corrette.                                                                       & S \\
    TUO11 & Si verifica che lo stato dell'ordine sia modificato e settato su "CLOSED" una volta confermata la ricezione dall'acquirente.      & S \\
    TUO12 & Si verifica che il venditore abbia ricevuto i fondi correttamente una volta confermata la ricezione dall'acquirente.              & S \\
    TUO13 & Si verifica il caso fallimentare in cui l'ID dell'ordine da sbloccare è errato.                                                   & S \\
    TUO14 & Si verifica il caso fallimentare in cui il codice di sblocco non coincide con quello registrato in blockchain\glo{}.              & S \\
    TUO15 & Si verifica il caso fallimentare in cui l'ordine viene confermato da un utente con indirizzo differente da quello del compratore. & S \\
    TUO16 & Si verifica che il rimborso sia richiesto dal compratore una volta settato su "CANCELLED" lo stato dell'ordine.                   & S \\
  \end{tabular}
\end{table}
\begin{center}
  \textit{\small Continua nella pagina successiva}
\end{center}
\begin{table}[H]
  \centering
  \renewcommand{\arraystretch}{1.8}
  \rowcolors{2}{green!100!black!40}{green!100!black!30}
  \begin{tabular}{c|p{10cm}|c}
    \rowcolor[HTML]{125E28}
    \color[HTML]{FFFFFF}\textbf{ID Test}
         & \multicolumn{1}{c}{\color[HTML]{FFFFFF}\textbf{Descrizione}}
         & \color[HTML]{FFFFFF}\textbf{Stato}                                                                                                               \\
    \hline
    TUO17 & Si verifica che il rimborso sia richiesto dal venditore una volta settato su "CANCELLED" lo stato dell'ordine.                              & S \\
    TUO18 & Si verifica che i soldi siano restituiti correttamente al compratore quando è richiesto il rimborso dal compratore.                         & S \\
    TUO19 & Si verifica che i soldi siano restituiti correttamente al compratore quando è richiesto il rimborso dal venditore.                          & S \\
    TUO20 & Si verifica il caso fallimentare in cui non è possibile chiedere il rimborso di un ordine chiuso.                                           & S \\
    TUO21 & Si verifica il caso fallimentare in cui un utente con indirizzo diverso da quello del compratore o venditore richiede il rimborso.          & S \\
    TUO22 & Si verifica che venga restituito correttamente il saldo del contratto.                                                                      & S \\
    TUO23 & Si verifica che venga restituito correttamente l'address dell'acquirente.                                                                   & S \\
    TUO24 & Si verifica che venga restituito correttamente l'address del venditore.                                                                     & S \\
    TUO25 & Si verifica che venga restituito correttamente l'ammontare da pagare.                                                                       & S \\
    TUO26 & Si verifica che venga restituito correttamente lo stato dell'ordine.                                                                        & S \\
    TUO27 & Si verifica che venga restituito correttamente l'ordine partendo dal suo ID.                                                                & S \\
    TUO28 & Si verifica che vengano restituiti correttamente tutti gli ordini di un acquirente.                                                         & S \\
    TUO29 & Si verifica che vengano restituiti correttamente tutti gli ordini di un venditore.                                                          & S \\
  \end{tabular}
  \caption{Test di unità - Solidity - OrderManager}
\end{table}



\paragraph{Tracciamento test di unità - Solidity - OrderManager}\label{paragraph:tracciamento_TUO}

\begin{table}[H]
  \centering
  \renewcommand{\arraystretch}{1.8}
  \rowcolors{2}{green!100!black!40}{green!100!black!30}
  \begin{tabular}{c|p{15cm}}
    \rowcolor[HTML]{125E28}
    \color[HTML]{FFFFFF}\textbf{ID Test}
         & \multicolumn{1}{c}{\color[HTML]{FFFFFF}\textbf{Metodo}}                                                                                       \\
    \hline
    TUO01 & smart-contract/test/1\_OrderManager.test.js:it("should have the initial order number to 0", ...)                                                                                        \\
    TUO02 & smart-contract/test/1\_OrderManager.test.js:describe("single payment order",\newline()=$>$\{it('order created correctly', ...)                                      \\
    TUO03 & smart-contract/test/1\_OrderManager.test.js:describe("single payment order",\newline()=$>$\{it("the contract is filled with the correct amount", ...) \\
    TUO04 & smart-contract/test/1\_OrderManager.test.js:describe("single payment order",\newline()=$>$\{describe("failure cases",()=$>$\{it("user tries to insert an amount less than the required amount", ...)                                                                                       \\
    TUO05 & smart-contract/test/1\_OrderManager.test.js:describe("single payment order",\newline()=$>$\{describe("failure cases",()=$>$\{it("user tries to order his item, or send funds to itself", ...)                                \\
    TUO06 & smart-contract/test/1\_OrderManager.test.js:describe("single payment order",\newline()=$>$\{describe("failure cases",()=$>$\{it("user tries to pass value equal to zero", ...)                                  \\
    TUO07 & smart-contract/test/1\_OrderManager.test.js:describe("single payment order",\newline()=$>$\{describe("failure cases",()=$>$\{it("user tries to send negative coin value", ...)                                          \\
    TUO08 & smart-contract/test/1\_OrderManager.test.js:describe("single payment order",\newline()=$>$\{describe("failure cases",()=$>$\{it("user hasn't enough founds", ...)                                         \\
    TUO09 & smart-contract/test/1\_OrderManager.test.js:describe("single payment order",\newline()=$>$\{describe("failure cases",()=$>$\{it("front end require the creation of an order with the same order id", ...)                                \\
    TUO10 & smart-contract/test/1\_OrderManager.test.js:describe("single payment order",\newline()=$>$\{it('get the correct order infos', ...)                                   \\
    TUO11 & smart-contract/test/1\_OrderManager.test.js:describe("order confirmation",\newline()=$>$\{it("should update state", ...)                                                          \\
    TUO12 & smart-contract/test/1\_OrderManager.test.js:describe("order confirmation",\newline()=$>$\{it("should release payment to seller", ...)                                       \\                                                      
    TUO13 & smart-contract/test/1\_OrderManager.test.js:describe("order confirmation",\newline()=$>$\{describe("failure cases",async ()=$>$\{it("order id isn't correct", ...)                                                          \\
  \end{tabular}
\end{table}
\begin{center}
  \textit{\small Continua nella pagina successiva}
\end{center}
\begin{table}[H]
  \centering
  \renewcommand{\arraystretch}{1.8}
  \rowcolors{2}{green!100!black!40}{green!100!black!30}
  \begin{tabular}{c|p{15cm}}
    \rowcolor[HTML]{125E28}
    \color[HTML]{FFFFFF}\textbf{ID Test}
         & \multicolumn{1}{c}{\color[HTML]{FFFFFF}\textbf{Metodo}}                                                                                       \\
    \hline
    TUO14 & smart-contract/test/1\_OrderManager.test.js:describe("order confirmation",\newline()=$>$\{describe("failure cases",async ()=$>$\{it("order unlock code doesn't match with unlock code saved", ...)                                                               \\
    TUO15 & smart-contract/test/1\_OrderManager.test.js:describe("order confirmation",\newline()=$>$\{describe("failure cases",async ()=$>$\{it("order is unlock from an address different from buyer address", ...)                                                         \\
    TUO16 & smart-contract/test/1\_OrderManager.test.js:describe("order refund",\newline async ()=$>$\{describe("should set the cancelled state",()=$>$\{it("from buyer", ...)                                                                               \\
    TUO17 & smart-contract/test/1\_OrderManager.test.js:describe("order refund",\newline async ()=$>$\{describe("should set the cancelled state",()=$>$\{it("from seller", ...)                                                                             \\
    TUO18 & smart-contract/test/1\_OrderManager.test.js:describe("order refund",\newline async ()=$>$\{describe("should move funds back to the buyer",async ()=$>$\{it("from buyer [@skip-on-coverage]", ...)                                                                              \\
    TUO19 & smart-contract/test/1\_OrderManager.test.js:describe("order refund",\newline async ()=$>$\{describe("should move funds back to the buyer",async ()=$>$\{it("from seller", ...)                                                                            \\
    TUO20 & smart-contract/test/1\_OrderManager.test.js:describe("order refund",\newline async ()=$>$\{describe("failure cases",()=$>$\{it("should not refund an order already closed", ...)                                                 \\
    TUO21 & smart-contract/test/1\_OrderManager.test.js:describe("order refund",\newline async ()=$>$\{describe("failure cases",()=$>$\{it("should be the owner or the seller", ...)                                                 \\
    TUO22 & smart-contract/test/1\_OrderManager.test.js:describe('check getter functions',\newline async ()=$>$\{it("check contractBalance()", ...)                                                                                       \\
    TUO23 & smart-contract/test/1\_OrderManager.test.js:describe('check getter functions',\newline async ()=$>$\{it("check getOwnerAddress(string)", ...)                                                                                    \\
    TUO24 & smart-contract/test/1\_OrderManager.test.js:describe('check getter functions',\newline async ()=$>$\{it("check getSellerAddress(string)", ...)                                                                                   \\
    TUO25 & smart-contract/test/1\_OrderManager.test.js:describe('check getter functions',\newline async ()=$>$\{it("check getAmountToPay(string)", ...)                                                                                     \\
    TUO26 & smart-contract/test/1\_OrderManager.test.js:describe('check getter functions',\newline async ()=$>$\{it("check getOrderState(string)", ...)                                                                                      \\
    TUO27 & smart-contract/test/1\_OrderManager.test.js:describe('check getter functions',\newline async ()=$>$\{it("check getOrderById(string)", ...)                                                                                      \\
  \end{tabular}
\end{table}
\begin{center}
  \textit{\small Continua nella pagina successiva}
\end{center}
\begin{table}[H]
  \centering
  \renewcommand{\arraystretch}{1.8}
  \rowcolors{2}{green!100!black!40}{green!100!black!30}
  \begin{tabular}{c|p{15cm}}
    \rowcolor[HTML]{125E28}
    \color[HTML]{FFFFFF}\textbf{ID Test}
         & \multicolumn{1}{c}{\color[HTML]{FFFFFF}\textbf{Metodo}}                                                                                       \\
    \hline
    TUO28 & smart-contract/test/1\_OrderManager.test.js:describe('check getter functions',\newline async ()=$>$\{it("check getOrdersByBuyer(address)", ...)                                                                                       \\
    TUO29 & smart-contract/test/1\_OrderManager.test.js:describe('check getter functions',\newline async ()=$>$\{it("check getOrdersBySeller(address)", ...)                                                         \\                                                   
  \end{tabular}
  \caption{Tracciamento test di unità - Solidity - OrderManager}
\end{table}
%%%%%%%%%%%%%%%%%%%%%%%%%%%%%%%%%%%%%%%%
\subsubsection{Test di unità - Solidity - MoneyBox} \label{subsubsection:TUM}
\begin{table}[H]
  \centering
  \renewcommand{\arraystretch}{1.8}
  \rowcolors{2}{green!100!black!40}{green!100!black!30}
  \begin{tabular}{c|p{10cm}|c}
    \rowcolor[HTML]{125E28}
    \color[HTML]{FFFFFF}\textbf{ID Test}
         & \multicolumn{1}{c}{\color[HTML]{FFFFFF}\textbf{Descrizione}}
         & \color[HTML]{FFFFFF}\textbf{Stato}                                                                                                    \\
    \hline
    TUM01 & Si verifica che il contatore degli ordini sia inizializzato a zero.                                                               & S \\
    TUM02 & Si verifica che l'ordine creato sia corretto.                                                                                     & S \\
    TUM03 & Si verifica che il trasferimento di un nuovo importo nella MoneyBox\glo{} avvenga correttamente.                                                                & S \\
    TUM04 & Si verifica che il rimborso a partire dalla richiesta del proprietario dell'ordine avvenga correttamente.                                                  & S \\
    TUM05 & Si verifica che il rimborso a partire dalla richiesta del venditore dell'ordine avvenga correttamente.                        & S \\
    TUM06 & Si verifica il caso fallimentare in cui l'utente non ha abbastanza fondi.                    & S \\
    TUM07 & Si verifica il caso fallimentare in cui l'acquirente richiede un rimborso da una "CLOSED" MoneyBox\glo{}.                                        & S \\
    TUM08 & Si verifica il caso fallimentare in cui l'utente prova a inviare un ammontare negativo.                                           & S \\
    TUM09 & Si verifica che vengano ritornate correttamente tutte le transazioni partecipanti alla MoneyBox\glo{}.                                                         & S \\
    TUM10 & Si verifica che venga ritornato il corretto ammontare mancante per il completamento della MoneyBox\glo{}.     & S \\
    TUM11 & Si verifica che vengano ritornati correttamente tutti gli ordini relativi all'acquirente.      & S \\
    TUM12 & Si verifica che vengano ritornati correttamente tutti gli ordini relativi al venditore.              & S \\
  \end{tabular}
  \caption{Test di unità - Solidity - MoneyBox}
\end{table}



\paragraph{Tracciamento test di unità - Solidity - OrderManager}\label{paragraph:tracciamento_TUM}

\begin{table}[H]
  \centering
  \renewcommand{\arraystretch}{1.8}
  \rowcolors{2}{green!100!black!40}{green!100!black!30}
  \begin{tabular}{c|p{15cm}}
    \rowcolor[HTML]{125E28}
    \color[HTML]{FFFFFF}\textbf{ID Test}
         & \multicolumn{1}{c}{\color[HTML]{FFFFFF}\textbf{Metodo}}                                                                                       \\
    \hline
    TUM01 & smart-contract/test/2\_MoneyBox.test.js:it("should have the initial order number to 0", ...)                                                                                      \\
    TUM02 & smart-contract/test/2\_MoneyBox.test.js:describe("moneybox management",\newline()=$>$\{it('moneybox created correctly', ...)                                                         \\
    TUM03 & smart-contract/test/2\_MoneyBox.test.js:describe("moneybox management",\newline()=$>$\{it("new fee transfer into moneybox", ...)              \\
    TUM04 & smart-contract/test/2\_MoneyBox.test.js:describe("moneybox management",\newline()=$>$\{it("refund all fee transfers from moneybox owner", ...)                                   \\
    TUM05 & smart-contract/test/2\_MoneyBox.test.js:describe("moneybox management",\newline()=$>$\{it("refund all fee transfers from moneybox seller", ...)                                 \\
    TUM06 & smart-contract/test/2\_MoneyBox.test.js:describe("moneybox management",\newline()=$>$\{describe("failure cases",()=$>$\{it("should not have insufficient value", ...)                                   \\
    TUM07 & smart-contract/test/2\_MoneyBox.test.js:describe("moneybox management",\newline()=$>$\{describe("failure cases",()=$>$\{it("buyer tries to call refund from a closed moneybox", ...)                                         \\
    TUM08 & smart-contract/test/2\_MoneyBox.test.js:describe("moneybox management",\newline()=$>$\{describe("failure cases",()=$>$\{it("negative fee payment", ...)                                          \\
    TUM09 & smart-contract/test/2\_MoneyBox.test.js:describe("check getter functions",\newline()=$>$\{it("check getMoneyBoxPayments(string)", ...)                                \\
    TUM10 & smart-contract/test/2\_MoneyBox.test.js:describe("check getter functions",\newline()=$>$\{it("check getAmountToFill(string)", ...)                                    \\
    TUM11 & smart-contract/test/2\_MoneyBox.test.js:describe("check getter functions",\newline()=$>$\{it("check getAllBuyerOrders(supercontract, \_buyerAddress)", ...)                                                          \\
    TUM12 & smart-contract/test/2\_MoneyBox.test.js:describe("check getter functions",\newline()=$>$\{it("check getAllSellerOrders(OrderManager, address)", ...)                                                      \\
  \end{tabular}
  \caption{Tracciamento test di unità - Solidity - MoneyBox}
\end{table}