\section{Specifica dei test}\label{section:specifica_test}
Il codice utilizzato per l'identificazione dei test è specificato dettagliatamente nelle \docNameVersionNdP{}, mentre delle sigle utili per la comprensione delle tabelle seguenti sono:
\begin{itemize}
  \item \textbf{S} = test superato;
  \item \textbf{NI} = test non implementato.
\end{itemize}

%%%%%%%%%%%%%%%%%%%%%%%%%%%%%%%%%%%%%%%%%%%%%%%%%%%%%%%%%%%%%%%%%%%%%%%%%%%%%%%
\subsection{Test di accettazione}\label{subsection:test_accettazione}
I test di accettazione sono necessari per dimostrare che il prodotto soddisfi i requisiti minimi concordati con il proponente. \\
Essi si compongono dei test di sistema e vengono eseguiti durante il collaudo finale sia dai membri del
gruppo che dall'azienda proponente, sotto supervisione del gruppo stesso.

%%%%%%%%%%%%%%%%%%%%%%%%%%%%%%%%%%%%%%%%%%%%%%%%%%%%%%%%%%%%%%%%%%%%%%%%%%%%%%%
\subsection{Test di sistema}\label{subsection:test_sistema}
\begin{table}[H]
  \centering
  \renewcommand{\arraystretch}{1.8}
  \rowcolors{2}{green!100!black!40}{green!100!black!30}
  \begin{tabular}{c|p{8cm}|c|c}
    \rowcolor[HTML]{125E28}
    \color[HTML]{FFFFFF}\textbf{ID Test}
              & \multicolumn{1}{c}{\color[HTML]{FFFFFF}\textbf{Descrizione}}
              & \color[HTML]{FFFFFF}\textbf{ID Requisiti}
              & \color[HTML]{FFFFFF}\textbf{Stato}                                                                                                        \\
    \hline
    TS1F1     & Verificare che la richiesta di checkout da un E-Commerce\glo{} avvenga correttamente.                                     & R1F1     & NI \\
    TS1F2     & Verificare che l'utente visualizzi correttamente le diverse tipologie di pagamento.                                       & R1F2     & NI \\
    TS1F2.1   & Verificare che l'utente possa scegliere la tipologia di pagamento unico correttamente.                                    & R1F2.1   & NI \\
    TS2F2.2   & Verificare che l'utente possa scegliere la tipologia di pagamento MoneyBox\glo{} correttamente.                           & R2F2.2   & NI \\
    TS2F2.2.1 & Verificare che l'utente possa visualizzare lo stato di completamento della MoneyBox\glo{} correttamente.                  & R2F2.2.1 & NI \\
    TS2F2.2.2 & Verificare che l'utente possa copiare l'invito di partecipazione alla MoneyBox\glo{} correttamente.                       & R2F2.2.2 & NI \\
    TS3F2.2.3 & Verificare che l'utente possa visualizzare una traduzione visiva della MoneyBox\glo{} correttamente.                      & R3F2.2.3 & NI \\
    TS2F2.2.4 & Verificare che in caso di chiusura della MoneyBox\glo{} i soldi vengano restituiti correttamente.                         & R2F2.2.4 & NI \\
    TS2F2.2.5 & Verificare che l'utente possa visualizzare l'elenco delle transazione dei partecipanti alla MoneyBox\glo{} correttamente. & R2F2.2.5 & NI \\
  \end{tabular}
\end{table}
\begin{center}
  \textit{\small Continua nella pagina successiva}
\end{center}
\begin{table}[H]
  \centering
  \renewcommand{\arraystretch}{1.8}
  \rowcolors{2}{green!100!black!40}{green!100!black!30}
  \begin{tabular}{c|p{8cm}|c|c}
    \rowcolor[HTML]{125E28}
    \color[HTML]{FFFFFF}\textbf{ID Test}
              & \multicolumn{1}{c}{\color[HTML]{FFFFFF}\textbf{Descrizione}}
              & \color[HTML]{FFFFFF}\textbf{ID Requisiti}
              & \color[HTML]{FFFFFF}\textbf{Stato}                                                                                                                     \\
    \hline
    TS1F3     & Verificare che l'utente possa visualizzare il totale dell'ordine correttamente.                                                        & R1F3     & NI \\
    TS1F4     & Verificare che l'utente possa effettuare la connessione a Metamask\glo{} correttamente.                                                & R1F4     & NI \\
    TS1F5     & Verificare che l'utente possa pagare correttamente.                                                                                    & R1F5     & NI \\
    TS1F5.1   & Verificare che il versamento per il pagamento unico copra l'intera somma richiesta.                                                    & R1F5.1   & NI \\
    TS2F5.2   & Verificare che il versamento per il pagamento tramite MoneyBox\glo{} possa coprire solo una parte della somma richiesta.               & R2F5.2   & NI \\
    TS1F5.4   & Verificare che l'utente possa visualizzare correttamente un messaggio d'errore nel caso in cui la transazione fallisca.                & R1F5.4   & NI \\
    TS1F6     & Verificare che la richiesta di rimborso funzioni correttamente.                                                                        & R1F6     & NI \\
    TS1F7     & Verificare che il proprietario dell'ordine possa sbloccare, correttamente, i fondi dallo Smart Contract\glo{} dopo avvenuta ricezione. & R1F7     & NI \\
    TS1F7.1   & Verificare che il proprietario dell'ordine possa visualizzare correttamente il codice di sblocco.                                      & R1F7.1   & NI \\
    TS1F8     & Verificare che l'utente possa visualizzare le transazioni.                                                                             & R1F8     & NI \\
    TS2F8.1   & Verificare che il venditore possa visualizzare le transazioni in entrata pagate.                                                       & R2F8.1   & NI \\
    TS2F8.1.1 & Verificare che il venditore possa visualizzare le transazioni in entrata pagate ma non sbloccate.                                      & R2F8.1.1 & NI \\
    TS2F8.1.2 & Verificare che il venditore possa visualizzare le transazioni in entrata pagate e sbloccate.                                           & R2F8.1.2 & NI \\
    TS2F8.1.3 & Verificare che il venditore possa visualizzare le transazioni in entrata pagate cancellate.                                            & R2F8.1.3 & NI \\
    TS1F8.2   & Verificare che il proprietario dell'ordine possa visualizzare le transazioni in uscita.                                                & R1F8.2   & NI \\
  \end{tabular}
\end{table}
\begin{center}
  \textit{\small Continua nella pagina successiva}
\end{center}
\begin{table}[H]
  \centering
  \renewcommand{\arraystretch}{1.8}
  \rowcolors{2}{green!100!black!40}{green!100!black!30}
  \begin{tabular}{c|p{8cm}|c|c}
    \rowcolor[HTML]{125E28}
    \color[HTML]{FFFFFF}\textbf{ID Test}
              & \multicolumn{1}{c}{\color[HTML]{FFFFFF}\textbf{Descrizione}}
              & \color[HTML]{FFFFFF}\textbf{ID Requisiti}
              & \color[HTML]{FFFFFF}\textbf{Stato}                                                                                                          \\
    \hline
    TS2F8.2.1 & Verificare che il proprietario dell'ordine possa visualizzare le transazioni in uscita non pagate.                          & R2F8.2.1 & NI \\
    TS2F8.2.2 & Verificare che il proprietario dell'ordine possa visualizzare le transazioni in uscita pagate ma non sbloccate.             & R2F8.2.2 & NI \\
    TS2F8.2.3 & Verificare che il proprietario dell'ordine possa visualizzare le transazioni in uscita pagate e sbloccate.                  & R2F8.2.3 & NI \\
    TS2F8.2.4 & Verificare che il proprietario dell'ordine possa visualizzare le transazioni in uscita cancellate.                          & R2F8.2.4 & NI \\
    TS3F9     & Verificare che si possa convertire correttamente l'ammontare depositato in stable coin\glo{}.                               & R3F9     & NI \\
    TS3F10    & Verificare che, dopo la transazione, una fee percentuale venga destinata al fondo ShopChain correttamente.                  & R3F10    & NI \\
    TS1F11    & Verificare che l'utente possa visualizzare correttamente l'indirizzo del suo wallet\glo{}.                                  & R1F11    & NI \\
    TS1F11.1  & Verificare che l'utente possa visualizzare correttamente l'indirizzo del suo wallet\glo{} in forma testuale.                & R1F11.1  & NI \\
    TS1F11.2  & Verificare che l'utente possa visualizzare correttamente un avviso della mancata connessione a Metamask\glo{}.              & R1F11.2  & NI \\
    TS3F11.3  & Verificare che l'utente possa visualizzare correttamente l'indirizzo del suo wallet\glo{} sotto forma di sequenza di emoji. & R3F11.3  & NI \\
    TS3F12    & Verificare che gli utenti partecipanti ad una MoneyBox\glo{} possano ricevere una notifica al suo completamento.            & R3F12    & NI \\
    TS1F14    & Verificare che l'utente possa visualizzare correttamente lo stato di connessione a Metamask\glo{}.                          & R1F14    & NI \\
    TS1F14.1  & Verificare che l'utente possa visualizzare un suggerimento se connesso correttamente.                                       & R1F14.1  & NI \\
    TS1F14.2  & Verificare che l'utente possa visualizzare nel caso in cui non sia installato Metamask\glo{}.                               & R1F14.2  & NI \\
  \end{tabular}
\end{table}
\begin{center}
  \textit{\small Continua nella pagina successiva}
\end{center}
\begin{table}[H]
  \centering
  \renewcommand{\arraystretch}{1.8}
  \rowcolors{2}{green!100!black!40}{green!100!black!30}
  \begin{tabular}{c|p{8cm}|c|c}
    \rowcolor[HTML]{125E28}
    \color[HTML]{FFFFFF}\textbf{ID Test}
               & \multicolumn{1}{c}{\color[HTML]{FFFFFF}\textbf{Descrizione}}
               & \color[HTML]{FFFFFF}\textbf{ID Requisiti}
               & \color[HTML]{FFFFFF}\textbf{Stato}                                                                                                                      \\
    \hline
    TS1F14.3   & Verificare che l'utente possa visualizzare un errore nel caso in cui la blockchain\glo{} selezionata non sia corretta.                 & R1F14.3   & NI \\
    TS1F14.4   & Verificare che l'utente possa visualizzare un errore nel caso in cui non abbia connesso un account a ShopChain.                        & R1F14.4   & NI \\
    TS1F15     & Verificare che l'utente possa visualizzare un messaggio di avviso nel caso in cui la transazione sia già presente in blockchain\glo{}. & R1F15     & NI \\
    TS1F16     & Verificare che l'utente possa visualizzare una pagina con i dettagli dell'ordine pagato.                                               & R1F16     & NI \\
    TS1F16.1   & Verificare che l'utente possa visualizzare i dettagli dell'ordine pagato.                                                              & R1F16.1   & NI \\
    TS1F16.1.1 & Verificare che l'utente possa visualizzare l'id dell'ordine pagato.                                                                    & R1F16.1.1 & NI \\
    TS1F16.1.2 & Verificare che l'utente possa visualizzare l'indirizzo del venditore dell'ordine pagato.                                               & R1F16.1.2 & NI \\
    TS1F16.1.3 & Verificare che l'utente possa visualizzare l'ammontare dell'ordine pagato.                                                             & R1F16.1.3 & NI \\
    TS1F16.1.4 & Verificare che l'utente possa visualizzare lo stato dell'ordine pagato.                                                                & R1F16.1.4 & NI \\
    TS1F16.1.5 & Verificare che l'utente possa visualizzare la data dell'ordine pagato.                                                                 & R1F16.1.5 & NI \\
  \end{tabular}
  \caption{Test di sistema}
\end{table}

%%%%%%%%%%%%%%%%%%%%%%%%%%%%%%%%%%%%%%%%%%%%%%%%%%%%%%%%%%%%%%%%%%%%%%%%%%%%%%%
\subsection{Test d'integrazione}\label{subsection:test_integrazione}
\begin{table}[H]
  \centering
  \renewcommand{\arraystretch}{1.8}
  \rowcolors{2}{green!100!black!40}{green!100!black!30}
  \begin{tabular}{c|p{8cm}|c}
    \rowcolor[HTML]{125E28}
    \color[HTML]{FFFFFF}\textbf{Codice}
        & \multicolumn{1}{c}{\color[HTML]{FFFFFF}\textbf{Descrizione}}
        & \color[HTML]{FFFFFF}\textbf{Stato}                                                                                                   \\
    \hline
    TI1 & Verificare che il collegamento con Metamask\glo{} avvenga correttamente.                                                        & S \\
    TI2 & Verificare che il collegamento tra Front-End\glo{} e Smart Contract\glo{} tramite libreria Web3.js\glo{} avvenga correttamente. & S \\
  \end{tabular}
  \caption{Test d'integrazione}
\end{table}

\vspace{1cm}
%%%%%%%%%%%%%%%%%%%%%%%%%%%%%%%%%%%%%%%%%%%%%%%%%%%%%%%%%%%%%%%%%%%%%%%%%%%%%%%
\subsection{Test di unità}\label{subsection:test_unita}

\subsubsection{Test di unità - Frontend}\label{subsubsection:TUF}
\begin{table}[H]
  \centering
  \renewcommand{\arraystretch}{1.8}
  \rowcolors{2}{green!100!black!40}{green!100!black!30}
  \begin{tabular}{c|p{8cm}|c|c}
    \rowcolor[HTML]{125E28}
    \color[HTML]{FFFFFF}\textbf{Codice}
          & \multicolumn{1}{c}{\color[HTML]{FFFFFF}\textbf{Descrizione}}
          & \color[HTML]{FFFFFF}\textbf{Stato}                                                                                      \\
    \hline
    TUF01 & Si verifica che un'istanza di OrderManager venga creata correttamente (costruttore parametrizzato).                 & S \\
    TUF02 & Si verifica che un'istanza di OrderManager venga creata correttamente (costruttore di default).                     & S \\
    TUF03 & Si verifica che ritorni correttamente il contatore degli ordini (OrderManager - istanza definita).                  & S \\
    TUF04 & Si verifica che ritorni correttamente il contatore degli ordini (OrderManager - istanza non definita).              & S \\
    TUF05 & Si verifica che ritorni correttamente il bilancio presente nel contratto (OrderManager - istanza definita).         & S \\
    TUF06 & Si verifica che ritorni correttamente il bilancio presente nel contratto (OrderManager - istanza non definita).     & S \\
    TUF07 & Si verifica che un'istanza di MoneyBoxManager venga creata correttamente (costruttore parametrizzato).              & S \\
    TUF08 & Si verifica che un'istanza di MoneyBoxManager venga creata correttamente (costruttore di default).                  & S \\
    TUF09 & Si verifica che un'istanza di OrderManagerRepo venga creata correttamente.                                          & S \\
    TUF10 & Si verifica che ritorni correttamente il bilancio presente nel contratto (OrderManagerRepo - istanza definita).     & S \\
    TUF11 & Si verifica che ritorni correttamente il bilancio presente nel contratto (OrderManagerRepo - istanza non definita). & S \\
    TUF12 & Si verifica che ritorni correttamente il contatore degli ordini (OrderManagerRepo - istanza definita).              & S \\
  \end{tabular}
\end{table}
\begin{center}
  \textit{\small Continua nella pagina successiva}
\end{center}
\begin{table}[H]
  \centering
  \renewcommand{\arraystretch}{1.8}
  \rowcolors{2}{green!100!black!40}{green!100!black!30}
  \begin{tabular}{c|p{8cm}|c|c}
    \rowcolor[HTML]{125E28}
    \color[HTML]{FFFFFF}\textbf{Codice}
          & \multicolumn{1}{c}{\color[HTML]{FFFFFF}\textbf{Descrizione}}
          & \color[HTML]{FFFFFF}\textbf{Stato}                                                                             \\
    \hline
    TUF13 & Si verifica che ritorni correttamente il contatore degli ordini (OrderManagerRepo - istanza non definita). & S \\
    TUF14 & Si verifica che un'istanza di MoneyBoxManagerRepo venga creata correttamente.                              & S \\
    TUF15 & Si verifica che un'istanza di ContractStore venga creata correttamente.                                    & S \\
    TUF16 & Si verifica che un'istanza di Amount venga creata correttamente.                                           & S \\
    TUF17 & Si verifica che l'ammontare venga aggiornato correttamente.                                                & S \\
    TUF18 & Si verifica che il valore in FTM\glo{} ritornato sia corretto.                                             & S \\
    TUF19 & Si verifica che l'ammontare venga assegnato correttamente.                                                 & S \\
    TUF20 & Si verifica che un'istanza di MoneyBox venga creata correttamente.                                         & S \\
    TUF21 & Si verifica che l'ammontare deciso venga inserito correttamente all'interno della MoneyBox\glo{}.          & S \\
    TUF22 & Si verifica che ritorni correttamente la lista dei pagamenti relativi la MoneyBox\glo{}.                   & S \\
    TUF23 & Si verifica che un'istanza di Order venga creata correttamente.                                            & S \\
    TUF24 & Si verifica che un ordine venga costruito correttamente.                                                   & S \\
    TUF25 & Si verifica che ritorni correttamente l'ammontare relativo all'ordine.                                     & S \\
    TUF26 & Si verifica che l'ammontare relativo all'ordine venga sbloccato correttamente.                             & S \\
    TUF27 & Si verifica che l'ammontare relativo all'ordine venga rimborsato correttamente.                            & S \\
    TUF28 & Si verifica che lo stato dell'ordine venga aggiornato correttamente.                                       & S \\
  \end{tabular}
\end{table}
\begin{center}
  \textit{\small Continua nella pagina successiva}
\end{center}
\begin{table}[H]
  \centering
  \renewcommand{\arraystretch}{1.8}
  \rowcolors{2}{green!100!black!40}{green!100!black!30}
  \begin{tabular}{c|p{8cm}|c|c}
    \rowcolor[HTML]{125E28}
    \color[HTML]{FFFFFF}\textbf{Codice}
          & \multicolumn{1}{c}{\color[HTML]{FFFFFF}\textbf{Descrizione}}
          & \color[HTML]{FFFFFF}\textbf{Stato}                                                                         \\
    \hline
    TUF29 & Si verifica che l'ordine venga aggiornato correttamente.                                               & S \\
    TUF30 & Si verifica che la collezione di ordini aggiunga gli ordini correttamente.                             & S \\
    TUF31 & Si verifica che la collezione di ordini ritorni l'ordine corretto dato l'ID.                           & S \\
    TUF32 & Si verifica che la collezione di ordini ritorni gli ordini corretti dato il venditore.                 & S \\
    TUF33 & Si verifica che la collezione di ordini ritorni gli ordini corretti dato il proprietario degli ordini. & S \\
    TUF34 & Si verifica che la collezione di ordini venga aggiornata correttamente.                                & S \\
    TUF35 & Si verifica che un'istanza di OrderState venga creata correttamente.                                   & S \\
    TUF36 & Si verifica che lo stato dell'ordine sia "Not Created".                                                & S \\
    TUF37 & Si verifica che lo stato dell'ordine sia "Created".                                                    & S \\
    TUF38 & Si verifica che lo stato dell'ordine sia "Paid".                                                       & S \\
    TUF39 & Si verifica che lo stato dell'ordine sia "Closed".                                                     & S \\
    TUF40 & Si verifica che lo stato dell'ordine sia "Cancelled".                                                  & S \\
    TUF41 & Si verifica che lo stato dell'ordine venga modificato correttamente.                                   & S \\
    TUF42 & Si verifica che l'indirizzo del pagamento sia corretto.                                                & S \\
    TUF43 & Si verifica che l'ammontare del pagamento sia corretto.                                                & S \\
    TUF44 & Si verifica che la data del pagamento sia corretta.                                                    & S \\
    TUF45 & Si verifica che il pagamento venga creato correttamente.                                               & S \\
  \end{tabular}
\end{table}
\begin{center}
  \textit{\small Continua nella pagina successiva}
\end{center}
\begin{table}[H]
  \centering
  \renewcommand{\arraystretch}{1.8}
  \rowcolors{2}{green!100!black!40}{green!100!black!30}
  \begin{tabular}{c|p{8cm}|c|c}
    \rowcolor[HTML]{125E28}
    \color[HTML]{FFFFFF}\textbf{Codice}
          & \multicolumn{1}{c}{\color[HTML]{FFFFFF}\textbf{Descrizione}}
          & \color[HTML]{FFFFFF}\textbf{Stato}                                                                                \\
    \hline
    TUF46 & Si verifica che un'istanza di OrderRepo venga creata correttamente.                                           & S \\
    TUF47 & Si verifica che l'ordine venga creato correttamente (OrderRepo - istanza definita).                           & S \\
    TUF48 & Si verifica che l'ordine venga creato correttamente (OrderRepo - istanza non definita).                       & S \\
    TUF49 & Si verifica che l'ordine venga sbloccato correttamente (OrderRepo - istanza definita).                        & S \\
    TUF50 & Si verifica che l'ordine venga sbloccato correttamente (OrderRepo - istanza non definita).                    & S \\
    TUF51 & Si verifica che l'ordine venga rimborsato correttamente (OrderRepo - istanza definita).                       & S \\
    TUF52 & Si verifica che l'ordine venga rimborsato correttamente (OrderRepo - istanza non definita).                   & S \\
    TUF53 & Si verifica che ritorni l'ordine corretto dato l'ID (OrderRepo - istanza definita).                           & S \\
    TUF54 & Si verifica che ritorni l'ordine corretto dato l'ID (OrderRepo - istanza non definita).                       & S \\
    TUF55 & Si verifica che ritorni gli ordini corretti dato il venditore (OrderRepo - istanza definita).                 & S \\
    TUF56 & Si verifica che ritorni gli ordini corretti dato il venditore (OrderRepo - istanza non definita).             & S \\
    TUF57 & Si verifica che ritorni gli ordini corretti dato l'acquirente (OrderRepo - istanza definita).                 & S \\
    TUF58 & Si verifica che ritorni gli ordini corretti dato l'acquirente (OrderRepo - istanza non definita).             & S \\
    TUF59 & Si verifica che un'istanza di MoneyBoxOrderRepo venga creata correttamente.                                   & S \\
    TUF60 & Si verifica che un nuovo pagamento venga effettuato correttamente (MoneyBoxOrderRepo - istanza definita).     & S \\
    TUF61 & Si verifica che un nuovo pagamento venga effettuato correttamente (MoneyBoxOrderRepo - istanza non definita). & S \\
  \end{tabular}
\end{table}
\begin{center}
  \textit{\small Continua nella pagina successiva}
\end{center}
\begin{table}[H]
  \centering
  \renewcommand{\arraystretch}{1.8}
  \rowcolors{2}{green!100!black!40}{green!100!black!30}
  \begin{tabular}{c|p{8cm}|c|c}
    \rowcolor[HTML]{125E28}
    \color[HTML]{FFFFFF}\textbf{Codice}
          & \multicolumn{1}{c}{\color[HTML]{FFFFFF}\textbf{Descrizione}}
          & \color[HTML]{FFFFFF}\textbf{Stato}                                                                                                          \\
    \hline
    TUF62 & Si verifica che ritorni la lista dei pagamenti relativi ad una MoneyBox\glo{} correttamente (MoneyBoxOrderRepo - istanza definita).     & S \\
    TUF63 & Si verifica che ritorni la lista dei pagamenti relativi ad una MoneyBox\glo{} correttamente (MoneyBoxOrderRepo - istanza non definita). & S \\
    TUF64 & Si verifica che ritorni l'ammontare mancante al completamento di una MoneyBox\glo{} (MoneyBoxOrderRepo - istanza definita).             & S \\
    TUF65 & Si verifica che ritorni l'ammontare mancante al completamento di una MoneyBox\glo{} (MoneyBoxOrderRepo - istanza non definita).         & S \\
    TUF66 & Si verifica che un'istanza di OrderStore venga creata correttamente.                                                                    & S \\
    TUF67 & Si verifica che un ordine venga creato correttamente tramite OrderStore.                                                                & S \\
    TUF68 & Si verifica che un istanza di LockOverlayViewModel venga creata correttamente.                                                          & S \\
    TUF69 & Si verifica che il LockOverlayViewModel sia connesso al provider (provider corretto).                                                   & S \\
    TUF70 & Si verifica che il LockOverlayViewModel sia connesso al provider (provider non corretto).                                               & S \\
    TUF71 & Si verifica che makeAutoObservable sia chiamata alla creazione di ConnectMetamaskViewModel.                                             & S \\
    TUF72 & Si verifica che la funzione connect di ConnectMetamaskViewModel sia chiamata correttamente.                                             & S \\
    TUF73 & Si verifica che la severità di MetamaskErrorViewModel sia esattamente quella data dal provider.                                         & S \\
    TUF74 & Si verifica che il nome di MetamaskErrorViewModel sia esattamente quello dato dal provider.                                             & S \\
  \end{tabular}
\end{table}
\begin{center}
  \textit{\small Continua nella pagina successiva}
\end{center}
\begin{table}[H]
  \centering
  \renewcommand{\arraystretch}{1.8}
  \rowcolors{2}{green!100!black!40}{green!100!black!30}
  \begin{tabular}{c|p{8cm}|c|c}
    \rowcolor[HTML]{125E28}
    \color[HTML]{FFFFFF}\textbf{Codice}
          & \multicolumn{1}{c}{\color[HTML]{FFFFFF}\textbf{Descrizione}}
          & \color[HTML]{FFFFFF}\textbf{Stato}                                                                                                          \\
    \hline
    TUF75 & Si verifica che la descrizione di MetamaskErrorViewModel sia esattamente quella data dal provider.                                      & S \\
    TUF76 & Si verifica che makeAutoObservable sia chiamata alla creazione di NavViewModel.                                                         & S \\
    TUF77 & Si verifica che l'address sia corretto in caso di connessione al provider di NavViewModel.                                              & S \\
    TUF78 & Si verifica che l'address sia non corretto in caso di fallita connessione al provider di NavViewModel.                                  & S \\
    TUF79 & Si verifica che un istanza di ECommerceViewModel sia creata correttamente.                                                              & S \\
    TUF80 & Si verifica che l'istanza di ECommerceViewModel ritorni l'amount corretto.                                                              & S \\
    TUF81 & Si verifica che l'istanza di ECommerceViewModel ritorni i wei corretti.                                                                 & S \\
    TUF82 & Si verifica che l'istanza di ECommerceViewModel ritorni l'id corretto.                                                                  & S \\
    TUF83 & Si verifica che setAmount di ECommerceViewModel sia corretta.                                                                           & S \\
    TUF84 & Si verifica che handleSubmit di ECommerceViewModel sia corretta.                                                                        & S \\
    TUF85 & Si verifica che l'istanza di ChoiceViewModel sia creata correttamente.                                                                  & S \\
    TUF86 & Si verifica che createOrder di ChoiceViewModel sia corretta. & S \\
    TUF87 & Si verifica che canRedirect di ChoiceViewModel sia corretta. & S \\
    TUF88 & Si verifica che isBusy di ChoiceViewModel sia corretta. & S \\
    TUF89 & Si verifica che isFailed di ChoiceViewModel sia corretta. & S \\
    TUF90 & Si verifica che setAmount di ChoiceViewModel sia corretta. & S \\
  \end{tabular}
\end{table}
\begin{center}
  \textit{\small Continua nella pagina successiva}
\end{center}
\begin{table}[H]
  \centering
  \renewcommand{\arraystretch}{1.8}
  \rowcolors{2}{green!100!black!40}{green!100!black!30}
  \begin{tabular}{c|p{8cm}|c|c}
    \rowcolor[HTML]{125E28}
    \color[HTML]{FFFFFF}\textbf{Codice}
          & \multicolumn{1}{c}{\color[HTML]{FFFFFF}\textbf{Descrizione}}
          & \color[HTML]{FFFFFF}\textbf{Stato}                                                                                                          \\
    \hline
    TUF91 & Si verifica che setId di ChoiceViewModel sia corretta. & S \\
    TUF92 & Si verifica che un'istanza di PickAmountViewModel venga creata correttamente. & S \\
    TUF93 & Si verifica che createMoneyBox di PickAmountViewModel sia corretta. & S \\
    TUF94 & Si verifica che canRedirect di PickAmountViewModel ritorni false. & S \\
    TUF95 & Si verifica che isBusy di PickAmountViewModel ritorni false. & S \\
    TUF96 & Si verifica che setAmount di PickAmountViewModel sia corretta. & S \\
    TUF97 & Si verifica che setId di PickAmountViewModel sia corretta. & S \\
    TUF98 & Si verifica che setInitFTM di PickAmountViewModel sia corretta. & S \\
    TUF99 & Si verifica che un'istanza di TransactionInitViewModel sia creata correttamente. & S \\
    TUF100 & Si verifica che createMoneyBox di TransactionInitViewModel sia corretta. & S \\
    TUF101 & Si verifica che setAmount di TransactionInitViewModel sia corretta. & S \\
    TUF102 & Si verifica che setId di TransactionInitViewModel sia corretta. & S \\
    TUF103 & Si verifica che getSellerAddress di TransactionInitViewModel sia settato all'indirizzo corretto. & S \\
    TUF104 & Si verifica che un'istanza di MoneyBoxBalanceViewModel sia creata correttamente. & S \\
    TUF105 & Si verifica che isBusy di MoneyBoxBalanceViewModel ritorni false. & S \\
    TUF106 & Si verifica che balanceFTM di MoneyBoxBalanceViewModel sia settato al valore di default. & S \\
  \end{tabular}
\end{table}
\begin{center}
  \textit{\small Continua nella pagina successiva}
\end{center}
\begin{table}[H]
  \centering
  \renewcommand{\arraystretch}{1.8}
  \rowcolors{2}{green!100!black!40}{green!100!black!30}
  \begin{tabular}{c|p{8cm}|c|c}
    \rowcolor[HTML]{125E28}
    \color[HTML]{FFFFFF}\textbf{Codice}
          & \multicolumn{1}{c}{\color[HTML]{FFFFFF}\textbf{Descrizione}}
          & \color[HTML]{FFFFFF}\textbf{Stato}                                                                                                          \\
    \hline
    TUF107 & Si verifica che balanceWEI di MoneyBoxBalanceViewModel sia settato al valore di default. & S \\
    TUF108 & Si verifica che un'istanza di MoneyBoxCountViewModel sia creata correttamente. & S \\
    TUF109 & Si verifica che isBusy di MoneyBoxCountViewModel ritorni false. & S \\
    TUF110 & Si verifica che count di MoneyBoxCountViewModel sia impostato al valore di default. & S \\
    TUF111 & Si verifica che un'istanza di OrderBalanceViewModel sia creata correttamente. & S \\
    TUF112 & Si verifica che isBusy di OrderBalanceViewModel ritorni false. & S \\
    TUF113 & Si verifica che balanceFTM di OrderBalanceViewModel sia impostato al valore di default. & S \\
    TUF114 & Si verifica che balanceWEI di OrderBalanceViewModel sia impostato al valore di default. & S \\
    TUF115 & Si verifica che un'istanza di OrderCountViewModel sia creata correttamente. & S \\
    TUF116 & Si verifica che isBusy di OrderCountViewModel sia impostato a false. & S \\
    TUF117 & Si verifica che count di OrderCountViewModel sia impostato al valore di default. & S \\
    TUF118 & Si verifica che un'istanza di MoneyBoxDetailsViewModel sia creata correttamente. & S \\
    TUF119 & Si verifica che l'id di MoneyBoxDetailsViewModel venga impostato correttamente. & S \\
    TUF120 & Si verifica che ownerAddress di MoneyBoxDetailsViewModel sia impostato al valore di default. & S \\
    TUF121 & Si verifica che sellerAddress di MoneyBoxDetailsViewModel sia impostato al valore di default. & S \\
    TUF122 & Si verifica che ftm di MoneyBoxDetailsViewModel sia impostato al valore di default. & S \\
  \end{tabular}
\end{table}
\begin{center}
  \textit{\small Continua nella pagina successiva}
\end{center}
\begin{table}[H]
  \centering
  \renewcommand{\arraystretch}{1.8}
  \rowcolors{2}{green!100!black!40}{green!100!black!30}
  \begin{tabular}{c|p{8cm}|c|c}
    \rowcolor[HTML]{125E28}
    \color[HTML]{FFFFFF}\textbf{Codice}
          & \multicolumn{1}{c}{\color[HTML]{FFFFFF}\textbf{Descrizione}}
          & \color[HTML]{FFFFFF}\textbf{Stato}                                                                                                          \\
    \hline
    TUF123 & Si verifica che wei di MoneyBoxDetailsViewModel sia impostato al valore di default. & S \\
    TUF124 & Si verifica che filledFtm di MoneyBoxDetailsViewModel sia impostato al valore di default. & S \\
    TUF125 & Si verifica che filledWei di MoneyBoxDetailsViewModel sia impostato al valore di default. & S \\
    TUF126 & Si verifica che ftmToFill di MoneyBoxDetailsViewModel sia impostato al valore di default. & S \\
    TUF127 & Si verifica che weiToFill di MoneyBoxDetailsViewModel sia impostato al valore di default. & S \\
    TUF128 & Si verifica che state di MoneyBoxDetailsViewModel sia impostato al valore di default. & S \\
    TUF129 & Si verifica che isPaid di MoneyBoxDetailsViewModel sia impostato a false. & S \\
    TUF130 & Si verifica che isRefunded di MoneyBoxDetailsViewModel sia impostato a false. & S \\
    TUF131 & Si verifica che isUnlocked di MoneyBoxDetailsViewModel sia impostato a false. & S \\
    TUF132 & Si verifica che un'istanza di OrderDetailsViewModel sia creata correttamente & S \\
    TUF133 & Si verifica che l'id di OrderDetailsViewModel venga impostato correttamente. & S \\
    TUF134 & Si verifica che ownerAddress di OrderDetailsViewModel sia impostato al valore di default. & S \\
    TUF135 & Si verifica che sellerAddress di OrderDetailsViewModel sia impostato al valore di default. & S \\
    TUF136 & Si verifica che ftm di OrderDetailsViewModel sia impostato al valore di default. & S \\
    TUF137 & Si verifica che wei di OrderDetailsViewModel sia impostato al valore di default. & S \\
    TUF138 & Si verifica che state di OrderDetailsViewModel sia impostato al valore di default. & S \\
  \end{tabular}
\end{table}
\begin{center}
  \textit{\small Continua nella pagina successiva}
\end{center}
\begin{table}[H]
  \centering
  \renewcommand{\arraystretch}{1.8}
  \rowcolors{2}{green!100!black!40}{green!100!black!30}
  \begin{tabular}{c|p{8cm}|c|c}
    \rowcolor[HTML]{125E28}
    \color[HTML]{FFFFFF}\textbf{Codice}
          & \multicolumn{1}{c}{\color[HTML]{FFFFFF}\textbf{Descrizione}}
          & \color[HTML]{FFFFFF}\textbf{Stato}                                                                                                          \\
    \hline
    TUF139 & Si verifica che isPaid di OrderDetailsViewModel sia impostato a false. & S \\
    TUF140 & Si verifica che code di OrderDetailsViewModel sia impostato al valore di default. & S \\
    TUF141 & Si verifica che un'istanza di TransactionListElViewModel sia creata correttamente. & S \\
    TUF142 & Si verifica che id di TransactionListElViewModel sia impostato al valore di default. & S \\
    TUF143 & Si verifica che il tipo di ordine di TransactionListElViewModel sia corretto. & S \\
    TUF144 & Si verifica che lo stato dell'ordine di TransactionListElViewModel sia corretto. & S \\
    TUF145 & Si verifica che transaction di TransactionListElViewModel ritorni l'ordine corretto. & S \\
  \end{tabular}
  \caption{Test di unità - Frontend}
\end{table}

\paragraph{Tracciamento test di unità - Frontend}\label{paragraph:tracciamento_TUF}
\begin{table}[H]
  \centering
  \renewcommand{\arraystretch}{1.8}
  \rowcolors{2}{green!100!black!40}{green!100!black!30}
  \begin{tabular}{c|p{15cm}}
    \rowcolor[HTML]{125E28}
    \color[HTML]{FFFFFF}\textbf{ID Test}
          & \multicolumn{1}{c}{\color[HTML]{FFFFFF}\textbf{Metodo}}                                                                                                                                                    \\
    \hline
    TUF01 & shopchain-frontend/src/core/modules/contract/domain/\_\_test\_\_/OrderManager.test.ts:\newline describe("should create an instance of OrderManager",()=$>$\{\newline it("constructor with parameter", ...) \\
    TUF02 & shopchain-frontend/src/core/modules/contract/domain/\_\_test\_\_/OrderManager.test.ts:\newline describe("should create an instance of OrderManager",()=$>$\{ it("default constructor", ...)                \\
    TUF03 & shopchain-frontend/src/core/modules/contract/domain/\_\_test\_\_/OrderManager.test.ts:\newline describe("should get order count",()=$>$\{it("defined instance", ...)                                       \\
    TUF04 & shopchain-frontend/src/core/modules/contract/domain/\_\_test\_\_/OrderManager.test.ts:\newline describe("should get order count",()=$>$\{it("undefined instance", ...)                                     \\
    TUF05 & shopchain-frontend/src/core/modules/contract/domain/\_\_test\_\_/OrderManager.test.ts:\newline describe("should get contract balance",()=$>$\{it("defined instance", ...)                                  \\
    TUF06 & shopchain-frontend/src/core/modules/contract/domain/\_\_test\_\_/OrderManager.test.ts:\newline describe("should get contract balance",()=$>$\{it("undefined instance", ...)                                \\
  \end{tabular}
\end{table}
\begin{center}
  \textit{\small Continua nella pagina successiva}
\end{center}
\begin{table}[H]
  \centering
  \renewcommand{\arraystretch}{1.8}
  \rowcolors{2}{green!100!black!40}{green!100!black!30}
  \begin{tabular}{c|p{15cm}}
    \rowcolor[HTML]{125E28}
    \color[HTML]{FFFFFF}\textbf{ID Test}
          & \multicolumn{1}{c}{\color[HTML]{FFFFFF}\textbf{Metodo}}                                                                                                                                                         \\
    \hline
    TUF07 & shopchain-frontend/src/core/modules/contract/domain/\_\_test\_\_/\newline MoneyBoxManager.test.ts:describe("should create an instance of MoneyBoxManager",\newline()=$>$\{it("constructor with parameter", ...) \\
    TUF08 & shopchain-frontend/src/core/modules/contract/domain/\_\_test\_\_/\newline MoneyBoxManager.test.ts:describe("should create an instance of MoneyBoxManager",\newline()=$>$\{it("default constructor", ...)        \\
    TUF09 & shopchain-frontend/src/core/modules/contract/repo/implementations/\_\_test\_\_/\newline OrderManagerRepo.test.ts:it("should create an instance of OrderManagerRepo", ...)                                       \\
    TUF10 & shopchain-frontend/src/core/modules/contract/repo/implementations/\_\_test\_\_/\newline OrderManagerRepo.test.ts:describe("should get contract balance",()=$>$\{\newline it("defined instance", ...)            \\
    TUF11 & shopchain-frontend/src/core/modules/contract/repo/implementations/\_\_test\_\_/\newline OrderManagerRepo.test.ts:describe("should get contract balance",()=$>$\{\newline it("undefined instance", ...)          \\
    TUF12 & shopchain-frontend/src/core/modules/contract/repo/implementations/\_\_test\_\_/\newline OrderManagerRepo.test.ts:describe("should get order count",()=$>$\{\newline it("defined instance", ...)                 \\
    TUF13 & shopchain-frontend/src/core/modules/contract/repo/implementations/\_\_test\_\_/\newline OrderManagerRepo.test.ts:describe("should get order count",()=$>$\{\newline it("undefined instance", ...)               \\
    TUF14 & shopchain-frontend/src/core/modules/contract/repo/implementations/\_\_test\_\_/\newline MoneyBoxManagerRepo.test.ts:it("should create an instance of MoneyBoxManagerRepo", ...)                                 \\
    TUF15 & shopchain-frontend/src/core/modules/contract/store/\_\_test\_\_/\newline ContractStore.test.ts:it("should create an instance of ContractStore", ...)                                                            \\
    TUF16 & shopchain-frontend/src/core/modules/order/domain/\_\_test\_\_/Amount.test.ts:\newline it("should create an instance of Amount", ...)                                                                            \\
    TUF17 & shopchain-frontend/src/core/modules/order/domain/\_\_test\_\_/Amount.test.ts:\newline it("should update the amount", ...)                                                                                       \\
    TUF18 & shopchain-frontend/src/core/modules/order/domain/\_\_test\_\_/Amount.test.ts:\newline it("should get the FTM value correctly", ...)                                                                             \\
    TUF19 & shopchain-frontend/src/core/modules/order/domain/\_\_test\_\_/Amount.test.ts:\newline it("should get the amount properly", ...)                                                                                 \\
    TUF20 & shopchain-frontend/src/core/modules/order/domain/\_\_test\_\_/MoneyBox.test.ts:\newline it("test constructor", ...)                                                                                             \\
  \end{tabular}
\end{table}
\begin{center}
  \textit{\small Continua nella pagina successiva}
\end{center}
\begin{table}[H]
  \centering
  \renewcommand{\arraystretch}{1.8}
  \rowcolors{2}{green!100!black!40}{green!100!black!30}
  \begin{tabular}{c|p{15cm}}
    \rowcolor[HTML]{125E28}
    \color[HTML]{FFFFFF}\textbf{ID Test}
          & \multicolumn{1}{c}{\color[HTML]{FFFFFF}\textbf{Metodo}}                                                                                        \\
    \hline
    TUF21 & shopchain-frontend/src/core/modules/order/domain/\_\_test\_\_/MoneyBox.test.ts:\newline it("amount to fill is correct", ...)                   \\
    TUF22 & shopchain-frontend/src/core/modules/order/domain/\_\_test\_\_/MoneyBox.test.ts:\newline it("payments", ...)                                    \\
    TUF23 & shopchain-frontend/src/core/modules/order/domain/\_\_test\_\_/Order.test.ts:\newline it("test order constructor", ...)                         \\
    TUF24 & shopchain-frontend/src/core/modules/order/domain/\_\_test\_\_/Order.test.ts:\newline it("create", ...)                                         \\
    TUF25 & shopchain-frontend/src/core/modules/order/domain/\_\_test\_\_/Order.test.ts:\newline it("amount", ...)                                         \\
    TUF26 & shopchain-frontend/src/core/modules/order/domain/\_\_test\_\_/Order.test.ts:\newline it("unlock", ...)                                         \\
    TUF27 & shopchain-frontend/src/core/modules/order/domain/\_\_test\_\_/Order.test.ts:\newline it("refund", ...)                                         \\
    TUF28 & shopchain-frontend/src/core/modules/order/domain/\_\_test\_\_/Order.test.ts:\newline it("should update the order state using the setter", ...) \\
    TUF29 & shopchain-frontend/src/core/modules/order/domain/\_\_test\_\_/Order.test.ts:\newline it("should update the order", ...)                        \\
    TUF30 & shopchain-frontend/src/core/modules/order/domain/\_\_test\_\_/OrderCollection.test.ts:\newline it("test addOrder", ...)                        \\
    TUF31 & shopchain-frontend/src/core/modules/order/domain/\_\_test\_\_/OrderCollection.test.ts:\newline it("test getById", ...)                         \\
    TUF32 & shopchain-frontend/src/core/modules/order/domain/\_\_test\_\_/OrderCollection.test.ts:\newline it("test getBySeller", ...)                     \\
    TUF33 & shopchain-frontend/src/core/modules/order/domain/\_\_test\_\_/OrderCollection.test.ts:\newline it("test getByOwner", ...)                      \\
    TUF34 & shopchain-frontend/src/core/modules/order/domain/\_\_test\_\_/OrderCollection.test.ts:\newline it("test update", ...)                          \\
    TUF35 & shopchain-frontend/src/core/modules/order/domain/\_\_test\_\_/OrderState.test.ts:\newline it("test OrderState constructor", ...)               \\
    TUF36 & shopchain-frontend/src/core/modules/order/domain/\_\_test\_\_/OrderState.test.ts:\newline it("isNotCreated", ...)                              \\
  \end{tabular}
\end{table}
\begin{center}
  \textit{\small Continua nella pagina successiva}
\end{center}
\begin{table}[H]
  \centering
  \renewcommand{\arraystretch}{1.8}
  \rowcolors{2}{green!100!black!40}{green!100!black!30}
  \begin{tabular}{c|p{15cm}}
    \rowcolor[HTML]{125E28}
    \color[HTML]{FFFFFF}\textbf{ID Test}
          & \multicolumn{1}{c}{\color[HTML]{FFFFFF}\textbf{Metodo}}                                                                                                                      \\
    \hline
    TUF37 & shopchain-frontend/src/core/modules/order/domain/\_\_test\_\_/OrderState.test.ts:\newline it("isCreated", ...)                                                               \\
    TUF38 & shopchain-frontend/src/core/modules/order/domain/\_\_test\_\_/OrderState.test.ts:\newline it("isPaid", ...)                                                                  \\
    TUF39 & shopchain-frontend/src/core/modules/order/domain/\_\_test\_\_/OrderState.test.ts:\newline it("isClosed", ...)                                                                \\
    TUF40 & shopchain-frontend/src/core/modules/order/domain/\_\_test\_\_/OrderState.test.ts:\newline it("isCancelled", ...)                                                             \\
    TUF41 & shopchain-frontend/src/core/modules/order/domain/\_\_test\_\_/OrderState.test.ts:\newline it("setState", ...)                                                                \\
    TUF42 & shopchain-frontend/src/core/modules/order/domain/\_\_test\_\_/Payment.test.ts:\newline it("from address is correct", ...)                                                    \\
    TUF43 & shopchain-frontend/src/core/modules/order/domain/\_\_test\_\_/Payment.test.ts:\newline it("amount is correct", ...)                                                          \\
    TUF44 & shopchain-frontend/src/core/modules/order/domain/\_\_test\_\_/Payment.test.ts:\newline it("timestamp is correct", ...)                                                       \\
    TUF45 & shopchain-frontend/src/core/modules/order/domain/\_\_test\_\_/Payment.test.ts:\newline it("created payment is correct", ...)                                                 \\
    TUF46 & shopchain-frontend/src/core/modules/order/repo/implementations/\_\_test\_\_/\newline OrderRepo.test.ts:it("should create an instance of OrderRepo", ...)                     \\
    TUF47 & shopchain-frontend/src/core/modules/order/repo/implementations/\_\_test\_\_/\newline OrderRepo.test.ts:describe("createOrder",()=$>$\{it("defined contract instance", ...)   \\
    TUF48 & shopchain-frontend/src/core/modules/order/repo/implementations/\_\_test\_\_/\newline OrderRepo.test.ts:describe("createOrder",()=$>$\{it("undefined contract instance", ...) \\
    TUF49 & shopchain-frontend/src/core/modules/order/repo/implementations/\_\_test\_\_/\newline OrderRepo.test.ts:describe("unlock",()=$>$\{it("defined contract instance", ...)        \\
    TUF50 & shopchain-frontend/src/core/modules/order/repo/implementations/\_\_test\_\_/\newline OrderRepo.test.ts:describe("unlock",()=$>$\{it("undefined contract instance", ...)      \\
    TUF51 & shopchain-frontend/src/core/modules/order/repo/implementations/\_\_test\_\_/\newline OrderRepo.test.ts:describe("refund",()=$>$\{it("defined contract instance", ...)        \\
    TUF52 & shopchain-frontend/src/core/modules/order/repo/implementations/\_\_test\_\_/\newline OrderRepo.test.ts:describe("refund",()=$>$\{it("undefined contract instance", ...)      \\
  \end{tabular}
\end{table}
\begin{center}
  \textit{\small Continua nella pagina successiva}
\end{center}
\begin{table}[H]
  \centering
  \renewcommand{\arraystretch}{1.8}
  \rowcolors{2}{green!100!black!40}{green!100!black!30}
  \begin{tabular}{c|p{15cm}}
    \rowcolor[HTML]{125E28}
    \color[HTML]{FFFFFF}\textbf{ID Test}
          & \multicolumn{1}{c}{\color[HTML]{FFFFFF}\textbf{Metodo}}                                                                                                                                                \\
    \hline
    TUF53 & shopchain-frontend/src/core/modules/order/repo/implementations/\_\_test\_\_/\newline OrderRepo.test.ts:describe("getOrderById",()=$>$\{it("defined contract instance", ...)                            \\
    TUF54 & shopchain-frontend/src/core/modules/order/repo/implementations/\_\_test\_\_/\newline OrderRepo.test.ts:describe("getOrderById",()=$>$\{it("undefined contract instance", ...)                          \\
    TUF55 & shopchain-frontend/src/core/modules/order/repo/implementations/\_\_test\_\_/\newline OrderRepo.test.ts:describe("getOrdersBySeller",()=$>$\{it("defined contract instance", ...)                       \\
    TUF56 & shopchain-frontend/src/core/modules/order/repo/implementations/\_\_test\_\_/\newline OrderRepo.test.ts:describe("getOrdersBySeller",()=$>$\{it("undefined contract instance", ...)                     \\
    TUF57 & shopchain-frontend/src/core/modules/order/repo/implementations/\_\_test\_\_/\newline OrderRepo.test.ts:describe("getOrdersByBuyer",()=$>$\{it("defined contract instance", ...)                        \\
    TUF58 & shopchain-frontend/src/core/modules/order/repo/implementations/\_\_test\_\_/\newline OrderRepo.test.ts:describe("getOrdersByBuyer",()=$>$\{it("undefined contract instance", ...)                      \\
    TUF59 & shopchain-frontend/src/core/modules/order/repo/implementations/\_\_test\_\_/\newline MoneyBoxOrderRepo.test.ts:it("should create an instance of MoneyBoxOrderRepo", ...)                               \\
    TUF60 & shopchain-frontend/src/core/modules/order/repo/implementations/\_\_test\_\_/\newline MoneyBoxOrderRepo.test.ts:describe("should do a newPayment",()=$>$\{it("defined contract instance", ...)          \\
    TUF61 & shopchain-frontend/src/core/modules/order/repo/implementations/\_\_test\_\_/\newline MoneyBoxOrderRepo.test.ts:describe("should do a newPayment",()=$>$\{it("undefined contract instance", ...)        \\
    TUF62 & shopchain-frontend/src/core/modules/order/repo/implementations/\_\_test\_\_/\newline MoneyBoxOrderRepo.test.ts:describe("should get payments",()=$>$\{it("defined contract instance", ...)             \\
    TUF63 & shopchain-frontend/src/core/modules/order/repo/implementations/\_\_test\_\_/\newline MoneyBoxOrderRepo.test.ts:describe("should get payments",()=$>$\{it("undefined contract instance", ...)           \\
    TUF64 & shopchain-frontend/src/core/modules/order/repo/implementations/\_\_test\_\_/\newline MoneyBoxOrderRepo.test.ts:describe("should get the amount to fill",()=$>$\{it("defined contract instance", ...)   \\
    TUF65 & shopchain-frontend/src/core/modules/order/repo/implementations/\_\_test\_\_/\newline MoneyBoxOrderRepo.test.ts:describe("should get the amount to fill",()=$>$\{it("undefined contract instance", ...) \\
  \end{tabular}
\end{table}
\begin{center}
  \textit{\small Continua nella pagina successiva}
\end{center}
\begin{table}[H]
  \centering
  \renewcommand{\arraystretch}{1.8}
  \rowcolors{2}{green!100!black!40}{green!100!black!30}
  \begin{tabular}{c|p{15cm}}
    \rowcolor[HTML]{125E28}
    \color[HTML]{FFFFFF}\textbf{ID Test}
          & \multicolumn{1}{c}{\color[HTML]{FFFFFF}\textbf{Metodo}}                                                                                          \\
    \hline
    TUF66 & shopchain-frontend/src/core/modules/order/store/implementations/\_\_test\_\_/\newline OrderStore.test.ts:it("should create an order store", ...) \\
    TUF67 & shopchain-frontend/src/core/modules/order/store/implementations/\_\_test\_\_/\newline OrderStore.test.ts:it("createOrder", ...)                  \\
    TUF68 & shopchain-frontend/src/application/layout/LockOverlay/LockOverlayViewModel\newline LockOverlayViewModel.test.ts:it("creates LockOverlayViewModel",...) \\
    TUF69 & shopchain-frontend/src/application/layout/LockOverlay/LockOverlayViewModel\newline LockOverlayViewModel.test.ts:it("successful", ()...) \\
    TUF70 & shopchain-frontend/src/application/layout/LockOverlay/LockOverlayViewModel\newline LockOverlayViewModel.test.ts:it("not successful",..) \\
    TUF71 & shopchain-frontend/src/application/layout/Nav/ConnectMetamask/\newline ConnectMetamaskViewModel/ConnectMetamaskViewModel.test.ts:it("should call makeAutoObservable at creation",...) \\
    TUF72 & shopchain-frontend/src/application/layout/Nav/ConnectMetamask/\newline ConnectMetamaskViewModel/ConnectMetamaskViewModel.test.ts:it("should call connect",...) \\
    TUF73 & shopchain-frontend/src/application/layout/Nav/MetamaskError/\newline MetamaskErrorViewModel/MetamaskErrorViewModel.test.ts:itit("should return severity",...) \\
    TUF74 & shopchain-frontend/src/application/layout/Nav/MetamaskError/\newline MetamaskErrorViewModel/MetamaskErrorViewModel.test.ts:itit("should return name",...) \\
    TUF75 & shopchain-frontend/src/application/layout/Nav/MetamaskError/\newline MetamaskErrorViewModel/MetamaskErrorViewModel.test.ts:itit("should return description",...) \\
    TUF76 & shopchain-frontend/src/application/layout/Nav/NavViewModel/NavViewModel.test.ts\newline :it("should call makeAutoObservable at creation",...) \\
    TUF77 & shopchain-frontend/src/application/layout/Nav/NavViewModel/NavViewModel.test.ts\newline :it("should return the address",...OK) \\
    TUF78 & shopchain-frontend/src/application/layout/Nav/NavViewModel/NavViewModel.test.ts\newline :it("should return the address",...FAIL) \\
    TUF79 & shopchain-frontend/src/application/pages/ECommerce/ECommerceViewModel.test.ts\newline :it("should create an instance",...) \\
  \end{tabular}
\end{table}
\begin{center}
  \textit{\small Continua nella pagina successiva}
\end{center}
\begin{table}[H]
  \centering
  \renewcommand{\arraystretch}{1.8}
  \rowcolors{2}{green!100!black!40}{green!100!black!30}
  \begin{tabular}{c|p{15cm}}
    \rowcolor[HTML]{125E28}
    \color[HTML]{FFFFFF}\textbf{ID Test}
          & \multicolumn{1}{c}{\color[HTML]{FFFFFF}\textbf{Metodo}}                                                                                          \\
    \hline
    TUF80 & shopchain-frontend/src/application/pages/ECommerce/ECommerceViewModel.test.ts\newline :it("should return amount",...) \\
    TUF81 & shopchain-frontend/src/application/pages/ECommerce/ECommerceViewModel.test.ts\newline :it("should return wei",...) \\
    TUF82 & shopchain-frontend/src/application/pages/ECommerce/ECommerceViewModel.test.ts\newline :it("should return id",...) \\
    TUF83 & shopchain-frontend/src/application/pages/ECommerce/ECommerceViewModel.test.ts\newline :it("should call setAmount",...) \\
    TUF84 & shopchain-frontend/src/application/pages/ECommerce/ECommerceViewModel.test.ts\newline :it("should call handleSubmit",...) \\
    TUF85 & shopchain-frontend/src/application/pages/TransactionInit/Choice/\newline ChoiceViewModel.test.ts:it("should create an instance",...) \\
    TUF86 & shopchain-frontend/src/application/pages/TransactionInit/Choice/\newline ChoiceViewModel.test.ts:it("should call createOrder",...) \\
    TUF87 & shopchain-frontend/src/application/pages/TransactionInit/Choice/\newline ChoiceViewModel.test.ts:it("should return canRedirect",...) \\
    TUF88 & shopchain-frontend/src/application/pages/TransactionInit/Choice/\newline ChoiceViewModel.test.ts:it("should return isBusy",...) \\
    TUF89 & shopchain-frontend/src/application/pages/TransactionInit/Choice/\newline ChoiceViewModel.test.ts:it("should return isFailed",...) \\
    TUF90 & shopchain-frontend/src/application/pages/TransactionInit/Choice/\newline ChoiceViewModel.test.ts:it("setAmount",...) \\
    TUF91 & shopchain-frontend/src/application/pages/TransactionInit/Choice/\newline ChoiceViewModel.test.ts:it("setId",...) \\
    TUF92 & shopchain-frontend/src/application/pages/TransactionInit/PickAmount/\newline PickAmountViewModel.test.ts:it("should create an instance",...) \\
    TUF93 & shopchain-frontend/src/application/pages/TransactionInit/PickAmount/\newline PickAmountViewModel.test.ts:it("should call createMoneyBox",...) \\
    TUF94 & shopchain-frontend/src/application/pages/TransactionInit/PickAmount/\newline PickAmountViewModel.test.ts:it("should return canRedirect",...) \\
    TUF95 & shopchain-frontend/src/application/pages/TransactionInit/PickAmount/\newline PickAmountViewModel.test.ts:it("should return isBusy",...) \\
  \end{tabular}
\end{table}
\begin{center}
  \textit{\small Continua nella pagina successiva}
\end{center}
\begin{table}[H]
  \centering
  \renewcommand{\arraystretch}{1.8}
  \rowcolors{2}{green!100!black!40}{green!100!black!30}
  \begin{tabular}{c|p{15cm}}
    \rowcolor[HTML]{125E28}
    \color[HTML]{FFFFFF}\textbf{ID Test}
          & \multicolumn{1}{c}{\color[HTML]{FFFFFF}\textbf{Metodo}}                                                                                          \\
    \hline
    TUF96 & shopchain-frontend/src/application/pages/TransactionInit/PickAmount/\newline PickAmountViewModel.test.ts:it("should call setAmount",...) \\
    TUF97 & shopchain-frontend/src/application/pages/TransactionInit/PickAmount/\newline PickAmountViewModel.test.ts:it("should call setId",...) \\
    TUF98 & shopchain-frontend/src/application/pages/TransactionInit/PickAmount/\newline PickAmountViewModel.test.ts:it("should call setInitFTM",...) \\
    TUF99 & shopchain-frontend/src/application/pages/TransactionInit/\newline TransactionInitViewModel.test.ts:it("should create an instance",...) \\
    TUF100 & shopchain-frontend/src/application/pages/TransactionInit/\newline TransactionInitViewModel.test.ts:it("should call createMoneyBox",...) \\
    TUF101 & shopchain-frontend/src/application/pages/TransactionInit/\newline TransactionInitViewModel.test.ts:it("should call setAmount",...) \\
    TUF102 & shopchain-frontend/src/application/pages/TransactionInit/\newline TransactionInitViewModel.test.ts:it("should call setId",...) \\
    TUF103 & shopchain-frontend/src/application/pages/TransactionInit/\newline TransactionInitViewModel.test.ts:it("should return the seller address",...) \\
    TUF104 & shopchain-frontend/src/application/pages/Home/MoneyBoxBalance/\newline MoneyBoxBalanceViewModel.test.ts:it("creates vm",...) \\
    TUF105 & shopchain-frontend/src/application/pages/Home/MoneyBoxBalance/\newline MoneyBoxBalanceViewModel.test.ts:it("isBusy",...) \\
    TUF106 & shopchain-frontend/src/application/pages/Home/MoneyBoxBalance/\newline MoneyBoxBalanceViewModel.test.ts:it("balanceFTM",...) \\
    TUF107 & shopchain-frontend/src/application/pages/Home/MoneyBoxBalance/\newline MoneyBoxBalanceViewModel.test.ts:it("balanceWEI",...) \\
    TUF108 & shopchain-frontend/src/application/pages/Home/MoneyBoxCount/\newline MoneyBoxCountViewModel.test.ts:it("creates vm",...) \\
    TUF109 & shopchain-frontend/src/application/pages/Home/MoneyBoxCount/\newline MoneyBoxCountViewModel.test.ts:it("isBusy",...) \\
    TUF110 & shopchain-frontend/src/application/pages/Home/MoneyBoxCount/\newline MoneyBoxCountViewModel.test.ts:it("count",...) \\
    TUF111 & shopchain-frontend/src/application/pages/Home/OrderBalance/\newline OrderBalanceViewModel.test.ts:it("creates vm",...) \\
  \end{tabular}
\end{table}
\begin{center}
  \textit{\small Continua nella pagina successiva}
\end{center}
\begin{table}[H]
  \centering
  \renewcommand{\arraystretch}{1.8}
  \rowcolors{2}{green!100!black!40}{green!100!black!30}
  \begin{tabular}{c|p{15cm}}
    \rowcolor[HTML]{125E28}
    \color[HTML]{FFFFFF}\textbf{ID Test}
          & \multicolumn{1}{c}{\color[HTML]{FFFFFF}\textbf{Metodo}}                                                                                          \\
    \hline
    TUF112 & shopchain-frontend/src/application/pages/Home/OrderBalance/\newline OrderBalanceViewModel.test.ts:it("isBusy",...) \\
    TUF113 & shopchain-frontend/src/application/pages/Home/OrderBalance/\newline OrderBalanceViewModel.test.ts:it("balanceFTM",...) \\
    TUF114 & shopchain-frontend/src/application/pages/Home/OrderBalance/\newline OrderBalanceViewModel.test.ts:it("balanceWEI",...) \\
    TUF115 & shopchain-frontend/src/application/pages/Home/OrderCount/\newline OrderCountViewModel.test.ts:it("creates vm",...) \\
    TUF116 & shopchain-frontend/src/application/pages/Home/OrderCount/\newline OrderCountViewModel.test.ts:it("isBusy",...) \\
    TUF117 & shopchain-frontend/src/application/pages/Home/OrderCount/\newline OrderCountViewModel.test.ts:it("count",...) \\
    TUF118 & shopchain-frontend/src/application/pages/MoneyBoxDetails/\newline MoneyBoxDetailsViewModel.test.ts:it("should create an instance",...) \\
    TUF119 & shopchain-frontend/src/application/pages/MoneyBoxDetails/\newline MoneyBoxDetailsViewModel.test.ts:it("should set the id",...) \\
    TUF120 & shopchain-frontend/src/application/pages/MoneyBoxDetails/\newline MoneyBoxDetailsViewModel.test.ts:it("should get ownerAddress",...) \\
    TUF121 & shopchain-frontend/src/application/pages/MoneyBoxDetails/\newline MoneyBoxDetailsViewModel.test.ts:it("should get sellerAddress",...) \\
    TUF122 & shopchain-frontend/src/application/pages/MoneyBoxDetails/\newline MoneyBoxDetailsViewModel.test.ts:it("should get ftm",...) \\
    TUF123 & shopchain-frontend/src/application/pages/MoneyBoxDetails/\newline MoneyBoxDetailsViewModel.test.ts:it("should get wei",...) \\
    TUF124 & shopchain-frontend/src/application/pages/MoneyBoxDetails/\newline MoneyBoxDetailsViewModel.test.ts:it("should get filledFtm",...) \\
    TUF125 & shopchain-frontend/src/application/pages/MoneyBoxDetails/\newline MoneyBoxDetailsViewModel.test.ts:it("should get filledWei",...) \\
    TUF126 & shopchain-frontend/src/application/pages/MoneyBoxDetails/\newline MoneyBoxDetailsViewModel.test.ts:it("should get ftmToFill",...) \\
    TUF127 & shopchain-frontend/src/application/pages/MoneyBoxDetails/\newline MoneyBoxDetailsViewModel.test.ts:it("should get weiToFill",...) \\
  \end{tabular}
\end{table}
\begin{center}
  \textit{\small Continua nella pagina successiva}
\end{center}
\begin{table}[H]
  \centering
  \renewcommand{\arraystretch}{1.8}
  \rowcolors{2}{green!100!black!40}{green!100!black!30}
  \begin{tabular}{c|p{15cm}}
    \rowcolor[HTML]{125E28}
    \color[HTML]{FFFFFF}\textbf{ID Test}
          & \multicolumn{1}{c}{\color[HTML]{FFFFFF}\textbf{Metodo}}                                                                                          \\
    \hline
    TUF128 & shopchain-frontend/src/application/pages/MoneyBoxDetails/\newline MoneyBoxDetailsViewModel.test.ts:it("should get state",...) \\
    TUF129 & shopchain-frontend/src/application/pages/MoneyBoxDetails/\newline MoneyBoxDetailsViewModel.test.ts:it("should get isPaid",...) \\
    TUF130 & shopchain-frontend/src/application/pages/MoneyBoxDetails/\newline MoneyBoxDetailsViewModel.test.ts:it("should get isRefunded",...) \\
    TUF131 & shopchain-frontend/src/application/pages/MoneyBoxDetails/\newline MoneyBoxDetailsViewModel.test.ts:it("should get isUnlocked",...) \\
    TUF132 & shopchain-frontend/src/application/pages/OrderDetails/\newline OrderDetailsViewModel.test.ts:it("should create an instance",...) \\
    TUF133 & shopchain-frontend/src/application/pages/OrderDetails/\newline OrderDetailsViewModel.test.ts:it("should set the id",...) \\
    TUF134 & shopchain-frontend/src/application/pages/OrderDetails/\newline OrderDetailsViewModel.test.ts:it("should get ownerAddress",...) \\
    TUF135 & shopchain-frontend/src/application/pages/OrderDetails/\newline OrderDetailsViewModel.test.ts:it("should get sellerAddress",...) \\
    TUF136 & shopchain-frontend/src/application/pages/OrderDetails/\newline OrderDetailsViewModel.test.ts:it("should get ftm",...) \\
    TUF137 & shopchain-frontend/src/application/pages/OrderDetails/\newline OrderDetailsViewModel.test.ts:it("should get wei",...) \\
    TUF138 & shopchain-frontend/src/application/pages/OrderDetails/\newline OrderDetailsViewModel.test.ts:it("should get state",...) \\
    TUF139 & shopchain-frontend/src/application/pages/OrderDetails/\newline OrderDetailsViewModel.test.ts:it("should get isPaid",...) \\
    TUF140 & shopchain-frontend/src/application/pages/OrderDetails/\newline OrderDetailsViewModel.test.ts:it("should get code",...) \\
    TUF141 & shopchain-frontend/src/application/pages/TransactionIn/TransactionListEl\newline TransactionListElViewModel.test.ts:it("should create an instance",...) \\
    TUF142 & shopchain-frontend/src/application/pages/TransactionIn/TransactionListEl\newline TransactionListElViewModel.test.ts:it("should return the order id",...) \\
    TUF143 & shopchain-frontend/src/application/pages/TransactionIn/TransactionListEl\newline TransactionListElViewModel.test.ts:it("should return the correct order type",...) \\
  \end{tabular}
\end{table}
\begin{center}
  \textit{\small Continua nella pagina successiva}
\end{center}
\begin{table}[H]
  \centering
  \renewcommand{\arraystretch}{1.8}
  \rowcolors{2}{green!100!black!40}{green!100!black!30}
  \begin{tabular}{c|p{15cm}}
    \rowcolor[HTML]{125E28}
    \color[HTML]{FFFFFF}\textbf{ID Test}
          & \multicolumn{1}{c}{\color[HTML]{FFFFFF}\textbf{Metodo}}                                                                                          \\
    \hline
    TUF144 & shopchain-frontend/src/application/pages/TransactionIn/TransactionListEl\newline TransactionListElViewModel.test.ts:it("should return the correct order status",...) \\
    TUF145 & shopchain-frontend/src/application/pages/TransactionIn/TransactionListEl\newline TransactionListElViewModel.test.ts:it("should return the order",...) \\
  \end{tabular}
  \caption{Tracciamento test di unità - Frontend}
\end{table}

%%%%%%%%%%%%%%%%%%%%%%%%%%%%%

\subsubsection{Test di unità - Solidity - OrderManager}\label{subsubsection:TUO}
\begin{table}[H]
  \centering
  \renewcommand{\arraystretch}{1.8}
  \rowcolors{2}{green!100!black!40}{green!100!black!30}
  \begin{tabular}{c|p{10cm}|c}
    \rowcolor[HTML]{125E28}
    \color[HTML]{FFFFFF}\textbf{ID Test}
          & \multicolumn{1}{c}{\color[HTML]{FFFFFF}\textbf{Descrizione}}
          & \color[HTML]{FFFFFF}\textbf{Stato}                                                                                                \\
    \hline
    TUO01 & Si verifica che il contatore degli ordini sia inizializzato a zero.                                                           & S \\
    TUO02 & Si verifica che l'ordine creato sia correttamente.                                                                            & S \\
    TUO03 & Si verifica che il contratto abbia ricevuto l'ammontare corretto.                                                             & S \\
    TUO04 & Si verifica il caso fallimentare in cui l'utente tenta di inviare un importo inferiore a quello richiesto.                    & S \\
    TUO05 & Si verifica il caso fallimentare in cui l'utente prova a registrare un ordine che ha come venditore se stesso.                & S \\
    TUO06 & Si verifica il caso fallimentare in cui l'utente prova a creare un ordine con un importo pari a zero.                         & S \\
    TUO07 & Si verifica il caso fallimentare in cui l'utente prova a creare un ordine con un importo negativo.                            & S \\
    TUO08 & Si verifica il caso fallimentare in cui l'utente prova a creare un ordine non avendo i fondi necessari.                       & S \\
    TUO09 & Si verifica il caso fallimentare in cui il frontend\glo{} richiede la creazione di un ordine con un ID precedentemente usato. & S \\
    TUO10 & Si verifica che le informazioni di ritorno dell'ordine siano corrette.                                                        & S \\
    TUO11 & Si verifica che lo stato dell'ordine sia modificato e settato su "CLOSED" una volta confermata la ricezione dall'acquirente.  & S \\
    TUO12 & Si verifica che il venditore abbia ricevuto i fondi correttamente una volta confermata la ricezione dall'acquirente.          & S \\
  \end{tabular}
\end{table}
\begin{center}
  \textit{\small Continua nella pagina successiva}
\end{center}
\begin{table}[H]
  \centering
  \renewcommand{\arraystretch}{1.8}
  \rowcolors{2}{green!100!black!40}{green!100!black!30}
  \begin{tabular}{c|p{10cm}|c}
    \rowcolor[HTML]{125E28}
    \color[HTML]{FFFFFF}\textbf{ID Test}
          & \multicolumn{1}{c}{\color[HTML]{FFFFFF}\textbf{Descrizione}}
          & \color[HTML]{FFFFFF}\textbf{Stato}                                                                                                    \\
    \hline
    TUO13 & Si verifica il caso fallimentare in cui l'ID dell'ordine da sbloccare sia errato.                                                 & S \\
    TUO14 & Si verifica il caso fallimentare in cui il codice di sblocco non coincide con quello registrato in blockchain\glo{}.              & S \\
    TUO15 & Si verifica il caso fallimentare in cui l'ordine viene confermato da un utente con indirizzo differente da quello del compratore. & S \\
    TUO16 & Si verifica che lo stato dell'ordine venga settato su "CANCELLED" alla richiesta di rimborso da parte del compratore.             & S \\
    TUO17 & Si verifica che lo stato dell'ordine venga settato su "CANCELLED" alla richiesta di rimborso da parte del venditore.              & S \\
    TUO18 & Si verifica che i soldi siano restituiti correttamente al compratore quando è richiesto il rimborso dal compratore.               & S \\
    TUO19 & Si verifica che i soldi siano restituiti correttamente al compratore quando è richiesto il rimborso dal venditore.                & S \\
    TUO20 & Si verifica che non sia possibile chiedere il rimborso di un ordine chiuso.                                                       & S \\
    TUO21 & Si verifica che un utente con indirizzo diverso da quello del compratore o venditore non possa richiedere il rimborso.            & S \\
    TUO22 & Si verifica che venga restituito correttamente il saldo del contratto.                                                            & S \\
    TUO23 & Si verifica che venga restituito correttamente l'indirizzo del wallet\glo{} dell'acquirente.                                      & S \\
    TUO24 & Si verifica che venga restituito correttamente l'indirizzo del wallet\glo{} del venditore.                                        & S \\
    TUO25 & Si verifica che venga restituito correttamente l'ammontare da pagare.                                                             & S \\
    TUO26 & Si verifica che venga restituito correttamente lo stato dell'ordine.                                                              & S \\
    TUO27 & Si verifica che vengano restituiti correttamente tutti i dati dell'ordine, partendo dal suo ID.                                   & S \\
    TUO28 & Si verifica che vengano restituiti correttamente tutti gli ordini di un acquirente.                                               & S \\
  \end{tabular}
\end{table}
\begin{center}
  \textit{\small Continua nella pagina successiva}
\end{center}
\begin{table}[H]
  \centering
  \renewcommand{\arraystretch}{1.8}
  \rowcolors{2}{green!100!black!40}{green!100!black!30}
  \begin{tabular}{c|p{10cm}|c}
    \rowcolor[HTML]{125E28}
    \color[HTML]{FFFFFF}\textbf{ID Test}
          & \multicolumn{1}{c}{\color[HTML]{FFFFFF}\textbf{Descrizione}}
          & \color[HTML]{FFFFFF}\textbf{Stato}                                                     \\
    \hline
    TUO29 & Si verifica che vengano restituiti correttamente tutti gli ordini di un venditore. & S \\
  \end{tabular}
  \caption{Test di unità - Solidity - OrderManager}
\end{table}

\paragraph{Tracciamento test di unità - Solidity - OrderManager}\label{paragraph:tracciamento_TUO}

\begin{table}[H]
  \centering
  \renewcommand{\arraystretch}{1.8}
  \rowcolors{2}{green!100!black!40}{green!100!black!30}
  \begin{tabular}{c|p{15cm}}
    \rowcolor[HTML]{125E28}
    \color[HTML]{FFFFFF}\textbf{ID Test}
          & \multicolumn{1}{c}{\color[HTML]{FFFFFF}\textbf{Metodo}}                                                                                                                                                   \\
    \hline
    TUO01 & smart-contract/test/1\_OrderManager.test.js:it("should have the initial order number to 0", ...)                                                                                                          \\
    TUO02 & smart-contract/test/1\_OrderManager.test.js:describe("single payment order",\newline()=$>$\{it('order created correctly', ...)                                                                            \\
    TUO03 & smart-contract/test/1\_OrderManager.test.js:describe("single payment order",\newline()=$>$\{it("the contract is filled with the correct amount", ...)                                                     \\
    TUO04 & smart-contract/test/1\_OrderManager.test.js:describe("single payment order",\newline()=$>$\{describe("failure cases",()=$>$\{it("user tries to insert an amount less than the required amount", ...)      \\
    TUO05 & smart-contract/test/1\_OrderManager.test.js:describe("single payment order",\newline()=$>$\{describe("failure cases",()=$>$\{it("user tries to order his item, or send funds to itself", ...)             \\
    TUO06 & smart-contract/test/1\_OrderManager.test.js:describe("single payment order",\newline()=$>$\{describe("failure cases",()=$>$\{it("user tries to pass value equal to zero", ...)                            \\
    TUO07 & smart-contract/test/1\_OrderManager.test.js:describe("single payment order",\newline()=$>$\{describe("failure cases",()=$>$\{it("user tries to send negative coin value", ...)                            \\
    TUO08 & smart-contract/test/1\_OrderManager.test.js:describe("single payment order",\newline()=$>$\{describe("failure cases",()=$>$\{it("user hasn't enough founds", ...)                                         \\
    TUO09 & smart-contract/test/1\_OrderManager.test.js:describe("single payment order",\newline()=$>$\{describe("failure cases",()=$>$\{it("front end require the creation of an order with the same order id", ...) \\
    TUO10 & smart-contract/test/1\_OrderManager.test.js:describe("single payment order",\newline()=$>$\{it('get the correct order infos', ...)                                                                        \\
    TUO11 & smart-contract/test/1\_OrderManager.test.js:describe("order confirmation",\newline()=$>$\{it("should update state", ...)                                                                                  \\
    TUO12 & smart-contract/test/1\_OrderManager.test.js:describe("order confirmation",\newline()=$>$\{it("should release payment to seller", ...)                                                                     \\
  \end{tabular}
\end{table}
\begin{center}
  \textit{\small Continua nella pagina successiva}
\end{center}
\begin{table}[H]
  \centering
  \renewcommand{\arraystretch}{1.8}
  \rowcolors{2}{green!100!black!40}{green!100!black!30}
  \begin{tabular}{c|p{15cm}}
    \rowcolor[HTML]{125E28}
    \color[HTML]{FFFFFF}\textbf{ID Test}
          & \multicolumn{1}{c}{\color[HTML]{FFFFFF}\textbf{Metodo}}                                                                                                                                                  \\
    \hline
    TUO13 & smart-contract/test/1\_OrderManager.test.js:describe("order confirmation",\newline()=$>$\{describe("failure cases",async ()=$>$\{it("order id isn't correct", ...)                                       \\
    TUO14 & smart-contract/test/1\_OrderManager.test.js:describe("order confirmation",\newline()=$>$\{describe("failure cases",async ()=$>$\{it("order unlock code doesn't match with unlock code saved", ...)       \\
    TUO15 & smart-contract/test/1\_OrderManager.test.js:describe("order confirmation",\newline()=$>$\{describe("failure cases",async ()=$>$\{it("order is unlock from an address different from buyer address", ...) \\
    TUO16 & smart-contract/test/1\_OrderManager.test.js:describe("order refund",\newline async ()=$>$\{describe("should set the cancelled state",()=$>$\{it("from buyer", ...)                                       \\
    TUO17 & smart-contract/test/1\_OrderManager.test.js:describe("order refund",\newline async ()=$>$\{describe("should set the cancelled state",()=$>$\{it("from seller", ...)                                      \\
    TUO18 & smart-contract/test/1\_OrderManager.test.js:describe("order refund",\newline async ()=$>$\{describe("should move funds back to the buyer",async ()=$>$\{it("from buyer [@skip-on-coverage]", ...)        \\
    TUO19 & smart-contract/test/1\_OrderManager.test.js:describe("order refund",\newline async ()=$>$\{describe("should move funds back to the buyer",async ()=$>$\{it("from seller", ...)                           \\
    TUO20 & smart-contract/test/1\_OrderManager.test.js:describe("order refund",\newline async ()=$>$\{describe("failure cases",()=$>$\{it("should not refund an order already closed", ...)                         \\
    TUO21 & smart-contract/test/1\_OrderManager.test.js:describe("order refund",\newline async ()=$>$\{describe("failure cases",()=$>$\{it("should be the owner or the seller", ...)                                 \\
    TUO22 & smart-contract/test/1\_OrderManager.test.js:describe('check getter functions',\newline async ()=$>$\{it("check contractBalance()", ...)                                                                  \\
    TUO23 & smart-contract/test/1\_OrderManager.test.js:describe('check getter functions',\newline async ()=$>$\{it("check getOwnerAddress(string)", ...)                                                            \\
    TUO24 & smart-contract/test/1\_OrderManager.test.js:describe('check getter functions',\newline async ()=$>$\{it("check getSellerAddress(string)", ...)                                                           \\
    TUO25 & smart-contract/test/1\_OrderManager.test.js:describe('check getter functions',\newline async ()=$>$\{it("check getAmountToPay(string)", ...)                                                             \\
    TUO26 & smart-contract/test/1\_OrderManager.test.js:describe('check getter functions',\newline async ()=$>$\{it("check getOrderState(string)", ...)                                                              \\
    TUO27 & smart-contract/test/1\_OrderManager.test.js:describe('check getter functions',\newline async ()=$>$\{it("check getOrderById(string)", ...)                                                               \\
  \end{tabular}
\end{table}
\begin{center}
  \textit{\small Continua nella pagina successiva}
\end{center}
\begin{table}[H]
  \centering
  \renewcommand{\arraystretch}{1.8}
  \rowcolors{2}{green!100!black!40}{green!100!black!30}
  \begin{tabular}{c|p{15cm}}
    \rowcolor[HTML]{125E28}
    \color[HTML]{FFFFFF}\textbf{ID Test}
          & \multicolumn{1}{c}{\color[HTML]{FFFFFF}\textbf{Metodo}}                                                                                          \\
    \hline
    TUO28 & smart-contract/test/1\_OrderManager.test.js:describe('check getter functions',\newline async ()=$>$\{it("check getOrdersByBuyer(address)", ...)  \\
    TUO29 & smart-contract/test/1\_OrderManager.test.js:describe('check getter functions',\newline async ()=$>$\{it("check getOrdersBySeller(address)", ...) \\
  \end{tabular}
  \caption{Tracciamento test di unità - Solidity - OrderManager}
\end{table}
%%%%%%%%%%%%%%%%%%%%%%%%%%%%%%%%%%%%%%%%
\subsubsection{Test di unità - Solidity - MoneyBox} \label{subsubsection:TUM}
\begin{table}[H]
  \centering
  \renewcommand{\arraystretch}{1.8}
  \rowcolors{2}{green!100!black!40}{green!100!black!30}
  \begin{tabular}{c|p{10cm}|c}
    \rowcolor[HTML]{125E28}
    \color[HTML]{FFFFFF}\textbf{ID Test}
          & \multicolumn{1}{c}{\color[HTML]{FFFFFF}\textbf{Descrizione}}
          & \color[HTML]{FFFFFF}\textbf{Stato}                                                                                                 \\
    \hline
    TUM01 & Si verifica che il contatore degli ordini sia inizializzato a zero.                                                            & S \\
    TUM02 & Si verifica che la MoneyBox\glo{} venga creata correttamente.                                                                  & S \\
    TUM03 & Si verifica che il proprietario dell'ordine possa pagare una parte dell'importo totale alla creazione della MoneyBox\glo{}.    & S \\
    TUM04 & Si verifica che il trasferimento di un nuovo importo nella MoneyBox\glo{} avvenga correttamente.                               & S \\
    TUM05 & Si verifica che il rimborso a partire dalla richiesta del proprietario della MoneyBox\glo{} avvenga correttamente.             & S \\
    TUM06 & Si verifica che il rimborso a partire dalla richiesta del venditore dell'ordine avvenga correttamente.                         & S \\
    TUM07 & Si verifica il caso fallimentare in cui l'utente non invia l'importo stabilito.                                                & S \\
    TUM08 & Si verifica il caso fallimentare in cui l'acquirente richieda il rimborso di una MoneyBox\glo{} chiusa.                        & S \\
    TUM09 & Si verifica il caso fallimentare in cui l'utente cerca di creare un nuovo pagamento con un ammontare negativo.                 & S \\
    TUM10 & Si verifica che vengano ritornate correttamente tutte le transazioni partecipanti alla MoneyBox\glo{}.                         & S \\
    TUM11 & Si verifica che venga ritornato il corretto ammontare mancante per il completamento della MoneyBox\glo{}.                      & S \\
    TUM12 & Si verifica che vengano ritornati correttamente tutti gli ordini relativi all'acquirente (registrati in entrambi i contratti). & S \\
  \end{tabular}
\end{table}
\begin{center}
  \textit{\small Continua nella pagina successiva}
\end{center}
\begin{table}[H]
  \centering
  \renewcommand{\arraystretch}{1.8}
  \rowcolors{2}{green!100!black!40}{green!100!black!30}
  \begin{tabular}{c|p{10cm}|c}
    \rowcolor[HTML]{125E28}
    \color[HTML]{FFFFFF}\textbf{ID Test}
          & \multicolumn{1}{c}{\color[HTML]{FFFFFF}\textbf{Descrizione}}
          & \color[HTML]{FFFFFF}\textbf{Stato}                                                                                               \\
    \hline
    TUM13 & Si verifica che vengano ritornati correttamente tutti gli ordini relativi al venditore (registrati in entrambi i contratti). & S \\
    TUM14 & Si verifica che vengano ritornati correttamente tutti gli ordini MoneyBox\glo{} relativi all'indirizzo di un partecipante.   & S \\
  \end{tabular}
  \caption{Test di unità - Solidity - MoneyBox}
\end{table}



\paragraph{Tracciamento test di unità - Solidity - OrderManager}\label{paragraph:tracciamento_TUM}

\begin{table}[H]
  \centering
  \renewcommand{\arraystretch}{1.8}
  \rowcolors{2}{green!100!black!40}{green!100!black!30}
  \begin{tabular}{c|p{15cm}}
    \rowcolor[HTML]{125E28}
    \color[HTML]{FFFFFF}\textbf{ID Test}
          & \multicolumn{1}{c}{\color[HTML]{FFFFFF}\textbf{Metodo}}                                                                                                                              \\
    \hline
    TUM01 & smart-contract/test/2\_MoneyBox.test.js:it("should have the initial order number to 0", ...)                                                                                         \\
    TUM02 & smart-contract/test/2\_MoneyBox.test.js:describe("moneybox management",\newline()=$>$\{it('moneybox created correctly', ...)                                                         \\
    TUM03 & smart-contract/test/2\_MoneyBox.test.js:describe("moneybox management",\newline()=$>$\{it("user sends import at moneybox creation", ...)                                             \\
    TUM04 & smart-contract/test/2\_MoneyBox.test.js:describe("moneybox management",\newline()=$>$\{it("new fee transfer into moneybox", ...)                                                     \\
    TUM05 & smart-contract/test/2\_MoneyBox.test.js:describe("moneybox management",\newline()=$>$\{it("refund all fee transfers from moneybox owner", ...)                                       \\
    TUM06 & smart-contract/test/2\_MoneyBox.test.js:describe("moneybox management",\newline()=$>$\{it("refund all fee transfers from moneybox seller", ...)                                      \\
    TUM07 & smart-contract/test/2\_MoneyBox.test.js:describe("moneybox management",\newline()=$>$\{describe("failure cases",()=$>$\{it("should not have insufficient value", ...)                \\
    TUM08 & smart-contract/test/2\_MoneyBox.test.js:describe("moneybox management",\newline()=$>$\{describe("failure cases",()=$>$\{it("buyer tries to call refund from a closed moneybox", ...) \\
    TUM09 & smart-contract/test/2\_MoneyBox.test.js:describe("moneybox management",\newline()=$>$\{describe("failure cases",()=$>$\{it("negative fee payment", ...)                              \\
    TUM10 & smart-contract/test/2\_MoneyBox.test.js:describe("check getter functions",\newline()=$>$\{it("check getMoneyBoxPayments(string)", ...)                                               \\
    TUM11 & smart-contract/test/2\_MoneyBox.test.js:describe("check getter functions",\newline()=$>$\{it("check getAmountToFill(string)", ...)                                                   \\
  \end{tabular}
\end{table}
\begin{center}
  \textit{\small Continua nella pagina successiva}
\end{center}
\begin{table}[H]
  \centering
  \renewcommand{\arraystretch}{1.8}
  \rowcolors{2}{green!100!black!40}{green!100!black!30}
  \begin{tabular}{c|p{15cm}}
    \rowcolor[HTML]{125E28}
    \color[HTML]{FFFFFF}\textbf{ID Test}
          & \multicolumn{1}{c}{\color[HTML]{FFFFFF}\textbf{Metodo}}                                                                                                     \\
    \hline
    TUM12 & smart-contract/test/2\_MoneyBox.test.js:describe("check getter functions",\newline()=$>$\{it("check getAllBuyerOrders(supercontract, \_buyerAddress)", ...) \\
    TUM13 & smart-contract/test/2\_MoneyBox.test.js:describe("check getter functions",\newline()=$>$\{it("check getAllSellerOrders(OrderManager, address)", ...)        \\
    TUM14 & smart-contract/test/2\_MoneyBox.test.js:describe("check getter functions",\newline()=$>$\{it("check getMoneyBoxesByParticipantAddress(address)", ...)       \\
  \end{tabular}
  \caption{Tracciamento test di unità - Solidity - MoneyBox}
\end{table}