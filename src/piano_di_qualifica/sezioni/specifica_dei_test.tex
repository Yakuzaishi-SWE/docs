\section{Specifica dei test}\label{section:specifica_test}

%%%%%%%%%%%%%%%%%%%%%%%%%%%%%%%%%%%%%%%%%%%%%%%%%%%%%%%%%%%%%%%%%%%%%%%%%%%%%%%
\subsection{Test di accettazione}\label{subsection:test_accettazione}
I test di accettazione sono necessari per dimostrare che il prodotto soddisfi i requisiti minimi concordati con il proponente. \\
Essi si compongono dei test di sistema e vengono eseguiti durante il collaudo finale sia dai membri del 
gruppo che dall'azienda proponente, sotto supervisione del gruppo stesso.

%%%%%%%%%%%%%%%%%%%%%%%%%%%%%%%%%%%%%%%%%%%%%%%%%%%%%%%%%%%%%%%%%%%%%%%%%%%%%%%
\subsection{Test di sistema}\label{subsection:test_sistema}
\begin{table}[H]
  \centering
  \renewcommand{\arraystretch}{1.8}
  \rowcolors{2}{green!100!black!40}{green!100!black!30}
  \begin{tabular}{c|p{8cm}|c|c}
    \rowcolor[HTML]{125E28}
    \color[HTML]{FFFFFF}\textbf{ID Test}
    & \multicolumn{1}{c}{\color[HTML]{FFFFFF}\textbf{Descrizione}}
    & \color[HTML]{FFFFFF}\textbf{ID Requisiti}
    & \color[HTML]{FFFFFF}\textbf{Stato}\\
    \hline
    TS1F1 &	Verificare che la richiesta di checkout da un E-Commerce avvenga correttamente. &	R1F1 &	NI \\
    TS1F2 &	Verificare che l'utente visualizzi correttamente le diverse tipologie di pagamento. &	R1F2 &	NI \\
    TS1F2.1 &	Verificare che l'utente possa scegliere la tipologia di pagamento unico correttamente. &	R1F2.1 &	NI \\
    TS1F2.2 &	Verificare che l'utente possa scegliere la tipologia di pagamento MoneyBox correttamente. &	R1F2.2 &	NI \\
    TS2F2.2.1 &	Verificare che l'utente possa visualizzare lo stato di completamento della MoneyBox correttamente. &	R2F2.2.1 &	NI \\
    TS2F2.2.2 &	Verificare che l'utente possa visualizzare l'invito di partecipazione alla MoneyBox correttamente. &	R2F2.2.2 &	NI \\
    TS3F2.2.3 &	Verificare che l'utente possa visualizzare una traduzione visiva della MoneyBox correttamente. &	R3F2.2.3 &	NI \\
    TS2F2.2.4 &	Verificare che in caso di chiusura della MoneyBox i soldi vengano restituiti correttamente. &	R2F2.2.4 &	NI \\
    TS2F2.2.5 &	Verificare che l'utente possa visualizzare l'elenco delle transazione dei partecipanti alla MoneyBox correttamente. &	R2F2.2.5 &	NI \\
    TS1F3 &	Verificare che l'utente possa visualizzare il totale dell'ordine correttamente. &	R1F3 &	NI \\
    TS1F4 &	Verificare che l'utente possa effettuare la connessione a Metamask correttamente. &	R1F4 &	NI \\
  \end{tabular}
\end{table}
\begin{center}
  \textit{\small Continua nella pagina successiva}
\end{center}
\begin{table}[H]
  \centering
  \renewcommand{\arraystretch}{1.8}
  \rowcolors{2}{green!100!black!40}{green!100!black!30}
  \begin{tabular}{c|p{8cm}|c|c}
    \rowcolor[HTML]{125E28}
    \color[HTML]{FFFFFF}\textbf{ID Test}
    & \multicolumn{1}{c}{\color[HTML]{FFFFFF}\textbf{Descrizione}}
    & \color[HTML]{FFFFFF}\textbf{ID Requisiti}
    & \color[HTML]{FFFFFF}\textbf{Stato}\\
    \hline
    TS1F5 &	Verificare che l'utente possa pagare correttamente. &	R1F5 &	NI \\
    TS1F5.1 &	Verificare che il versamento per il pagamento unico copra l'intera somma richiesta. &	R1F5.1 &	NI \\
    TS2F5.2 &	Verificare che il versamento per il pagamento tramite MoneyBox possa coprire solo una parte della somma richiesta. &	R2F5.2 &	NI \\
    TS1F6 &	Verificare che la richiesta di rimborso funzioni correttamente. &	R1F6 &	NI \\
    TS1F7 &	Verificare che l'acquirente possa sbloccare, correttamente, i fondi dallo Smart Contract dopo avvenuta ricezione. &	R1F7 &	NI \\
    TS1F8 &	Verificare che l'utente possa visualizzare le transazioni. &	R1F8 &	NI \\
    TS2F8.1 &	Verificare che il venditore possa visualizzare le transazioni in entrata pagate. & R2F8.1 &	NI \\
    TS2F8.1.1 & Verificare che il venditore possa visualizzare le transazioni in entrata pagate ma non sbloccate. & R2F8.1.1 &	NI \\
    TS2F8.1.2 & Verificare che il venditore possa visualizzare le transazioni in entrata pagate e sbloccate. & R2F8.1.2 &	NI \\
    TS2F8.1.3 & Verificare che il venditore possa visualizzare le transazioni in entrata pagate cancellate. & R2F8.1.3 &	NI \\
    TS1F8.2 &	Verificare che il proprietario dell'ordine possa visualizzare le transazioni in uscita. &	R1F8.2 & NI \\
    TS2F8.2.1 &	Verificare che il proprietario dell'ordine possa visualizzare le transazioni in uscita non pagate. & R2F8.2.1 &	NI \\
    TS2F8.2.2 &	Verificare che il proprietario dell'ordine possa visualizzare le transazioni in uscita pagate ma non sbloccate. &	R2F8.2.2 &	NI \\
    TS2F8.2.3 &	Verificare che il proprietario dell'ordine possa visualizzare le transazioni in uscita pagate e sbloccate. & R2F8.2.3 &	NI \\
    TS2F8.2.4 &	Verificare che il proprietario dell'ordine possa visualizzare le transazioni in uscita cancellate. &	R2F8.2.4 &	NI \\
  \end{tabular}
\end{table}
\begin{center}
  \textit{\small Continua nella pagina successiva}
\end{center}
\begin{table}[H]
  \centering
  \renewcommand{\arraystretch}{1.8}
  \rowcolors{2}{green!100!black!40}{green!100!black!30}
  \begin{tabular}{c|p{8cm}|c|c}
    \rowcolor[HTML]{125E28}
    \color[HTML]{FFFFFF}\textbf{ID Test}
    & \multicolumn{1}{c}{\color[HTML]{FFFFFF}\textbf{Descrizione}}
    & \color[HTML]{FFFFFF}\textbf{ID Requisiti}
    & \color[HTML]{FFFFFF}\textbf{Stato}\\
    \hline
    TS2F9 &	Verificare che si possa convertire correttamente l'ammontare depositato in stable coin. &	R2F9 &	NI \\
    TS3F10 &	Verificare che, dopo la transazione, una fee percentuale venga destinata al fondo ShopChain correttamente. & R3F10 &	NI \\
    TS1F11 &	Verificare che la richiesta di un ordine venga caricata correttamente su una blockchain pubblica. &	R1V1 &	NI \\
    TS1F12 &	Verificare che il pagamento e le transazioni siano gestite correttamente dallo smart contract. &	R1V2 &	NI \\
  \end{tabular}
  \caption{Test di sistema}
\end{table}

%%%%%%%%%%%%%%%%%%%%%%%%%%%%%%%%%%%%%%%%%%%%%%%%%%%%%%%%%%%%%%%%%%%%%%%%%%%%%%%
\subsection{Test d'integrazione}\label{subsection:test_integrazione}
\begin{table}[H]
  \centering
  \renewcommand{\arraystretch}{1.8}
  \rowcolors{2}{green!100!black!40}{green!100!black!30}
  \begin{tabular}{c|p{8cm}|c}
    \rowcolor[HTML]{125E28}
    \color[HTML]{FFFFFF}\textbf{Codice}
    & \multicolumn{1}{c}{\color[HTML]{FFFFFF}\textbf{Descrizione}}
    & \color[HTML]{FFFFFF}\textbf{Stato}\\
    \hline
    TI1 & Verificare che il collegamento con Metamask avvenga correttamente. & NI \\
    TI2 & Verificare che il collegamento tra Front-End\glo{} e Smart Contract tramite libreria Web3.js avvenga correttamente. & NI \\
  \end{tabular}
  \caption{Test d'integrazione}
\end{table}

%%%%%%%%%%%%%%%%%%%%%%%%%%%%%%%%%%%%%%%%%%%%%%%%%%%%%%%%%%%%%%%%%%%%%%%%%%%%%%%
\subsection{Test di unità}\label{subsection:test_unita}
Per garantire il corretto funzionamento di ogni minimo componente autonomo del sistema vengono eseguiti i test di unità. \\
I test in questione verranno stabiliti durante il periodo di \textit{Progettazione di dettaglio e codifica requisiti obbligatori}.
