\section{Introduzione}\label{section:introduzione}

\subsection{Scopo del documento}\label{subsection:scopo_documento}
Questo documento ha lo scopo di descrivere le strategie che il gruppo usa per perseguire gli obbiettivi di qualità da applicare al progetto.\\
Esso deve implementare degli standard che permettano il miglioramento continuo, tramite misurazioni periodiche dei risultati ottenuti, sfruttandoli per ottenere azioni migliorative.\\
All'interno del \docNamePdP{} vengono raccolte le definizioni dei test, il loro stato e il loro tracciamento rispetto ai requisiti individuati nell'\docNameVersionAdR{}.

\subsection{Scopo del capitolato}\label{subsection:scopo_capitolato}
L'avvento delle tecnologie BlockChain\glo{} ha portato e porterà nei prossimi anni a grandi cambiamenti nella società.\\
In particolare, ha aperto le porte a una nuova forma di finanza, la cosiddetta “DeFi” (Finanza Decentralizzata) che ha permesso a chiunque sia dotato di connessione internet di creare un Wallet\glo{} e possedere quindi criptovalute\glo{}.\\
Questo ha delineato due profili critici strettamente legati: da un lato il controllo del proprio portafoglio è passato completamente nelle mani dell'utente, dall'altro lato questo comporta la mancanza di un ente terzo che si occupi di gestire transazioni e offrire garanzie.\\
Nel capitolato in questione si vuole proprio risolvere questo problema, in uno scenario che comprende un e-commerce\glo{} basato su BlockChain\glo{} in cui si vuole tutelare entrambe le parti coinvolte in un acquisto tramite criptovalute.\\
Il fine del progetto è la realizzazione di un prototipo di una piattaforma integrabile con un “crypto-ecommerce”\glo{}, che si occupi di gestire gli ordini dalle fasi di pagamento alla consegna.

\subsection{Glossario}\label{subsection:glossario}
\gloDesc{}

\subsection{Riferimenti}\label{subsection:riferimenti}
\subsubsection{Riferimenti normativi}\label{subsubsection:riferimenti_normativi}
\begin{itemize}
  \item \textbf{\docNameVersionNdP{}};
  \item \textbf{Regolamento del progetto didattico:}\\\url{https://www.math.unipd.it/~tullio/IS-1/2021/Dispense/PD2.pdf}
\end{itemize}

\subsubsection{Riferimenti informativi}\label{subsubsection:riferimenti_informativi}
\begin{itemize}
  \item \textbf{\docNameVersionAdR{}};
  \item \textbf{Capitolato d'appalto C2 - ShopChain:}\\\url{https://www.math.unipd.it/~tullio/IS-1/2021/Progetto/C2.pdf}
  \item \textbf{Software Engineering - Ian Sommerville - 10th Edition:}
  \begin{itemize}
    \item Capitolo 24 - Quality management.
  \end{itemize}
  \item \textbf{Qualità di prodotto - slide T12 del corso di Ingegneria del Software:}\\\url{https://www.math.unipd.it/~tullio/IS-1/2021/Dispense/T12.pdf}
  \item \textbf{Qualità di processo - slide T13 del corso Ingegneria del Software:}\\\url{https://www.math.unipd.it/~tullio/IS-1/2021/Dispense/T13.pdf}
  \item \textbf{Verifica e validazione: introduzione - slide T14 del corso Ingegneria del Software:}\\\url{https://www.math.unipd.it/~tullio/IS-1/2021/Dispense/T14.pdf}
  \item \textbf{Verifica e validazione: analisi statica - slide T15 del corso Ingegneria del Software:}\\\url{https://www.math.unipd.it/~tullio/IS-1/2021/Dispense/T15.pdf}
  \item \textbf{Verifica e validazione: analisi dinamica - slide T16 del corso Ingegneria del Software:}\\\url{https://www.math.unipd.it/~tullio/IS-1/2021/Dispense/T16.pdf}
  \item \textbf{ISO/IEC 9126:}\\\url{http://www.colonese.it/00-Manuali_Pubblicatii/07-ISO-IEC9126_v2.pdf}
  \item \textbf{ISO/IEC 12207:1997:}\\\url{https://www.math.unipd.it/~tullio/IS-1/2009/Approfondimenti/ISO_12207-1995.pdf}
  \item \textbf{Indice di Gulpease:}\\\url{https://it.wikipedia.org/wiki/Indice_Gulpease}
\end{itemize}