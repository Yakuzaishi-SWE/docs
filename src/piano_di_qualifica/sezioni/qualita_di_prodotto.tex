\section{Qualità di prodotto}\label{section:qualita_prodotto}
Per garantire la qualità di prodotto, è stato scelto di utilizzare come riferimento lo standard ISO/IEC 9126\glo{}.
In questa sezione vengono esposti i valori ottimali e accettabili riguardanti le metriche selezionate dal gruppo \groupName{}.
Per consultare nel dettaglio le metriche sotto riportate, fare riferimento al documento \docNameVersionNdP{}.\\

\begin{table}[H]
  \centering
  \renewcommand{\arraystretch}{1.8}
  \rowcolors{2}{green!100!black!40}{green!100!black!30}
  \begin{tabular}{c|p{8cm}|p{2cm}}
    \rowcolor[HTML]{125E28}
    \color[HTML]{FFFFFF}\textbf{Obiettivo} &
    \multicolumn{1}{c}{\color[HTML]{FFFFFF}\textbf{Descrizione}} &
    \multicolumn{1}{c}{\color[HTML]{FFFFFF}\textbf{Metriche}} \\
    \hline
    \rowcolor[HTML]{6BC26B}
    \multicolumn{3}{c}{\textbf{Monitoraggio documentazione}}                                                                                                                                                           \\
    \hline
    \textbf{Leggibilità documenti}   & I documenti devono essere comprensibili agli utenti.                                                                                                  & MPD01                   \\
    \textbf{Correttezza linguistica} & Tutti gli errori grammaticali devono essere corretti.                                                                                                 & MPD02                   \\
    \hline
    \rowcolor[HTML]{6BC26B}
    \multicolumn{3}{c}{\textbf{Monitoraggio software}}                                                                                                                                                                 \\
    \hline
    \textbf{Funzionalità}            & Capacità del prodotto di offrire tutte le funzioni individuate nell'\docNameAdR{}, perseguendo accuratezza e adeguatezza.                             & MPD03 MPD04 MPD05       \\
    \textbf{Usabilità}               & Capacità di essere comprensibile in modo da rendere piacevole l'esperienza dell'utente. Le funzionalità devono essere compatibili con le aspettative. & MPD07                   \\
    \textbf{Portabilità}             & Capacità di poter funzionare in diversi ambienti di esecuzione. Gli obiettivi da perseguire sono adattabilità e sostituibilità.                       & MPD08                   \\
  \end{tabular}
  \caption{Obiettivi di qualità di prodotto}
\end{table}

%%%%%%%%%%%%%%%%%%%%%%%%%%%%%%%%%%%%%%%%%%%%%%%%%%%%%%%%%%%%%%%%%%%%%%%%%%%%%%%%%%%%%%%%%%%

\subsection{Metriche utilizzate}\label{subsection:metriche_prodotto}
\begin{table}[H]
  \centering
  \renewcommand{\arraystretch}{1.8}
  \rowcolors{2}{green!100!black!40}{green!100!black!30}
  \begin{tabular}{c|p{6cm}|c|c}
    \rowcolor[HTML]{125E28}
    \color[HTML]{FFFFFF}\textbf{Codice}
          & \multicolumn{1}{c}{\color[HTML]{FFFFFF}\textbf{Nome Metrica}}
          & \color[HTML]{FFFFFF}\textbf{Valore accettabile}
          & \color[HTML]{FFFFFF}\textbf{Valore ottimale}                    \\
    \hline
    \rowcolor[HTML]{6BC26B}
    \multicolumn{4}{c}{\textbf{Documenti}}                                  \\
    \hline
    MPD01 & Indice di Gulpease                & $\ge 60\%$   & $\ge 80\%$   \\
    MPD02 & Errori ortografici                & 0            & 0            \\
    \hline
    \rowcolor[HTML]{6BC26B}
    \multicolumn{4}{c}{\textbf{Software}}                                   \\
    \hline
    % Funzionalità
    MPD03 & Copertura requisiti obbligatori   & $100\%$      & $100\%$      \\
    MPD04 & Copertura requisiti desiderabili  & $90\%$       & $100\%$      \\
    MPD05 & Copertura requisiti opzionali     & $80\%$       & $100\%$      \\
    % Usabilità
    MPD06 & Facilità di utilizzo              & $5$ click    & $4$ click    \\
    % Portabilità
    MPD07 & Versioni browser supportate       & $\ge 80\%$   & $100\%$      \\
    \end{tabular}
  \caption{Metriche di qualità di prodotto}
\end{table}

