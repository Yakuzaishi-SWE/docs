\section{Qualità di processo}\label{section:qualita_processo}
Per garantire la qualità dei processi, è stato scelto di utilizzare come riferimento lo standard 
\href{https://www.math.unipd.it/~tullio/IS-1/2009/Approfondimenti/ISO_12207-1995.pdf}{ISO/IEC 12207:1997}, 
dal quale sono stati semplificati e adattati alle necessità alcuni dei processi in esso elencati.\\
In questa sezione vengono esposti i valori ottimali e accettabili riguardanti le metriche scelte,
per consultare nel dettaglio le metriche sotto riportate, fare riferimento al documento \docNameVersionNdP{}.

\subsection{Obiettivi di qualità di processo}\label{subsection:obiettivi_processo}
\begin{table}[H]
  \centering
  \renewcommand{\arraystretch}{1.8}
  \rowcolors{2}{green!100!black!40}{green!100!black!30}
  \begin{tabular}{c|p{6cm}|p{4cm}}
    \rowcolor[HTML]{125E28}
    \color[HTML]{FFFFFF}\textbf{Obiettivo}
    & \multicolumn{1}{c}{\color[HTML]{FFFFFF}\textbf{Descrizione}}
    & \multicolumn{1}{c}{\color[HTML]{FFFFFF}\textbf{Metriche}}\\
    \hline
    \rowcolor[HTML]{6BC26B}
    \multicolumn{3}{c}{\textbf{Processi primari}}\\
    \hline
    \textbf{Fornitura} & Processo che consiste nel decidere procedure e risorse adatte a soddisfare le necessità del cliente.  &  \\
    \textbf{Sviluppo} & Processo che ha lo scopo di sviluppare un prodotto software che indirizzi le esigenze del cliente. &  \\
    \hline
    \rowcolor[HTML]{6BC26B}
    \multicolumn{3}{c}{\textbf{Processi di supporto}}\\
    \hline
    \textbf{Verifica} & Processo che ha lo scopo di confermare che ciascun servizio realizzato soddisfi i requisiti specificati. &  \\
    \textbf{Gestione della qualità} & Processo che ha lo scopo di assicurare che il prodotto e i servizi offerti siano conformi agli standard definiti. & \\
    \hline
    \rowcolor[HTML]{6BC26B}
    \multicolumn{3}{c}{\textbf{Processi organizzativi}}\\
    \textbf{Gestione organizzativa} & Processo che si occupa di esporre le modalità di coordinamento del gruppo. &  \\
  \end{tabular}
  \caption{Obiettivi di qualità di processo}
\end{table}

%%%%%%%%%%%%%%%%%%%%%%%%%%%%%%%%%%%%%%%%%%%%%%%%%%%%%%%%%%%%%%%%%%%%%%%%%%%%%%%%%%%%%%%%%%%

\subsection{Metriche utilizzate}\label{subsection:metriche_processo}

\subsubsection{Processi primari}\label{subsubsection:metriche_processi_primari}
\begin{table}[H]
  \centering
  \renewcommand{\arraystretch}{1.8}
  \rowcolors{2}{green!100!black!40}{green!100!black!30}
  \begin{tabular}{c|p{6cm}|c|c}
    \rowcolor[HTML]{125E28}
    \color[HTML]{FFFFFF}\textbf{Codice}
    & \multicolumn{1}{c}{\color[HTML]{FFFFFF}\textbf{Nome metrica}}
    & \color[HTML]{FFFFFF}\textbf{Valore accettabile}
    & \color[HTML]{FFFFFF}\textbf{Valore ottimale}\\
    \hline
    \rowcolor[HTML]{6BC26B}
    \multicolumn{4}{c}{\textbf{Fornitura}}\\
    \hline
    %https://www.pmbypm.com/earned-value-management-formulas/
    %https://shiyamtj.wordpress.com/2018/09/26/requirement-stability-index/
    %Budget At Completion (BAC) = Preventivo iniziale
    MPC01 & Cost Variance (CV) & $\le 10\%$ & $\le 0\%$ \\
    MPC02 & Schedule Variance (SV) & $\le 10\%$ & $\le 0\%$ \\
    MPC03 & Estimated At Completion (EAC) & $\ge BAC - 3\%; \le BAC + 3\%$ & $= BAC$ \\
    MPC04 & Estimate To Complete (ETC) & $\ge 0$ & $\le EAC$ \\
    MPC05 & Earned Value (EV) & $> 0$ & $\le EAC$ \\
    MPC06 & Actual Cost (AC) & $\ge 0$ & $\le EAC$ \\
    MPC07 & Planned Value (PV) & $\ge 0$ & $\le BAC$ \\
    \hline
    \rowcolor[HTML]{6BC26B}
    \multicolumn{4}{c}{\textbf{Sviluppo}}\\
    \hline
    MPC09 & Obligatory Requirements Satisfied & $100\%$ & $100\%$ \\
    MPC10 & Optional Requirements Satisfied & $70\%$ & $100\%$ \\
    MPC11 & Requirements Stability Index (RSI) & $80\%$ & $100\%$ \\ 
  \end{tabular}
  \caption{Metriche di qualità dei processi primari}
\end{table}
%%%%%%%%%%%%%%%%%%%%%%%%%%%%%%%%%%%%%%%%%%%%%%%%%%%%%%%%%%%%%%%%%%%%%%%%%%%%%%%%%%%%%%%%%%%
\subsubsection{Processi di supporto}\label{subsubsection:metriche_processi_supporto}
\begin{table}[H]
  \centering
  \renewcommand{\arraystretch}{1.8}
  \rowcolors{2}{green!100!black!40}{green!100!black!30}
  \begin{tabular}{c|p{6cm}|c|c}
    \rowcolor[HTML]{125E28}
    \color[HTML]{FFFFFF}\textbf{Codice}
    & \multicolumn{1}{c}{\color[HTML]{FFFFFF}\textbf{Nome metrica}}
    & \color[HTML]{FFFFFF}\textbf{Valore accettabile}
    & \color[HTML]{FFFFFF}\textbf{Valore ottimale}\\
    \hline
    \rowcolor[HTML]{6BC26B}
    \multicolumn{4}{c}{\textbf{Verifica}}\\
    \hline
    MPC12 & Passed Tests & $\ge 80\%$ & $100\%$ \\
    MPC13 & Code Coverage & $\ge 70\%$ & $100\%$ \\
    \hline
    \rowcolor[HTML]{6BC26B}
    \multicolumn{4}{c}{\textbf{Gestione della qualità}}\\
    \hline
    MPC14 & Metrics Satisfied & $\ge 85$ & $100\%$\\
  \end{tabular}
  \caption{Metriche di qualità dei processi di supporto}
\end{table}
%%%%%%%%%%%%%%%%%%%%%%%%%%%%%%%%%%%%%%%%%%%%%%%%%%%%%%%%%%%%%%%%%%%%%%%%%%%%%%%%%%%%%%%%%%%
\subsubsection{Processi organizzativi}\label{subsubsection:metriche_processi_organizzativi}
\begin{table}[H]
  \centering
  \renewcommand{\arraystretch}{1.8}
  \rowcolors{2}{green!100!black!40}{green!100!black!30}
  \begin{tabular}{c|p{6cm}|c|c}
    \rowcolor[HTML]{125E28}
    \color[HTML]{FFFFFF}\textbf{Codice}
    & \multicolumn{1}{c}{\color[HTML]{FFFFFF}\textbf{Nome metrica}}
    & \color[HTML]{FFFFFF}\textbf{Valore accettabile}
    & \color[HTML]{FFFFFF}\textbf{Valore ottimale}\\
    \hline
    \rowcolor[HTML]{6BC26B}
    \multicolumn{4}{c}{\textbf{Gestione organizzativa}}\\
    \hline
    MPC15 & Risks Found & $\le 5$ & $0$ \\
  \end{tabular}
  \caption{Metriche di qualità dei processi organizzativi}
\end{table}