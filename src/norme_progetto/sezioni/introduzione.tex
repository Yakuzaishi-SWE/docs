\section{Introduzione}
\subsection {Scopo del documento}
Il seguente documento stabilisce tutte le regole e le procedure fondamentali che ogni membro del gruppo Yakuzaishi deve rispettare al fine di ottenere un' organizzazione uniforme dei file prodotti.
Durante l'intera durata del progetto tutti i membri del gruppo si impegneranno a visionare e a rispettare le norme che vi sono descritte.
Il documento allo stato attuale risulta incompleto in quanto redatto secondo un approccio di tipo incrementale, il quale consiste nell' aggiornare il documento passo a passo al seguito di ogni decisione presa dal gruppo.
In questo modo verranno prima redatte le norme ritenute più urgenti, e solo successivamente quelle necessarie alle attività che si svolgeranno in seguito.
Le norme già presenti possono subire cambiamenti (aggiunte, rimozioni, modifiche), e dovranno essere comunicate dal \textit{Responsabile di Progetto} a ciasun membro del gruppo.
\subsection{Scopo del capitolato}
L'avvento delle tecnologie BlockChain\glo\ ha portato e porterà nei prossimi anni a grandi cambiamenti nella società. 
In particolare, ha aperto le porte a una nuova forma di finanza, la cosiddetta “DeFi” (Finanza Decentralizzata) che ha permesso a chiunque sia dotato di connessione internet di creare un Wallet\glo\ e possedere quindi criptovalute\glo.
Questo ha delineato due profili critici strettamente legati; da un lato il controllo del proprio portafoglio è passato completamente nelle mani dell'utente, dall'altro lato questo comporta la mancanza di un ente terzo che si occupi di gestire transazioni e offrire garanzie.
\newline
Nel capitolato in questione si vuole proprio risolvere questo problema, in uno scenario che comprende un e-commerce\glo\ basato su BlockChain\glo\ in cui si vuole tutelare entrambe le parti coinvolte in un acquisto tramite criptovalute.
\newline
Il fine del progetto è la realizzazione di un prototipo di una piattaforma integrabile con un “crypto-ecommerce”\glo\, che si occupi di gestire gli ordini dalle fasi di pagamento alla consegna.
\subsection{Glossario}
Per evitare ambiguità relativamente a termini specifici propri dell' ambito del capitolato scelto e delle tenologie utilizzate, è stato redatto il Glossario.
In questo documento sono riportati tutti i termini di particolare importanza, essi sono contrassegnati da una 'G' a pedice.
\subsection{Riferimenti}
\subsubsection{Riferimenti normativi}
\begin{itemize}
    \item \textbf{Capitolato d'appalto C2 - ShopChain:} \\
    \url{https://www.math.unipd.it/~tullio/IS-1/2021/Progetto/C2.pdf}
\end{itemize}
\subsubsection{Riferimenti informativi}
\begin {itemize}
    \item Corso tenuto alla laurea Magistrale in informatica dell' Università di Verona riguardante le blockchain\glo: \\
    \url{https://univr.cloud.panopto.eu/Panopto/Pages/Sessions/List.aspx?folderID=1c8bb888-fca4-48bd-85af-acc700e40484}
    
    \item Software Engineering - Ian Sommerville - 10 th Edition:\\
    Parte 4 - Software management:
    \begin{itemize}
        \item Capitolo 25 - Configuration management:
    
    \begin{itemize}
        \item Paragrafo 25.1 - Version management (da pag. 735 a 740);
        \item Paragrafo 25.2 - System building (da pag. 740 a 745).
    \end{itemize}
\end{itemize}
    \item Guida \LaTeX:\\
    \url{http://www.lorenzopantieri.net/LaTeX_files/LaTeXimpaziente.pdf}
\end{itemize}