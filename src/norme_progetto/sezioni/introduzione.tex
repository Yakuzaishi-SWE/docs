\section{Introduzione} \label{section:introduzione}
\subsection{Scopo del documento}
Le \docNameNdP{}, rappresentate da questo documento, sanciscono il way of working del gruppo \groupName{} nello svolgimento del progetto \projectName{}.\\
Vengono qui raccolte regole e procedure di lavoro la cui approvazione è stata decisa dall'intero gruppo, per garantire comprensione e soprattutto assimilazione di quanto scritto.\\
La scelta di adottare standard all'interno delle procedure di lavoro è dettata dalla necessità di perseguire efficienza e al contempo regolare e quantificare tutto ciò che viene svolto e prodotto dai membri del gruppo in maniera organizzata.\\

\subsection{Glossario}
\gloDesc{}

\subsection{Evoluzione del documento}
Le \docNameNdP{} maturano con il susseguirsi di processi, ma in ogni momento rappresenteranno la raccolta di best practices che il gruppo \groupName{} si impegna a seguire.\\
Come ci si aspetta da ogni documento, subirà variazioni e sarà manutenuto costantemente; occorrerà quindi riferirsi al registro delle modifiche per essere certi di visionare la versione più aggiornata.

\subsection{Riferimenti}

\subsubsection{Riferimenti normativi}
\begin{enumerate}
	\item \href{https://github.com/CareMessagePlatform/jasmine-styleguide}{Style guide per il framework di testing Jasmine};
	\item \href{https://github.com/airbnb/javascript}{Style guide airbnb per JavaScript}.
\end{enumerate}

\subsubsection{Riferimenti informativi}
\begin{enumerate}
	\item \textsc{I. Sommerville}, \guillemotleft Requirements engineering\guillemotright, in \textit{Software engineering}, Pearson, 9\textsuperscript{a} edizione;
	\item \guillemotleft Primary life cycle processes, Supporting life cycle processes, Organizational life cycle processes\guillemotright, in \href{https://www.math.unipd.it/~tullio/IS-1/2009/Approfondimenti/ISO_12207-1995.pdf}{AS/NZS ISO/IEC 12207:1997};
	\item \href{https://www.cs.purdue.edu/homes/apm/courses/BITSC461-fall03/miller-guidelines/IEEE830-1998.html}{Standard IEEE 830-1998 commentato}.
\end{enumerate}


