\setcounter{secnumdepth}{5}
\makeatletter
\renewcommand\subparagraph{
\@startsection {subparagraph}{5}{0mm}{-\baselineskip}{.5\baselineskip}{\normalfont \normalsize \bfseries }}
\makeatother

\section{Processi primari}\label{section:processi_primari}
\subsection{Fornitura}

\subsection{Sviluppo}
    \subsubsection{Scopo}
    Scopo del processo di sviluppo è quello di descrivere i compiti e le attività da svolgere relative allo sviluppo del prodotto software.
    In questa sezione vengono dunque descritte le attività, le norme e le convenzioni adottate per la composizione di tale processo.

    \subsubsection{Aspettative}
    Le aspettative nell’applicazione del processo di sviluppo sono:
    \begin{itemize}
        \item Determinare vincoli tecnologici;
        \item Determinare gli obiettivi di sviluppo;
        \item Determinare vincoli di design;
        \item Realizzare un prodotto finale che superi i test e soddisfi requisiti e richieste del proponente.
    \end{itemize}

    \subsubsection{Descrizione}
    Il processo di sviluppo contiene le attività e i compiti dello sviluppatore, tra cui le attività per l’analisi dei requisiti, la progettazione, 
    la codifica, l’integrazione, il test, l’installazione e l’accettazione relative ai prodotti software.

    \subsubsection{Attività}
    Di seguito sono elencate e successivamente trattate le attività caratterizzanti tale processo:
    \begin{itemize}
        \item Analisi dei requisiti;
        \item Progettazione;
        \item Codifica.
    \end{itemize}
    
    \paragraph{Analisi dei requisiti}
        \subparagraph{Scopo}
        Lo scopo dell’analisi dei requisiti è quello di individuare i requisiti diretti e indiretti, impliciti ed espliciti che il proponente richiede 
        per la realizzazione del prodotto e i vari casi d'uso del prodotto stesso.
    
        \subparagraph{Descrizione}
        i requisiti vengono raccolti da diverse fonti quali:
       \begin{itemize}
           \item Lettura investigativa del capitolato;
           \item Confronto interno tra i memmbri del gruppo;
           \item Confronto con il proponente;
           \item Analisi dei casi d'uso.
       \end{itemize}

        \subparagraph{Aspettative}
        L’obiettivo dell’attività di analisi dei requisiti consiste nella creazione della documentazione formale contenente tutti i requisiti 
        richiesti dal proponente, che siano essi stati definiti all'inizio o che siano stati concordati in fase di sviluppo.

        \subparagraph{Classificazione dei requisiti}
        Ciascun requisito viene classificato mediante l'uso di un codice identificativo. Per tale codice è stato definito uno standard dal gruppo che viene spiegato nel seguito:\\\\
        \textbf{Codice Identificativo:}\\
        \begin{center}
            \textbf{\LARGE{R[Importanza][Tipologia][Codice]}}
        \end{center}
        dove:

        \textbf{Importanza}: indica l'importanza associata ad un requisito e può assumere un valore numerico compreso tra 1 e 3 con il seguente significato:
            \begin{itemize}[label={}]
                \item \textbf{1} : Requisito obbligatorio;
                \item \textbf{2} : Requisito desiderabile ma non obbligatorio;
                \item \textbf{3} : Requisito opzionale.
            \end{itemize}

        \textbf{Tipologia}: rappresenta una tipologia di requisito e può assumere uno fra i seguenti valori letterali:
        \begin{itemize}[label={}]
            \item \textbf{V} : Vincolo;
            \item \textbf{P} : Prestazionale;
            \item \textbf{Q} : Qualitativo;
            \item \textbf{F} : Funzionale.
        \end{itemize}

        \textbf{Codice}: è un identificativo univoco del requisito espresso in forma gerarchica padre/figlio.
        \\
        \\
        Una volta associato ad un requisito, l'identificativo è immutabile.
        Ogni requisito deve inoltre essere accompagnato da una serie di informazioni aggiuntive:
        \begin{itemize}
            \item \textbf{Descrizione}: Descrizione sintetica ed esplicativa del requisito in questione;
            \item \textbf{Classificazione}: Al fine di rendere la tabella più leggibile viene rimarcata ancora una volta l'importanza del requisito anche se già presente nel codice identificativo;
            \item \textbf{Fonte}: L'origine del requisito viene riportata in questa sezione.
        \end{itemize}

        %Aggiungere esempio tabella

        \subparagraph{Classificazione dei casi d'uso}



