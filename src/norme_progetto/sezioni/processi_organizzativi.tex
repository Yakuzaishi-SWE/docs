\section{Processi organizzativi}\label{section:Processi organizzativi}
I processi organizzativi sono tutti quelli che riguardano l'organizzazione di \groupName. Essi sono:
\begin{itemize}
\item Processo di gestione;
\item Processo di impiego delle infrastrutture interne;
\item Processo di miglioramento;
\item Processo di formazione.
\end{itemize}

\subsection{Gestione}
Il processo di gestione si occupa della gestione del progetto. Una delle attività chiave di questo processo è quella di definire i ruoli assunti dai membri di \groupName{} nel corso dello svolgimento delle attività riportate nel documento \docNameVersionPdP{} descritto nella sezione \ref{paragraph:PdP} \nameref{paragraph:PdP}.\\
Ciascun ruolo viene ricoperto, a rotazione, da ogni singolo componente del gruppo, in modo da essere svolto per un tempo equo.

\subsubsection{Ruoli di progetto}
\paragraph{Analista}
Si occupa di individuare, analizzare e documentare i servizi che il sistema deve fornire. 
Le sue responsabilità sono:
\begin{itemize}
	\item Analizzare la complessità del problema generale e delle funzionalità da implementare;
	\item Modellare concettualmente il sistema;
	\item Individuare i requisiti del progetto e suddividerli in categorie;
	\item Redigere i documenti \docNameVersionAdR{} e \docNameVersionSdF.
\end{itemize}

\paragraph{Responsabile di progetto}
Il suo compito consiste nel garantire lo svolgimento delle attività pianificate entro i tempi e le modalità previste dal gruppo, nonchè nel garantire un indice di qualità elevato nel prodotto finale. \\
Rappresenta il gruppo \groupName{} nelle comunicazioni esterne con committenti e proponenti. Le sue responsabilità sono:
\begin{itemize}
	\item Gestire l'assegnazione dei compiti agli altri membri del gruppo;
	\item Organizzare il lavoro in modo da minimizzare la probabilità che si verifichino problemi;
	\item Approvare preventivo e consuntivo e inserirli nel documento \docNameVersionPdP;
	\item Approvare la documentazione nella fase finale del processo di verifica;
	\item Redigere l'Organigramma e il \docNameVersionPdP.
\end{itemize}


\paragraph{Programmatore}
Il compito del \roleProgrammer{} consiste nella scrittura del codice richiesto dallo svolgimento del progetto \projectName. Le sue responsabilità sono:
\begin{itemize}
	
	\item Scrivere codice documentato, manutenibile e versionato;
	\item Rendere l'attività di verifica semplice svolgendo il proprio compito con gli strumenti e le modalità previsti da questo documento.
\end{itemize}

\paragraph{Verificatore}
Il \roleVerifier{} è atto alla verifica del progetto. 
Le sue responsabilità sono: 
\begin{itemize}
	\item Controllare che vengano rispettate le norme di progetto, inserite in questo documento \docNameVersionNdP{};
	\item Assicurarsi della conformità di ogni stadio del ciclo di vita del prodotto;
	\item Segnalare eventuali anomalie riscontrate nella verifica in modo che vengano corrette tempestivamente;
	\item Inserire gli esiti documentati delle verifiche svolte all'interno del documento \docNameVersionPdQ.
\end{itemize}

\paragraph{Progettista}
Il \roleDesigner{} si occupa di definire la struttura architetturale del sistema. È responsabile delle attività di progettazione, che devono portare alla realizzazione di un prodotto in grado di soddisfare i requisiti individuati dagli analisti. Le sue responsabilità sono:
\begin{itemize}
	\item Studiare l'architettura più adatta al prodotto da realizzare;
	\item Garantire la qualità del prodotto istanziando un adeguato \docNameVersionPdQ{} in cui redige le sezioni programmatiche del documento;
	\item Identificare l'architettura ad alto livello del sistema, ovvero la \textit{Technology Baseline}, e l'architettura a livello di componenti del sistema, ovvero la  \textit{Product Baseline}.
\end{itemize}

\paragraph{Amministratore di progetto}
L'\roleAdministrator{} è responsabile degli strumenti necessari all'ambiente di lavoro di \groupName.
Le sue responsabilità sono: 
\begin{itemize}
	\item Gestire il sistema di archiviazione e versionamento di documentazione e codice;
	\item Gestire il sistema di configurazione e versionamento del prodotto;
	\item Mantenere efficiente l'ambiente di sviluppo, fornendo strumenti adeguati ai membri del gruppo;
	\item Redigere le \docNameVersionNdP{} che l'intero gruppo deve seguire;
	\item È responsabile della redazione e attuazione di piani e procedure di gestione per la qualità, previste nel \docNameVersionPdQ{};
	\item Collaborare con il \roleProjectManager{} alla redazione del \docNameVersionPdP.
\end{itemize}

\subsubsection{Rotazione dei ruoli}
È prevista una rotazione dei ruoli a cadenza periodica da parte del gruppo \groupName{}. \\
L'attribuzione dei ruoli viene svolta secondi i seguenti criteri:
\begin{itemize}
	\item \textbf{Equità: }ciascun membro del gruppo dovrà svolgere un numero di ore per ogni ruolo equo rispetto agli altri membri;
	\item \textbf{Continuità: }i ruoli sono fatti ruotare in concomitanza all'ingresso di un nuovo periodo, in modo da garantire che il lavoro svolto dai membri di \groupName{} abbia una continuità nel corso di ciascun periodo. I periodi sono descritti nel \docNameVersionPdP{};
	\item \textbf{Assenza di conflitti: }poichè ciascun membro può ricoprire più ruoli in un periodo, si cerca di assegnarli in modo che non vi sia conflitto di interesse fra di essi. Vige in ogni caso la regola che un \roleProjectManager{} non può approvare il proprio lavoro di verifica se assume anche il ruolo di \roleVerifier{}, mentre un \roleVerifier{} non può verificare il proprio codice o i propri testi qualora abbia precedentemente assunto il ruolo di \roleAnalyst{} o \roleProgrammer{} nello stesso periodo.
\end{itemize}
Le tabelle e i grafici relativi all'assegnazione dei ruoli e la loro durata sono presenti nel \docNameVersionPdP{}.

\subsection{Impiego delle infrastrutture interne}
\subsubsection{Gestione delle comunicazioni}
\paragraph{Comunicazioni interne}
Le comunicazioni interne avvengono esclusivamente attraverso l'applicazione Slack. All'interno dell'applicazione è stato creato un canale di comunicazione per ogni attività attualmente in corso, nel quale i componenti del gruppo possono discutere di argomenti ad esso associati. Attualmente i canali utilizzati sono i seguenti:
\begin{itemize}
	\item \textbf{\#generale:} utilizzato per discutere di problematiche che il gruppo BitCraft si trova ad affrontare e per cui non è presente un canale dedicato;
	\item \textbf{\#incontri:} utilizzato per per decidere data, ora e luogo degli incontri;
	\item \textbf{\#latex:} utilizzato per discutere le problematiche relative a \LaTeX{} e template;
	\item \textbf{\#commits:} aggiornato da un bot con i commit caricati tramite push. Non si deve scrivere in questo canale in quanto dedicato solo ai commit;
	\item \textbf{\#issues:} aggiornato da un bot con le issue aggiunte alla issue board di GitLab. Non si deve scrivere in questo canale in quanto dedicato solo alle issue;
	\item \textbf{\#norme\_di\_progetto:} utilizzato per discutere di tutto ciò che riguarda le \docNameNdP;
	\item \textbf{\#piano\_di\_progetto:} utilizzato per discutere di tutto ciò che riguarda il \docNamePdP;
	\item \textbf{\#piano\_di\_qualifica:} utilizzato per discutere di tutto ciò che riguarda il \docNamePdQ;
	\item \textbf{\#analisi\_dei\_requisiti:} utilizzato per discutere di tutto ciò che riguarda l'\docNameAdR;
	\item \textbf{\#verifica:} utilizzato dai verificatori che inoltreranno il commit della verifica fatta su un documento, con un messaggio nella forma seguente:
	\begin{itemize}
		\item{} [tag\_documento][VERIFY] Verificato nome e numero sezione/nome file di codice;
		\item Priorità[Critical/Major/Minor/Low];
		\item @redattore/i o programmatore/i che hanno scritto il documento/codice.
	\end{itemize}
	\item \textbf{\#spam:} utilizzato per comunicazioni che escono dagli ambiti sopra citati.
\end{itemize}
\paragraph{Comunicazioni esterne}
Il \roleProjectManager{} si occupa delle comunicazioni esterne, che avvengono esclusivamente attraverso l'indirizzo di posta elettronica del gruppo \groupName:
\begin{center} 
	\groupEmail
\end{center}
Egli dovrà avvisare il team, tramite i canali di comunicazione interna, delle discussioni avvenute con entità esterne e darne un resoconto dettagliato. 
\subsubsection{Gestione degli incontri}\label{section:incontri}
\paragraph{Incontri Interni}
Il \roleProjectManager{} ha il compito di organizzare gli incontri interni utilizzando l'apposito canale di comunicazione. Egli dovrà accertarsi della partecipazione di almeno 5 membri del gruppo (eccezione fatta per gli incontri con il proponente, in cui potrà essere presente una rappresentanza del gruppo) affinchè si possa stabilire una data per l'incontro. \\Quando si verifica questa condizione si può procedere a fissare l'evento su Google Calendar tramite la mail del gruppo, che lo notificherà al team attraverso la sua integrazione all'interno del canale "\#incontri" su Slack. \\Per ogni incontro viene steso un verbale che raccoglie tutti gli argomenti trattati durante la riuniuone. Il verbale segue il template descritto nella sezione \ref{paragraph:strumenti} \nameref{paragraph:strumenti}. \\Si definiscono alcune sezioni del verbale:
\begin{itemize}
	\item \textbf{Argomenti principali:} vengono descritti i punti focali da discutere nella riunione;
	\item \textbf{Verbale e azioni da intraprendere:} vengono descritte le discussioni svolte, le decisioni prese e le eventuali attività da intraprendere conseguenti a tali decisioni;
\end{itemize}

\paragraph{Incontri Esterni}
Il \roleProjectManager{} ha il compito di organizzare gli incontri esterni con il proponente o con il committente. Un incontro esterno viene richiesto se vi è la necessità impellente di chiarimento di una problematica riscontrata da parte del gruppo. In questo caso il \roleProjectManager{} provvede a formalizzare una richiesta all'entità esterna (proponente o committente) tramite le mail di contatto fornite. \\Se la richiesta verrà accettata, verranno comunicate ai membri del gruppo, tramite canali di comunicazione interni, le informazioni riguardanti data, ora e luogo dell'incontro. Come per gli incontri interni, deve essere scritto un verbale che descriva gli argomenti trattati.

\subsubsection{Gestione degli strumenti di coordinamento}

\paragraph{Ticketing}\label{paragraph:ticketing}

Per la suddivisione delle issue viene utilizzato GitLab issue board. Il \roleProjectManager{} avrà il compito di creare ed assegnare le varie issue ai membri del gruppo. GitLab issue board permette una buona gestione delle issue, infatti:
\begin{itemize}
	\item Le issue possono trovarsi in quattro stati diversi: Open, To Do, Doing, Closed;
	\item Ogni membro del gruppo ha la possibilità di spostare le proprie issue nelle diverse colonne che rappresentano gli stati;
	\item Le issue possono avere diverse etichette, tra cui:
	\begin{itemize}
		\item Etichette per tipologia di documento (NdP, SdF, PdP, AdR, PdQ, Verbale);
		\item Un'etichetta per il template;
		\item Un'etichetta per gli script;
		\item Un'etichetta per l'attività di verifica; 
		\item Un'etichetta per segnalare i bug da risolvere;
		\item Etichette per gli stati To Do e Doing.
	\end{itemize}
	\item Le issue hanno una descrizione e possono ricevere commenti da parte degli altri componenti del gruppo;
	\item All'interno di una issue possono essere presenti delle task sotto forma di checklist nella sezione commenti (per aggiungere un elemento alla checklist è si può utilizzare il tasto apposito o anteporre la stringa "* [ ]" alla task);
	\item Quando viene creata una issue e assegnata ad un membro, quest'ultimo riceve una segnalazione via email.
\end{itemize}
In particolare, una issue deve avere un breve titolo significativo e ulteriori informazioni che devono essere scritte nella descrizione. Se una issue è particolarmente estesa, devono essere create delle task in cui suddividerla.\\
Le issue possono essere chiuse tramite commit. La procedura è indicata nell'apposita sottosezione \ref{paragraph:Strumenti_versionamento} \nameref{paragraph:Strumenti_versionamento}.
\\Le issue si trovano in uno dei 4 stati precedentemente indicati quando sono situate nella corrispondente colonna della board. Ogni colonna ha un suo significato:
\begin{itemize}
	\item \textbf{Open:} sono presenti le issue appena create che non sono ancora state assegnate;
	\item \textbf{To Do:} sono presenti le issue assegnate e da portare a termine entro la data indicata nella issue stessa;
	\item \textbf{Doing:} sono presenti le issue che sono in esecuzione;
	\item \textbf{Closed:} sono presenti le issue portate a termine.
\end{itemize}
Ogni membro del gruppo ha il dovere di tenere aggiornate le sue attività e deve rispettare le scadenze delle varie issue. Se si completa una task bisogna spuntarla come completata all'interno della issue.\\
Inoltre è possibile creare delle milestone a cui le issue possono essere assegnate che permettono di stabilire un punto di riferimento sulla linea temporale entro il quale è previsto il completamento di tali issue. È compito del \roleProjectManager{} definire una milestone a metà e alla fine di ciascun periodo previsto dal \docNameVersionPdP{}.

\subsubsection{Gestione degli strumenti di versionamento} \label{paragraph:Strumenti_versionamento}

\paragraph{Repository}
Per il versionamento dei file è stata scelta la piattaforma GitLab. L'\roleAdministrator{} ha creato il repository e ha aggiunto come collaboratori tutti i membri del gruppo \groupName{}.\\

\paragraph{Tipi di file e .gitignore}
Nelle cartelle che contengono i documenti, sono presenti solo i file .tex e .png. Nel file .gitignore sono state aggiunte tutte le estensioni che devono essere ignorate da un commit, al fine di evitare la presenza di file indesiderati o non rilevanti che non devono essere versionati.

\paragraph{Aggiornamento}
Per l'aggiornamento del repository si applica la seguente procedura:

\begin{itemize}
	\item \textbf{"git status":} verifica quali file sono stati modificati, creati o eliminati e possono essere aggiunti all'area di stage;
	\item \textbf{"git add nome\_file":} vengono aggiunti all'area di stage i file modificati, creati o eliminati che si vogliono versionare in remoto. Per aggiungere all'area di stage tutti i file interessati con un unico comando è possibile usare il comando "git add .";
	\item \textbf{"git commit -m '[label documento][label intervento]breve descrizione delle modifiche'":} le modifiche aggiunte all'area di stage verranno associate ad un id identificativo con un messaggio riassuntivo delle modifiche. \\Ogni commit ha le seguenti caratteristiche:
	\begin{itemize}
		\item Deve registrare modifiche atomiche(riguardanti il minor numero possibile di file);
		\item Se vengono aggiunte sezioni al documento, senza modificarne contenuti già presenti, si deve fare un commit soltanto con l'etichetta del tipo di documento;
		\item Nella descrizione delle modifiche, se ci si riferisce ad una specifica sezione o sottosezione del documento, deve essere indicata con il seguente formato "indice\_sezione - nome\_sezione";
		\item È possibile chiudere le issue inserendo nel messaggio una sottostringa contenente una parola chiave seguita da un '\#' e dal numero della issue da chiudere. Le parole chiave sono: close, closes, closed, fix, fixes, fixed, resolve, resolves, resolved.
	\end{itemize}
	\item \textbf{"git pull":} il repository locale viene aggiornato con le modifiche presenti in remoto fatte dagli altri collaboratori del repository. Può generare conflitti da risolvere con un merge;
	\item \textbf{"git push":} vengono caricati in remoto tutti i commit aggiunti dall'ultimo comando push eseguito.\\
	Quando fare push:
	\begin{itemize}
		\item Per chiudere issue;
		\item Se il proprio lavoro serve ad un altro membro;
		\item A fine giornata;
		\item Se bisogna avvisare il \roleProjectManager{};
		\item Se vengono aggiunte parole del \docNameVersionGlo{}.
	\end{itemize}
\end{itemize}

Altri comandi utili:

\begin{itemize}
	\item \textbf{"git commit --amend -m 'messaggio che sovrascrive il messaggio precedente'":} questo comando può essere utilizzato per correggere il messaggio di un commit appena fatto nel caso di errori di battitura;
\end{itemize}

Le label utilizzate nei commit sono:
\begin{itemize}
	\item Tipo di documento
	\begin{itemize}
		\item \textbf{[PdP]:} \docNamePdP;
		\item \textbf{[PdQ]:} \docNamePdQ;
		\item \textbf{[NdP]:} \docNameNdP;
		\item \textbf{[AdR]:} \docNameAdR;
		\item \textbf{[Glo]:} \docNameGlo;
		\item \textbf{[SdF]:} \docNameSdF;
		\item \textbf{[Script]:} script;
		\item \textbf{[Template]:} template.
	\end{itemize}
	
	\item Tipologia di intervento:
	\begin{itemize}
		\item \textbf{[FIX]:} apportate modifiche da parte di un Redattore su contenuti già scritti dal medesimo;
		\item \textbf{[VERIFY]:} documento posto a verifica con aggiunta di commenti per le modifiche da apportare.
	\end{itemize}
\end{itemize}

\subsection{Trattamento dei rischi}
Il \roleProjectManager{} è responsabile del trattamento dei rischi.
Per far ciò egli deve:
\begin{itemize}
	\item Monitorare i rischi già identificati nella tabella di analisi del rischi del \docNameVersionPdP;
	\item Aggiungere nuovi rischi nella tabella di analisi dei rischi nel \docNameVersionPdP{} qualora venissero identificati.
\end{itemize}

\subsubsection{Struttura del rischio}
Ogni rischio ha le seguenti proprietà che permettono la sua identificazione e categorizzazione in modo rapido:
\begin{itemize}
\item	\textbf{Nome del rischio: }il nome assegnato ad esso;
\item	\textbf{Descrizione: }una descrizione del rischio;
\item	\textbf{Rilevamento: }modalità di rilevamento del problema;
\item	\textbf{Pericolosità: }grado di pericolosità del rischio, può essere:
	\begin{itemize}
		\item Basso;
		\item Medio-basso;
		\item Medio;
		\item Medio-alto;
		\item Alto.
	\end{itemize}
\item	\textbf{Occorrenza: }frequenza di occorrenza del rischio, può essere:
\begin{itemize}
	\item Bassa;
	\item Medio-bassa;
	\item Media;
	\item Medio-alta;
	\item Alta.
\end{itemize}
\end{itemize}
Ciascun rischio viene inserito nella tabella di analisi dei rischi. In tale tabella sono previste per ciascuno di essi fino a 4 righe:
\begin{itemize}
	\item \textbf{Rischio: }riga obbligatoria che ha per colonne le proprietà del rischio;
	\item \textbf{Prevenzione: }riga facoltativa con un'unica colonna in cui si descrivono le strategie impiegate per ridurre la probabilità che il rischio si verifichi;
	\item \textbf{Minimizzazione: }riga facoltativa con un'unica colonna in cui si descrivono le strategie che permettono di ridurre i danni generati dal problema;
	\item \textbf{Piano di contingenza: }riga obbligatoria con un'unica colonna in cui si descrive la strategia di trattamento del rischio qualora esso dovesse verificarsi.
\end{itemize}

\subsubsection{Attualizzazione dei rischi}
Mano a mano che si avanza con lo sviluppo è prevista la redazione e il continuo aggiornamento di una tabella di attualizzazione dei rischi nel \docNameVersionPdP{} con una suddivisione data dai periodi di progetto. In tale tabella si riportano gli episodi in cui i rischi si verificano e il livello di adeguatezza del piano di contingenza nel loro trattamento. Qualora tale piano non dovesse risultare adeguato al trattamento del problema sarà compito del \roleProjectManager{} migliorarlo.
Per monitorare la qualità del trattamento dei rischi il gruppo \groupName{} ha deciso di impiegare le seguenti metriche, che permettono di valutare l'esaustività della tabella di di analisi dei rischi l'efficacia dei piani di contingenza:
\begin{itemize}
	\item \textbf{Numero di Rischi Imprevisti (NRI):}\label{metrics:NRI} Permette di valutare e comprendere l'esaustività della tabella di analisi dei rischi presente nel \docNameVersionPdP{}. Al termine di ogni periodo del progetto, il \roleProjectManager{} stila un bilancio del numero di nuovi rischi che si sono manifestati, non precedentemente previsti dall'analisi dei rischi; \\
	\item \textbf{Successo del Piano di Contingenza (SPC):}\label{metrics:SPC} Permette di rilevare quanto sono stati efficaci i piani di contingenza previsti rispetto ai rischi che si sono verificati all'interno di un periodo. Questa metrica viene calcolata al termine di ogni periodo ed è definita nel seguente modo: 
	$$
	SPC[\%]=\frac{\#PDC\_positivi}{\#PDC\_totali} 
	$$ 
	dove:
	\begin{itemize}
		\item \textbf{\#PDC\_positivi: }indica il numero dei piani di contingenza che hanno avuto successo rispetto ai rischi che si sono presentati durante il periodo;
		\item \textbf{\#PDC\_totali: }indica il numero totale di rischi che si sono presentati durante il periodo.
	\end{itemize}
	Questa metrica viene calcolata manualmente ed è agevolata dalla consultazione dell'appendice Attualizzazione dei Rischi presente all'interno del \docNameVersionPdP{}.
	Nel caso in cui non si verificasse nemmeno un rischio durante il periodo corrente, la misurazione di questa metrica verrà considerata come accettata.
\end{itemize}

\subsection{Miglioramento}
\groupName{} si impegna a utilizzare le metriche derivate dalle analisi riportate nel \docNameVersionPdQ{} e ogni forma di feedback al fine di migliorare i propri processi di sviluppo.

\subsection{Formazione}\label{subsection:formazione}
Lo sviluppo di un prodotto quale \projectName{} richiede a \groupName{} di implementare il processo di formazione in maniera continua. Durante il processo di pianificazione si è pertanto deciso di inserire dei periodi di formazione che permettano di sviluppare competenze il più possibile adeguate al lavoro da svolgere.\\
\groupName{} si impegna a rendere tali competenze comuni fra i membri del gruppo, cercando di rendere la formazione cooperativa tramite l'utilizzo degli appositi canali di Slack e la presentazione di ciò che si è appreso agli altri membri durante le riunioni.