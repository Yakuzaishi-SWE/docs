\section{Processi di supporto}\label{section:processi_supporto}
\subsection{Documentazione}
\subsubsection{Scopo}\label{subsubsection: scopo}
Scopo di questa sezione è quello di normare la documentazione dei vari processi e le attività di sviluppo.
Ci occuperemo quindi di definire le norme per la definizione della struttura che i vari documenti redatti dal gruppo Yakuzaishi devono avere.
\subsection{Gestione della configurazione}\label{subsection:gestione_configurazione}
\subsubsection{Scopo}
\subsubsection{Aspettative}
\subsubsection{Descrizione}
\subsubsection{Versionamento}
\paragraph{Codice di versione}
\paragraph{Sistemi software utilizzati}
\subsubsection{Struttura del repository}
\subsubsection{Metriche}
\subsubsection{Strumenti}

\subsection{Gestione della qualità}\label{subsection: gestione_qualita}
\subsubsection{Scopo}
Lo scopo di questa sezione è la definizione del modo in cui il gruppo si impegna nella gestione della qualità.
\subsubsection{Aspettative}
Le aspettative del team di sviluppo durante questo processo sono le seguenti:
\begin {itemize}
    \item Conseguimento della qualità del prodotto, in linea con le richieste del proponente;
    \item Buona qualità di organizzazione del gruppo;
    \item Prova oggettiva della qualità del prodotto;
\end {itemize}
\subsubsection{Descrizione}
La gestione di qualità è un processo che viene descritto nel \textit{Piano di Qualifica}, esso infatti ha il compito di determinare le metriche e le modalità usate per valutare la qualità di prodotti e processi.
Il team di sviluppo ritiene fondamentale raggiungere una buona qualità del prodotto, e ha ritenuto che il modo migliore per raggiungere questo obiettivo è mediante un approccio di tipo sistematico, ovvero assegnndo a ciascun componente del gruppo un compito ed un ruolo, svolgendo così diversi processi simultaneamente e ottenendo così un ottimizzazione delle risorse, effettuando infine dei test sul prodotto finito. 
\subsubsection{Attività}
Affinchè il prodotto software e la documentazione redatta raggiungano una buona qualità, ogni componente del gruppo Yakuzaishi deve:
\begin{itemize}
    \item Attenersi al \textit{Piano di Qualifica};
    \item Porsi obiettivi incrementali, in modo da ridurre il margine di errore;
    \item Creare un avanzamento continuo della formazione personale, mediante conoscienze pregresse, conoscienze di altri membri del gruppo o mediante autoformazione. 
 \end {itemize}   
\paragraph{Classificazione delle metriche}
...Da discuterne con gli altri...
\subsubsection{Metriche}
Non ci sono metriche particolari.
\subsubsection{Strumenti}
Non ci sono strumenti particolari.

\subsection{Verifica}

\subsection{Validazione}\label{subsection: validazione}
\subsubsection{Scopo}
Scopo di questa sezione è definire come il team di sviluppo ha deciso di affrontare il processo di validazione, ovvero per assicurarsi che il prodotto sviluppato sia in linea con i requisiti concordati con il proponente e che soddisfi i bisogni del committente.
\subsubsection{Aspettative}
Le aspettative del team di sviluppo tramite il seguente processo sono le seguenti: 
\begin{itemize}
    \item Assicurarsi che il prodotto software sia conforme con i requisiti riportati nell' \textit{Analisi dei Requisiti};
    \item  Dimostrare la correttezza delle attività svolte in fase di verifica.
\end{itemize}
\subsubsection{Descrizione}
Tale processo consiste nell' esaminazione del prodotto nella fase di verifica e si assicura che esso sia in linea con i requisiti concordati con i proponente e che soddisfi i bisongni del committente.
Sarà il \textit{Responsabile di Progetto} che avrà la responsabilità di controllare i risultati decidendo se:
\begin{itemize}
    \item Accettare il prodotto;
    \item Rifiutarlo richiedendo una nuova verifica.
\end{itemize}    
\subsubsection{Metriche}
Durante il seguente processo non sono state adottate metriche particolari.
\subsubsection{Strumenti}
Il seguente processo non fa uso di strumenti particolari.


