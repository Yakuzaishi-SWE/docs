\section{Processi di supporto}\label{section:processi_supporto}
\subsection{Documentazione}\label{subsection: documentazione}
\subsubsection{Scopo}\label{subsubsection: scopo}
Scopo di questa sezione è quello di normare la documentazione dei vari processi e le attività di sviluppo.
Ci occuperemo quindi di definire le norme per la definizione della struttura che i vari documenti redatti dal gruppo Yakuzaishi devono avere.
\subsubsection{Aspettative}
Le aspettative del gruppo Yakuzaishi in questo processo sono:
\begin{itemize}
    \item Delineare una valida e chiara struttura dei documenti;
    \item Definire una convenzione che accomuni tutti i tipi di documentazione redatta.
 \end {itemize}
\subsubsection{Descrizione}
Lo scopo della documentazione è quello di trascrivere i fatti accaduti e le decisioni prese dal gruppo durante l' intera durata del progetto.
\subsubsection{Ciclo di vita del documento}
Le fasi del ciclo di vita del documento sono le seguenti:
\begin {itemize}
    \item Il documento viene pensato e vengono organizzate le varie parti;
    \item Viene creata la bozza e la struttura del documento;
    \item Viene redatto il contenuto del documento;
    \item Ogni sezione del documento è soggetta a revisioni correttive e per sistemare le varie parti;
    \item L' approvatore stabilisce che il documento è stato completato ed è pronto per il rilascio.
\end{itemize}
\subsubsection{Template}
Il team di sviluppo ha deciso di realizzare un template mediante l'utilizzo di \LaTeX in modo da avere una struttura uniforme di tutte le pagine nei vari documenti.
Un altro scopo del template è quello di velocizzare la stesura dei documenti stessi, dato che con la struttura già disponibile, i membri del gruppo che si occupano di redigere i documenti dovranno solamente occuparsi della stesura degli stessi, senza badare alla parte grafica.
Il template in particolare definisce la prima pagina, il registro delle modifiche e l'indicizzazione del documento.
\subsubsection{Struttura del documento}
\paragraph{Prima pagina}
La struttura di ogni prima pagina dei vari documenti è la seguente:
\begin {itemize}
    \item \textbf{Logo:} il logo del gruppo \groupName\ è collocato in alto ed è centralizzato;
    \item \textbf{Nome del documento:} sotto il logo e centralizzato vi è il nome del documento in questione;
    \item \textbf{Nome del capitolato:} sotto il nome del documento e centralizzato vi è il nome del capitolato scelto;
    \item \textbf{Informazioni sul documento:} esse comprendono:
        \begin{itemize}
            \item \textbf{Responsabile:} indica il nome del \roleProjectManager\ in quel preciso momento;
            \item \textbf{Redattori:} indica chi si è occupato della redazione del documento;
            \item \textbf{Verificatori: }indica chi sono stati i verificatori in quel preciso documento;
            \item \textbf{Uso:} indica se il documento è destinato a uso interno o esterno;
            \item \textbf{Destinatari:} indica a chi è destinato il documento;
            \item \textbf{Versione:} indica la versione del documento. 
        \end{itemize}
    \item \textbf{Sommario:} spiega brevemente le finalità del documento che si sta per redigere.        
\end {itemize}
\paragraph{Registro delle modifiche}
Il registro delle modifiche consiste nel file \textit{changelog.tex} ed è presente in ogni singolo documento.
Esso corrisponde ad una tabella nella quale si tiene traccia di ogni attività svolta (stesura, modifica, verifica, approvazione) e viene aggiornato ogni volta che avviene una delle attiità precedentemente citate.
La tabella del registro delle modifiche contiene le seguenti voci:
\begin{itemize}
    \item \textbf{Versione:} indica la versione attuale del documento;
    \item \textbf{Data:} indica la data in cui accade un determinato evento;
    \item \textbf{Autore:} indica chi ha apportato dei cambiamenti al documento;
    \item \textbf{Ruolo:} indica il ruolo che l'autore svolgeva in quel preciso momento;
    \item \textbf{Descrizione:} una breve descrizione della modifica effettuata.
\end {itemize}
\paragraph{Indice}
L'indice viene posto in seguito al registro delle modifiche e deve contenere tutte le varie sezioni in cui si compone il documento, compresi grafici e tabelle.
\paragraph{Struttura delle pagine}
Ogni pagina di contenuto è strutturata come segue:
\begin {itemize}
    \item \textbf{Intestazione} in alto composta da:
    \begin {itemize}
        \item \textbf{Nome del gruppo} in alto a sinistra;
        \item \textbf{Logo del gruppo} in alto a destra.
    \end{itemize}
    \item \textbf{Piè di pagina} in basso composto da:
    \begin{itemize}
        \item \textbf{Nome del documento} in basso a sinistra;
        \item \textbf{Numero di pagina} in basso a destra, contenente sia il numero di pagina corrente sia il numero di pagine totali.
    \end {itemize}
    \item \textbf{Contenuto} compreso tra l'intestazione e il piè di pagina.
\end {itemize}
\paragraph{Verbali}
I verbali vengono redatti successivamente alla riunione del gruppo oppure dopo un incontro con il proponente, essi contengono il rapporto dettagliato delle tematiche discusse durante il meeting. 
Essi prevedono una singola stesura, dato che contengono delle decisioni che non vengono modificate successivamente.
La struttura di un verbale è la seguente:
\begin {itemize}
    \item \textbf{Prima pagina:} contenente:
    \begin{itemize}
        \item \textbf{Logo del gruppo:} collocato in alto ed in posizione centrale;
        \item \textbf{Titolo:} collocato sotto al logo del gruppo in posizione centrale, contenente data e specificando se si tratta di un verbale interno o esterno;
        \item \textbf{Informazioni su documento:} contenenti:
        \begin{itemize}
            \item \textbf{Responsabile};
            \item \textbf{Redattori};
            \item \textbf{Verificatori};
            \item \textbf{Uso:} specificando se è interno o esterno;
            \item \textbf{Destinatari}.
        \end{itemize}
        \item \textbf{Sommario:} contenente una breve descrizione del documento.
    \end {itemize}
    \item \textbf{Contenuti:} collocati nella seconda pagina del verbale, si tratta di un indice che raccoglie i vari punti di cui si tratta nel verbale;
    \item \textbf{Pagine di contenuto:} situate dopo l'indice esse contengono:
    \begin {itemize}
        \item \textbf{Informazioni sulla riunione:} contenenti data, ora di inizio, ora di fine, luogo e partecipanti dell'incontro;
        \item \textbf{Ordine del giorno:} contenente i temi trattati durante la riunione.
        \item \textbf{Resoconto:} contenente le decisioni prese;
        \item \textbf{Tracciamento delle decisioni:} contiene le decisioni prese, riportate in forma tabellare.
    \end {itemize}
\end {itemize}
\subsubsection{Convenzioni}
\paragraph{Nomi dei file}
Di seguito viene descritta la rappresentazione dei nomi dei file, i quali sono validi per tutti i documenti:
\begin{itemize}
    \item I nomi dei file iniziano tutti con la lettera minuscola;
    \item Se il nome comprende più parole allora ognuna di esse è separata dal simbolo di underscore;
\end {itemize}
Esempi corretti:
\begin{itemize}
    \item introduzione;
    \item norme\textunderscore di\textunderscore progetto.
\end {itemize}
Esempi non corretti:
\begin{itemize}
    \item Norme\textunderscore di\textunderscore Progetto (sono presenti lettere maiuscole);
    \item NormeDiProgetto (sono presenti lettere maiuscole e sono assenti i caratteri separatori).
\end {itemize}

\paragraph{Stile di testo}
Nella sezione sottostante vengono riportati i vari stili di testo utilizzati nei documenti e i contesti in cui essi sono utilizzati:
\begin {itemize}
    \item \textbf{Grassetto:} lo stile grassetto viene utilizzato per indicare i termini negli elenchi puntati e per i titoli delle sezioni;
    \item \textbf{Corsivo:} lo stile corsivo viene utilizzato per indicare il nome del gruppo, per il nome del proponente e per le parole di particolare rilevanza all' interno dei documenti;
    \item \textbf{Link:} i link rappresentano dei collegamenti esterni al documento, essi verranno rappresentati di colore blu e sottolineati;
    \item \textbf{Nome dei documenti:} essi vengono rappresentati in corsivo con la lettera maiuscola tranne la preposizione, se si fa un riferimento specifico ad un particolare documento bisogna indicarne anche la versione (sempre in corsivo).
    Se il nome del documento è presente nel titolo allora non verrà applicato il corsivo, ma il grassetto.
    \item \textbf{Collegamenti interni:} le parole che si riferiscono ad una parte del documento vanno sottolineate.
\end {itemize}
\paragraph{Glossario}
Le norme relative al \textit{Glossario} sono:
\begin{itemize}
    \item Ogni parola presente nel \textit{Glossario} viene contrassegnata con una 'G' a pedice;
    \item Non vengono riportate nel \textit{Glossario} le parole presenti in un titolo o in una didascalia;
    \item Se un termine compare nella sua stessa definizione all' interno del \textit{Glossario} esso non viene contrassegnato.
\end{itemize}
\paragraph{Elenchi puntati e numerati}
Di seguito viene la descrizione di come il team di sviluppo ha deciso di utilizzare gli elenchi puntati e numerati.
\begin {itemize}
    \item Ogni punto dell' elenco inizia con la lettera maiuscola;
    \item Alla fine di ogni punto vi è un ';';
    \item Dopo l'ultima voce vi è un '.';
    \item Se vi è un concetto da spiegare esso viene scritto in grassetto seguito da ':' e segue la spiegazione di esso.
\end {itemize}
\paragraph{Sigle}
Le sigle presenti nei vari documenti rappresentano i ruoli che ogni membro del gruppo a rotazione deve svolgere:
\begin{itemize}
    \item \textbf{RE:} Responsabile di Progetto;
    \item \textbf{AM:} Amministratore;
    \item \textbf{AN:} Analista;
    \item \textbf{PT:} Progettista;
    \item \textbf{PR:} Programmatore;
    \item \textbf{VE:} Verificatore.
\end {itemize}
... inserire altre sigle usate negli altri documenti...
\paragraph{Data}
Il team di sviluppo ha deciso di adottare la seguente convenzione per la rappresentazione delle date che compaiono nei vari documenti:\\
\textbf{YYYY-MM-DD}\\
dove \textbf{YYYY} indica l'anno, \textbf{MM} indica il mese e \textbf{DD} indica il giorno.

\subsubsection{Elementi grafici}
\paragraph{Tabelle}
Ad eccezione del registro delle modifiche, le tabelle di ogni documento seguono le seguenti convenzioni:
\begin{itemize}
    \item Ogni tabella contiene al di sotto di essa, in posizione centrale, una didascalia descrittiva;
    \item Ogni tabella viene identificata con un numero progressivo a partire da 1, che la identifica univocamente all' interno del documento.
\end{itemize}
\paragraph{Immagini}
Le immagini sono collocate al centro della pagina in questione e contengono un numero e una descrizione esplicativa.
\paragraph{Diagrammi}
Sia i diagrammi UML\glo\ che i diagrammi di Gantt\glo\ vengono riportati come immagini, quindi sono soggetti alle regole sopra riportate.
\subsubsection{Metriche}
... da discuterne con gli altri ...
\subsubsection{Strumenti}
Seguono gli strumenti utilizzati in questa fase. (Da inserire passo a passo)
\begin{itemize}
    \item \LaTeX: linguaggio di markup per la preparazione di testi, basato sul programma di composizione tipografica TEX;\\
         \url{https://www.latex-project.org}
    
\end{itemize}



\subsection{Gestione della configurazione}\label{subsection:gestione_configurazione}
    \subsubsection{Scopo}
    Scopo di questa sezione è definire come il team Yakuzaishi ha deciso di affrontare la tematica della gestione della configurazione, ovvero come il team di sviluppo ha deciso di mantenere tracciata la documentazione redatta.

    \subsubsection{Aspettative}
    Le aspettative del gruppo Yakuzaishi nell'utilizzo di questo processo sono:
    \begin{itemize}
        \item Possibilità di tracciare tutte le modifiche effettuate;
        \item Possibilità di ripristino, qualora fosse necessario, ad una versione precedente;
        \item Possibilità di condivisione dei file configurati tra tutti i membri del gruppo;
        \item Possibilità di individuare e correggere eventuali errori e/o conflitti.
    \end{itemize}

    \subsubsection{Descrizione}
    Il processo di gestione della configurazione ha lo scopo di mantenere organizzata e traccibile la documentazione redatta ed il codice sviluppato, creando una storia per ciascun file prodotto.
    In particolare si vuole gestire la struttura e la disposizione delle varie parti di ogni file all'interno di repository\glo\ facilmente accessibili e navigabili.

    \subsubsection{Versionamento}
        \paragraph{Codice di versione}
        Ogni versione di documento è identificata tramite un codice numerico di tre cifre:
        \begin{center}
            \Large \textbf{[X].[Y].[Z]}
        \end{center}
        il cui significato viene spiegato nel seguito:
        \begin{itemize}
            \item \textbf{X}: versione stabile, sottoposta ad approvazione del responsabile del documento;
            \item \textbf{Y}: versione controllata, sottoposta a revisione da parte del verificatore del documento;
            \item \textbf{Z}: versione modificata dal redattore del documento, seguita da una verifica di quanto scritto.
        \end{itemize}
        Per quanto riguarda il prodotto software, oltre a quanto descritto precedentemente, vengono aggiunti ancora 2 parametri trasformando il codice in un codice numerico di 5 cifre come segue:
        \begin{center}
            \Large \textbf{[X].[Y].[Z]-[A].[B]}
        \end{center}
        in cui il significato dei primi tre numeri rimane inalterato rispetto a quanto spiegato prima, mentre per i due successivamente aggiunti:
        \begin{itemize}
            \item \textbf{A}: indica una versione completa e funzionante del prodotto che supera tutti i test, soddisfa le metriche e implementa i requisiti obbligatori;
            \item \textbf{B}: cresce al raggiungimento degli obiettivi degli incrementi pianificati nel Piano di Progetto.
        \end{itemize}

            \subparagraph{Metriche del codice di versione}
            Le cifre del codice di versione precedentemente descritte seguono una particolare metrica di avanzamento:
            \begin{enumerate}
                \item Tutte le cifre iniziano dal valore 0;
                \item Ciascuna cifra aumenta di un'unità ogni qual volta viene compiuta un'operazione sul documento e in particolare:
                \begin{itemize}
                    \item Se la cifra \textbf{X} viene modificata, le cifre \textbf{Y} e \textbf{Z} ritornano al valore 0;
                    \item Se la cifra \textbf{Y} viene modificata, la cifra \textbf{Z} ritorna al valore 0;
                    \item Le cifre \textbf{A} e \textbf{B} possono solo incrementare il loro valore.
                \end{itemize}
            \end{enumerate}

        \paragraph{Sistemi software utilizzati}
        Utilizziamo un sistema software per gestire le versioni? Se si, quale?\\

        Per gestire i repository\glo\ Git\glo\ si è scelto di utilizzare il servizio offerto da GitHub\glo\ in quanto:
        \begin{itemize}
            \item Gran parte dei membri del gruppo hanno già precedentemente familiarizzato con questo software;
            \item Offre un servizio multipiattaforma dall'utilizzo tramite linea di comando, alla web app, fino ad arrivare alle applicazioni desktop e mobile;
            \item Possibilità di suddividere il tempo in milestone\glo;
            \item Possibilità di organizzare e suddividere il lavoro tramite la creazione di issues\glo\ assegnabili alle milestone\glo.
        \end{itemize} 

    \subsubsection{Struttura del repository}
    Al fine di organizzare meglio il lavoro si è deciso di creare due repository\glo\ distinti, entrambi pubblici, con lo scopo di tenere separati i documenti ed il software:
    \begin{itemize}
        \item \textbf{Yakuzaishi-SWE/docs}: per il versionamento dei documenti;
        \item \textbf{Yakuzaishi-SWE/shopchain}: per il versionamento del codice.
    \end{itemize}

        \paragraph{Yakuzaishi-SWE/docs}
        L'organizzazione della repo è così riassunta:
        \begin{itemize}
            \item \textbf{Branch\glo\ master}: è il branch principale in cui è presente la sola documentazione, in formato tex, pronta alla revisione;
            \item \textbf{Branch\glo\ gh-pages} è un branch nel quale vengono caricati i file compilati in formato pdf della documentazione presente nel master. I file prensenti in questo branch vengono automaticamente aggiunti e resi consultabili in formato pdf alla seguente pagina web:
            \begin{center}
                \url{https://yakuzaishi-swe.github.io/docs/};
            \end{center}
            \item \textbf{Branch\glo\ derivanti dal master}: sono diversi in numero e ciascuno di questi ha un nome parlante riferito al singolo documento. Ognuno è dedicato alla stesura del documento da cui prende il nome e, quando il lavoro su uno di questi branch sarà finito e sottoposto a revisione, il branch in questione verrà unito al master.
        \end{itemize}

        Nel main sono attualmente presenti:
        \begin{enumerate}
            \item una cartella src all'interno della quale sono presenti:
            \begin{itemize}
                \item Una cartella per ciascun documento tra quelli da consegnare;
                \item Una cartella contenente i verbali interni ed esterni;
                \item una cartella per il glossario;
                \item una cartella per il template \LaTeX dei documenti.
            \end{itemize}
            \item un file .gitignore che dichiara esplicitamente l’estensione dei file da non tracciare poiché poco utili allo scopo del repository\glo;
            \item un makefile (da spiegare bene quando sarà completamente sistemato);
            \item un readmefile.md classico della gran parte dei repository\glo\ git contenente una breve descrizione di quello che lo stesso repository\glo\ contiene.
        \end{enumerate}
            \subparagraph{Gestione dei cambiamenti in Yakuzaishi-SWE/docs}
            La separazione del flusso di lavoro tra i vari documenti da redarre permette una notevole diminuzione dei conflitti. Il punto focale è che il branch master rimanga pulito da ogni tipo di errore, per cui non è utilizzabile da nessun membro del gruppo fino a che ciascun responsabile non abbia dato l’approvazione al corrispettivo documento. Solo in quel momento è permesso il merge\glo\ di uno dei branch minori nel master. I cambiamenti da gestire sui documenti possono essere:
            \begin{itemize}
                \item \textbf{Modifiche minori}: riguardano errori grammaticali, lessicali o di sintassi, che possono essere corretti dai redattori o dai verificatori senza l’approvazione del responsabile;
                \item \textbf{Modifiche generali}: riguardano cambiamenti più generali come la struttura del documento o convenzioni da utilizzare e richiedono il consulto con il responsabile, il quale potrà accettare o declinare la proposta di modifica.
            \end{itemize}

        \paragraph{Yakuzaishi-SWE/shopchain}
        Ancora da definire, o se è stata definita non ne sono ancora al corrente...



    \subsubsection{Metriche}
    Il processo di gestione della configurazione non fa uso di metriche qualitative particolari.
    
    \subsubsection{Strumenti}
    Non sono stati identificati degli strumenti particolari per la gestione della configurazione.



\subsection{Gestione della qualità}\label{subsection: gestione_qualita}
\subsubsection{Scopo}
Lo scopo di questa sezione è la definizione del modo in cui il gruppo si impegna nella gestione della qualità.
\subsubsection{Aspettative}
Le aspettative del team di sviluppo durante questo processo sono le seguenti:
\begin {itemize}
    \item Conseguimento della qualità del prodotto, in linea con le richieste del proponente;
    \item Buona qualità di organizzazione del gruppo;
    \item Prova oggettiva della qualità del prodotto;
\end {itemize}
\subsubsection{Descrizione}
La gestione di qualità è un processo che viene descritto nel \textit{Piano di Qualifica}, esso infatti ha il compito di determinare le metriche e le modalità usate per valutare la qualità di prodotti e processi.
Il team di sviluppo ritiene fondamentale raggiungere una buona qualità del prodotto, e ha ritenuto che il modo migliore per raggiungere questo obiettivo è mediante un approccio di tipo sistematico, ovvero assegnndo a ciascun componente del gruppo un compito ed un ruolo, svolgendo così diversi processi simultaneamente e ottenendo così un ottimizzazione delle risorse, effettuando infine dei test sul prodotto finito. 
\subsubsection{Attività}
Affinchè il prodotto software e la documentazione redatta raggiungano una buona qualità, ogni componente del gruppo Yakuzaishi deve:
\begin{itemize}
    \item Attenersi al \textit{Piano di Qualifica};
    \item Porsi obiettivi incrementali, in modo da ridurre il margine di errore;
    \item Creare un avanzamento continuo della formazione personale, mediante conoscienze pregresse, conoscienze di altri membri del gruppo o mediante autoformazione. 
 \end {itemize}   
\paragraph{Classificazione delle metriche}
...Da discuterne con gli altri...
\subsubsection{Metriche}
Non ci sono metriche particolari.
\subsubsection{Strumenti}
Non ci sono strumenti particolari.



\subsection{Verifica}



\subsection{Validazione}\label{subsection: validazione}
\subsubsection{Scopo}
Scopo di questa sezione è definire come il team di sviluppo ha deciso di affrontare il processo di validazione, ovvero per assicurarsi che il prodotto sviluppato sia in linea con i requisiti concordati con il proponente e che soddisfi i bisogni del committente.
\subsubsection{Aspettative}
Le aspettative del team di sviluppo tramite il seguente processo sono le seguenti: 
\begin{itemize}
    \item Assicurarsi che il prodotto software sia conforme con i requisiti riportati nell' \textit{Analisi dei Requisiti};
    \item  Dimostrare la correttezza delle attività svolte in fase di verifica.
\end{itemize}
\subsubsection{Descrizione}
Tale processo consiste nell' esaminazione del prodotto nella fase di verifica e si assicura che esso sia in linea con i requisiti concordati con i proponente e che soddisfi i bisongni del committente.
Sarà il \textit{Responsabile di Progetto} che avrà la responsabilità di controllare i risultati decidendo se:
\begin{itemize}
    \item Accettare il prodotto;
    \item Rifiutarlo richiedendo una nuova verifica.
\end{itemize}    
\subsubsection{Metriche}
Durante il seguente processo non sono state adottate metriche particolari.
\subsubsection{Strumenti}
Il seguente processo non fa uso di strumenti particolari.


