\newglossaryentry{UML}
{
	name={UML},
	description={Sta per Unified Modeling Language, tradotto in italiano come "linguaggio di modellizzazione unificato", è un linguaggio di modellazione e di specifica basato sul paradigma orientato agli oggetti}
}

\newglossaryentry{Proof of concept}
{
	name={Proof of concept},
	description={Prototipo dell'architettura di sistema, utile a dimostrare in una prima fase la fattibilità del sistema e le sue funzionalità}
}

\newglossaryentry{Lista di controllo}
{
	name={Lista di controllo},
	description={Elenco esaustivo di cose da fare o da verificare per eseguire una determinata attività. È spesso utilizzato anche il termine anglosassone 'checklist'. Ricorrere ad una lista di controllo minimizza il possibile insuccesso dovuto, tra gli altri, ai potenziali limiti della memoria e dell'attenzione umana. Il ricorso ad una lista di controllo aiuta a garantire la coerenza e la completezza nello svolgimento di un compito}
}

\newglossaryentry{Bug}
{
	name={Bug},
	description={Errore di funzionamento di un sistema o di un programma}
}

\newglossaryentry{Framework}
{
	name={Framework},
	description={Architettura logica di supporto sulla quale un software può essere progettato e realizzato, spesso facilitandone lo sviluppo da parte del programmatore}
}

\newglossaryentry{Diagrammi di Gant}
{
	name={Diagrammi di Gant},
	description={Il diagramma di Gantt è usato principalmente nelle attività di project management ed è costruito partendo da un asse orizzontale, a rappresentazione dell'arco temporale totale del progetto, suddiviso in fasi incrementali (giorni, settimane, mesi), e da un asse verticale, a rappresentazione delle mansioni o attività che costituiscono il progetto.}
}

\newglossaryentry{Milestone}
{
	name={Milestone},
	description={Letteralmente significa pietra miliare. Indica importanti traguardi intermedi nello svolgimento del progetto. Molto spesso sono rappresentate da eventi}
}
		
\newglossaryentry{Repository}
{
	name={Repository},
	description={Piattaforma software che permette di conservare una significativa quantità di informazioni. Sui dati memorizzati in questi archivi possono essere svolte numerose operazioni di protezione, classificazione, elaborazione dei documenti. In questi sistemi la gestione delle risorse informative è centralizzata e viene realizzata in un ambiente accessibile da più macchine hardware}
}

\newglossaryentry{Git}
{
	name={Git},
	description={Software per il controllo di versione distribuito utilizzabile da interfaccia a riga di comando, creato da Linus Torvalds nel 2005}
}

\newglossaryentry{Issues}
{
	name={Issues},
	description={Lista di punti che possono essere usati per tenere traccia di bug, miglioramenti o altre richieste di progetto}
}

\newglossaryentry{Branch}
{
	name={Branch},
	description={Letteralmente significa ramo. Utilizzati in Git per l'implementazione di funzionalità tra loro isolate, cioè sviluppate in modo indipendente l'una dall'altra ma a partire dalla medesima radice}
}

\newglossaryentry{Merge}
{
	name={Merge},
	description={Funzione avanzata di fusione tra branch in uno nuovo}
}

\newglossaryentry{Github projects board}
{
	name={Github projects board},
	description={Aiutano a organizzare e dare priorità al tuo lavoro. Puoi creare schede di progetto per lavori su funzionalità specifiche, roadmap complete o persino liste di controllo per il rilascio. Con le schede di progetto, hai la flessibilità di creare flussi di lavoro personalizzati che si adattano alle tue esigenze.}
}

\newglossaryentry{Google meet}
{
	name={Google meet},
	description={Applicazione di teleconferenza sviluppata da Google}
}