\newglossaryentry{UML}
{
	name={UML},
	description={Sta per Unified Modeling Language, tradotto in italiano come "linguaggio di modellizzazione unificato", è un linguaggio di modellazione e di specifica basato sul paradigma orientato agli oggetti}
}

\newglossaryentry{Proof of concept}
{
	name={Proof of concept},
	description={Prototipo dell'architettura di sistema, utile a dimostrare in una prima fase la fattibilità del sistema e le sue funzionalità}
}

\newglossaryentry{Lista di controllo}
{
	name={Lista di controllo},
	description={Elenco esaustivo di cose da fare o da verificare per eseguire una determinata attività. È spesso utilizzato anche il termine anglosassone 'checklist'. Ricorrere ad una lista di controllo minimizza il possibile insuccesso dovuto, tra gli altri, ai potenziali limiti della memoria e dell'attenzione umana. Il ricorso ad una lista di controllo aiuta a garantire la coerenza e la completezza nello svolgimento di un compito}
}

\newglossaryentry{Bug}
{
	name={Bug},
	description={Errore di funzionamento di un sistema o di un programma}
}

\newglossaryentry{Framework}
{
	name={Framework},
	description={Architettura logica di supporto sulla quale un software può essere progettato e realizzato, spesso facilitandone lo sviluppo da parte del programmatore}
}

\newglossaryentry{Diagrammi di Gantt}
{
	name={Diagrammi di Gantt},
	description={Il diagramma di Gantt è usato principalmente nelle attività di project management ed è costruito partendo da un asse orizzontale, a rappresentazione dell'arco temporale totale del progetto, suddiviso in fasi incrementali (giorni, settimane, mesi), e da un asse verticale, a rappresentazione delle mansioni o attività che costituiscono il progetto.}
}

\newglossaryentry{Milestone}
{
	name={Milestone},
	description={Letteralmente significa pietra miliare. Indica importanti traguardi intermedi nello svolgimento del progetto. Molto spesso sono rappresentate da eventi}
}
		
\newglossaryentry{Repository}
{
	name={Repository},
	description={Piattaforma software che permette di conservare una significativa quantità di informazioni. Sui dati memorizzati in questi archivi possono essere svolte numerose operazioni di protezione, classificazione, elaborazione dei documenti. In questi sistemi la gestione delle risorse informative è centralizzata e viene realizzata in un ambiente accessibile da più macchine hardware}
}

\newglossaryentry{Git}
{
	name={Git},
	description={Software per il controllo di versione distribuito utilizzabile da interfaccia a riga di comando, creato da Linus Torvalds nel 2005}
}

\newglossaryentry{Issue}
{
	name={Issue},
	description={Lista di punti che possono essere usati per tenere traccia di bug, miglioramenti o altre richieste di progetto}
}

\newglossaryentry{Branch}
{
	name={Branch},
	description={Letteralmente significa ramo. Utilizzati in Git per l'implementazione di funzionalità tra loro isolate, cioè sviluppate in modo indipendente l'una dall'altra ma a partire dalla medesima radice}
}

\newglossaryentry{Merge}
{
	name={Merge},
	description={Funzione avanzata di fusione tra branch in uno nuovo}
}

\newglossaryentry{Github projects board}
{
	name={Github projects board},
	description={Aiutano a organizzare e dare priorità al tuo lavoro. Puoi creare schede di progetto per lavori su funzionalità specifiche, roadmap complete o persino liste di controllo per il rilascio. Con le schede di progetto, hai la flessibilità di creare flussi di lavoro personalizzati che si adattano alle tue esigenze.}
}

\newglossaryentry{Google meet}
{
	name={Google meet},
	description={Applicazione gratuita di teleconferenza sviluppata da Google}
}

\newglossaryentry{IDE}
{
	name={IDE},
	description={sta per Integrated Development Environment, è un ambiente di sviluppo ovvero un software che, in fase di programmazione, supporta i programmatori nello sviluppo e debugging del codice sorgente di un programma segnalando errori di sintassi del codice direttamente in fase di scrittura, oltre a tutta una serie di strumenti e funzionalità di supporto alla fase stessa di sviluppo e debugging.}
}

\newglossaryentry{Open source}
{
	name={Open source},
	description={È un software per computer rilasciato con una licenza in cui il detentore del copyright concede agli utenti i diritti di utilizzare, studiare, modificare e distribuire il software e il suo codice sorgente a chiunque e per qualsiasi scopo}
}

\newglossaryentry{Runtime System}
{
	name={Runtime System},
	description={È un software che fornisce i servizi necessari all'esecuzione di un programma, pur non facendo parte in senso stretto del sistema operativo}
}

\newglossaryentry{Frontend}
{
	name={Frontend},
	description={parte visibile di un sito o di un’applicazione}
}

\newglossaryentry{Interfaccia utente}
{
	name={Interfaccia utente},
	description={anche conosciuta come UI (dall'inglese User Interface), è un'interfaccia uomo-macchina, ovvero ciò che si frappone tra una macchina e un utente, consentendone l'interazione reciproca}
}

\newglossaryentry{Token}
{
	name={Token},
	description={Sono le monete digitali create a supporto di un progetto blockchain\glo\ basate su una valuta digitale preesistente, tipicamente l'Ethereum\glo}
}

\newglossaryentry{SonarCloud}
{
	name={SonarCloud},
	description={È una piattaforma per l’ispezione continua della qualit`a del codice per eseguire revisioni automatiche con analisi statiche del codice per rilevare bug ed errori di codice}
}

\newglossaryentry{Backend}
{
	name={Backend},
	description={Si riferisce a qualsiasi parte di un sito Web o di un programma software che gli utenti non vedono}
}

%\newglossaryentry{TERMINE}
%{
%	name={TERMINE},
%	description={DESCRIZIONE}
%}