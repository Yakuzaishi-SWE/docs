\newglossaryentry{Proof of work}
{
	name={Proof of work},
	description=P{rotocollo di validazione dei blocchi di una blockchain nato nel 2008 con Bitcoin. È un metodo che incentiva i miner\glo\ a competere tra loro nell’elaborazione dei blocchi. La competizione consiste nel risolvere un problema matematico che richiede una grande potenza di calcolo (esempi di problemi possono essere trovare l’input di una funzione di hash partendo da un output; fare una scomposizione in numeri primi).
	}
}

\newglossaryentry{Miner}
{
	name={Miner},
	description={Nodo speciale del network che mette a disposizione i suoi computer per il processo di mining\glo}
}

\newglossaryentry{Mining}
{
	name={Mining},
	description={Parola coniata nell’ambito della verifica delle transazioni Bitcoin}
}

\newglossaryentry{Proof of authority}
{
	name={Proof of authority},
	description={Algoritmo di consenso nato nel 2017 per la validazione dei blocchi di una blockchain che offre prestazioni migliori rispetto all’algoritmo PoW relativamente al numero di transazioni al secondo. Si basa sull’elezione di un numero limitato di nodi validatori scelti arbitrariamente. Viene utilizzato principalmente nelle reti blockchain private (ad esempio di un’azienda)}
}

\newglossaryentry{Proof of stake}
{
	name={Proof of stake},
	description={Algoritmo di consenso introdotto nel 2011 con l’obiettivo di risolvere i problemi relativi al PoW. Si basa su un processo di elezione pseudo-casuale per selezionare un nodo che agirà da validatore del blocco successivo.  I nodi che vogliono partecipare al processo di forging (L’inserimento di nuovi blocchi viene chiamato “forging” (forgiare dall’inglese) invece di mining (come avviene in PoW)) devono congelare una certa somma di monete all’interno della rete (è letteralmente il significato di staking). Il criterio di selezione si basa su una combinazione di fattori che possono includere periodo di staking, randomizzazione e fondi di proprietà del nodo}
}

\newglossaryentry{Stable coin}
{
	name={Stable coin},
	description={Sono asset digitali progettati per simulare il valore delle valute fiat come il dollaro o l'euro. Rispetto a criptovalute come Bitcoin e Ethereum famose per la loro volatilità rispetto alle fiat, le stable coin vedono movimenti di prezzo trascurabili e tracciano strettamente il valore dell'asset o della moneta fiat sottostante che simulano}
}

\newglossaryentry{Cardano}
{
	name={Cardano},
	description={Blockchain proof-of-stake fondata nel 2017 che dà il nome all’omonima moneta che ha come abbreviazione ADA, ereditata dal nome del matematico Ada Lovelace del XIX secolo}
}

\newglossaryentry{Caduceo}
{
	name={Caduceo},
	description={Bastone alato con due serpenti attorcigliati intorno a esso}
}

\newglossaryentry{GMail}
{
	name={GMail},
	description={Servizio gratuito di posta elettronica supportato da pubblicità fornito da Google}
}

\newglossaryentry{GitHub}
{
	name={GitHub},
	description={Servizio di controllo di versione e hosting per progetti software}
}

\newglossaryentry{Telegram}
{
	name={Telegram},
	description={Servizio di messaggistica istantanea e broadcasting basato su cloud ed erogato senza fini di lucro dalla società Telegram LLC}
}

\newglossaryentry{Discord}
{
	name={Discord},
	description={Piattaforma di VoIP, messaggistica istantanea e distribuzione digitale progettata per la comunicazione tra comunità di utenti che comunicano con chiamate vocali, videochiamate, messaggi di testo, media e file in chat private o come membri di un server}
}

\newglossaryentry{Google Drive}
{
	name={Google Drive},
	description={Servizio web, in ambiente cloud computing\glo, di memorizzazione e sincronizzazione online introdotto da Google}
}

\newglossaryentry{Cloud computing}
{
	name={Cloud computing},
	description={Paradigma di erogazione di servizi offerti su richiesta da un fornitore a un cliente finale attraverso la rete internet, a partire da un insieme di risorse preesistenti, configurabili e disponibili in remoto}
}

\newglossaryentry{Latex}
{
	name={Latex},
	description={Linguaggio di marcatura per la preparazione di testi, basato sul programma di composizione tipografica TEX}
}

%\newglossaryentry{TERMINE}
%{
%	name={TERMINE},
%	description={DESCRIZIONE}
%}