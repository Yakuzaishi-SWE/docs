\newglossaryentry{Angular}
{
	name={Angular},
	description={Framework open source\glo\ per lo sviluppo di applicazioni web}
}

\newglossaryentry{BlockChain}
{
	name={BlockChain},
	description={Letteralmente significa “catena di blocchi”, è una struttura dati decentralizzata i cui dati sono raggruppati in blocchi concatenati in ordine cronologico e la cui integrità è garantita dall’uso della crittografia. Il suo contenuto è “immutabile”, una volta scritto tramite un processo normato non è più modificabile né eliminabile, a meno di non invalidare l’intero processo}
}

\newglossaryentry{Criptovaluta}
{
	name={Criptovaluta},
	description={Traduzione in italiano del termine inglese cryptocurrency e si riferisce ad una rappresentazione digitale di valore basata sulla crittografia. Si tratta di una risorsa digitale paritaria e decentralizzata. Al mondo esistono oltre 13000 criptovalute\glo}
}

\newglossaryentry{Cripto e-commerce}
{
	name={Cripto e-commerce},
	description={E-Commerce in cui si paga in criptovalute\glo}
}

\newglossaryentry{E-Commerce}
{
	name={E-Commerce},
	description={Insieme di attività di vendita e acquisto di prodotti effettuato tramite internet}
}

\newglossaryentry{Ethereum}
{
	name={Ethereum},
	description={blockchain\glo\ più famosa dopo quella di Bitcoin\glo, ma a differenza di quest’ultima è programmabile, questo significa che può essere usata per molte cose oltre alla gestione di asset digitali e pagamenti (attualmente sono state implementate molte applicazioni, anche complesse)}
}

\newglossaryentry{Java spring}
{
	name={Java spring},
	description={Framework applicativo open source\glo\ che fornisce supporto infrastrutturale per lo sviluppo di applicazioni Java}
}

\newglossaryentry{QR code}
{
	name={QR code},
	description={Codice a barre bidimensionale, ossia a matrice, composto da moduli neri disposti all'interno di uno schema bianco di forma quadrata, impiegato in genere per memorizzare informazioni destinate a essere lette tramite un apposito lettore ottico o anche smartphon}
}

\newglossaryentry{Smart contract}
{
	name={Smart contract},
	description={Sono essenzialmente dei programmi salvati sulla blockchain\glo\ che convertono i tradizionali contratti nella loro controparte digitale. Hanno quindi il vero e proprio obiettivo di digitalizzare i termini di un accordo in codice che viene eseguito quando i termini del contratto vengono rispettati}
}

\newglossaryentry{Solidity}
{
	name={Solidity},
	description={Linguaggio di programmazione ad alto livello orientato agli oggetti per scrivere smart contract. Viene utilizzato su varie piattaforme blockchain\glo , in particolare Ethereum}
}

\newglossaryentry{Wallet}
{
	name={Wallet},
	description={Letteralmente un portafoglio per le criptovalute\glo , che si traduce fisicamente in un dispositivo, un supporto fisico, un programma o un servizio che memorizza le chiavi pubbliche e/o private per le transazioni di criptovaluta\glo}
}	

\newglossaryentry{PostgreSQL}
{
	name={PostgreSQL},
	description={Database relazionale ad oggetti open source\glo}
}

\newglossaryentry{DBMS}
{
	name={DBMS},
	description={Sta per Data Base Memory System, è un sistema software progettato per consentire la creazione, la manipolazione e l'interrogazione efficiente di database, ospitato su architettura hardware dedicata oppure su semplice computer}
}

\newglossaryentry{Flutter}
{
	name={Flutter},
	description={Framework open source\glo\ di Google per la creazione di applicazioni multipiattaforma compilate in modo nativo da un’unica base di codice}
}

\newglossaryentry{Polygon}
{
	name={Polygon},
	description={Framework che ha lo scopo di creare e connettere blockchain\glo\ compatibili con Ethereum. Prevede quindi delle soluzioni scalabili a supporto di ecosistemi multi-chain (ovvero con più blockchain\glo)}
}

\newglossaryentry{Avalanche}
{
	name={Avalanche},
	description={Blockchain\glo\ che si basa sul layer 1\glo\ nata per le applicazioni decentralizzate e blockchain\glo\ personalizzate. Una delle principali rivali di Ethereum, infatti è molto popolare per i suoi smart contract e la velocità di transazioni al secondo. La sua abbreviazione è AVAX}
}

\newglossaryentry{Layer 1}
{
	name={Layer 1},
	description={Insieme di soluzioni adottate per migliorare il protocollo di validazione di base e renderlo più scalabile. I due soluzioni più utilizzate sono la modifica del tipo di protocollo di consenso (ad esempio, passaggio da PoW a PoS) e lo Sharding (divisione di una singola transazione in parti più piccole, chiamate “Shards” (dall’inglese frammenti), che possono essere processate in parallelo dalla rete)}
}

\newglossaryentry{Bitcoin}
{
	name={Bitcoin},
	description={Criptovaluta\glo. È un sistema di pagamento valutario internazionale creato nel 2009 da un anonimo inventore (o gruppo di inventori), noto con lo pseudonimo di Satoshi Nakamoto, che sviluppò un'idea da lui stesso presentata su Internet a fine 2008 nel famoso documento:“Bitcoin: A Peer-to-Peer Electronic Cash System”. Per convenzione se il termine Bitcoin è utilizzato con l'iniziale maiuscola si riferisce alla tecnologia e alla rete, mentre se minuscola (bitcoin) si riferisce alla valuta in sé}
}

\newglossaryentry{Fantom}
{
	name={Fantom},
	description={Blockchain\glo\ relativamente recente nata nel dicembre 2019 e si basa su un proprio algoritmo di consenso personalizzato. Vanta come punto di forza la velocità relativa alle transazioni, dichiara inferiore ai due secondi dagli sviluppatori stessi}
}

\newglossaryentry{Time series database}
{
	name={Time series database},
	description={Sistema software ottimizzato per l'archiviazione e la gestione di serie temporali attraverso coppie associate di tempo e valore}
}

\newglossaryentry{}
{
	name={},
	description={Sta per Explorator Data Analysis, ovvero la prima fase esplorativa dei dati}
}

\newglossaryentry{Machine learning}
{
	name={Machine learning},
	description={Variante alla programmazione tradizionale nella quale in una macchina si predispone l'abilità di apprendere qualcosa dai dati in maniera autonoma}
}

\newglossaryentry{Intelligenza artificiale}
{
	name={Intelligenza artificiale},
	description={Abilità di una macchina di mostrare capacità umane quali il ragionamento, l'apprendimento, la pianificazione e la creatività}
}

\newglossaryentry{HTML}
{
	name={HTML},
	description={Sta per HyperText Markup Language, linguaggio usato per l'impaginazione di documenti ipertestuali disponibili nel web}
}

\newglossaryentry{CSS}
{
	name={CSS},
	description={Sta per Cascading Style Sheets, linguaggio usato per definire la formattazione di documenti ipertestuali disponibili nel web}
}

\newglossaryentry{JavaScript}
{
	name={JavaScript},
	description={Linguaggio di programmazione orientato agli oggetti e agli eventi, comunemente utilizzato nella programmazione Web per la creazione di effetti dinamici interattivi tramite funzioni di script invocate da eventi innescati a loro volta in vari modi dall'utente sulla pagina web in uso}
}

\newglossaryentry{D3.js}
{
	name={D3.js},
	description={Libreria JavaScript per creare visualizzazioni dinamiche ed interattive partendo da dati organizzati, visibili attraverso un comune browser}
}

\newglossaryentry{CSV}
{
	name={CSV},
	description={Sta per comma-separated values, Formato di file basato su file di testo utilizzato per l'importazione ed esportazione di una tabella di dati}
}

\newglossaryentry{UPS}
{
	name={UPS},
	description={Sta per Uninterruptible Power Supply, definito in italiano come gruppo di continuità, è un'apparecchiatura elettrica utilizzata per ovviare a repentine anomalie nella fornitura di energia elettrica normalmente utilizzata}
}

\newglossaryentry{Modbus}
{
	name={Modbus},
	description={Protocollo di comunicazione seriale per mettere in comunicazione i propri controllori logici programmabili (PLC). È diventato uno standard de facto nella comunicazione di tipo industriale}
}

\newglossaryentry{BLE}
{
	name={BLE},
	description={Sta per Bluetooth Low Energy, Tecnologia wireless personal area network. Rispetto al Bluetooth "classico", il Bluetooth Low Energy ha lo scopo di fornire un consumo energetico e un costo notevolmente ridotto, mantenendo un intervallo di comunicazione simile}
}

%\newglossaryentry{TERMINE}
%{
%	name={TERMINE},
%	description={DESCRIZIONE}
%}