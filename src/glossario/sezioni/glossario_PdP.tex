\newglossaryentry{RTB}
{
	name={RTB},
	description={Sta per Requirement and Technology Baseline. Fissa i requisiti che il fornitore si impegna a soddisfare, in accordo con il proponente, motivando le tecnologie, i framework, e le librerie selezionate per la realizzazione del prodotto dimostrandone adeguatezza e fattibilità, tramite un Proof of Concept}
}

\newglossaryentry{PB}
{
	name={PB},
	description={Sta per Product Baseline. Illustra la baseline architetturale (design e codice) del prodotto in maniera definitiva}
}

\newglossaryentry{CA}
{
	name={CA},
	description={Sta per Customer Acceptance. Revisione finale, si svolge alla presenza del proponente, incentrata su collaudo dimostrativo del prodotto realizzato}
}

\newglossaryentry{Visual Studio Code}
{
	name={Visual Studio Code},
	description={IDE\glo{} libero e gratuito sviluppato da Microsoft per Windows, Linux e macOS}
}

\newglossaryentry{Linguaggio di markup }
{
	name={Linguaggio di markup},
	description={Insieme di regole che descrivono i meccanismi di rappresentazione (strutturali, semantici, presentazionali) o layout di un testo}
}

\newglossaryentry{Continous integration}
{
	name={Continous integration},
	description={Pratica che si applica in contesti in cui lo sviluppo del software avviene attraverso un sistema di controllo versione. Consiste nell'allineamento frequente dagli ambienti di lavoro degli sviluppatori verso l'ambiente condiviso}
}

\newglossaryentry{Scrum}
{
	name={Scrum},
	description={Framework agile per la gestione del ciclo di sviluppo del software, iterativo ed incrementale, concepito per gestire progetti e prodotti software o applicazioni di sviluppo}
}

\newglossaryentry{Sprint}
{
	name={Sprint},
	description={Intervallo di tempo fisso ripetibile durante il quale viene creato un prodotto "Fatto" del valore più alto possibile}
}

\newglossaryentry{FTM}
{
	name={FTM},
	description={Simbolo utilizzato per la criptovaluta Fantom}
}

% \newglossaryentry{TERMINE}
% {
% 	name={TERMINE},
% 	description={DESCRIZIONE}
% }