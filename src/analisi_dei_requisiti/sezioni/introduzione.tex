\section{Introduzione} \label{section:introduzione}

\subsection{Scopo del Documento}
Il presente documento ha lo scopo di raccogliere i casi d'uso ed i requisiti individuati dal gruppo
durante la fase di analisi, riguardo il capitolato d'appalto proposto dall'azienda Sync Lab (C2).
La formulazione segue le regole esposte nel documento \docNameVersionNdP{}.

\subsection{Scopo del Prodotto}
L'obiettivo richiesto dall'azienda proponente è la realizzazione di una web app\glo{} che permetta la gestione dei pagamenti per una piattaforma e-commerce\glo{}, tramite l'uso di smart contracts\glo{} basati sulla blockchain\glo{} Fantom\glo{}.
La decisione di sviluppare l'applicazione per la blockchain\glo{} Fantom\glo{} è stata presa dopo una attenta analisi delle possibili blockchain\glo{} adatte alla costruzione di smart contract\glo{}.
L'applicazione sarà inoltre dotata di una funzione MoneyBox\glo{} per gestire pagamenti da più utenti per uno stesso prodotto.

\subsection{Glossario}
Nel documento possono essere presenti termini non ampiamente conosciuti o dal significato ambiguo.
E' perciò stato creato un glossario contente tali termini con il loro significato specifico.
Un termine è presente all'interno del glossario se seguito da una G corsiva in pedice.

\subsection{Riferimenti}

\paragraph{Normativi}
\begin{itemize}
    \item \textbf{\docNameVersionNdP{}};
    \item \textbf{Regolamento del progetto didattico}:
          \begin{center}
              \href{https://www.math.unipd.it/~tullio/IS-1/2021/Dispense/PD2.pdf}{https://www.math.unipd.it/~tullio/IS-1/2021/Dispense/PD2.pdf}
          \end{center}
    \item \textbf{Documentazione Capitolato C2 - ShopChain}:
          \begin{center}
              \href{https://www.math.unipd.it/~tullio/IS-1/2021/Progetto/C2.pdf}{https://www.math.unipd.it/~tullio/IS-1/2021/Progetto/C2.pdf}
          \end{center}
\end{itemize}

\paragraph{Informativi}
\begin{itemize}
    \item \textbf{\docNameVersionPdP{}};
    \item \textbf{Analisi dei requisiti - slide T7 del corso di Ingegneria del Software}:
          \begin{center}
              \href{https://www.math.unipd.it/~tullio/IS-1/2021/Dispense/T07.pdf}{https://www.math.unipd.it/~tullio/IS-1/2021/Dispense/T07.pdf}
          \end{center}
    \item \textbf{Analisi e descrizione delle funzionalità: diagrammi dei casi d'uso - slide P4 del corso di Ingegneria del Software}:
          \begin{center}
              \href{https://www.math.unipd.it/~rcardin/swea/2022/Diagrammi%20Use%20Case.pdf}{https://www.math.unipd.it/~rcardin/swea/2022/Diagrammi\%20Use\%20Case.pdf}
          \end{center}
    \item \textbf{Corso su Blockchain di UNIVR fornito dall'azienda}:
          \begin{center}
              \href{https://univr.cloud.panopto.eu/Panopto/Pages/Sessions/List.aspx?folderID=1c8bb888-fca4-48bd-85af-acc700e40484}{https://univr.cloud.panopto.eu/Panopto/Pages/Sessions/List.aspx?folderID=1c8bb888-fca4-48bd-85af-acc700e40484}
          \end{center}
    \item \textbf{Documentazione del linguaggio Solidity\glo{}}:
          \begin{center}
              \href{https://docs.soliditylang.org/en/v0.7.4/introduction-to-smart-contracts.html}{https://docs.soliditylang.org/en/v0.7.4/introduction-to-smart-contracts.html}
          \end{center}
\end{itemize}

\clearpage


