\section{Introduzione}

\subsection{Scopo del Documento}
Il presente documento ha lo scopo di raccogliere i casi d'uso ed i requisiti individuati dal gruppo
durante la fase di analisi, riguardo il capitolato d'appalto proposto dall'azienda Sync Lab (C2). La formulazione di questi segue le regole esposte nel documento
NormeDiProgetto.

\subsection{Scopo del Prodotto}
L'obiettivo richiesto dalla azienda proponente è la realizzazione di una webApp\glo\ che permetta la gestione dei pagamenti per una piattaforma E-Commerce, tramite l'uso di Smart Contracts\glo\ basati sulla Blockchain\glo\ Fantom\glo.
La decisione di sviluppare l'applicazione per la blockchain Fantom è stata presa dopo una attenta analisi delle possibili blockchain adatte alla costruzione di Smart Contract. La suddetta analisi è consultabile nell'appendice \ref{appendix:studio}
L'applicazione sarà inoltre dotata di una funzione MoneyBox\glo\ per gestire pagamenti da più utenti per uno stesso prodotto.

\subsection{Glossario}
Nel documento possono essere presenti termini non ampiamente conosciuti o dal significato ambiguo. E' perciò stato creato
un Glossario contente tali termini con il loro significato specifico. Un termine è presente
all’interno del Glossario se seguito da una G corsiva in pedice.

\subsection{Riferimenti}

\paragraph{Normativi}
\begin{itemize}
    \item Norme di Progetto
    \item Piano di Progetto
    \item Documentazione Capitolato C2
\end{itemize}

\paragraph{Informativi}
\begin{itemize}
    \item Slide del corso di Ingegneria del software A.A.21/22 - Analisi dei requisiti\footnote{\href{}{link alle slide}}.
    \item Slide del corso di Ingegneria del software A.A.21/22 - Diagrammi delle classi\footnote{\href{}{link alle slide}}.
    \item Corso su Blockchain di UNIVR fornito dall'azienda\footnote{\href{https://univr.cloud.panopto.eu/Panopto/Pages/Sessions/List.aspx?folderID=1c8bb888-fca4-48bd-85af-acc700e40484 }{Corso Blockchain}}.
    \item Documentazione del linguaggio Solidity\glo \footnote{\href{https://docs.soliditylang.org/en/v0.7.4/introduction-to-smart-contracts.html}{Documentazione Solidity}}.
\end{itemize}

\clearpage


