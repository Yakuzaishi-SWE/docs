\section{Casi d'uso}

\subsection{Scopo}
Lo scopo di questa sezione è la descrizione in elenco di tutti i casi d’uso individuati dal gruppo, in
riferimento alle funzionalità dell’applicazione.

\subsection{Attori}
Come accordato con il proponente, la webapp deve essere usata sia dal venditore che dal cliente,
sono quindi presenti tre attori nella gerarchia: la piattaforma E-Commerce da cui parte l'ordine, Metamask\glo, l'utente generico.
L'utente generico ha poi due specializzazioni possibili: l'utente propietario dell'ordine e l'utente venditore.

\subsection{UC1 - Inizializzazione delle Transazione}

\begin{itemize}
    \item Attore primario: E-Commerce.
    \item Attore secondario: utente propietario dell'ordine.
    \item Precondizioni: il sistema è raggiungibile e funzionante.
    \item Postcondizioni: la transazione viene caricata dnel sistema, l'utente viene reindirizzato alla pagina di checkout.
    \item Scenario principale: L'E-Commerce delega il pagamento comunicando la somma richiesta e l'indirizzo del destinatario.
\end{itemize}

\subsection{UC2 - Tipo Pagamento}

\begin{itemize}
    \item Attore primario: Utente propietario dell'ordine.
    \item Precondizioni: l'E-Commerce ha iniziato la transazione [UC1].
    \item Postcondizioni: viene visualizzato il riepilogo dell'ordine e scelta la modalità di pagamento.
    \item Scenario principale: l'utente seleziona la tiopologia del pagamento tra quelle disponibili.
    \item Generalizzazioni: l'utente sceglie una delle seguenti opzioni\begin{itemize}
        \item pagamento unico [UC2.1].
        \item pagamento moneybox [UC2.2].
    \end{itemize}
\end{itemize}

\paragraph{UC2.1 - Selezionato Pagamento Unico}

\begin{itemize}
    \item Attore primario: propietario dell'ordine.
    \item Precondizioni: l'E-Commerce ha iniziato la transazione [UC1].
    \item Postcondizioni: l'utente ha selezionato pagamento unico come metodo di pagamento e ha pagato con lo stesso.
    \item Scenario principale: \begin{enumerate}
        \item l'utente seleziona pagamento unico come metodo di pagamento.
        \item l'utente effettua il pagamento [UC5].
    \end{enumerate}
\end{itemize}

\paragraph{UC2.2 -  Selezionato Pagamento MoneyBox}

\begin{itemize}
    \item Attore primario: utente generico, utente propietario dell'ordine.
    \item Precondizioni: l'E-Commerce ha iniziato la transazione [UC1].
    \item Postcondizioni: l'utente ha selezionato pagamento MoneyBox come metodo di pagamento e ha pagato con lo stesso.
    \item Scenario principale:\begin{enumerate}
        \item l'utente seleziona pagamento MoneyBox come metodo di pagamento.
        \item l'utente visualizza lo stato di completamento [UC2.2.1].
        \item l'utente visualizza l'invito alla partecipazione della MoneyBox [UC2.2.2].
        \item l'utente, opzionalmente, partecipa alla MoneyBox [UC2.2.3].
        \item l'utente propietario dell'ordine, opzionalmente, può chiudere una MoneyBox [UC2.2.4].
    \end{enumerate}
\end{itemize}

\paragraph{UC2.2.1 - Visualizzazione Stato Completamento}

\begin{itemize}
    \item Attore primario: utente generico.
    \item Precondizioni: l'E-Commerce ha iniziato la transazione [UC1], l'utente dispone dell'invito valido ad una MoneyBox.
    \item Postcondizioni: l'utente ha visualizzato lo stato del completamento.
    \item Scenario principale: \begin{enumerate}
        \item l'utente visualizza lo stato del completamento in formato percentuale e il saldo mancante.
        \item l'utente visualizza una lista delle transazioni avvenute [UC2.2.1.1].
    \end{enumerate}
    \end{itemize}

\paragraph{UC2.2.2 - Visualizzazione Invito Partecipazione MoneyBox}

\begin{itemize}
    \item Attore primario: utente generico.
    \item Precondizioni: l'E-Commerce ha iniziato la transazione [UC1], l'utente dispone dell'invito valido ad una MoneyBox.
    \item Postcondizioni: l'utente ha visualizzato l'invito alla MoneyBox.
    \item Scenario principale: l'utente visualizza l'invito alla MoneyBox.
\end{itemize}

\paragraph{UC2.2.3 - Scelta Quota Partecipazione}

\begin{itemize}
    \item Attore primario: utente generico.
    \item Precondizioni: l'E-Commerce ha iniziato la transazione [UC1], l'utente dispone dell'invito valido ad una MoneyBox.
    \item Postcondizioni: l'utente ha selezionato la quota da versare e ha effettuato il pagamento.
    \item Scenario principale:\begin{enumerate}
        \item l'utente seleziona la quota da versare, compresa tra zero escluso e il minimo tra il saldo disponibile nel wallet e il rimanente della MoneyBox.
        \item l'utente effettua il pagamento [UC5].
    \end{enumerate}
\end{itemize}

\paragraph{UC2.2.4 - Chiusura MoneyBox}

\begin{itemize}
    \item Attore primario: utente propietario.
    \item Precondizioni: l'utente ha connesso il proprio Metamask [UC4].
    \item Postcondizioni: l'utente ha chiuso la MoneyBox e i fondi vengono restituiti.
    \item Scenario principale: \begin{enumerate}
        \item l'utente chiude la MoneyBox.
        \item viene annullata ogni transazione e quindi effettuati i rimborsi.
    \end{enumerate}
\end{itemize}

\subsection{UC3 - Visualizzazione Riepilogo Ordine}

\begin{itemize}
    \item Attore primario: utente generico.
    \item Precondizioni: l'E-Commerce ha iniziato la transazione [UC1].
    \item Postcondizioni: l'utente ha visualizzato il riepilogo dell'ordine.
    \item Scenario principale: \begin{enumerate}
        \item l'utente visualizza il totale dell'ordine.
        \item l'utente visualizza un elenco degli item ordinati.
    \end{enumerate}
\end{itemize}

\subsection{UC4 - Connessione Metamask}

\begin{itemize}
    \item Attore primario: utente generico.
    \item Precondizioni: il sistema è raggiungibile e funzionante.
    \item Postcondizioni: l'utente ha connesso Metamask a ShopChain.
    \item Scenario principale: \begin{enumerate}
        \item l'utente visualizza il pop-up di Metamask per la connessione a ShopChain.
        \item l'utente autorizza la connessione a ShopChain.
    \end{enumerate}
    \item Estensioni: \begin{enumerate}
        \item Nel caso in cui l'estensione Metamask risultasse assente nel browser: \begin{itemize}
            \item la connessione non ha successo.
            \item viene visualizzato errore estensione mancante [UC4.1].
        \end{itemize}
        \item Nel caso in cui l'utente abbia connesso il proprio wallet ma non abbia configurato la stessa blockchain di ShopChain: \begin{itemize}
            \item la connessione non viene completata.
            \item viene visualizzato errore blockchain non registrata [UC4.2].
        \end{itemize}
        \item Nel caso in cui l'utente abbia configurato la stessa blockchain di ShopChain ma non selezionato un account compatibile: \begin{itemize}
            \item la connessione non viene completata.
            \item viene visualizzato errore account non compatibile [UC4.3].
        \end{itemize}
        \item Nel caso l'utente rifiuti la connessione, visualizza errore connessione non completata [UC4.4].
    \end{enumerate}
\end{itemize}

\paragraph{UC4.1 - Visualizzazione Errore Estensione Mancante}

\begin{itemize}
    \item Attore primario: utente generico,
    \item Precondizioni: l'utente utilizza un browser sprovvisto di estensione Metamask.
    \item Postcondizioni: l'utente ha visualizzato l'errore e la connessione fallisce.
    \item Scenario principale: \begin{enumerate}
        \item l'utente visualizza un messaggio di errore per mancata connessione.
        \item l'utente viene invitato a scaricare l'estensione Metamask.
        \item l’utente clicca ”OK” per continuare.
    \end{enumerate}
\end{itemize}

\paragraph{UC4.2 - Visualizzazione Errore Blockchain non registrata}

\begin{itemize}
    \item Attore primario: utente generico,
    \item Precondizioni:l'utente ha connesso il proprio wallet ma non ha configurato la stessa blockchain di ShopChain.
    \item Postcondizioni: l'utente ha visualizzato l'errore e la connessione fallisce.
    \item Scenario principale: \begin{enumerate}
        \item l'utente visualizza un messaggio di errore per mancata connessione.
        \item l'utente visualizza i dati per configurare la blockchain.
        \item l’utente clicca ”OK” per continuare.
    \end{enumerate}
\end{itemize}

\paragraph{UC4.3 - Visualizzazione Errore account non compatibile}

\begin{itemize}
    \item Attore primario: utente generico,
    \item Precondizioni: l'utente ha configurato la stessa blockchain di ShopChain ma non ha selezionato un account compatibile.
    \item Postcondizioni: l'utente ha visualizzato l'errore e la connessione fallisce.
    \item Scenario principale: \begin{enumerate}
        \item l'utente visualizza un messaggio di errore per mancata connessione.
        \item l'utente viene invitato a cambiare account o a registrare un account compatibile.
        \item l’utente clicca ”OK” per continuare.
    \end{enumerate}
\end{itemize}

\paragraph{UC4.4 - Visualizzazione Errore connessione non completata}

\begin{itemize}
    \item Attore primario: utente generico,
    \item Precondizioni: l'utente rifiuta la connessione.
    \item Postcondizioni: l'utente ha visualizzato l'errore e la connessione fallisce.
    \item Scenario principale: \begin{enumerate}
        \item l'utente visualizza un messaggio di errore per mancata connessione.
        \item l'utente viene invitato a ritentare.
        \item l’utente clicca ”OK” per continuare.
    \end{enumerate}
\end{itemize}

\subsection{UC5 - Pagamento}

\begin{itemize}
    \item Attore primario: utente generico.
    \item Attore secondario: Metamask.
    \item Precondizioni: l'utente ha scelto il Tipo di Pagamento [UC2], l'utente ha connesso il proprio wallet tramite Metamask [UC4].
    \item Postcondizioni: l'utente ha effettuato il pagamento.
    \item Scenario principale: \begin{enumerate}
        \item l'utente visualizza il pop-up di Metamask con i dettagli della transazione, solo in caso di saldo sufficente.
        \item l'utente autorizza la transazione.
        \item l'utente visualizza un messaggio di conferma di avvenuto pagamento.
    \end{enumerate}
    \item Estensioni: \begin{enumerate}
        \item Nel caso l'utente rifiuti il pagamento: \begin{itemize}
            \item la transazione viene annullata.
            \item l'utente visualizza errore transazione annullata [UC5.1].
        \end{itemize}
        \item Nel caso venisse raggiunta la somma totale dovuta, viene generato e mostrato il codice di sblocco [UC5.2].
    \end{enumerate}
\end{itemize}

\paragraph{UC5.1 - Visualizzazione errore transazione annullata}

\begin{itemize}
    \item Attore primario: utente generico.
    \item Precondizioni: l'utente ha rifiutato la transazione.
    \item Postcondizioni: l'utente ha visualizzato l'errore e la transazione fallisce.
    \item Scenario principale: \begin{enumerate}
        \item l'utente visualizza un messaggio di errore per rifiuto della transazione.
        \item l’utente clicca ”OK” per continuare.
    \end{enumerate}
\end{itemize}

\paragraph{UC5.2 - Generazione Codice di Sblocco}

\begin{itemize}
    \item Attore primario: utente generico.
    \item Precondizioni: il versamento raggiunge la somma totale dovuta.
    \item Postcondizioni: l'utente ha visualizzato il codice di sblocco.
    \item Scenario principale: \begin{enumerate}
        \item viene generato il codice di sblocco.
        \item l'utente visualizza il codice di sblocco.
    \end{enumerate}
\end{itemize}

\subsection{UC6 - Sblocco Ordine}

\begin{itemize}
    \item Attore primario: utente propietario dell'ordine.
    \item Attore secondario: utente venditore.
    \item Precondizioni: il sistema ha generato il codice di sblocco [UC5.2]. 
    \item Postcondizioni: l'utente propietario ha sbloccato l'ordine e il denaro è stato trasferito all'utente venditore.
    \item Scenario principale: \begin{enumerate}
        \item l'utente inserisce il codice di sblocco.
        \item il sistema trasferisce il denaro sul wallet dell'utente venditore.
    \end{enumerate}
    \item Estensioni: \begin{itemize}
        \item ? nel caso di mancato inserimento del codice di sblocco, entro i termini prestabiliti, viene visualizzato un messaggio di errore [UC6.1].
        \item rifiuta? %[UC6.2]
        \end{itemize}
\end{itemize}

\subsection{UC7 - Visualizzazione Transazioni}
%baseline 2 incrementi, poc 3, requisiti obbl 2, opzionali 1, +1 bonus

\begin{itemize}
    \item Attore primario: utente venditore.
    \item Precondizioni: il sistema è raggiungibile e funzionante.
    \item Postcondizioni: l'utente ha selezionato lo stato delle transazioni che vuole visualizzare.
    \item Scenario principale: l'utente sceglie lo stato delle transazioni da visualizzare tra quelli disponibili.
    \item Generalizzazioni: l'utente sceglie una delle seguenti opzioni\begin{itemize}
        \item visualizza transazioni non completate [UC7.1].
        \item visualizza transazioni pagate [UC7.2].
        \item visualizza transazioni pagate e sbloccate [UC7.3].
    \end{itemize}
\end{itemize}

\paragraph{UC7.1 - Visualizza Transazioni Non Completate}

\begin{itemize}
    \item Attore primario: utente venditore.
    \item Precondizioni: l'utente ha selezionato visualizza transazioni non completate.
    \item Postcondizioni: l'utente ha visualizzato la lista di transazioni non completate.
    \item Scenario principale: l'utente visualizza la lista di transazioni non completate.
\end{itemize}

\paragraph{UC7.2 - Visualizza Transazioni Pagate}

\begin{itemize}
    \item Attore primario: utente venditore.
    \item Precondizioni: l'utente ha selezionato visualizza transazioni pagate.
    \item Postcondizioni: l'utente ha visualizzato la lista di transazioni pagate.
    \item Scenario principale: l'utente visualizza la lista di transazioni pagate.
\end{itemize}

\paragraph{UC7.3 - Visualizza Transazioni Pagate e Sbloccate}

\begin{itemize}
    \item Attore primario: utente venditore.
    \item Precondizioni: l'utente ha selezionato visualizza transazioni pagate e sbloccate.
    \item Postcondizioni: l'utente ha visualizzato la lista di transazioni pagate e sbloccate.
    \item Scenario principale: l'utente visualizza la lista di transazioni pagate e sbloccate.
\end{itemize}