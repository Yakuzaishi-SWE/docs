\section{Casi d'uso}

\subsection{Scopo}
Lo scopo di questa sezione è la descrizione in elenco di tutti i casi d’uso individuati dal gruppo, in
riferimento alle funzionalità dell’applicazione.

\subsection{Attori}
Come accordato con il proponente, la webapp deve essere usata sia dal venditore che dal cliente,
sono quindi presenti quattro attori nella gerarchia: la piattaforma E-Commerce da cui parte l'ordine, Metamask, l'utente seller, l'utente customer.

\subsection{UC1 - lorem ipsum}

\begin{itemize}
    \item Attore primario:
    \item Precondizioni:
    \item Postcondizioni:
    \item Scenario principale: \begin{enumerate}
        \item 1
        \item 2
    \end{enumerate}
    \item Generalizzazioni: \begin{itemize}
        \item 1
        \item 2
    \end{itemize}
    \item Estensioni: \begin{enumerate}
        \item 1
    \end{enumerate}
\end{itemize}

\paragraph{UC1.1 - lorem ipsum}

\begin{itemize}
    \item Attore primario:
    \item Precondizioni:
    \item Postcondizioni:
    \item Scenario principale: \begin{enumerate}
        \item 1
        \item 2
    \end{enumerate}
    \item Generalizzazioni: \begin{itemize}
        \item 1
        \item 2
    \end{itemize}
    \item Estensioni: \begin{enumerate}
        \item 1
    \end{enumerate}
\end{itemize}

\paragraph{UC1.2 - lorem ipsum}

\begin{itemize}
    \item Attore primario:
    \item Precondizioni:
    \item Postcondizioni:
    \item Scenario principale: \begin{enumerate}
        \item 1
        \item 2
    \end{enumerate}
    \item Generalizzazioni: \begin{itemize}
        \item 1
        \item 2
    \end{itemize}
    \item Estensioni: \begin{enumerate}
        \item 1
    \end{enumerate}
\end{itemize}

