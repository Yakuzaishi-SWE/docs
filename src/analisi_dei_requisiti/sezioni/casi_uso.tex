\section{Casi d'uso}

\subsection{Scopo}
Lo scopo di questa sezione è la descrizione in elenco di tutti i casi d’uso individuati dal gruppo, in
riferimento alle funzionalità dell’applicazione.

\subsection{Attori}
Come accordato con il proponente, la webapp deve essere usata sia dal venditore che dal cliente,
sono quindi presenti tre attori nella gerarchia: la piattaforma E-Commerce da cui parte l'ordine, Metamask\glo, l'utente generico.
L'utente generico ha poi due specializzazioni possibili: l'utente propietario dell'ordine e l'utente venditore.

\subsection{UC1 - Inizializzazione delle Transazione}

\begin{itemize}
    \item Attore primario: E-Commerce.
    \item Attore secondario: Utente Customer.
    \item Precondizioni: il sistema è raggiungibile e funzionante.
    \item Postcondizioni: la transazione viene caricata dnel sistema, l'utente viene reindirizzato alla pagina di checkout.
    \item Scenario principale: L'E-Commerce delega il pagamento comunicando la somma richiesta e l'indirizzo del destinatario.
\end{itemize}

\paragraph{UC1.1 - }

\subsection{UC2 - Tipo Pagamento}

\begin{itemize}
    \item Attore primario: Utente propietario dell'ordine.
    \item Precondizioni: l'E-Commerce ha iniziato la transazione [UC1].
    \item Postcondizioni: viene visualizzato il riepilogo dell'ordine e scelta la modalità di pagamento.
    \item Scenario principale: l'utente seleziona la tiopologia del pagamento tra quelle disponibili.
    \item Generalizzazioni: l'utente sceglie una delle seguenti opzioni\begin{itemize}
        \item pagamento unico [UC2.1].
        \item pagamento moneybox UC[2.2].
    \end{itemize}
\end{itemize}

\paragraph{UC2.1 - Selezionato Pagamento Unico}

\begin{itemize}
    \item Attore primario: propietario dell'ordine.
    \item Precondizioni: l'E-Commerce ha iniziato la transazione [UC1].
    \item Postcondizioni: l'utente ha selezionato pagamento unico come metodo di pagamento e ha pagato con lo stesso.
    \item Scenario principale: \begin{enumerate}
        \item l'utente seleziona pagamento unico come metodo di pagamento.
        \item l'utente effettua il pagamento [UC4].
    \end{enumerate}
\end{itemize}

\paragraph{UC2.2 -  Selezionato Pagamento MoneyBox}

\begin{itemize}
    \item Attore primario: utente customer.
    \item Precondizioni: l'E-Commerce ha iniziato la transazione [UC1].
    \item Postcondizioni: l'utente ha selezionato pagamento MoneyBox come metodo di pagamento e ha pagato con lo stesso.
    \item Scenario principale:\begin{enumerate}
        \item l'utente seleziona pagamento MoneyBox come metodo di pagamento.
        \item l'utente visualizza lo stato di completamento [UC2.2.1].
        \item l'utente visualizza l'invito alla partecipazione della MoneyBox [2.2.2].
        \item l'utente partecipa alla MoneyBox [2.2.3].
    \end{enumerate}
\end{itemize}

\paragraph{UC2.2.1 - Visualizzazione Stato Completamento}

\begin{itemize}
    \item Attore primario: utente customer.
    \item Precondizioni: l'E-Commerce ha iniziato la transazione [UC1], l'utente dispone dell'invito valido ad una MoneyBox.
    \item Postcondizioni: l'utente ha visualizzato lo stato del completamento.
    \item Scenario principale: \begin{enumerate}
        \item l'utente visualizza lo stato del completamento in formato percentuale e il saldo mancante.
        \item l'utente visualizza una lista delle transazioni avvenute [UC2.2.1.1].
    \end{enumerate}
    \end{itemize}

\paragraph{UC2.2.2 - Visualizzazione Invito Partecipazione MoneyBox}

\begin{itemize}
    \item Attore primario: utente customer.
    \item Precondizioni: l'E-Commerce ha iniziato la transazione [UC1], l'utente dispone dell'invito valido ad una MoneyBox.
    \item Postcondizioni: l'utente ha visualizzato l'invito alla MoneyBox.
    \item Scenario principale: l'utente visualizza l'invito alla MoneyBox.
\end{itemize}

\paragraph{UC2.2.3 - Scelta Quota Partecipazione}

\begin{itemize}
    \item Attore primario: utente customer.
    \item Precondizioni: l'E-Commerce ha iniziato la transazione [UC1], l'utente dispone dell'invito valido ad una MoneyBox.
    \item Postcondizioni: l'utente ha selezionato la quota da versare e ha effettuato il pagamento.
    \item Scenario principale:\begin{enumerate}
        \item l'utente seleziona la quota da versare, compresa tra zero escluso e il minimo tra il saldo disponibile nel wallet e il rimanente della MoneyBox.
        \item l'utente effettua il pagamento [UC4].
    \end{enumerate}
\end{itemize}

\subsection{UC3 - Visualizzazione Riepilogo Ordine}

\begin{itemize}
    \item Attore primario: utente customer.
    \item Precondizioni: l'E-Commerce ha iniziato la transazione [UC1].
    \item Postcondizioni: l'utente ha visualizzato il riepilogo dell'ordine.
    \item Scenario principale: \begin{enumerate}
        \item l'utente visualizza il totale dell'ordine.
        \item l'utente visualizza un elenco degli item ordinati.
    \end{enumerate}
\end{itemize}

\subsection{UC4 - Pagamento}

\begin{itemize}
    \item Attore primario: utente customer.
    \item Attore secondario: Metamask.
    \item Precondizioni: l'utente ha scelto il Tipo di Pagamento [UC2].
    \item Postcondizioni: l'utente ha effettuato il pagamento.
    \item Scenario principale: \begin{enumerate}
        \item l'utente effettuata una verifica del saldo del wallet.
        \item l'utente visualizza il Pop-Up\glo\ di Metamask.
        \item l'utente autorizza il pagamento.
        \item l'utente visualizza un messagio di conferma di avvenuto pagamento.
    \end{enumerate}
    \item Generalizzazioni: \begin{itemize}
        \item 
        \item 
    \end{itemize}
    \item Estensioni: \begin{enumerate}
        \item 1
    \end{enumerate}
\end{itemize}



\begin{itemize}
    \item Attore primario:
    \item Precondizioni:
    \item Postcondizioni:
    \item Scenario principale: \begin{enumerate}
        \item 1
        \item 2
    \end{enumerate}
    \item Generalizzazioni: \begin{itemize}
        \item 1
        \item 2
    \end{itemize}
    \item Estensioni: \begin{enumerate}
        \item 1
    \end{enumerate}
\end{itemize}


\begin{itemize}
    \item Attore primario:
    \item Precondizioni:
    \item Postcondizioni:
    \item Scenario principale: \begin{enumerate}
        \item 1
        \item 2
    \end{enumerate}
    \item Generalizzazioni: \begin{itemize}
        \item 1
        \item 2
    \end{itemize}
    \item Estensioni: \begin{enumerate}
        \item 1
    \end{enumerate}
\end{itemize}



\begin{itemize}
    \item Attore primario:
    \item Precondizioni:
    \item Postcondizioni:
    \item Scenario principale: \begin{enumerate}
        \item 1
        \item 2
    \end{enumerate}
    \item Generalizzazioni: \begin{itemize}
        \item 1
        \item 2
    \end{itemize}
    \item Estensioni: \begin{enumerate}
        \item 1
    \end{enumerate}
\end{itemize}



\begin{itemize}
    \item Attore primario:
    \item Precondizioni:
    \item Postcondizioni:
    \item Scenario principale: \begin{enumerate}
        \item 1
        \item 2
    \end{enumerate}
    \item Generalizzazioni: \begin{itemize}
        \item 1
        \item 2
    \end{itemize}
    \item Estensioni: \begin{enumerate}
        \item 1
    \end{enumerate}
\end{itemize}


