\section{Requisiti} \label{section:requisiti}

\subsection{Introduzione}
Il gruppo \groupName{} per classificare e assegnare i requisiti del capitolato ha seguito quanto definito nelle \docNameVersionNdP{}.

\subsection{Requisiti funzionali} \label{subsection:requisiti_funzionali}
\begin{table}[H]
    \centering
    \renewcommand{\arraystretch}{1.8}
    \rowcolors{2}{green!100!black!40}{green!100!black!30}
    \begin{tabular}{c | c | p{6cm} | c}
        \rowcolor[HTML]{125E28}
        \multicolumn{1}{c}{\color[HTML]{FFFFFF} \textbf{Codice}} & 
		\multicolumn{1}{c}{\color[HTML]{FFFFFF} \textbf{Classificazione}} & 
		\multicolumn{1}{c}{\color[HTML]{FFFFFF} \textbf{Descrizione}} & 
		\multicolumn{1}{c}{\color[HTML]{FFFFFF} \textbf{Fonti}} \\
        \hline
        R1F1 & Obbligatorio & Si deve processare una richiesta di checkout da un E-Commerce. & Capitolato \\
        R1F2 & Obbligatorio & L'utente deve poter scegliere la tipologia di pagamento. & Verbale Esterno 15.11.21 \\
        R1F2.1 & Obbligatorio & L'utente deve poter scegliere la tipologia di pagamento unico. & Verbale Esterno 15.11.21\\
        R1F2.2 & Obbligatorio & L'utente deve poter scegliere la tipologia di pagamento Moneybox. & Verbale Esterno 15.11.21\\
        R2F2.2.1 & Desiderabile & Si deve poter visualizzare lo stato di completamento della MoneyBox. & UC2.2.1 \\
        R2F2.2.2 & Desiderabile & Si deve poter visualizzare l'invito di partecipazione alla MoneyBox. & UC2.2.2 \\
        R3F2.2.3 & Opzionale & L'utente deve poter visualizzare una traduzione visiva dell'indirizzo Moneybox. & Verbale Esterno 17.01.22 \\
        R2F2.2.4 & Desiderabile & Si deve poter chiudere la MoneyBox restituendo i soldi. & UC2.2.4 \\
        R2F2.2.5 & Desiderabile & L'utente deve poter visualizzare l'elenco delle transazioni partecipanti. & Capitolato \\
        R1F3 & Obbligatorio & L'utente deve poter visualizzare il totale dell'ordine. & Capitolato \\
        R1F4 & Obbligatorio & L'utente deve poter effettuare la connessione a Metamask. & Capitolato \\
        R1F5 & Obbligatorio & L'utente deve poter pagare. & Capitolato \\
    \end{tabular}
\end{table}
\begin{center}
    \textit{\small Continua nella pagina successiva}
\end{center}
\begin{table}[H]
    \centering
    \renewcommand{\arraystretch}{1.8}
    \rowcolors{2}{green!100!black!40}{green!100!black!30}
    \begin{tabular}{c | c | p{5cm} | c}
        \rowcolor[HTML]{125E28} 
        \multicolumn{1}{c}{\color[HTML]{FFFFFF} \textbf{Codice}} & 
		\multicolumn{1}{c}{\color[HTML]{FFFFFF} \textbf{Classificazione}} & 
		\multicolumn{1}{c}{\color[HTML]{FFFFFF} \textbf{Descrizione}} & 
		\multicolumn{1}{c}{\color[HTML]{FFFFFF} \textbf{Fonti}} \\
        \hline
        R1F5.1 & Obbligatorio & L'utente che ha scelto il metodo di pagamento unico deve pagare interamente la somma richiesta. & Verbale Esterno 15.11.21 \\
        R2F5.2 & Desiderabile & L'utente che partecipa al pagamento MoneyBox deve poter pagare una parte della somma richiesta. & Verbale Esterno 15.11.21 \\
        R1F6 & Obbligatorio & Si deve trattenere una fee percentuale destinata al fondo ShopChain. & Capitolato \\
        R1F6.1 & Obbligatorio & Parte della fee sarà destinata al pagamento delle gas fee\glo{}. & Capitolato \\
        R1F7 & Obbligatorio & Il proprietario dell'ordine deve poter confermare la ricezione e quindi sbloccare i fondi dallo Smart Contract. & Capitolato \\
        R1F8 & Obbligatorio & Il venditore deve poter visualizzare le transazioni. & Capitolato \\
        R2F8.1 & Desiderabile & Il venditore deve poter visualizzare le transazioni non completate. & Capitolato \\
        R1F8.2 & Obbligatorio & Il venditore deve poter visualizzare le transazioni completate. & Capitolato \\
        R2F8.2.1 & Desiderabile & Il venditore deve poter visualizzare le transazioni completate e non sbloccate. & Capitolato \\
        R1F8.2.2 & Obbligatorio & Il venditore deve poter visualizzare le transazioni completate e sbloccate. & Capitolato \\
        R2F9 & Desiderabile & Si deve convertire l'ammontare depositato sullo smart contract in stable coin. & Capitolato \\ 
    \end{tabular}
    \caption{Requisiti funzionali}
\end{table}

\subsection{Requisiti di qualità} \label{subsection:requisiti_qualita}
\begin{table}[H]
    \centering
    \renewcommand{\arraystretch}{1.8}
    \rowcolors{2}{green!100!black!40}{green!100!black!30}
    \begin{tabular}{c | c | p{6cm} | c}
        \rowcolor[HTML]{125E28}
        \multicolumn{1}{c}{\color[HTML]{FFFFFF} \textbf{Codice}} & 
		\multicolumn{1}{c}{\color[HTML]{FFFFFF} \textbf{Classificazione}} & 
		\multicolumn{1}{c}{\color[HTML]{FFFFFF} \textbf{Descrizione}} & 
		\multicolumn{1}{c}{\color[HTML]{FFFFFF} \textbf{Fonti}} \\
        \hline
        R1Q1 & Obbligatorio & Il progetto deve essere accessibile pubblicamente su GitHub\glo{} o su un altra repository\glo{} pubblica. & Capitolato \\
        R1Q2 & Obbligatorio & Il prodotto deve essere sviluppato in modo concorde a quanto stabilito nelle Norme di Progetto\glo{}. & Capitolato \\
        R1Q3 & Obbligatorio & Devono essere realizzati test di unità ed integrazione con una copertura minima dell'80\%. & Capitolato\\
        R1Q4 & Obbligatorio & Deve essere fornita una completa documentazione sulle scelte implementative e progettuali effettuate. & Capitolato \\
    \end{tabular}
    \caption{Requisiti di qualità}
\end{table}

\subsection{Requisiti di vincolo} \label{subsection:requisiti_vincolo}
\begin{table}[H]
    \centering
    \renewcommand{\arraystretch}{1.8}
    \rowcolors{2}{green!100!black!40}{green!100!black!30}
    \begin{tabular}{c | c | p{6cm} | c}
        \rowcolor[HTML]{125E28}
        \multicolumn{1}{c}{\color[HTML]{FFFFFF} \textbf{Codice}} & 
		\multicolumn{1}{c}{\color[HTML]{FFFFFF} \textbf{Classificazione}} & 
		\multicolumn{1}{c}{\color[HTML]{FFFFFF} \textbf{Descrizione}} & 
		\multicolumn{1}{c}{\color[HTML]{FFFFFF} \textbf{Fonti}} \\
        \hline
		R1V1 & Obbligatorio & La richiesta di un ordine dev'essere caricato su una blockchain pubblica. & Capitolato \\
        R1V2 & Obbligatorio & L'avvenuto pagamento e le transazioni devono essere gestite e verificate tramite Smart Contract. & Capitolato \\
        R1V3 & Obbligatorio & L'applicazione per l'interazione con la piattaforma dev'essere sviluppata attraverso l'uso di tecnologie web. & Capitolato \\ 
        R2V4 & Desiderabile & La blockchain di riferimento scelta sarà Fantom. & Verbale Esterno 10.11.21 \\
        R3V5 & Opzionale & L'applicativo deve essere utilizzabile anche da dispositivi mobili, come smartphone e tablet. & Capitolato \\                       
    \end{tabular}
\end{table}
\begin{center}
    \textit{\small Continua nella pagina successiva}
\end{center}
\begin{table}[H]
    \centering
    \renewcommand{\arraystretch}{1.8}
    \rowcolors{2}{green!100!black!40}{green!100!black!30}
    \begin{tabular}{c | c | p{5cm} | c}
        \rowcolor[HTML]{125E28} 
        \multicolumn{1}{c}{\color[HTML]{FFFFFF} \textbf{Codice}} & 
		\multicolumn{1}{c}{\color[HTML]{FFFFFF} \textbf{Classificazione}} & 
		\multicolumn{1}{c}{\color[HTML]{FFFFFF} \textbf{Descrizione}} & 
		\multicolumn{1}{c}{\color[HTML]{FFFFFF} \textbf{Fonti}} \\
        \hline
        R1V6 & Obbligatorio & Gli smart contract verranno sviluppati in Solidity. & Capitolato \\
        R1V7 & Obbligatorio & Il front-end dell'applicazione verrà sviluppato in React. & Capitolato \\ 
        R1V8 & Obbligatorio & Il back-end verrà sviluppato usando un server Node.js. & Capitolato \\
        R1V9 & Obbligatorio & Il back-end verrà sviluppato usufruendo del framework\glo{} Express.js\glo{}. & Capitolato \\
        R3V10 & Opzionale & Come database di supporto verrà utilizzato Redis\glo{}. & Verbale Interno 07.02.22 \\
    \end{tabular}
    \caption{Requisiti di vincolo}
\end{table}

\subsection{Requisiti prestazionali} \label{subsection:requisiti_prestazionali}

L'azienda non ha posto requisiti di tipo prestazionale per la applicazione,
nonostante questo il gruppo presterà attenzione alle prestazioni durante lo sviluppo.

\clearpage
