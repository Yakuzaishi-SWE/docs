\section{Requisiti} \label{section:requisiti}

\subsection{Introduzione}
Il gruppo \groupName{} per classificare e assegnare i requisiti del capitolato ha seguito quanto definito nelle \docNameVersionNdP{}.

\subsection{Requisiti funzionali} \label{subsection:requisiti_funzionali}
\begin{table}[H]
    \centering
    \renewcommand{\arraystretch}{1.8}
    \rowcolors{2}{green!100!black!40}{green!100!black!30}
    \begin{tabular}{c | c | p{6cm} | c }
        \rowcolor[HTML]{125E28}
        \multicolumn{1}{c}{\color[HTML]{FFFFFF} \textbf{Codice}}          &
        \multicolumn{1}{c}{\color[HTML]{FFFFFF} \textbf{Classificazione}} &
        \multicolumn{1}{c}{\color[HTML]{FFFFFF} \textbf{Descrizione}}     &
        \multicolumn{1}{c}{\color[HTML]{FFFFFF} \textbf{Fonti}}                                                                                                                                                                  \\
        \hline
        R1F1                                                              & Obbligatorio & Si deve processare una richiesta di checkout da un e-commerce\glo{}.                          & \Shortunderstack{Capitolato\\UC1\\}                        \\
        R1F2                                                              & Obbligatorio & L'utente deve poter scegliere la tipologia di pagamento.                                      & \Shortunderstack{Verbale Esterno 15.11.21\\UC2.2\\}        \\
        R1F2.1                                                            & Obbligatorio & L'utente deve poter scegliere la tipologia di pagamento unico.                                & \Shortunderstack{Verbale Esterno 15.11.21\\UC2.2.1\\}      \\
        R2F2.2                                                            & Desiderabile & L'utente deve poter scegliere la tipologia di pagamento MoneyBox\glo{}.                       & \Shortunderstack{Verbale Esterno 15.11.21\\UC2.2.2\\}      \\
        R2F2.2.1                                                          & Desiderabile & Si deve poter visualizzare lo stato di completamento della MoneyBox\glo{}.                    & UC3.1                                 \\
        R2F2.2.2                                                          & Desiderabile & Si deve poter visualizzare l'invito di partecipazione alla MoneyBox\glo{}.                    & UC3.3                                 \\
        R3F2.2.3                                                          & Opzionale    & L'utente deve poter visualizzare una traduzione visiva dell'indirizzo MoneyBox\glo{}.         & \Shortunderstack{Verbale Esterno 17.01.22\\UC3.5\\}        \\
        R2F2.2.4                                                          & Desiderabile & Si deve poter chiudere la MoneyBox\glo{} restituendo i soldi.                                 & UC8, UC9                                  \\
        R2F2.2.5                                                          & Desiderabile & L'utente deve poter visualizzare l'elenco delle transazioni partecipanti alla MoneyBox\glo{}. & \Shortunderstack{Capitolato\\UC3.2\\} \\
        R1F3                                                              & Obbligatorio & L'utente deve poter visualizzare il totale dell'ordine.                                       & \Shortunderstack{Capitolato\\UC2.1\\} \\
        R1F4                                                              & Obbligatorio & L'utente deve poter effettuare la connessione a Metamask\glo{}.                               & \Shortunderstack{Capitolato\\UC4\\}   \\
        R1F5                                                              & Obbligatorio & L'utente deve poter pagare.                                                                   & \Shortunderstack{Capitolato\\UC2.3\\} \\
    \end{tabular}
\end{table}
\begin{center}
    \textit{\small Continua nella pagina successiva}
\end{center}
\begin{table}[H]
    \centering
    \renewcommand{\arraystretch}{1.8}
    \rowcolors{2}{green!100!black!40}{green!100!black!30}
    \begin{tabular}{c | c | p{6cm} | c}
        \rowcolor[HTML]{125E28}
        \multicolumn{1}{c}{\color[HTML]{FFFFFF} \textbf{Codice}}          &
        \multicolumn{1}{c}{\color[HTML]{FFFFFF} \textbf{Classificazione}} &
        \multicolumn{1}{c}{\color[HTML]{FFFFFF} \textbf{Descrizione}}     &
        \multicolumn{1}{c}{\color[HTML]{FFFFFF} \textbf{Fonti}}                                                                                                                                                                                          \\
        \hline
        R1F5.1                                                            & Obbligatorio & L'utente che ha scelto il metodo di pagamento unico deve pagare interamente la somma richiesta.                       & \Shortunderstack{Verbale Esterno 15.11.21\\UC2.3.1\\}      \\
        R2F5.2                                                            & Desiderabile & L'utente che partecipa al pagamento MoneyBox\glo{} deve poter pagare una parte della somma richiesta.                 & \Shortunderstack{Verbale Esterno 15.11.21\\UC2.3.2\\}      \\
        R1F5.3                                                            & Obbligatorio & L'utente deve poter visualizzare il codice di sblocco generato dal sistema.                                           & UC2.4                                 \\
        R1F5.4                                                            & Obbligatorio & L'utente deve poter visualizzare un messaggio d'errore nel caso in cui la transazione fallisca.                       & UC2.3.3                               \\
        R1F6                                                              & Obbligatorio & Il proprietario dell'ordine o il venditore devono poter richiedere il rimborso della transazione.                     & \Shortunderstack{Capitolato\\UC8, UC9\\}   \\
        R1F7                                                              & Obbligatorio & Il proprietario dell'ordine deve poter confermare la ricezione e quindi sbloccare i fondi dallo smart contract\glo{}. & \Shortunderstack{Capitolato\\UC6\\}   \\
        R1F8                                                              & Obbligatorio & L'utente deve poter visualizzare le transazioni.                                                                      & \Shortunderstack{Capitolato\\UC7\\}   \\
        R2F8.1                                                            & Desiderabile & Il venditore deve poter visualizzare le transazioni in entrata pagate.                                                & \Shortunderstack{Capitolato\\UC7.1\\} \\
        R2F8.1.1                                                          & Desiderabile & Il venditore deve poter visualizzare le transazioni in entrata pagate non sbloccate.                                  & UC7.1.1                               \\
        R2F8.1.2                                                          & Desiderabile & Il venditore deve poter visualizzare le transazioni in entrata pagate e sbloccate.                                    & UC7.1.2                               \\
        R2F8.1.3                                                          & Desiderabile & Il venditore deve poter visualizzare le transazioni in entrata pagate ma cancellate.                                  & UC7.1.3                               \\
        R1F8.2                                                            & Obbligatorio & Il proprietario dell'ordine deve poter visualizzare le transazioni in uscita.                                         & UC7.2                                 \\
    \end{tabular}
\end{table}
\begin{center}
    \textit{\small Continua nella pagina successiva}
\end{center}
\begin{table}[H]
    \centering
    \renewcommand{\arraystretch}{1.8}
    \rowcolors{2}{green!100!black!40}{green!100!black!30}
    \begin{tabular}{c | c | p{6cm} | c}
        \rowcolor[HTML]{125E28}
        \multicolumn{1}{c}{\color[HTML]{FFFFFF} \textbf{Codice}}          &
        \multicolumn{1}{c}{\color[HTML]{FFFFFF} \textbf{Classificazione}} &
        \multicolumn{1}{c}{\color[HTML]{FFFFFF} \textbf{Descrizione}}     &
        \multicolumn{1}{c}{\color[HTML]{FFFFFF} \textbf{Fonti}}                                                                                                                                                                  \\
        \hline
        R2F8.2.1                                                          & Desiderabile & Il proprietario dell'ordine deve poter visualizzare le transazioni in uscita non pagate.                   & UC7.2.1                  \\
        R2F8.2.2                                                          & Desiderabile & Il proprietario dell'ordine deve poter visualizzare le transazioni in uscita pagate ma non sbloccate.      & UC7.2.2                  \\
        R2F8.2.3                                                          & Desiderabile & Il proprietario dell'ordine deve poter visualizzare le transazioni in uscita pagate e sbloccate.           & UC7.2.3                  \\
        R2F8.2.4                                                          & Desiderabile & Il proprietario dell'ordine deve poter visualizzare le transazioni in uscita cancellate.                   & UC7.2.4                  \\
        R2F9                                                              & Desiderabile & Si deve convertire l'ammontare depositato sullo smart contract\glo{} in stable coin\glo{}.                 & Capitolato               \\
        R3F10                                                             & Opzionale    & La piattaforma deve trattenere una piccola somma in percentuale di un ordine destinata al fondo ShopChain. & Verbale Esterno 15.11.21 \\
        R1F11                                                             & Obbligatorio & L'utente deve poter visualizzare il suo indirizzo wallet.                                                  & UC5                      \\
        R1F11.1                                                           & Obbligatorio & L'utente deve poter visualizzare il suo indirizzo wallet in forma testuale.                                & UC5.1                    \\
        R1F11.2                                                           & Obbligatorio & L'utente deve poter visualizzare un avviso della mancata connessione a Metamask\glo{}.                     & UC5.3                    \\
        R3F11.3                                                           & Opzionale    & L'utente deve poter visualizzare il suo indirizzo wallet sotto forma di stringa di emoji.                  & UC5.2                    \\
        R3F12                                                             & Opzionale    & Gli utenti partecipanti devono ricevere una notifica al completamento della MoneyBox\glo{}.                & Verbale Interno 02.03.22 \\
        R1F13                                                             & Obbligatorio & L'utente deve poter visualizzare il dettaglio dell'errore della transazione.                               & UC2.3.3                  \\
    \end{tabular}
\end{table}
\begin{center}
    \textit{\small Continua nella pagina successiva}
\end{center}
\begin{table}[H]
    \centering
    \renewcommand{\arraystretch}{1.8}
    \rowcolors{2}{green!100!black!40}{green!100!black!30}
    \begin{tabular}{c | c | p{6cm} | c}
        \rowcolor[HTML]{125E28}
        \multicolumn{1}{c}{\color[HTML]{FFFFFF} \textbf{Codice}}          &
        \multicolumn{1}{c}{\color[HTML]{FFFFFF} \textbf{Classificazione}} &
        \multicolumn{1}{c}{\color[HTML]{FFFFFF} \textbf{Descrizione}}     &
        \multicolumn{1}{c}{\color[HTML]{FFFFFF} \textbf{Fonti}}                                                                                                                                                                   \\
        \hline
        R1F14                                                             & Obbligatorio & L'utente deve poter visualizzare lo stato di connessione a Metamask\glo{}.                                       & UC4                      \\
        R1F14.1                                                           & Obbligatorio & L'utente deve poter visualizzare un suggerimento nel caso sia connesso correttamente.                      & UC4                      \\
        R1F14.2                                                           & Obbligatorio & L'utente deve poter visualizzare un errore nel caso non sia installato Metamask\glo{}.                                             & UC4.1   \\
        R1F14.3                                                           & Obbligatorio & L'utente deve poter visualizzare un errore nel caso in cui la blockchain selezionata non sia corretta.                       & UC4.2   \\
        R1F14.4                                                           & Obbligatorio & L'utente deve poter visualizzare un errore nel caso in cui non abbia connesso un account a ShopChain.                        & UC4.3   \\
        R1F15                                                             & Obbligatorio & L'utente deve poter visualizzare un messaggio di avviso nel caso in cui la transazione sia già presente in blockchain\glo{}. & UC2.2.3 \\
    \end{tabular}
    \caption{Requisiti funzionali}
\end{table}

\subsection{Requisiti di qualità} \label{subsection:requisiti_qualita}
\begin{table}[H]
    \centering
    \renewcommand{\arraystretch}{1.8}
    \rowcolors{2}{green!100!black!40}{green!100!black!30}
    \begin{tabular}{c | c | p{6cm} | c}
        \rowcolor[HTML]{125E28}
        \multicolumn{1}{c}{\color[HTML]{FFFFFF} \textbf{Codice}}          &
        \multicolumn{1}{c}{\color[HTML]{FFFFFF} \textbf{Classificazione}} &
        \multicolumn{1}{c}{\color[HTML]{FFFFFF} \textbf{Descrizione}}     &
        \multicolumn{1}{c}{\color[HTML]{FFFFFF} \textbf{Fonti}}                                                                                                                                                                \\
        \hline
        R1Q1                                                              & Obbligatorio & Il progetto deve essere accessibile pubblicamente su GitHub\glo{} o su un altra repository\glo{} pubblica.      & Capitolato        \\
        R1Q2                                                              & Obbligatorio & Il prodotto deve essere sviluppato in modo concorde a quanto stabilito nelle \docNameVersionNdP{}.              & Capitolato        \\
        R1Q3                                                              & Obbligatorio & Devono essere realizzati test di unità ed integrazione con una copertura minima dell'80\%.                      & Capitolato        \\
        R1Q4                                                              & Obbligatorio & Deve essere fornita una completa documentazione sulle scelte implementative e progettuali effettuate.           & Capitolato        \\
    \end{tabular}
\end{table}
\begin{center}
    \textit{\small Continua nella pagina successiva}
\end{center}
\begin{table}[H]
    \centering
    \renewcommand{\arraystretch}{1.8}
    \rowcolors{2}{green!100!black!40}{green!100!black!30}
    \begin{tabular}{c | c | p{6cm} | c}
        \rowcolor[HTML]{125E28}
        \multicolumn{1}{c}{\color[HTML]{FFFFFF} \textbf{Codice}}          &
        \multicolumn{1}{c}{\color[HTML]{FFFFFF} \textbf{Classificazione}} &
        \multicolumn{1}{c}{\color[HTML]{FFFFFF} \textbf{Descrizione}}     &
        \multicolumn{1}{c}{\color[HTML]{FFFFFF} \textbf{Fonti}}                                                                                                                                                                   \\
        \hline
        R1Q5                                                              & Obbligatorio & Deve essere fornito un manuale per l'utilizzo.                                                                  & Capitolato        \\
        R1Q6                                                              & Obbligatorio & Deve essere fornito un manuale per la manutenzione e l'estensione dell'applicazione.                            & Capitolato        \\
        R1Q7                                                              & Obbligatorio & Lo sviluppo del codice TypeScript\glo{} deve essere supportato dal software di analisi del codice ESLint\glo{}. & Decisione Interna \\
    \end{tabular}
    \caption{Requisiti di qualità}
\end{table}

\subsection{Requisiti di vincolo} \label{subsection:requisiti_vincolo}
\begin{table}[H]
    \centering
    \renewcommand{\arraystretch}{1.8}
    \rowcolors{2}{green!100!black!40}{green!100!black!30}
    \begin{tabular}{c | c | p{6cm} | c}
        \rowcolor[HTML]{125E28}
        \multicolumn{1}{c}{\color[HTML]{FFFFFF} \textbf{Codice}}          &
        \multicolumn{1}{c}{\color[HTML]{FFFFFF} \textbf{Classificazione}} &
        \multicolumn{1}{c}{\color[HTML]{FFFFFF} \textbf{Descrizione}}     &
        \multicolumn{1}{c}{\color[HTML]{FFFFFF} \textbf{Fonti}}                                                                                                                                                                     \\
        \hline
        R1V1                                                              & Obbligatorio & La richiesta di un ordine dev'essere caricato su una blockchain\glo{} pubblica.                               & Capitolato               \\
        R1V2                                                              & Obbligatorio & L'avvenuto pagamento e le transazioni devono essere gestite e verificate tramite smart contract\glo{}.        & Capitolato               \\
        R1V3                                                              & Obbligatorio & L'applicazione per l'interazione con la piattaforma dev'essere sviluppata attraverso l'uso di tecnologie web. & Capitolato               \\
        R2V4                                                              & Desiderabile & La blockchain\glo{} di riferimento scelta sarà Fantom\glo{}.                                                  & Verbale Esterno 01.12.21 \\
        R3V5                                                              & Opzionale    & L'applicativo deve essere utilizzabile anche da dispositivi mobili, come smartphone e tablet.                 & Capitolato               \\
        R1V6                                                              & Obbligatorio & Gli smart contract\glo{} verranno sviluppati in Solidity\glo{}.                                               & Capitolato               \\
        R1V7                                                              & Obbligatorio & Il front-end\glo{} dell'applicazione verrà sviluppato in React\glo{} tipizzato utilizzando TypeScript\glo{}.  & Capitolato               \\
        R1V8                                                              & Obbligatorio & Lo stile dell'applicazione verrà sviluppato in Sass\glo{}.                                                    & Decisione Interna        \\
    \end{tabular}
\end{table}
\begin{center}
    \textit{\small Continua nella pagina successiva}
\end{center}
\begin{table}[H]
    \centering
    \renewcommand{\arraystretch}{1.8}
    \rowcolors{2}{green!100!black!40}{green!100!black!30}
    \begin{tabular}{c | c | p{6cm} | c}
        \rowcolor[HTML]{125E28}
        \multicolumn{1}{c}{\color[HTML]{FFFFFF} \textbf{Codice}}          &
        \multicolumn{1}{c}{\color[HTML]{FFFFFF} \textbf{Classificazione}} &
        \multicolumn{1}{c}{\color[HTML]{FFFFFF} \textbf{Descrizione}}     &
        \multicolumn{1}{c}{\color[HTML]{FFFFFF} \textbf{Fonti}}                                                                                                                                                                   \\
        \hline
        R1V9                                                              & Obbligatorio & Per l'interazione con la blockchain\glo{} verrà utilizzata la libreria JavaScript\glo{} Web3.js\glo{}.        & Capitolato               \\
        R1V10                                                             & Obbligatorio & Per il testing in locale dello smart contract\glo{} verrà utilizzato Ganache\glo{}.                           & Decisione Interna        \\
        R1V11                                                             & Obbligatorio & Per il caricamento dello smart contract\glo{} e l'esecuzione dei test verrà utilizzato Truffle\glo{}.         & Decisione Interna        \\
        R1V12                                                             & Obbligatorio & Per la scrittura dei test si utilizzerà il framework\glo{} JavaScript\glo{} Mocha\glo{}.                      & Decisione Interna        \\
        R1V13                                                             & Obbligatorio & Per la scrittura dei test si utilizzerà la libreria JavaScript\glo{} Chai\glo{}.                              & Decisione Interna        \\
        R1V14                                                             & Obbligatorio & L'applicativo deve essere compatibile con il browser Google Chrome\glo{} dalla versione 98.  & Capitolato \\
        R1V15                                                             & Obbligatorio & L'applicativo deve essere compatibile con il browser Microsoft Edge\glo{} dalla versione 98. & Capitolato \\
        R1V16                                                             & Obbligatorio & L'applicativo deve essere compatibile con il browser Firefox\glo{} dalla versione 97.        & Capitolato \\
        R1V17                                                             & Obbligatorio & L'applicativo deve essere compatibile con il browser Brave\glo{} dalla versione 1.36.        & Capitolato \\
    \end{tabular}
    \caption{Requisiti di vincolo}
\end{table}

\subsubsection{Requisiti sistemi operativi}

Non è stato individuato nessun requisito di vincolo riguardante i sistemi operativi da supportare poiché l'applicativo da sviluppare viene eseguito su browser.

\subsection{Requisiti prestazionali} \label{subsection:requisiti_prestazionali}

L'azienda non ha posto requisiti di tipo prestazionale per la applicazione.
Nonostante questo, a seguito di un approfondito studio e confronto riguardo le possibili 
blockchain\glo{}\footnote{Presente in verbale\_interno\_2021\_11\_16.pdf} da utilizzare, è stata selezionata Fantom\glo{} 
proprio per il suo veloce algoritmo di consenso, il quale si presta perfettamente a soddisfare piccoli pagamenti
come nel caso di un e-commerce\glo{}.
