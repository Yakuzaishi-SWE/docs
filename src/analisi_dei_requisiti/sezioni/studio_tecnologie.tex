\section{Studio delle tecnologie}

\subsection{Cos'è una blockchain}
La blockchain, che letteralmente significa "catena di blocchi", è una struttura dati condivisa e
immutabile. Viene definita come un registro digitale le cui voci sono raggruppate a blocchi che
vengono concatenati in ordine cronologico e la cui integrità è garantita dall'uso della crittografia.
Si basa sul concetto in cui quando il contenuto viene scritto (tramite un processo normato), questo non
è più modificabile nè eliminabile, pena l'invalidazione dell'intero processo.\\
Le caratteristiche di base che caratterizzano questa tecnologia sono: digitalizzazione dei dati, decentalizzazione,
disintermediazione (non è richiesto nessun intermediario), tracciabilità dei trasferimenti, trasparenza/verificabilità,
immutabilità e programmabilità dei trasferimenti. Grazie a tali proprietà, la blockchain si pone come una valida
alternativa in termini di sicurezza, affidabilità, trasparenza e costi alle banche dati e ai documenti gestiti in maniera
centralizzata da autorità riconosciute e regolamentate (come banche, pubbliche amministrazioni, ecc...).

\subsection{Ethereum}
Ethereum è la blockchain più famosa dopo quella di Bitcoin, prende la struttura di base di quest'ultima e aggiunge delle
importanti modifiche. In particolare si basa su una blockchain programmabile, aprendo alla possibilità di effettuare altre attività
oltre a semplici transazioni di criptovalute, parliamo in particolare di servizi finanziari, giochi o app decentralizzate, quindi
completamente libere e non soggiogate ad un ente centrale.

\subsection{Differenze tra blockchain di layer 0, 1, 2}
Le blockchain possono essere catalogate in base al tipo di architettura su cui si basano, in particolare possono basarsi su un layer 0, 1 o 2.
In base a questo tipo di differenziazione offrono un diverso grado di astrazione e funzionalità. Come funziona per una linguaggio di programmazione
più si sale di livello, più ci si allontana dallo "strato fisico".
    
    \subsubsection{Layer 0}
    è il livello più basso, quindi permette di agire direttamente sull’architettura blockchain senza dover mettere mano a layer e protocolli sottostanti. 
    Qui inoltre è possibile creare dApps (decentralized applications), validare schemi di dati, “coniare” criptovalute e altro ancora.
    Tutto l’hardware, server, nodi e qualsiasi device collegato ai nodi fa parte di questo livello. 
    A questo livello agiscono direttamente gli algoritmi di consenso, come proof-of-work, proof-of-stake, proof-of-activity.

    \subsubsection{Layer 1}
    Si occupano principalmente di rendere il protocollo base più scalabile. Si basa su due principali soluzioni:
    \begin{itemize}
        \item \textbf{Consensus protocol changes}: primo grande ed evidente cambiamento c’è stato con Ethereum, in cui i progetti hanno iniziato a spostarsi 
        dal protocollo di consenso Proof-of-Work (PoW) a un più veloce e leggero Proof-of-Stake (PoS). Questo è avvenuto su Ethereum con il protocollo Casper.
        \item \textbf{Sharding}: si basa sul semplice concetto di dividere le transazioni in “frammenti” più piccoli (traduzione di “shards”) che possono essere 
        processati parallelamente dalla rete.
    \end{itemize}
    \paragraph{Pro}
    \begin{itemize}
        \item Aggiunge semplicemente un livello superiore all’architettura, senza apportare grandi modifiche;
        \item Permette di eseguire un maggior numero di operazioni a parità di tempo;
    \end{itemize}
    \paragraph{Contro}
    \begin{itemize}
        \item Inefficient Consensus Protocol: alcuni protocolli di consenso non sono adatti a gestire grandi traffici di transazioni e conseguentemente a un'adozione di massa;
        \item Excessive Workload: il carico di lavoro su questo livello può diventare eccessivo e provocare grandi rallentamenti (è quello che succede con Ethereum e le sue conseguenti
        elevate gas fee).        
    \end{itemize}

    \subsubsection{Layer 2}
    Sono lo strato più alto di astrazione, e attualmente si pongono l'obiettivo di risolvere gran parte dei problemi presenti nel layer 1. Le soluzioni si dividono in due possibilità:
    \begin{itemize}
        \item \textbf{State Channels}: ha come componente principale un canale full-duplex tra due partecipanti che permette di svolgere compiti che tipicamente sono on-blockchain, off-blockchain. 
        Questo permette di diminuire di molto i tempi di attesa visto che viene a mancare la componente terza, come ad esempio le azioni dei miner.
        Come funziona:
        \begin{itemize}
            \item Una porzione di blockchain viene ritagliata (viene usata l’espressione “sealed off, ovvero sigillata) tramite multi-firma o sotto smart-contract previo accordo tra le due parti coinvolte;
            \item Quest’ultime possono interagire tra loro senza preoccuparsi di firmare le transazioni ecc…
            \item Quando l’interazione è finita, lo stato finale viene registrato sulla blockchain
        \end{itemize}
        La metafora più facile è quella del conto, è come se una persona aprisse un conto con un’altra e poi lo saldasse “a rate”. Invece di registrare ogni singola rata sulla rete, viene 
        registrato solo il saldo finale.\\
        Le soluzioni di questo tipo più comuni sono: “Bitcoin’s Lightning Network” (permette di eseguire tante microtransazioni) e “Ethereum’s Raiden Network” (permette anche di eseguire smart contracts 
        sui canali). Entrambe utilizzano Hashed Timelock Contracts (HTLCs) per creare state channels.
    \end{itemize}
    \begin{itemize}
        \item \textbf{Nested blockchains}: si basa sulla soluzione chiamata Plasma, la quale si basa sulle seguenti caratteristiche:
        \begin{itemize}
            \item la rete principale (la main chain) stabilisce le regole di base dell’intero sistema e non prende parte alle operazioni svolte a meno che non debba risolvere dispute;
            \item il sistema si basa su un sistema in cui la main chain si appoggia a multipli livelli di blockchain. Questi livelli sono connessi tra loro in modo da formare una struttura del tipo padre-figlio. 
            La chain padre delega il lavoro alla chain figlia in grado di eseguire quella particolare istruzione e che ritornerà poi il risultato indietro al padre.
            \item questo sistema permette di ridurre il carico della root chain e se costruita e usata correttamente è in grado di aumentare la scalabilità della rete esponenzialmente     
        \end{itemize}
    \end{itemize}
    \paragraph{Pro}
    \begin{itemize}
        \item Evita problemi di disordine per quanto riguarda i compiti di una blockchain;
        \item Evita di pagare inutili fees su microoperazioni e microtransazioni;
    \end{itemize}
    \paragraph{Contro}
    \begin{itemize}
        \item Minore controllo sui dati a basso livello;        
    \end{itemize}

\subsection{Blockchain alternative}
Visto i grandi difetti presenti nella blockchain Ethereum, in particolare il costo elevato delle sue gas fee e la lenta scalabilità, abbiamo preso in esame diverse blockchain e valutato la migliore 
rispettivamente alle esigenze del problema. Qui sotto sono riportati i pro e contro di ogni soluzione.
    \subsubsection{Avalanche}
    Avalanche è un protocollo che integra diverse blockchain, che permette la creazione di asset tokenizzati, l'esecuzione di smart contract e anche la creazione di blockchain private. 
    AVAX è il suo token di riferimento, utilizzato per governance, pagamento commissioni e gas per i contratti.
    \paragraph{Pro}
    \begin{itemize}
        \item Blockchain ad alta scalabilità;
        \item Fee molto basse;
    \end{itemize}
    \paragraph{Contro}
    \begin{itemize}
        \item Tecnologia meno diffusa e conosciuta rispetto ad Ethereum
        \item Progetto discretamente complesso.
    \end{itemize}
    
    \subsubsection{Solana}
    Solana è una blockchain open source, permissionless che garantisce un alto numero di transazioni al secondo. Queste caratteristiche sono ottenute principalmente dall’architettura a basso livello
    che utilizza ad esempio un proprio algoritmo di consenso “Proof-of-History” e una cifratura SHA-256.
    \paragraph{Pro}
    \begin{itemize}
        \item Transazioni molto veloci ($>$ 710000 tps);
        \item Fee molto basse;
        \item Progetto recente ma dal grande potenziale;
    \end{itemize}
    \paragraph{Contro}
    \begin{itemize}
        \item Non è una blockchain Ethereum compatibile, quindi gli smart contract non possono essere scritti in Solidity. Il linguaggio necessario assomiglia al C e a RUST ed è più complesso;
        \item I suoi punti di forza la rendono anche una scelta azzardata per un primo approccio al mondo delle blockchain;
    \end{itemize}
    
    \subsubsection{Moonriver}
    Questo tipo di soluzione non presenta particolari caratteristiche se non quella di implementare gli smart contract su una parachain. Una parachain è una blockchain “specializzata” che si connette ad esempio a Polkadot (un altro esempio è Kusama), quest’ultimo è un protocollo di rete che permette di trasferire non solo tokens ma anche dati tra più blockchain. Quindi Polkadot è a tutti gli effetti un’applicazione che permette di avere registri e computazioni cross-chain.
    Le operazioni tra queste vengono eseguite in parallelo e non sequenzialmente su una relay chain che funge da main chain del sistema, una chain rimane connessa finchè è in grado di compilare in codice Wasm* e aderisce alla Relay Chain API.
    \paragraph{Pro}
    \begin{itemize}
        \item Sviluppo su una parachain, tecnologia molto interessante e valida per impieghi futuri;
        \item Fee basse rispetto ad Ethereum;
        \item Ambiente Ethereum compatibile, quindi possibilità di usare Solidity per lo sviluppo di Smart Contract
        \item Progetto recente ma dal grande potenziale;
    \end{itemize}
    \paragraph{Contro}
    \begin{itemize}
        \item Blockchain molto meno conosciuta, quindi probabilmente si avrà una community di supporto più piccola
    \end{itemize}
    
    \subsubsection{Polygon}
    è un protocollo e un framework per costruire e connettere Ethereum-compatible blockchain. Nasce per risolvere i principali problemi legati alla rete Ethereum, ovvero basso throughput, gas fee elevate, mancanza di sovranità.
    Nonostante siano nate delle blockchain ethereum compatibili che sfruttano le qualità migliori dell’ecosistema ethereum, non c’è ancora un protocollo o framework che si occupi di metterle in comunicazione ad alto livello. Proprio a questo scopo nasce Polygon.

    \paragraph{Pro}
    \begin{itemize}
        \item Possibilità di sviluppare su una tecnologia ad alto livello;
        \item Ambiente Ethereum compatibile, quindi possibilità di usare Solidity per lo sviluppo di Smart Contract
    \end{itemize}
    \paragraph{Contro}
    \begin{itemize}
        \item Blockchain molto meno conosciuta, quindi probabilmente si avrà una community di supporto più piccola
        \item Polygon si appoggia comunque a Ethereum, questo significa che ottimizza le transazioni solo tra le chain connesse, in caso contrario si hanno comunque fee alte.
    \end{itemize}

    \subsubsection{Scelta: Fantom}
    è una piattaforma che ha i suoi punti di forza su high-performance, scalabilità e sicurezza negli smart contract. Nasce per risolvere le limitazioni delle precedenti piattaforme blockchain (come molte altre sue coetanee).
    Fantom è permissionless, decentralized e open-source.\\
    Usa un suo algoritmo di consenso aBFT (Lachesis) che permette di avere delle eccellenti prestazioni per quanto riguarda velocità, costo e sicurezza.
    E’ un tipo di algoritmo di consenso che permette di creare delle applicazioni peer-to-peer senza che lo sviluppatore debba creare il proprio network layer. Lachesis è:
    \begin{itemize}
        \item \textbf{asincrono}: i partecipanti hanno la libertà di processare comandi in momenti diversi;
        \item \textbf{leaderless}: nessun partecipante nella rete ha un ruolo speciale;
        \item \textbf{byzantine fault-tolerant}: supporta fino a un terzo dei nodi difettosi inclusi comportamenti dannosi; 
        \item \textbf{finale}: l’output dell’algoritmo di consenso può essere usato immediatamente, non c’è bisogno di aspettare per la conferma di blocchi (tempo medio di 1-2 secondi)
    \end{itemize}
    
    \paragraph{Pro}
    \begin{itemize}
        \item Transazioni molto veloci;
        \item Fee molto basse;
        \item Blockchain Ethereum virtual machine compatibile, quindi possibilità di usare Solidity come linguaggio per gli smart contract;
        \item Modulare, ovvero permette di convertire dApps sviluppare sulla rete Ethereum, di essere adattate in poco tempo.
    \end{itemize}
    \paragraph{Contro}
    \begin{itemize}
        \item Meno conosciuta rispetto ad Ethereum;
    \end{itemize}