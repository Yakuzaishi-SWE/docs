\section{Capitolato C2 - Shop Chain}\label{section:c2}

\subsection{Informazioni generali}
\begin{itemize}
	\item \textbf{Nome:} \textit{Shop Chain - Exchange Platform on	BlockChain};
	\item \textbf{Proponente:} \textit{Sync Lab};
	\item \textbf{Committente:} \textit{Prof. Tullio Vardanega e Prof. Riccardo Cardin}.
\end{itemize}

\subsection{Descrizione generale}
L'avvento delle tecnologie BlockChain\glo\ ha portato e porterà nei prossimi anni a grandi cambiamenti nella società. 
In particolare, ha aperto le porte a una nuova forma di finanza, la cosiddetta “DeFi” (Finanza Decentralizzata) che ha permesso a chiunque sia dotato di connessione internet di creare un Wallet\glo\ e possedere quindi criptovalute\glo.
Questo ha delineato due profili critici strettamente legati; da un lato il controllo del proprio portafoglio è passato completamente nelle mani dell'utente, dall'altro lato questo comporta la mancanza di un ente terzo che si occupi di gestire transazioni e offrire garanzie.
\newline
Nel capitolato in questione si vuole proprio risolvere questo problema, in uno scenario che comprende un e-commerce\glo\ basato su BlockChain in cui si vuole tutelare entrambe le parti coinvolte in un acquisto tramite criptovalute.

\subsection{Finalità del progetto}
Fondamentalmente si parla della realizzazione di un prototipo di una piattaforma integrabile con un “crypto-ecommerce”\glo\, che si occupi di gestire gli ordini dalle fasi di pagamento alla consegna.
\newline
Gli obietti sono riassunti dalle seguenti sezioni:
\begin{itemize}
	\item \textbf{Gestione dell'ordine}:
		\begin{itemize}
			\item Caricamento dell'ordine sulla BlockChain che deve avvenire tramite la firma di uno {smart contract}\glo\ tra acquirente e venditore;
			\item Verifica del pagamento da parte dell'acquirente;
			\item Elaborazione e spedizione dell'ordine;
			\item Verifica ricezione dell'ordine tramite QR code\glo\ usando una clausula dello smart contract;
			\item Invio del denaro sul wallet del venditore.
		\end{itemize}
	\item \textbf{Realizzazione degli smart contract} che avranno lo scopo di:
		\begin{itemize}
			\item registrare l'ordine (direttamente sulla BlockChain);
			\item trattenere l'ammontare dell'ordine in criptovalute;
			\item fungere da prova d'acquisto a tutela delle parti coinvolte.
		\end{itemize}
	\item \textbf{Applicazione web} divisa in due parti:
		\begin{itemize}
			\item parte admin che deve fornire le seguenti funzionalità:
			\begin{itemize}
				\item visualizzazione e gestione dello stato degli ordini;
			\end{itemize}
			\item parte cliente che deve permettere semplicemente di:
			\begin{itemize}
				\item effettuare pagamenti;
			\end{itemize}
		\end{itemize}
	\item \textbf{Applicazione mobile (o Web App)} con lo scopo di:
		\begin{itemize}
			\item registrare tramite QR code la corretta ricezione dell'ordine, consentendo lo sblocco del denaro depositato sullo smart contract.
		\end{itemize}
\end{itemize}

\subsection{Tecnologie utilizzate}
Il committente non impone nessun vincolo riguardo le tecnologie necessarie ai fini del progetto, ma ha espresso comunque delle preferenze:
\begin{itemize}
	\item \textbf{utilizzo di BlockChain pubblica (come ad esempio Ethereum\glo)} con linguaggio Solidity\glo\ da usare per la scrittura degli {smart contract}\glo;
	\item \textbf{Java Spring\glo\ e Angular\glo} per lo sviluppo rispettivamente delle parti di Back-end e di Front-end della componente Web Application del sistema;
	\item \textbf{PostgresSQL\glo} come DBMS\glo\ di riferimento;
	\item \textbf{Flutter\glo} come linguaggio di programmazione per lo sviluppo dell'app mobile.
\end{itemize}

\subsection{Aspetti positivi}
\begin{itemize}
	\item Il progetto offre la possibilità di trattare argomenti molto recenti e di grande rilevanza come la struttura BlockChain e relativo mondo;
	\item Gli obiettivi solo molto chiari e specifici;
	\item Totale libertà nella scelta delle tecnologie da utilizzare;
	\item L'azienda committente è molto disponibile e fornisce materiale di formazione.
\end{itemize}

\subsection{Criticità e fattori di rischio}
\begin{itemize}
	\item Gli argomenti nonostante siano molto interessanti sono poco conosciuti dalla maggior parte del gruppo;
	\item Il progetto richiederà un discreto investimento di risorse relativamente alla formazione sugli argomenti;
\end{itemize}

\subsection{Conclusione}
Il capitolato in questione ha guadagnato fin da subito l'attenzione della maggioranza del gruppo, in particolare per la possibilità di affrontare e approfondire la tematica della BlockChain, tecnologia sulla quale il gruppo crede vivamente in un'adozione futura.
La libertà di progettazione e selezione delle tecnologie da utilizzare è stato un altro motivo di coesione nella scelta della preferenza. Il fatto di non essere vincolati riguardo l'utilizzo della rete Ethereum, per esempio, offre la possibilità di valutare BlockChain
alternative come Polygon\glo, Avalanche\glo, Fantom\glo\ e confrontarne eventuali pregi e difetti, con l'obiettivo di un futuro utilizzo reale.
\newline
Inoltre il capitolato permetterà al gruppo di acquisire o consolidare le proprie conoscenze anche in ambito web, sia in ambito front-end che back-end. A seguito di quanto evidenziato, il gruppo \textit{Yakuzaishi} ha deciso di designare questo capitolato come prima scelta,
accompagnandola con una forte motivazione collettiva e personale nell'apprendiamento delle tecnologie sopra descritte.
