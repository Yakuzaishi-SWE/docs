\section{Capitolato C6 - Soldino}\label{section:c6}

\subsection{Descrizione generale}
Soldino è concepita come una piattaforma, idealmente fornita da un ente \es{Governo}, la quale deve permettere alle aziende la registrazione e la vendita di beni e servizi, in un sistema che pone particolare attenzione alla gestione dell’IVA. \\
L’ente superiore deve poter coniare e ridistribuire moneta ai cittadini, che viene usata per comprare/vendere beni e pagare tasse.

\subsection{Obiettivo del progetto}
Il capitolato prevede la costruzione di una Dapp sulla EVM che consenta il tracciamento automatico dell’IVA durante l’acquisto e la vendita di beni/servizi. Deve essere inoltre facilitato lo svolgimento delle usuali attività di gestione tasse \es{rendiconto trimestrale, gestione pagamenti, cambio dei tassi}.\\
Gli attori della piattaforma sono di tre tipi:
\begin{itemize}
	\item \textbf{Il Governo:} ha la possibilità di gestire la creazione e la distribuzione della moneta (tramite token) agli utenti connessi alla piattaforma e di tenere conto di tutti gli iscritti ad essa;
	\item\textbf{ Un’azienda:} può registrare la propria attività sulla piattaforma e successivamente essere in grado di commercializzare la sua attività;
	\item \textbf{Un privato cittadino:} può acquistare beni o servizi sulla piattaforma utilizzando token Cubit. Deve inoltre poter decidere di aprire un’ attività, effettuandone la registrazione.
\end{itemize}

\subsection{Tecnologie utilizzate}
\begin{itemize}
	\item \textbf{Solidity}: linguaggio utilizzato per la scrittura di smart contracts, il cui codice viene eseguito sulla EVM. Gli smart contracts permettono lo scambio di token tra parti;
	
	\item \textbf{Ethereum}: la piattaforma su cui vengono scritte le Dapps. Lo scopo di queste Dapps è di unire parti interessate a scambi (di beni/servizi o soldi) senza che sia presente un intermediario centralizzato \es{sistema delle banche, o sistema client-server};
	
	\item \textbf{HTML, CSS, Javascript}: linguaggi web per la creazione della UI relativa alla Dapp. Vengono posti dei vincoli sulle tecnologie specifiche e sui design pattern da adottare: ES8 per Javascript, framework React/Redux e Sass.
\end{itemize}
\subsection{Valutazione finale}
Le tecnologie e i temi sono altamente attuali e hanno colto l’interesse del gruppo; il progetto è atto a risolvere problemi importanti e ha un bacino di utenti molto ampio.\\
Non si può però tralasciare il fatto che la tecnologia blockchain sia ancora giovane e poco matura, e che non si sia ancora definita con un tratto distintivo pur avendo grande potenziale.\\
Nel capitolato si spazia in un mondo vasto di tecnologie e spunti di studio, addirittura troppi per riuscire ad essere contenuti in un progetto di questa taglia, che è quindi infine stato considerato troppo oneroso in termini di risorse e costi.


