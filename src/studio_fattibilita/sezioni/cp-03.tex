\section{Capitolato scelto C3 - CC4D}\label{section:c3}

\subsection{Informazioni generali}
	\begin {itemize}
		\item \textbf{Nome:} \textit{CC4D};
		\item \textbf{Proponente:} \textit{San Marco Informatica S.p.A.};
		\item \textbf{Committente:} \textit{Prof. Tullio Vardanega e Prof. Riccardo Cardin}.
	\end{itemize}

\subsection{Descrizione del capitolato}
	Il controllo statistico consente di contenere l'esito di un processo all'interno di specifici limiti, determinati attraverso lo studio della variazione naturale dei limiti del processo.\\
	Uno degli strumenti utilizzati a tale scopo sono le carte di controllo, le quali permettono di identificare visivamente tramite grafici tali variazioni.\\
	Questi controlli verranno effettuati su diverse misurazioni di pezzi prodotti da determinate macchine produttive.

\subsection{Finalità del progetto}
	Il capitolato richiede la creazione di:
	\begin{itemize}
		\item \textbf{Console Amministrativa}: WEBAPP per il censimento delle macchine produttive e le relative caratteristiche le quali determineranno la regolarità delle misurazioni ricevute;
		\item \textbf{API Raccolta Dati}: API per l'immisione delle misurazioni di una determinata caratteristica e salvataggio in un time series database\glo;
		\item \textbf{Motore di Calcolo}: responsabile della valutazione della conformità delle nuove misurazioni ed eventuale ricalcolo dei limiti sulla base delle ultime N rilevazioni;
		\item \textbf{Visualizzazione Dati}: WEBAPP per la visualizzazione a rotazione delle molteplici carte di controllo.
	\end{itemize}

\subsection{Tecnologie interessate}
	\begin{itemize}
		\item \textbf{React/Angular/Vue}: Frameworks consigliati per la WEBAPP admin e utente;
		\item \textbf{d3js}: Libreria grafica per la realizzazione dei grafici delle carte di controllo;
		\item \textbf{Java/NodeJS}: Linguaggi consigliati per il motore di calcolo e API collezione dati;
		\item \textbf{TSDB}: Time Series Database a scelta tra: MongoDb, MariaDb ColumnStore (MariaDb), TimeScale (Postgresql), ClickHouse.
	\end{itemize}

\subsection{Aspetti Positivi}
	Il sistema oltre al particolare caso proposto, in quanto generico potrebbe risultare utile per una vastità di applicazioni differenti.

\subsection{Criticità e fattori di rischio}
	Lo scoglio più grande deriva delle conoscenze che vanno oltre la nostra preparazione statistica di base, per la realizzazione ottimale del motore di calcolo.

\subsection{Conclusioni}
	Il team di sviluppo inizialmente espresse un grande interesse e curiosità verso questo capitolato nonostante le sue difficoltà, non è stato scelto in quanto non ha retto a confronto delle altre scelte. 
%\textbf{GLOSSARIO}: time series database: è un software ottimizzato per salvare e fornire dati definiti come coppia tempo e valore. 